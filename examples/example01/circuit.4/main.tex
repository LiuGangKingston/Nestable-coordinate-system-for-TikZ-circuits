% This is circuit 4 of example 01 of
% https://github.com/LiuGangKingston/Nestable-coordinate-system-for-TikZ-circuits.git


\documentclass[tikz,border=5mm]{standalone}
\usepackage[siunitx]{circuitikz}
\usetikzlibrary{shapes,arrows,positioning}
%   This is an accessory  file for 
%   https://github.com/LiuGangKingston/Nestable-coordinate-system-for-Tikz-circuits.git
%            Version 1.0
%   free for non-commercial use.
%   Please send us emails for any problems/suggestions/comments.
%   Please be advised that none of us accept any responsibility
%   for any consequences arising out of the usage of this
%   software, especially for damage.
%   For usage, please refer to the README file and the following lines.
%   This code was written by
%        Gang Liu (gl.cell@outlook)
%                 (http://orcid.org/0000-0003-1575-9290)
%          and
%        Shiwei Huang (huang937@gmail.com)
%   Copyright (c) 2021
%
%
%  The following command is to get the x-component and y-component 
%  of a coordinate. The command is
%  \getxyofcoordinate{the coordinate}{x-component}{y-component};
\newcommand{\getxyofcoordinate}[3]{%
\coordinate (tempcoord) at ($#1$);
\path (tempcoord) node {};
\pgfgetlastxy{\tempx}{\tempy};
\pgfmathsetmacro{#2}{\tempx}
\pgfmathsetmacro{#3}{\tempy}
}


%  The following command is the same as above but for given unit.
%  The command is
%  \getxyingivenunit{the unit like cm}{the coordinate}{x-component}{y-component};
\newcommand{\getxyingivenunit}[4]{%
\coordinate (tempcoord) at (1#1,1#1);
\path (tempcoord) node {};
\pgfgetlastxy{\tempxunit}{\tempyunit};
\coordinate (tempcoord) at ($#2$);
\path (tempcoord) node {};
\pgfgetlastxy{\tempx}{\tempy};
\pgfmathsetmacro{#3}{\tempx/\tempxunit}
\pgfmathsetmacro{#4}{\tempy/\tempyunit}
}


%  The following command is to print the value of a coordinate with some words at the first coordinate postion 
%  The command is
%  \printcoordinateat{the first coordinate}{the words}{the coordinate};
\newcommand{\printcoordinateat}[3]{%
\getxyingivenunit{cm}{#3}{\tempxx}{\tempyy}
\node at #1 {#2 ($\tempxx$, $\tempyy$).};
}


%  The following command is to print a keyworded coordinate system as a background.
%  The command is
%  \coordinatebackground{the KEYWORD}
%                                            {the first letter in both x and y directions}
%                                       {the second letter in both x and y directions}
%                                             {the last letter in both x and y directions};
\newcommand{\coordinatebackground}[4]{
\pgfmathsetmacro{\colourpercent}{30}
\foreach \i in {#2,#3,...,#4} 
{\node [black!\colourpercent] at (#1ppp\i\i) {\i};}
\foreach \i in {#2,#4} 
{\node [white] at (#1ppp\i\i) {\i};}
\coordinatebackgroundxy{#1}{#2}{#3}{#4}{#2}{#3}{#4};
}


%  The following command is to print a keyworded coordinate system as a background.
%  The command is
%  \coordinatebackgroundxy{the KEYWORD}
%                                                {the first letter in the x direction}
%                                           {the second letter in the x direction}
%                                                 {the last letter in the x direction}
%                                                {the first letter in the y direction}
%                                           {the second letter in the y direction}
%                                                 {the last letter in the y direction};
\newcommand{\coordinatebackgroundxy}[7]{
\pgfmathsetmacro{\bordercolourpercent}{60}
\pgfmathsetmacro{\colourpercent}{30}

\foreach \i in {#2,#3,...,#4} 
\foreach \j in {#5} 
\foreach \k in {#7} 
{\draw [dashed,black!\colourpercent] (#1ppp\i\j) -- (#1ppp\i\k);}

\foreach \i in {#5,#6,...,#7} 
\foreach \j in {#2} 
\foreach \k in {#4} 
{\draw [dashed,black!\colourpercent] (#1ppp\j\i) -- (#1ppp\k\i);}

\foreach \i in {#2,#4} 
\foreach \j in {#5} 
\foreach \k in {#7} 
{\draw [dashed,black!\bordercolourpercent] (#1ppp\i\j) -- (#1ppp\i\k);}

\foreach \i in {#5,#7} 
\foreach \j in {#2} 
\foreach \k in {#4} 
{\draw [dashed,black!\bordercolourpercent] (#1ppp\j\i) -- (#1ppp\k\i);}

\foreach \i in {#2,#3,...,#4} 
\foreach \j in {#5} 
\foreach \k in {#7} 
{
\node [black!\bordercolourpercent] at ($(#1ppp\i\j) + (0,-.2)$) {\i};
\node [black!\bordercolourpercent] at ($(#1ppp\i\k) + (0,.2)$) {\i};
}

\foreach \i in {#5,#6,...,#7} 
\foreach \j in {#2} 
\foreach \k in {#4} 
{
\node [black!\bordercolourpercent] at ($(#1ppp\k\i) + (.2,0)$) {\i};
\node [black!\bordercolourpercent] at ($(#1ppp\j\i) + (-.2,0)$) {\i};
}

}






\begin{document}

\ctikzset{
/tikz/circuitikz/bipoles/length=1cm
}



 
 
\begin{circuitikz} [scale=0.8]
 
%%%%%% The next line is for circuit 4.
% https://github.com/LiuGangKingston/Nestable-coordinate-system-for-Tikz-circuits.git
% https://github.com/LiuGangKingston/Nestable-coordinate-system-for-Tikz-circuits.git

https://github.com/LiuGangKingston/Nestable-coordinate-system-for-Tikz-circuits.git
https://github.com/LiuGangKingston/Nestable-coordinate-system-for-Tikz-circuits.git

% https://github.com/LiuGangKingston/Nestable-coordinate-system-for-Tikz-circuits.git
% https://github.com/LiuGangKingston/Nestable-coordinate-system-for-Tikz-circuits.git


\pgfmathsetmacro{\totalgangliuxxx}{26}
\pgfmathsetmacro{\totalgangliuyyy}{26}
\pgfmathsetmacro{\gangliuxxxspacing}{1}
\pgfmathsetmacro{\gangliuyyyspacing}{1}
\pgfmathsetmacro{\gangliuxxxa}{-8}
\pgfmathsetmacro{\gangliuyyya}{-8}

\pgfmathsetmacro{\gangliuxxxb}{\gangliuxxxa + \gangliuxxxspacing + 0.0 }
\pgfmathsetmacro{\gangliuxxxc}{\gangliuxxxb + \gangliuxxxspacing + 0.0 }
\pgfmathsetmacro{\gangliuxxxd}{\gangliuxxxc + \gangliuxxxspacing + 0.0 }
\pgfmathsetmacro{\gangliuxxxe}{\gangliuxxxd + \gangliuxxxspacing + 0.0 }
\pgfmathsetmacro{\gangliuxxxf}{\gangliuxxxe + \gangliuxxxspacing + 0.0 }
\pgfmathsetmacro{\gangliuxxxg}{\gangliuxxxf + \gangliuxxxspacing + 0.0 }
\pgfmathsetmacro{\gangliuxxxh}{\gangliuxxxg + \gangliuxxxspacing + 0.0 }
\pgfmathsetmacro{\gangliuxxxi}{\gangliuxxxh + \gangliuxxxspacing + 0.0 }
\pgfmathsetmacro{\gangliuxxxj}{\gangliuxxxi + \gangliuxxxspacing + 0.0 }
\pgfmathsetmacro{\gangliuxxxk}{\gangliuxxxj + \gangliuxxxspacing + 0.0 }
\pgfmathsetmacro{\gangliuxxxl}{\gangliuxxxk + \gangliuxxxspacing + 0.0 }
\pgfmathsetmacro{\gangliuxxxm}{\gangliuxxxl + \gangliuxxxspacing + 0.0 }
\pgfmathsetmacro{\gangliuxxxn}{\gangliuxxxm + \gangliuxxxspacing + 0.0 }
\pgfmathsetmacro{\gangliuxxxo}{\gangliuxxxn + \gangliuxxxspacing + 0.0 }
\pgfmathsetmacro{\gangliuxxxp}{\gangliuxxxo + \gangliuxxxspacing + 0.0 }
\pgfmathsetmacro{\gangliuxxxq}{\gangliuxxxp + \gangliuxxxspacing + 0.0 }
\pgfmathsetmacro{\gangliuxxxr}{\gangliuxxxq + \gangliuxxxspacing + 0.0 }
\pgfmathsetmacro{\gangliuxxxs}{\gangliuxxxr + \gangliuxxxspacing + 0.0 }
\pgfmathsetmacro{\gangliuxxxt}{\gangliuxxxs + \gangliuxxxspacing + 0.0 }
\pgfmathsetmacro{\gangliuxxxu}{\gangliuxxxt + \gangliuxxxspacing + 0.0 }
\pgfmathsetmacro{\gangliuxxxv}{\gangliuxxxu + \gangliuxxxspacing + 0.0 }
\pgfmathsetmacro{\gangliuxxxw}{\gangliuxxxv + \gangliuxxxspacing + 0.0 }
\pgfmathsetmacro{\gangliuxxxx}{\gangliuxxxw + \gangliuxxxspacing + 0.0 }
\pgfmathsetmacro{\gangliuxxxy}{\gangliuxxxx + \gangliuxxxspacing + 0.0 }
\pgfmathsetmacro{\gangliuxxxz}{\gangliuxxxy + \gangliuxxxspacing + 0.0 }

\pgfmathsetmacro{\gangliuyyyb}{\gangliuyyya + \gangliuyyyspacing + 0.0 }
\pgfmathsetmacro{\gangliuyyyc}{\gangliuyyyb + \gangliuyyyspacing + 0.0 }
\pgfmathsetmacro{\gangliuyyyd}{\gangliuyyyc + \gangliuyyyspacing + 0.0 }
\pgfmathsetmacro{\gangliuyyye}{\gangliuyyyd + \gangliuyyyspacing + 0.0 }
\pgfmathsetmacro{\gangliuyyyf}{\gangliuyyye + \gangliuyyyspacing + 0.0 }
\pgfmathsetmacro{\gangliuyyyg}{\gangliuyyyf + \gangliuyyyspacing + 0.0 }
\pgfmathsetmacro{\gangliuyyyh}{\gangliuyyyg + \gangliuyyyspacing + 0.0 }
\pgfmathsetmacro{\gangliuyyyi}{\gangliuyyyh + \gangliuyyyspacing + 0.0 }
\pgfmathsetmacro{\gangliuyyyj}{\gangliuyyyi + \gangliuyyyspacing + 0.0 }
\pgfmathsetmacro{\gangliuyyyk}{\gangliuyyyj + \gangliuyyyspacing + 0.0 }
\pgfmathsetmacro{\gangliuyyyl}{\gangliuyyyk + \gangliuyyyspacing + 0.0 }
\pgfmathsetmacro{\gangliuyyym}{\gangliuyyyl + \gangliuyyyspacing + 0.0 }
\pgfmathsetmacro{\gangliuyyyn}{\gangliuyyym + \gangliuyyyspacing + 0.0 }
\pgfmathsetmacro{\gangliuyyyo}{\gangliuyyyn + \gangliuyyyspacing + 0.0 }
\pgfmathsetmacro{\gangliuyyyp}{\gangliuyyyo + \gangliuyyyspacing + 0.0 }
\pgfmathsetmacro{\gangliuyyyq}{\gangliuyyyp + \gangliuyyyspacing + 0.0 }
\pgfmathsetmacro{\gangliuyyyr}{\gangliuyyyq + \gangliuyyyspacing + 0.0 }
\pgfmathsetmacro{\gangliuyyys}{\gangliuyyyr + \gangliuyyyspacing + 0.0 }
\pgfmathsetmacro{\gangliuyyyt}{\gangliuyyys + \gangliuyyyspacing + 0.0 }
\pgfmathsetmacro{\gangliuyyyu}{\gangliuyyyt + \gangliuyyyspacing + 0.0 }
\pgfmathsetmacro{\gangliuyyyv}{\gangliuyyyu + \gangliuyyyspacing + 0.0 }
\pgfmathsetmacro{\gangliuyyyw}{\gangliuyyyv + \gangliuyyyspacing + 0.0 }
\pgfmathsetmacro{\gangliuyyyx}{\gangliuyyyw + \gangliuyyyspacing + 0.0 }
\pgfmathsetmacro{\gangliuyyyy}{\gangliuyyyx + \gangliuyyyspacing + 0.0 }
\pgfmathsetmacro{\gangliuyyyz}{\gangliuyyyy + \gangliuyyyspacing + 0.0 }

\coordinate (gangliupppaa) at (\gangliuxxxa, \gangliuyyya);
\coordinate (gangliupppab) at (\gangliuxxxa, \gangliuyyyb);
\coordinate (gangliupppac) at (\gangliuxxxa, \gangliuyyyc);
\coordinate (gangliupppad) at (\gangliuxxxa, \gangliuyyyd);
\coordinate (gangliupppae) at (\gangliuxxxa, \gangliuyyye);
\coordinate (gangliupppaf) at (\gangliuxxxa, \gangliuyyyf);
\coordinate (gangliupppag) at (\gangliuxxxa, \gangliuyyyg);
\coordinate (gangliupppah) at (\gangliuxxxa, \gangliuyyyh);
\coordinate (gangliupppai) at (\gangliuxxxa, \gangliuyyyi);
\coordinate (gangliupppaj) at (\gangliuxxxa, \gangliuyyyj);
\coordinate (gangliupppak) at (\gangliuxxxa, \gangliuyyyk);
\coordinate (gangliupppal) at (\gangliuxxxa, \gangliuyyyl);
\coordinate (gangliupppam) at (\gangliuxxxa, \gangliuyyym);
\coordinate (gangliupppan) at (\gangliuxxxa, \gangliuyyyn);
\coordinate (gangliupppao) at (\gangliuxxxa, \gangliuyyyo);
\coordinate (gangliupppap) at (\gangliuxxxa, \gangliuyyyp);
\coordinate (gangliupppaq) at (\gangliuxxxa, \gangliuyyyq);
\coordinate (gangliupppar) at (\gangliuxxxa, \gangliuyyyr);
\coordinate (gangliupppas) at (\gangliuxxxa, \gangliuyyys);
\coordinate (gangliupppat) at (\gangliuxxxa, \gangliuyyyt);
\coordinate (gangliupppau) at (\gangliuxxxa, \gangliuyyyu);
\coordinate (gangliupppav) at (\gangliuxxxa, \gangliuyyyv);
\coordinate (gangliupppaw) at (\gangliuxxxa, \gangliuyyyw);
\coordinate (gangliupppax) at (\gangliuxxxa, \gangliuyyyx);
\coordinate (gangliupppay) at (\gangliuxxxa, \gangliuyyyy);
\coordinate (gangliupppaz) at (\gangliuxxxa, \gangliuyyyz);
\coordinate (gangliupppba) at (\gangliuxxxb, \gangliuyyya);
\coordinate (gangliupppbb) at (\gangliuxxxb, \gangliuyyyb);
\coordinate (gangliupppbc) at (\gangliuxxxb, \gangliuyyyc);
\coordinate (gangliupppbd) at (\gangliuxxxb, \gangliuyyyd);
\coordinate (gangliupppbe) at (\gangliuxxxb, \gangliuyyye);
\coordinate (gangliupppbf) at (\gangliuxxxb, \gangliuyyyf);
\coordinate (gangliupppbg) at (\gangliuxxxb, \gangliuyyyg);
\coordinate (gangliupppbh) at (\gangliuxxxb, \gangliuyyyh);
\coordinate (gangliupppbi) at (\gangliuxxxb, \gangliuyyyi);
\coordinate (gangliupppbj) at (\gangliuxxxb, \gangliuyyyj);
\coordinate (gangliupppbk) at (\gangliuxxxb, \gangliuyyyk);
\coordinate (gangliupppbl) at (\gangliuxxxb, \gangliuyyyl);
\coordinate (gangliupppbm) at (\gangliuxxxb, \gangliuyyym);
\coordinate (gangliupppbn) at (\gangliuxxxb, \gangliuyyyn);
\coordinate (gangliupppbo) at (\gangliuxxxb, \gangliuyyyo);
\coordinate (gangliupppbp) at (\gangliuxxxb, \gangliuyyyp);
\coordinate (gangliupppbq) at (\gangliuxxxb, \gangliuyyyq);
\coordinate (gangliupppbr) at (\gangliuxxxb, \gangliuyyyr);
\coordinate (gangliupppbs) at (\gangliuxxxb, \gangliuyyys);
\coordinate (gangliupppbt) at (\gangliuxxxb, \gangliuyyyt);
\coordinate (gangliupppbu) at (\gangliuxxxb, \gangliuyyyu);
\coordinate (gangliupppbv) at (\gangliuxxxb, \gangliuyyyv);
\coordinate (gangliupppbw) at (\gangliuxxxb, \gangliuyyyw);
\coordinate (gangliupppbx) at (\gangliuxxxb, \gangliuyyyx);
\coordinate (gangliupppby) at (\gangliuxxxb, \gangliuyyyy);
\coordinate (gangliupppbz) at (\gangliuxxxb, \gangliuyyyz);
\coordinate (gangliupppca) at (\gangliuxxxc, \gangliuyyya);
\coordinate (gangliupppcb) at (\gangliuxxxc, \gangliuyyyb);
\coordinate (gangliupppcc) at (\gangliuxxxc, \gangliuyyyc);
\coordinate (gangliupppcd) at (\gangliuxxxc, \gangliuyyyd);
\coordinate (gangliupppce) at (\gangliuxxxc, \gangliuyyye);
\coordinate (gangliupppcf) at (\gangliuxxxc, \gangliuyyyf);
\coordinate (gangliupppcg) at (\gangliuxxxc, \gangliuyyyg);
\coordinate (gangliupppch) at (\gangliuxxxc, \gangliuyyyh);
\coordinate (gangliupppci) at (\gangliuxxxc, \gangliuyyyi);
\coordinate (gangliupppcj) at (\gangliuxxxc, \gangliuyyyj);
\coordinate (gangliupppck) at (\gangliuxxxc, \gangliuyyyk);
\coordinate (gangliupppcl) at (\gangliuxxxc, \gangliuyyyl);
\coordinate (gangliupppcm) at (\gangliuxxxc, \gangliuyyym);
\coordinate (gangliupppcn) at (\gangliuxxxc, \gangliuyyyn);
\coordinate (gangliupppco) at (\gangliuxxxc, \gangliuyyyo);
\coordinate (gangliupppcp) at (\gangliuxxxc, \gangliuyyyp);
\coordinate (gangliupppcq) at (\gangliuxxxc, \gangliuyyyq);
\coordinate (gangliupppcr) at (\gangliuxxxc, \gangliuyyyr);
\coordinate (gangliupppcs) at (\gangliuxxxc, \gangliuyyys);
\coordinate (gangliupppct) at (\gangliuxxxc, \gangliuyyyt);
\coordinate (gangliupppcu) at (\gangliuxxxc, \gangliuyyyu);
\coordinate (gangliupppcv) at (\gangliuxxxc, \gangliuyyyv);
\coordinate (gangliupppcw) at (\gangliuxxxc, \gangliuyyyw);
\coordinate (gangliupppcx) at (\gangliuxxxc, \gangliuyyyx);
\coordinate (gangliupppcy) at (\gangliuxxxc, \gangliuyyyy);
\coordinate (gangliupppcz) at (\gangliuxxxc, \gangliuyyyz);
\coordinate (gangliupppda) at (\gangliuxxxd, \gangliuyyya);
\coordinate (gangliupppdb) at (\gangliuxxxd, \gangliuyyyb);
\coordinate (gangliupppdc) at (\gangliuxxxd, \gangliuyyyc);
\coordinate (gangliupppdd) at (\gangliuxxxd, \gangliuyyyd);
\coordinate (gangliupppde) at (\gangliuxxxd, \gangliuyyye);
\coordinate (gangliupppdf) at (\gangliuxxxd, \gangliuyyyf);
\coordinate (gangliupppdg) at (\gangliuxxxd, \gangliuyyyg);
\coordinate (gangliupppdh) at (\gangliuxxxd, \gangliuyyyh);
\coordinate (gangliupppdi) at (\gangliuxxxd, \gangliuyyyi);
\coordinate (gangliupppdj) at (\gangliuxxxd, \gangliuyyyj);
\coordinate (gangliupppdk) at (\gangliuxxxd, \gangliuyyyk);
\coordinate (gangliupppdl) at (\gangliuxxxd, \gangliuyyyl);
\coordinate (gangliupppdm) at (\gangliuxxxd, \gangliuyyym);
\coordinate (gangliupppdn) at (\gangliuxxxd, \gangliuyyyn);
\coordinate (gangliupppdo) at (\gangliuxxxd, \gangliuyyyo);
\coordinate (gangliupppdp) at (\gangliuxxxd, \gangliuyyyp);
\coordinate (gangliupppdq) at (\gangliuxxxd, \gangliuyyyq);
\coordinate (gangliupppdr) at (\gangliuxxxd, \gangliuyyyr);
\coordinate (gangliupppds) at (\gangliuxxxd, \gangliuyyys);
\coordinate (gangliupppdt) at (\gangliuxxxd, \gangliuyyyt);
\coordinate (gangliupppdu) at (\gangliuxxxd, \gangliuyyyu);
\coordinate (gangliupppdv) at (\gangliuxxxd, \gangliuyyyv);
\coordinate (gangliupppdw) at (\gangliuxxxd, \gangliuyyyw);
\coordinate (gangliupppdx) at (\gangliuxxxd, \gangliuyyyx);
\coordinate (gangliupppdy) at (\gangliuxxxd, \gangliuyyyy);
\coordinate (gangliupppdz) at (\gangliuxxxd, \gangliuyyyz);
\coordinate (gangliupppea) at (\gangliuxxxe, \gangliuyyya);
\coordinate (gangliupppeb) at (\gangliuxxxe, \gangliuyyyb);
\coordinate (gangliupppec) at (\gangliuxxxe, \gangliuyyyc);
\coordinate (gangliuppped) at (\gangliuxxxe, \gangliuyyyd);
\coordinate (gangliupppee) at (\gangliuxxxe, \gangliuyyye);
\coordinate (gangliupppef) at (\gangliuxxxe, \gangliuyyyf);
\coordinate (gangliupppeg) at (\gangliuxxxe, \gangliuyyyg);
\coordinate (gangliupppeh) at (\gangliuxxxe, \gangliuyyyh);
\coordinate (gangliupppei) at (\gangliuxxxe, \gangliuyyyi);
\coordinate (gangliupppej) at (\gangliuxxxe, \gangliuyyyj);
\coordinate (gangliupppek) at (\gangliuxxxe, \gangliuyyyk);
\coordinate (gangliupppel) at (\gangliuxxxe, \gangliuyyyl);
\coordinate (gangliupppem) at (\gangliuxxxe, \gangliuyyym);
\coordinate (gangliupppen) at (\gangliuxxxe, \gangliuyyyn);
\coordinate (gangliupppeo) at (\gangliuxxxe, \gangliuyyyo);
\coordinate (gangliupppep) at (\gangliuxxxe, \gangliuyyyp);
\coordinate (gangliupppeq) at (\gangliuxxxe, \gangliuyyyq);
\coordinate (gangliuppper) at (\gangliuxxxe, \gangliuyyyr);
\coordinate (gangliupppes) at (\gangliuxxxe, \gangliuyyys);
\coordinate (gangliupppet) at (\gangliuxxxe, \gangliuyyyt);
\coordinate (gangliupppeu) at (\gangliuxxxe, \gangliuyyyu);
\coordinate (gangliupppev) at (\gangliuxxxe, \gangliuyyyv);
\coordinate (gangliupppew) at (\gangliuxxxe, \gangliuyyyw);
\coordinate (gangliupppex) at (\gangliuxxxe, \gangliuyyyx);
\coordinate (gangliupppey) at (\gangliuxxxe, \gangliuyyyy);
\coordinate (gangliupppez) at (\gangliuxxxe, \gangliuyyyz);
\coordinate (gangliupppfa) at (\gangliuxxxf, \gangliuyyya);
\coordinate (gangliupppfb) at (\gangliuxxxf, \gangliuyyyb);
\coordinate (gangliupppfc) at (\gangliuxxxf, \gangliuyyyc);
\coordinate (gangliupppfd) at (\gangliuxxxf, \gangliuyyyd);
\coordinate (gangliupppfe) at (\gangliuxxxf, \gangliuyyye);
\coordinate (gangliupppff) at (\gangliuxxxf, \gangliuyyyf);
\coordinate (gangliupppfg) at (\gangliuxxxf, \gangliuyyyg);
\coordinate (gangliupppfh) at (\gangliuxxxf, \gangliuyyyh);
\coordinate (gangliupppfi) at (\gangliuxxxf, \gangliuyyyi);
\coordinate (gangliupppfj) at (\gangliuxxxf, \gangliuyyyj);
\coordinate (gangliupppfk) at (\gangliuxxxf, \gangliuyyyk);
\coordinate (gangliupppfl) at (\gangliuxxxf, \gangliuyyyl);
\coordinate (gangliupppfm) at (\gangliuxxxf, \gangliuyyym);
\coordinate (gangliupppfn) at (\gangliuxxxf, \gangliuyyyn);
\coordinate (gangliupppfo) at (\gangliuxxxf, \gangliuyyyo);
\coordinate (gangliupppfp) at (\gangliuxxxf, \gangliuyyyp);
\coordinate (gangliupppfq) at (\gangliuxxxf, \gangliuyyyq);
\coordinate (gangliupppfr) at (\gangliuxxxf, \gangliuyyyr);
\coordinate (gangliupppfs) at (\gangliuxxxf, \gangliuyyys);
\coordinate (gangliupppft) at (\gangliuxxxf, \gangliuyyyt);
\coordinate (gangliupppfu) at (\gangliuxxxf, \gangliuyyyu);
\coordinate (gangliupppfv) at (\gangliuxxxf, \gangliuyyyv);
\coordinate (gangliupppfw) at (\gangliuxxxf, \gangliuyyyw);
\coordinate (gangliupppfx) at (\gangliuxxxf, \gangliuyyyx);
\coordinate (gangliupppfy) at (\gangliuxxxf, \gangliuyyyy);
\coordinate (gangliupppfz) at (\gangliuxxxf, \gangliuyyyz);
\coordinate (gangliupppga) at (\gangliuxxxg, \gangliuyyya);
\coordinate (gangliupppgb) at (\gangliuxxxg, \gangliuyyyb);
\coordinate (gangliupppgc) at (\gangliuxxxg, \gangliuyyyc);
\coordinate (gangliupppgd) at (\gangliuxxxg, \gangliuyyyd);
\coordinate (gangliupppge) at (\gangliuxxxg, \gangliuyyye);
\coordinate (gangliupppgf) at (\gangliuxxxg, \gangliuyyyf);
\coordinate (gangliupppgg) at (\gangliuxxxg, \gangliuyyyg);
\coordinate (gangliupppgh) at (\gangliuxxxg, \gangliuyyyh);
\coordinate (gangliupppgi) at (\gangliuxxxg, \gangliuyyyi);
\coordinate (gangliupppgj) at (\gangliuxxxg, \gangliuyyyj);
\coordinate (gangliupppgk) at (\gangliuxxxg, \gangliuyyyk);
\coordinate (gangliupppgl) at (\gangliuxxxg, \gangliuyyyl);
\coordinate (gangliupppgm) at (\gangliuxxxg, \gangliuyyym);
\coordinate (gangliupppgn) at (\gangliuxxxg, \gangliuyyyn);
\coordinate (gangliupppgo) at (\gangliuxxxg, \gangliuyyyo);
\coordinate (gangliupppgp) at (\gangliuxxxg, \gangliuyyyp);
\coordinate (gangliupppgq) at (\gangliuxxxg, \gangliuyyyq);
\coordinate (gangliupppgr) at (\gangliuxxxg, \gangliuyyyr);
\coordinate (gangliupppgs) at (\gangliuxxxg, \gangliuyyys);
\coordinate (gangliupppgt) at (\gangliuxxxg, \gangliuyyyt);
\coordinate (gangliupppgu) at (\gangliuxxxg, \gangliuyyyu);
\coordinate (gangliupppgv) at (\gangliuxxxg, \gangliuyyyv);
\coordinate (gangliupppgw) at (\gangliuxxxg, \gangliuyyyw);
\coordinate (gangliupppgx) at (\gangliuxxxg, \gangliuyyyx);
\coordinate (gangliupppgy) at (\gangliuxxxg, \gangliuyyyy);
\coordinate (gangliupppgz) at (\gangliuxxxg, \gangliuyyyz);
\coordinate (gangliupppha) at (\gangliuxxxh, \gangliuyyya);
\coordinate (gangliuppphb) at (\gangliuxxxh, \gangliuyyyb);
\coordinate (gangliuppphc) at (\gangliuxxxh, \gangliuyyyc);
\coordinate (gangliuppphd) at (\gangliuxxxh, \gangliuyyyd);
\coordinate (gangliuppphe) at (\gangliuxxxh, \gangliuyyye);
\coordinate (gangliuppphf) at (\gangliuxxxh, \gangliuyyyf);
\coordinate (gangliuppphg) at (\gangliuxxxh, \gangliuyyyg);
\coordinate (gangliuppphh) at (\gangliuxxxh, \gangliuyyyh);
\coordinate (gangliuppphi) at (\gangliuxxxh, \gangliuyyyi);
\coordinate (gangliuppphj) at (\gangliuxxxh, \gangliuyyyj);
\coordinate (gangliuppphk) at (\gangliuxxxh, \gangliuyyyk);
\coordinate (gangliuppphl) at (\gangliuxxxh, \gangliuyyyl);
\coordinate (gangliuppphm) at (\gangliuxxxh, \gangliuyyym);
\coordinate (gangliuppphn) at (\gangliuxxxh, \gangliuyyyn);
\coordinate (gangliupppho) at (\gangliuxxxh, \gangliuyyyo);
\coordinate (gangliuppphp) at (\gangliuxxxh, \gangliuyyyp);
\coordinate (gangliuppphq) at (\gangliuxxxh, \gangliuyyyq);
\coordinate (gangliuppphr) at (\gangliuxxxh, \gangliuyyyr);
\coordinate (gangliuppphs) at (\gangliuxxxh, \gangliuyyys);
\coordinate (gangliupppht) at (\gangliuxxxh, \gangliuyyyt);
\coordinate (gangliuppphu) at (\gangliuxxxh, \gangliuyyyu);
\coordinate (gangliuppphv) at (\gangliuxxxh, \gangliuyyyv);
\coordinate (gangliuppphw) at (\gangliuxxxh, \gangliuyyyw);
\coordinate (gangliuppphx) at (\gangliuxxxh, \gangliuyyyx);
\coordinate (gangliuppphy) at (\gangliuxxxh, \gangliuyyyy);
\coordinate (gangliuppphz) at (\gangliuxxxh, \gangliuyyyz);
\coordinate (gangliupppia) at (\gangliuxxxi, \gangliuyyya);
\coordinate (gangliupppib) at (\gangliuxxxi, \gangliuyyyb);
\coordinate (gangliupppic) at (\gangliuxxxi, \gangliuyyyc);
\coordinate (gangliupppid) at (\gangliuxxxi, \gangliuyyyd);
\coordinate (gangliupppie) at (\gangliuxxxi, \gangliuyyye);
\coordinate (gangliupppif) at (\gangliuxxxi, \gangliuyyyf);
\coordinate (gangliupppig) at (\gangliuxxxi, \gangliuyyyg);
\coordinate (gangliupppih) at (\gangliuxxxi, \gangliuyyyh);
\coordinate (gangliupppii) at (\gangliuxxxi, \gangliuyyyi);
\coordinate (gangliupppij) at (\gangliuxxxi, \gangliuyyyj);
\coordinate (gangliupppik) at (\gangliuxxxi, \gangliuyyyk);
\coordinate (gangliupppil) at (\gangliuxxxi, \gangliuyyyl);
\coordinate (gangliupppim) at (\gangliuxxxi, \gangliuyyym);
\coordinate (gangliupppin) at (\gangliuxxxi, \gangliuyyyn);
\coordinate (gangliupppio) at (\gangliuxxxi, \gangliuyyyo);
\coordinate (gangliupppip) at (\gangliuxxxi, \gangliuyyyp);
\coordinate (gangliupppiq) at (\gangliuxxxi, \gangliuyyyq);
\coordinate (gangliupppir) at (\gangliuxxxi, \gangliuyyyr);
\coordinate (gangliupppis) at (\gangliuxxxi, \gangliuyyys);
\coordinate (gangliupppit) at (\gangliuxxxi, \gangliuyyyt);
\coordinate (gangliupppiu) at (\gangliuxxxi, \gangliuyyyu);
\coordinate (gangliupppiv) at (\gangliuxxxi, \gangliuyyyv);
\coordinate (gangliupppiw) at (\gangliuxxxi, \gangliuyyyw);
\coordinate (gangliupppix) at (\gangliuxxxi, \gangliuyyyx);
\coordinate (gangliupppiy) at (\gangliuxxxi, \gangliuyyyy);
\coordinate (gangliupppiz) at (\gangliuxxxi, \gangliuyyyz);
\coordinate (gangliupppja) at (\gangliuxxxj, \gangliuyyya);
\coordinate (gangliupppjb) at (\gangliuxxxj, \gangliuyyyb);
\coordinate (gangliupppjc) at (\gangliuxxxj, \gangliuyyyc);
\coordinate (gangliupppjd) at (\gangliuxxxj, \gangliuyyyd);
\coordinate (gangliupppje) at (\gangliuxxxj, \gangliuyyye);
\coordinate (gangliupppjf) at (\gangliuxxxj, \gangliuyyyf);
\coordinate (gangliupppjg) at (\gangliuxxxj, \gangliuyyyg);
\coordinate (gangliupppjh) at (\gangliuxxxj, \gangliuyyyh);
\coordinate (gangliupppji) at (\gangliuxxxj, \gangliuyyyi);
\coordinate (gangliupppjj) at (\gangliuxxxj, \gangliuyyyj);
\coordinate (gangliupppjk) at (\gangliuxxxj, \gangliuyyyk);
\coordinate (gangliupppjl) at (\gangliuxxxj, \gangliuyyyl);
\coordinate (gangliupppjm) at (\gangliuxxxj, \gangliuyyym);
\coordinate (gangliupppjn) at (\gangliuxxxj, \gangliuyyyn);
\coordinate (gangliupppjo) at (\gangliuxxxj, \gangliuyyyo);
\coordinate (gangliupppjp) at (\gangliuxxxj, \gangliuyyyp);
\coordinate (gangliupppjq) at (\gangliuxxxj, \gangliuyyyq);
\coordinate (gangliupppjr) at (\gangliuxxxj, \gangliuyyyr);
\coordinate (gangliupppjs) at (\gangliuxxxj, \gangliuyyys);
\coordinate (gangliupppjt) at (\gangliuxxxj, \gangliuyyyt);
\coordinate (gangliupppju) at (\gangliuxxxj, \gangliuyyyu);
\coordinate (gangliupppjv) at (\gangliuxxxj, \gangliuyyyv);
\coordinate (gangliupppjw) at (\gangliuxxxj, \gangliuyyyw);
\coordinate (gangliupppjx) at (\gangliuxxxj, \gangliuyyyx);
\coordinate (gangliupppjy) at (\gangliuxxxj, \gangliuyyyy);
\coordinate (gangliupppjz) at (\gangliuxxxj, \gangliuyyyz);
\coordinate (gangliupppka) at (\gangliuxxxk, \gangliuyyya);
\coordinate (gangliupppkb) at (\gangliuxxxk, \gangliuyyyb);
\coordinate (gangliupppkc) at (\gangliuxxxk, \gangliuyyyc);
\coordinate (gangliupppkd) at (\gangliuxxxk, \gangliuyyyd);
\coordinate (gangliupppke) at (\gangliuxxxk, \gangliuyyye);
\coordinate (gangliupppkf) at (\gangliuxxxk, \gangliuyyyf);
\coordinate (gangliupppkg) at (\gangliuxxxk, \gangliuyyyg);
\coordinate (gangliupppkh) at (\gangliuxxxk, \gangliuyyyh);
\coordinate (gangliupppki) at (\gangliuxxxk, \gangliuyyyi);
\coordinate (gangliupppkj) at (\gangliuxxxk, \gangliuyyyj);
\coordinate (gangliupppkk) at (\gangliuxxxk, \gangliuyyyk);
\coordinate (gangliupppkl) at (\gangliuxxxk, \gangliuyyyl);
\coordinate (gangliupppkm) at (\gangliuxxxk, \gangliuyyym);
\coordinate (gangliupppkn) at (\gangliuxxxk, \gangliuyyyn);
\coordinate (gangliupppko) at (\gangliuxxxk, \gangliuyyyo);
\coordinate (gangliupppkp) at (\gangliuxxxk, \gangliuyyyp);
\coordinate (gangliupppkq) at (\gangliuxxxk, \gangliuyyyq);
\coordinate (gangliupppkr) at (\gangliuxxxk, \gangliuyyyr);
\coordinate (gangliupppks) at (\gangliuxxxk, \gangliuyyys);
\coordinate (gangliupppkt) at (\gangliuxxxk, \gangliuyyyt);
\coordinate (gangliupppku) at (\gangliuxxxk, \gangliuyyyu);
\coordinate (gangliupppkv) at (\gangliuxxxk, \gangliuyyyv);
\coordinate (gangliupppkw) at (\gangliuxxxk, \gangliuyyyw);
\coordinate (gangliupppkx) at (\gangliuxxxk, \gangliuyyyx);
\coordinate (gangliupppky) at (\gangliuxxxk, \gangliuyyyy);
\coordinate (gangliupppkz) at (\gangliuxxxk, \gangliuyyyz);
\coordinate (gangliupppla) at (\gangliuxxxl, \gangliuyyya);
\coordinate (gangliuppplb) at (\gangliuxxxl, \gangliuyyyb);
\coordinate (gangliuppplc) at (\gangliuxxxl, \gangliuyyyc);
\coordinate (gangliupppld) at (\gangliuxxxl, \gangliuyyyd);
\coordinate (gangliuppple) at (\gangliuxxxl, \gangliuyyye);
\coordinate (gangliuppplf) at (\gangliuxxxl, \gangliuyyyf);
\coordinate (gangliuppplg) at (\gangliuxxxl, \gangliuyyyg);
\coordinate (gangliuppplh) at (\gangliuxxxl, \gangliuyyyh);
\coordinate (gangliupppli) at (\gangliuxxxl, \gangliuyyyi);
\coordinate (gangliuppplj) at (\gangliuxxxl, \gangliuyyyj);
\coordinate (gangliuppplk) at (\gangliuxxxl, \gangliuyyyk);
\coordinate (gangliupppll) at (\gangliuxxxl, \gangliuyyyl);
\coordinate (gangliuppplm) at (\gangliuxxxl, \gangliuyyym);
\coordinate (gangliupppln) at (\gangliuxxxl, \gangliuyyyn);
\coordinate (gangliuppplo) at (\gangliuxxxl, \gangliuyyyo);
\coordinate (gangliuppplp) at (\gangliuxxxl, \gangliuyyyp);
\coordinate (gangliuppplq) at (\gangliuxxxl, \gangliuyyyq);
\coordinate (gangliuppplr) at (\gangliuxxxl, \gangliuyyyr);
\coordinate (gangliupppls) at (\gangliuxxxl, \gangliuyyys);
\coordinate (gangliuppplt) at (\gangliuxxxl, \gangliuyyyt);
\coordinate (gangliuppplu) at (\gangliuxxxl, \gangliuyyyu);
\coordinate (gangliuppplv) at (\gangliuxxxl, \gangliuyyyv);
\coordinate (gangliuppplw) at (\gangliuxxxl, \gangliuyyyw);
\coordinate (gangliuppplx) at (\gangliuxxxl, \gangliuyyyx);
\coordinate (gangliuppply) at (\gangliuxxxl, \gangliuyyyy);
\coordinate (gangliuppplz) at (\gangliuxxxl, \gangliuyyyz);
\coordinate (gangliupppma) at (\gangliuxxxm, \gangliuyyya);
\coordinate (gangliupppmb) at (\gangliuxxxm, \gangliuyyyb);
\coordinate (gangliupppmc) at (\gangliuxxxm, \gangliuyyyc);
\coordinate (gangliupppmd) at (\gangliuxxxm, \gangliuyyyd);
\coordinate (gangliupppme) at (\gangliuxxxm, \gangliuyyye);
\coordinate (gangliupppmf) at (\gangliuxxxm, \gangliuyyyf);
\coordinate (gangliupppmg) at (\gangliuxxxm, \gangliuyyyg);
\coordinate (gangliupppmh) at (\gangliuxxxm, \gangliuyyyh);
\coordinate (gangliupppmi) at (\gangliuxxxm, \gangliuyyyi);
\coordinate (gangliupppmj) at (\gangliuxxxm, \gangliuyyyj);
\coordinate (gangliupppmk) at (\gangliuxxxm, \gangliuyyyk);
\coordinate (gangliupppml) at (\gangliuxxxm, \gangliuyyyl);
\coordinate (gangliupppmm) at (\gangliuxxxm, \gangliuyyym);
\coordinate (gangliupppmn) at (\gangliuxxxm, \gangliuyyyn);
\coordinate (gangliupppmo) at (\gangliuxxxm, \gangliuyyyo);
\coordinate (gangliupppmp) at (\gangliuxxxm, \gangliuyyyp);
\coordinate (gangliupppmq) at (\gangliuxxxm, \gangliuyyyq);
\coordinate (gangliupppmr) at (\gangliuxxxm, \gangliuyyyr);
\coordinate (gangliupppms) at (\gangliuxxxm, \gangliuyyys);
\coordinate (gangliupppmt) at (\gangliuxxxm, \gangliuyyyt);
\coordinate (gangliupppmu) at (\gangliuxxxm, \gangliuyyyu);
\coordinate (gangliupppmv) at (\gangliuxxxm, \gangliuyyyv);
\coordinate (gangliupppmw) at (\gangliuxxxm, \gangliuyyyw);
\coordinate (gangliupppmx) at (\gangliuxxxm, \gangliuyyyx);
\coordinate (gangliupppmy) at (\gangliuxxxm, \gangliuyyyy);
\coordinate (gangliupppmz) at (\gangliuxxxm, \gangliuyyyz);
\coordinate (gangliupppna) at (\gangliuxxxn, \gangliuyyya);
\coordinate (gangliupppnb) at (\gangliuxxxn, \gangliuyyyb);
\coordinate (gangliupppnc) at (\gangliuxxxn, \gangliuyyyc);
\coordinate (gangliupppnd) at (\gangliuxxxn, \gangliuyyyd);
\coordinate (gangliupppne) at (\gangliuxxxn, \gangliuyyye);
\coordinate (gangliupppnf) at (\gangliuxxxn, \gangliuyyyf);
\coordinate (gangliupppng) at (\gangliuxxxn, \gangliuyyyg);
\coordinate (gangliupppnh) at (\gangliuxxxn, \gangliuyyyh);
\coordinate (gangliupppni) at (\gangliuxxxn, \gangliuyyyi);
\coordinate (gangliupppnj) at (\gangliuxxxn, \gangliuyyyj);
\coordinate (gangliupppnk) at (\gangliuxxxn, \gangliuyyyk);
\coordinate (gangliupppnl) at (\gangliuxxxn, \gangliuyyyl);
\coordinate (gangliupppnm) at (\gangliuxxxn, \gangliuyyym);
\coordinate (gangliupppnn) at (\gangliuxxxn, \gangliuyyyn);
\coordinate (gangliupppno) at (\gangliuxxxn, \gangliuyyyo);
\coordinate (gangliupppnp) at (\gangliuxxxn, \gangliuyyyp);
\coordinate (gangliupppnq) at (\gangliuxxxn, \gangliuyyyq);
\coordinate (gangliupppnr) at (\gangliuxxxn, \gangliuyyyr);
\coordinate (gangliupppns) at (\gangliuxxxn, \gangliuyyys);
\coordinate (gangliupppnt) at (\gangliuxxxn, \gangliuyyyt);
\coordinate (gangliupppnu) at (\gangliuxxxn, \gangliuyyyu);
\coordinate (gangliupppnv) at (\gangliuxxxn, \gangliuyyyv);
\coordinate (gangliupppnw) at (\gangliuxxxn, \gangliuyyyw);
\coordinate (gangliupppnx) at (\gangliuxxxn, \gangliuyyyx);
\coordinate (gangliupppny) at (\gangliuxxxn, \gangliuyyyy);
\coordinate (gangliupppnz) at (\gangliuxxxn, \gangliuyyyz);
\coordinate (gangliupppoa) at (\gangliuxxxo, \gangliuyyya);
\coordinate (gangliupppob) at (\gangliuxxxo, \gangliuyyyb);
\coordinate (gangliupppoc) at (\gangliuxxxo, \gangliuyyyc);
\coordinate (gangliupppod) at (\gangliuxxxo, \gangliuyyyd);
\coordinate (gangliupppoe) at (\gangliuxxxo, \gangliuyyye);
\coordinate (gangliupppof) at (\gangliuxxxo, \gangliuyyyf);
\coordinate (gangliupppog) at (\gangliuxxxo, \gangliuyyyg);
\coordinate (gangliupppoh) at (\gangliuxxxo, \gangliuyyyh);
\coordinate (gangliupppoi) at (\gangliuxxxo, \gangliuyyyi);
\coordinate (gangliupppoj) at (\gangliuxxxo, \gangliuyyyj);
\coordinate (gangliupppok) at (\gangliuxxxo, \gangliuyyyk);
\coordinate (gangliupppol) at (\gangliuxxxo, \gangliuyyyl);
\coordinate (gangliupppom) at (\gangliuxxxo, \gangliuyyym);
\coordinate (gangliupppon) at (\gangliuxxxo, \gangliuyyyn);
\coordinate (gangliupppoo) at (\gangliuxxxo, \gangliuyyyo);
\coordinate (gangliupppop) at (\gangliuxxxo, \gangliuyyyp);
\coordinate (gangliupppoq) at (\gangliuxxxo, \gangliuyyyq);
\coordinate (gangliupppor) at (\gangliuxxxo, \gangliuyyyr);
\coordinate (gangliupppos) at (\gangliuxxxo, \gangliuyyys);
\coordinate (gangliupppot) at (\gangliuxxxo, \gangliuyyyt);
\coordinate (gangliupppou) at (\gangliuxxxo, \gangliuyyyu);
\coordinate (gangliupppov) at (\gangliuxxxo, \gangliuyyyv);
\coordinate (gangliupppow) at (\gangliuxxxo, \gangliuyyyw);
\coordinate (gangliupppox) at (\gangliuxxxo, \gangliuyyyx);
\coordinate (gangliupppoy) at (\gangliuxxxo, \gangliuyyyy);
\coordinate (gangliupppoz) at (\gangliuxxxo, \gangliuyyyz);
\coordinate (gangliuppppa) at (\gangliuxxxp, \gangliuyyya);
\coordinate (gangliuppppb) at (\gangliuxxxp, \gangliuyyyb);
\coordinate (gangliuppppc) at (\gangliuxxxp, \gangliuyyyc);
\coordinate (gangliuppppd) at (\gangliuxxxp, \gangliuyyyd);
\coordinate (gangliuppppe) at (\gangliuxxxp, \gangliuyyye);
\coordinate (gangliuppppf) at (\gangliuxxxp, \gangliuyyyf);
\coordinate (gangliuppppg) at (\gangliuxxxp, \gangliuyyyg);
\coordinate (gangliupppph) at (\gangliuxxxp, \gangliuyyyh);
\coordinate (gangliuppppi) at (\gangliuxxxp, \gangliuyyyi);
\coordinate (gangliuppppj) at (\gangliuxxxp, \gangliuyyyj);
\coordinate (gangliuppppk) at (\gangliuxxxp, \gangliuyyyk);
\coordinate (gangliuppppl) at (\gangliuxxxp, \gangliuyyyl);
\coordinate (gangliuppppm) at (\gangliuxxxp, \gangliuyyym);
\coordinate (gangliuppppn) at (\gangliuxxxp, \gangliuyyyn);
\coordinate (gangliuppppo) at (\gangliuxxxp, \gangliuyyyo);
\coordinate (gangliuppppp) at (\gangliuxxxp, \gangliuyyyp);
\coordinate (gangliuppppq) at (\gangliuxxxp, \gangliuyyyq);
\coordinate (gangliuppppr) at (\gangliuxxxp, \gangliuyyyr);
\coordinate (gangliupppps) at (\gangliuxxxp, \gangliuyyys);
\coordinate (gangliuppppt) at (\gangliuxxxp, \gangliuyyyt);
\coordinate (gangliuppppu) at (\gangliuxxxp, \gangliuyyyu);
\coordinate (gangliuppppv) at (\gangliuxxxp, \gangliuyyyv);
\coordinate (gangliuppppw) at (\gangliuxxxp, \gangliuyyyw);
\coordinate (gangliuppppx) at (\gangliuxxxp, \gangliuyyyx);
\coordinate (gangliuppppy) at (\gangliuxxxp, \gangliuyyyy);
\coordinate (gangliuppppz) at (\gangliuxxxp, \gangliuyyyz);
\coordinate (gangliupppqa) at (\gangliuxxxq, \gangliuyyya);
\coordinate (gangliupppqb) at (\gangliuxxxq, \gangliuyyyb);
\coordinate (gangliupppqc) at (\gangliuxxxq, \gangliuyyyc);
\coordinate (gangliupppqd) at (\gangliuxxxq, \gangliuyyyd);
\coordinate (gangliupppqe) at (\gangliuxxxq, \gangliuyyye);
\coordinate (gangliupppqf) at (\gangliuxxxq, \gangliuyyyf);
\coordinate (gangliupppqg) at (\gangliuxxxq, \gangliuyyyg);
\coordinate (gangliupppqh) at (\gangliuxxxq, \gangliuyyyh);
\coordinate (gangliupppqi) at (\gangliuxxxq, \gangliuyyyi);
\coordinate (gangliupppqj) at (\gangliuxxxq, \gangliuyyyj);
\coordinate (gangliupppqk) at (\gangliuxxxq, \gangliuyyyk);
\coordinate (gangliupppql) at (\gangliuxxxq, \gangliuyyyl);
\coordinate (gangliupppqm) at (\gangliuxxxq, \gangliuyyym);
\coordinate (gangliupppqn) at (\gangliuxxxq, \gangliuyyyn);
\coordinate (gangliupppqo) at (\gangliuxxxq, \gangliuyyyo);
\coordinate (gangliupppqp) at (\gangliuxxxq, \gangliuyyyp);
\coordinate (gangliupppqq) at (\gangliuxxxq, \gangliuyyyq);
\coordinate (gangliupppqr) at (\gangliuxxxq, \gangliuyyyr);
\coordinate (gangliupppqs) at (\gangliuxxxq, \gangliuyyys);
\coordinate (gangliupppqt) at (\gangliuxxxq, \gangliuyyyt);
\coordinate (gangliupppqu) at (\gangliuxxxq, \gangliuyyyu);
\coordinate (gangliupppqv) at (\gangliuxxxq, \gangliuyyyv);
\coordinate (gangliupppqw) at (\gangliuxxxq, \gangliuyyyw);
\coordinate (gangliupppqx) at (\gangliuxxxq, \gangliuyyyx);
\coordinate (gangliupppqy) at (\gangliuxxxq, \gangliuyyyy);
\coordinate (gangliupppqz) at (\gangliuxxxq, \gangliuyyyz);
\coordinate (gangliupppra) at (\gangliuxxxr, \gangliuyyya);
\coordinate (gangliuppprb) at (\gangliuxxxr, \gangliuyyyb);
\coordinate (gangliuppprc) at (\gangliuxxxr, \gangliuyyyc);
\coordinate (gangliuppprd) at (\gangliuxxxr, \gangliuyyyd);
\coordinate (gangliupppre) at (\gangliuxxxr, \gangliuyyye);
\coordinate (gangliuppprf) at (\gangliuxxxr, \gangliuyyyf);
\coordinate (gangliuppprg) at (\gangliuxxxr, \gangliuyyyg);
\coordinate (gangliuppprh) at (\gangliuxxxr, \gangliuyyyh);
\coordinate (gangliupppri) at (\gangliuxxxr, \gangliuyyyi);
\coordinate (gangliuppprj) at (\gangliuxxxr, \gangliuyyyj);
\coordinate (gangliuppprk) at (\gangliuxxxr, \gangliuyyyk);
\coordinate (gangliuppprl) at (\gangliuxxxr, \gangliuyyyl);
\coordinate (gangliuppprm) at (\gangliuxxxr, \gangliuyyym);
\coordinate (gangliuppprn) at (\gangliuxxxr, \gangliuyyyn);
\coordinate (gangliupppro) at (\gangliuxxxr, \gangliuyyyo);
\coordinate (gangliuppprp) at (\gangliuxxxr, \gangliuyyyp);
\coordinate (gangliuppprq) at (\gangliuxxxr, \gangliuyyyq);
\coordinate (gangliuppprr) at (\gangliuxxxr, \gangliuyyyr);
\coordinate (gangliuppprs) at (\gangliuxxxr, \gangliuyyys);
\coordinate (gangliuppprt) at (\gangliuxxxr, \gangliuyyyt);
\coordinate (gangliupppru) at (\gangliuxxxr, \gangliuyyyu);
\coordinate (gangliuppprv) at (\gangliuxxxr, \gangliuyyyv);
\coordinate (gangliuppprw) at (\gangliuxxxr, \gangliuyyyw);
\coordinate (gangliuppprx) at (\gangliuxxxr, \gangliuyyyx);
\coordinate (gangliupppry) at (\gangliuxxxr, \gangliuyyyy);
\coordinate (gangliuppprz) at (\gangliuxxxr, \gangliuyyyz);
\coordinate (gangliupppsa) at (\gangliuxxxs, \gangliuyyya);
\coordinate (gangliupppsb) at (\gangliuxxxs, \gangliuyyyb);
\coordinate (gangliupppsc) at (\gangliuxxxs, \gangliuyyyc);
\coordinate (gangliupppsd) at (\gangliuxxxs, \gangliuyyyd);
\coordinate (gangliupppse) at (\gangliuxxxs, \gangliuyyye);
\coordinate (gangliupppsf) at (\gangliuxxxs, \gangliuyyyf);
\coordinate (gangliupppsg) at (\gangliuxxxs, \gangliuyyyg);
\coordinate (gangliupppsh) at (\gangliuxxxs, \gangliuyyyh);
\coordinate (gangliupppsi) at (\gangliuxxxs, \gangliuyyyi);
\coordinate (gangliupppsj) at (\gangliuxxxs, \gangliuyyyj);
\coordinate (gangliupppsk) at (\gangliuxxxs, \gangliuyyyk);
\coordinate (gangliupppsl) at (\gangliuxxxs, \gangliuyyyl);
\coordinate (gangliupppsm) at (\gangliuxxxs, \gangliuyyym);
\coordinate (gangliupppsn) at (\gangliuxxxs, \gangliuyyyn);
\coordinate (gangliupppso) at (\gangliuxxxs, \gangliuyyyo);
\coordinate (gangliupppsp) at (\gangliuxxxs, \gangliuyyyp);
\coordinate (gangliupppsq) at (\gangliuxxxs, \gangliuyyyq);
\coordinate (gangliupppsr) at (\gangliuxxxs, \gangliuyyyr);
\coordinate (gangliupppss) at (\gangliuxxxs, \gangliuyyys);
\coordinate (gangliupppst) at (\gangliuxxxs, \gangliuyyyt);
\coordinate (gangliupppsu) at (\gangliuxxxs, \gangliuyyyu);
\coordinate (gangliupppsv) at (\gangliuxxxs, \gangliuyyyv);
\coordinate (gangliupppsw) at (\gangliuxxxs, \gangliuyyyw);
\coordinate (gangliupppsx) at (\gangliuxxxs, \gangliuyyyx);
\coordinate (gangliupppsy) at (\gangliuxxxs, \gangliuyyyy);
\coordinate (gangliupppsz) at (\gangliuxxxs, \gangliuyyyz);
\coordinate (gangliupppta) at (\gangliuxxxt, \gangliuyyya);
\coordinate (gangliuppptb) at (\gangliuxxxt, \gangliuyyyb);
\coordinate (gangliuppptc) at (\gangliuxxxt, \gangliuyyyc);
\coordinate (gangliuppptd) at (\gangliuxxxt, \gangliuyyyd);
\coordinate (gangliupppte) at (\gangliuxxxt, \gangliuyyye);
\coordinate (gangliuppptf) at (\gangliuxxxt, \gangliuyyyf);
\coordinate (gangliuppptg) at (\gangliuxxxt, \gangliuyyyg);
\coordinate (gangliupppth) at (\gangliuxxxt, \gangliuyyyh);
\coordinate (gangliupppti) at (\gangliuxxxt, \gangliuyyyi);
\coordinate (gangliuppptj) at (\gangliuxxxt, \gangliuyyyj);
\coordinate (gangliuppptk) at (\gangliuxxxt, \gangliuyyyk);
\coordinate (gangliuppptl) at (\gangliuxxxt, \gangliuyyyl);
\coordinate (gangliuppptm) at (\gangliuxxxt, \gangliuyyym);
\coordinate (gangliuppptn) at (\gangliuxxxt, \gangliuyyyn);
\coordinate (gangliupppto) at (\gangliuxxxt, \gangliuyyyo);
\coordinate (gangliuppptp) at (\gangliuxxxt, \gangliuyyyp);
\coordinate (gangliuppptq) at (\gangliuxxxt, \gangliuyyyq);
\coordinate (gangliuppptr) at (\gangliuxxxt, \gangliuyyyr);
\coordinate (gangliupppts) at (\gangliuxxxt, \gangliuyyys);
\coordinate (gangliuppptt) at (\gangliuxxxt, \gangliuyyyt);
\coordinate (gangliuppptu) at (\gangliuxxxt, \gangliuyyyu);
\coordinate (gangliuppptv) at (\gangliuxxxt, \gangliuyyyv);
\coordinate (gangliuppptw) at (\gangliuxxxt, \gangliuyyyw);
\coordinate (gangliuppptx) at (\gangliuxxxt, \gangliuyyyx);
\coordinate (gangliupppty) at (\gangliuxxxt, \gangliuyyyy);
\coordinate (gangliuppptz) at (\gangliuxxxt, \gangliuyyyz);
\coordinate (gangliupppua) at (\gangliuxxxu, \gangliuyyya);
\coordinate (gangliupppub) at (\gangliuxxxu, \gangliuyyyb);
\coordinate (gangliupppuc) at (\gangliuxxxu, \gangliuyyyc);
\coordinate (gangliupppud) at (\gangliuxxxu, \gangliuyyyd);
\coordinate (gangliupppue) at (\gangliuxxxu, \gangliuyyye);
\coordinate (gangliupppuf) at (\gangliuxxxu, \gangliuyyyf);
\coordinate (gangliupppug) at (\gangliuxxxu, \gangliuyyyg);
\coordinate (gangliupppuh) at (\gangliuxxxu, \gangliuyyyh);
\coordinate (gangliupppui) at (\gangliuxxxu, \gangliuyyyi);
\coordinate (gangliupppuj) at (\gangliuxxxu, \gangliuyyyj);
\coordinate (gangliupppuk) at (\gangliuxxxu, \gangliuyyyk);
\coordinate (gangliupppul) at (\gangliuxxxu, \gangliuyyyl);
\coordinate (gangliupppum) at (\gangliuxxxu, \gangliuyyym);
\coordinate (gangliupppun) at (\gangliuxxxu, \gangliuyyyn);
\coordinate (gangliupppuo) at (\gangliuxxxu, \gangliuyyyo);
\coordinate (gangliupppup) at (\gangliuxxxu, \gangliuyyyp);
\coordinate (gangliupppuq) at (\gangliuxxxu, \gangliuyyyq);
\coordinate (gangliupppur) at (\gangliuxxxu, \gangliuyyyr);
\coordinate (gangliupppus) at (\gangliuxxxu, \gangliuyyys);
\coordinate (gangliuppput) at (\gangliuxxxu, \gangliuyyyt);
\coordinate (gangliupppuu) at (\gangliuxxxu, \gangliuyyyu);
\coordinate (gangliupppuv) at (\gangliuxxxu, \gangliuyyyv);
\coordinate (gangliupppuw) at (\gangliuxxxu, \gangliuyyyw);
\coordinate (gangliupppux) at (\gangliuxxxu, \gangliuyyyx);
\coordinate (gangliupppuy) at (\gangliuxxxu, \gangliuyyyy);
\coordinate (gangliupppuz) at (\gangliuxxxu, \gangliuyyyz);
\coordinate (gangliupppva) at (\gangliuxxxv, \gangliuyyya);
\coordinate (gangliupppvb) at (\gangliuxxxv, \gangliuyyyb);
\coordinate (gangliupppvc) at (\gangliuxxxv, \gangliuyyyc);
\coordinate (gangliupppvd) at (\gangliuxxxv, \gangliuyyyd);
\coordinate (gangliupppve) at (\gangliuxxxv, \gangliuyyye);
\coordinate (gangliupppvf) at (\gangliuxxxv, \gangliuyyyf);
\coordinate (gangliupppvg) at (\gangliuxxxv, \gangliuyyyg);
\coordinate (gangliupppvh) at (\gangliuxxxv, \gangliuyyyh);
\coordinate (gangliupppvi) at (\gangliuxxxv, \gangliuyyyi);
\coordinate (gangliupppvj) at (\gangliuxxxv, \gangliuyyyj);
\coordinate (gangliupppvk) at (\gangliuxxxv, \gangliuyyyk);
\coordinate (gangliupppvl) at (\gangliuxxxv, \gangliuyyyl);
\coordinate (gangliupppvm) at (\gangliuxxxv, \gangliuyyym);
\coordinate (gangliupppvn) at (\gangliuxxxv, \gangliuyyyn);
\coordinate (gangliupppvo) at (\gangliuxxxv, \gangliuyyyo);
\coordinate (gangliupppvp) at (\gangliuxxxv, \gangliuyyyp);
\coordinate (gangliupppvq) at (\gangliuxxxv, \gangliuyyyq);
\coordinate (gangliupppvr) at (\gangliuxxxv, \gangliuyyyr);
\coordinate (gangliupppvs) at (\gangliuxxxv, \gangliuyyys);
\coordinate (gangliupppvt) at (\gangliuxxxv, \gangliuyyyt);
\coordinate (gangliupppvu) at (\gangliuxxxv, \gangliuyyyu);
\coordinate (gangliupppvv) at (\gangliuxxxv, \gangliuyyyv);
\coordinate (gangliupppvw) at (\gangliuxxxv, \gangliuyyyw);
\coordinate (gangliupppvx) at (\gangliuxxxv, \gangliuyyyx);
\coordinate (gangliupppvy) at (\gangliuxxxv, \gangliuyyyy);
\coordinate (gangliupppvz) at (\gangliuxxxv, \gangliuyyyz);
\coordinate (gangliupppwa) at (\gangliuxxxw, \gangliuyyya);
\coordinate (gangliupppwb) at (\gangliuxxxw, \gangliuyyyb);
\coordinate (gangliupppwc) at (\gangliuxxxw, \gangliuyyyc);
\coordinate (gangliupppwd) at (\gangliuxxxw, \gangliuyyyd);
\coordinate (gangliupppwe) at (\gangliuxxxw, \gangliuyyye);
\coordinate (gangliupppwf) at (\gangliuxxxw, \gangliuyyyf);
\coordinate (gangliupppwg) at (\gangliuxxxw, \gangliuyyyg);
\coordinate (gangliupppwh) at (\gangliuxxxw, \gangliuyyyh);
\coordinate (gangliupppwi) at (\gangliuxxxw, \gangliuyyyi);
\coordinate (gangliupppwj) at (\gangliuxxxw, \gangliuyyyj);
\coordinate (gangliupppwk) at (\gangliuxxxw, \gangliuyyyk);
\coordinate (gangliupppwl) at (\gangliuxxxw, \gangliuyyyl);
\coordinate (gangliupppwm) at (\gangliuxxxw, \gangliuyyym);
\coordinate (gangliupppwn) at (\gangliuxxxw, \gangliuyyyn);
\coordinate (gangliupppwo) at (\gangliuxxxw, \gangliuyyyo);
\coordinate (gangliupppwp) at (\gangliuxxxw, \gangliuyyyp);
\coordinate (gangliupppwq) at (\gangliuxxxw, \gangliuyyyq);
\coordinate (gangliupppwr) at (\gangliuxxxw, \gangliuyyyr);
\coordinate (gangliupppws) at (\gangliuxxxw, \gangliuyyys);
\coordinate (gangliupppwt) at (\gangliuxxxw, \gangliuyyyt);
\coordinate (gangliupppwu) at (\gangliuxxxw, \gangliuyyyu);
\coordinate (gangliupppwv) at (\gangliuxxxw, \gangliuyyyv);
\coordinate (gangliupppww) at (\gangliuxxxw, \gangliuyyyw);
\coordinate (gangliupppwx) at (\gangliuxxxw, \gangliuyyyx);
\coordinate (gangliupppwy) at (\gangliuxxxw, \gangliuyyyy);
\coordinate (gangliupppwz) at (\gangliuxxxw, \gangliuyyyz);
\coordinate (gangliupppxa) at (\gangliuxxxx, \gangliuyyya);
\coordinate (gangliupppxb) at (\gangliuxxxx, \gangliuyyyb);
\coordinate (gangliupppxc) at (\gangliuxxxx, \gangliuyyyc);
\coordinate (gangliupppxd) at (\gangliuxxxx, \gangliuyyyd);
\coordinate (gangliupppxe) at (\gangliuxxxx, \gangliuyyye);
\coordinate (gangliupppxf) at (\gangliuxxxx, \gangliuyyyf);
\coordinate (gangliupppxg) at (\gangliuxxxx, \gangliuyyyg);
\coordinate (gangliupppxh) at (\gangliuxxxx, \gangliuyyyh);
\coordinate (gangliupppxi) at (\gangliuxxxx, \gangliuyyyi);
\coordinate (gangliupppxj) at (\gangliuxxxx, \gangliuyyyj);
\coordinate (gangliupppxk) at (\gangliuxxxx, \gangliuyyyk);
\coordinate (gangliupppxl) at (\gangliuxxxx, \gangliuyyyl);
\coordinate (gangliupppxm) at (\gangliuxxxx, \gangliuyyym);
\coordinate (gangliupppxn) at (\gangliuxxxx, \gangliuyyyn);
\coordinate (gangliupppxo) at (\gangliuxxxx, \gangliuyyyo);
\coordinate (gangliupppxp) at (\gangliuxxxx, \gangliuyyyp);
\coordinate (gangliupppxq) at (\gangliuxxxx, \gangliuyyyq);
\coordinate (gangliupppxr) at (\gangliuxxxx, \gangliuyyyr);
\coordinate (gangliupppxs) at (\gangliuxxxx, \gangliuyyys);
\coordinate (gangliupppxt) at (\gangliuxxxx, \gangliuyyyt);
\coordinate (gangliupppxu) at (\gangliuxxxx, \gangliuyyyu);
\coordinate (gangliupppxv) at (\gangliuxxxx, \gangliuyyyv);
\coordinate (gangliupppxw) at (\gangliuxxxx, \gangliuyyyw);
\coordinate (gangliupppxx) at (\gangliuxxxx, \gangliuyyyx);
\coordinate (gangliupppxy) at (\gangliuxxxx, \gangliuyyyy);
\coordinate (gangliupppxz) at (\gangliuxxxx, \gangliuyyyz);
\coordinate (gangliupppya) at (\gangliuxxxy, \gangliuyyya);
\coordinate (gangliupppyb) at (\gangliuxxxy, \gangliuyyyb);
\coordinate (gangliupppyc) at (\gangliuxxxy, \gangliuyyyc);
\coordinate (gangliupppyd) at (\gangliuxxxy, \gangliuyyyd);
\coordinate (gangliupppye) at (\gangliuxxxy, \gangliuyyye);
\coordinate (gangliupppyf) at (\gangliuxxxy, \gangliuyyyf);
\coordinate (gangliupppyg) at (\gangliuxxxy, \gangliuyyyg);
\coordinate (gangliupppyh) at (\gangliuxxxy, \gangliuyyyh);
\coordinate (gangliupppyi) at (\gangliuxxxy, \gangliuyyyi);
\coordinate (gangliupppyj) at (\gangliuxxxy, \gangliuyyyj);
\coordinate (gangliupppyk) at (\gangliuxxxy, \gangliuyyyk);
\coordinate (gangliupppyl) at (\gangliuxxxy, \gangliuyyyl);
\coordinate (gangliupppym) at (\gangliuxxxy, \gangliuyyym);
\coordinate (gangliupppyn) at (\gangliuxxxy, \gangliuyyyn);
\coordinate (gangliupppyo) at (\gangliuxxxy, \gangliuyyyo);
\coordinate (gangliupppyp) at (\gangliuxxxy, \gangliuyyyp);
\coordinate (gangliupppyq) at (\gangliuxxxy, \gangliuyyyq);
\coordinate (gangliupppyr) at (\gangliuxxxy, \gangliuyyyr);
\coordinate (gangliupppys) at (\gangliuxxxy, \gangliuyyys);
\coordinate (gangliupppyt) at (\gangliuxxxy, \gangliuyyyt);
\coordinate (gangliupppyu) at (\gangliuxxxy, \gangliuyyyu);
\coordinate (gangliupppyv) at (\gangliuxxxy, \gangliuyyyv);
\coordinate (gangliupppyw) at (\gangliuxxxy, \gangliuyyyw);
\coordinate (gangliupppyx) at (\gangliuxxxy, \gangliuyyyx);
\coordinate (gangliupppyy) at (\gangliuxxxy, \gangliuyyyy);
\coordinate (gangliupppyz) at (\gangliuxxxy, \gangliuyyyz);
\coordinate (gangliupppza) at (\gangliuxxxz, \gangliuyyya);
\coordinate (gangliupppzb) at (\gangliuxxxz, \gangliuyyyb);
\coordinate (gangliupppzc) at (\gangliuxxxz, \gangliuyyyc);
\coordinate (gangliupppzd) at (\gangliuxxxz, \gangliuyyyd);
\coordinate (gangliupppze) at (\gangliuxxxz, \gangliuyyye);
\coordinate (gangliupppzf) at (\gangliuxxxz, \gangliuyyyf);
\coordinate (gangliupppzg) at (\gangliuxxxz, \gangliuyyyg);
\coordinate (gangliupppzh) at (\gangliuxxxz, \gangliuyyyh);
\coordinate (gangliupppzi) at (\gangliuxxxz, \gangliuyyyi);
\coordinate (gangliupppzj) at (\gangliuxxxz, \gangliuyyyj);
\coordinate (gangliupppzk) at (\gangliuxxxz, \gangliuyyyk);
\coordinate (gangliupppzl) at (\gangliuxxxz, \gangliuyyyl);
\coordinate (gangliupppzm) at (\gangliuxxxz, \gangliuyyym);
\coordinate (gangliupppzn) at (\gangliuxxxz, \gangliuyyyn);
\coordinate (gangliupppzo) at (\gangliuxxxz, \gangliuyyyo);
\coordinate (gangliupppzp) at (\gangliuxxxz, \gangliuyyyp);
\coordinate (gangliupppzq) at (\gangliuxxxz, \gangliuyyyq);
\coordinate (gangliupppzr) at (\gangliuxxxz, \gangliuyyyr);
\coordinate (gangliupppzs) at (\gangliuxxxz, \gangliuyyys);
\coordinate (gangliupppzt) at (\gangliuxxxz, \gangliuyyyt);
\coordinate (gangliupppzu) at (\gangliuxxxz, \gangliuyyyu);
\coordinate (gangliupppzv) at (\gangliuxxxz, \gangliuyyyv);
\coordinate (gangliupppzw) at (\gangliuxxxz, \gangliuyyyw);
\coordinate (gangliupppzx) at (\gangliuxxxz, \gangliuyyyx);
\coordinate (gangliupppzy) at (\gangliuxxxz, \gangliuyyyy);
\coordinate (gangliupppzz) at (\gangliuxxxz, \gangliuyyyz);

%\gangprintcoordinateat{(0,0)}{The last coordinate values: }{($(gangliupppzz)$)}; 



\pgfmathsetmacro{\totalgliuaxxx}{26}
\pgfmathsetmacro{\totalgliuayyy}{26}
\pgfmathsetmacro{\gliuaxxxspacing}{1}
\pgfmathsetmacro{\gliuayyyspacing}{1}
\pgfmathsetmacro{\gliuaxxxa}{\gangliuxxxk}
\pgfmathsetmacro{\gliuayyya}{\gangliuyyyj}

\pgfmathsetmacro{\gliuaxxxb}{\gliuaxxxa + \gliuaxxxspacing + 0.0 }
\pgfmathsetmacro{\gliuaxxxc}{\gliuaxxxb + \gliuaxxxspacing + 0.0 }
\pgfmathsetmacro{\gliuaxxxd}{\gliuaxxxc + \gliuaxxxspacing + 0.0 }
\pgfmathsetmacro{\gliuaxxxe}{\gliuaxxxd + \gliuaxxxspacing + 0.0 }
\pgfmathsetmacro{\gliuaxxxf}{\gliuaxxxe + \gliuaxxxspacing + 0.0 }
\pgfmathsetmacro{\gliuaxxxg}{\gliuaxxxf + \gliuaxxxspacing + 0.0 }
\pgfmathsetmacro{\gliuaxxxh}{\gliuaxxxg + \gliuaxxxspacing + 0.0 }
\pgfmathsetmacro{\gliuaxxxi}{\gliuaxxxh + \gliuaxxxspacing + 0.0 }
\pgfmathsetmacro{\gliuaxxxj}{\gliuaxxxi + \gliuaxxxspacing + 0.0 }
\pgfmathsetmacro{\gliuaxxxk}{\gliuaxxxj + \gliuaxxxspacing + 0.0 }
\pgfmathsetmacro{\gliuaxxxl}{\gliuaxxxk + \gliuaxxxspacing + 0.0 }
\pgfmathsetmacro{\gliuaxxxm}{\gliuaxxxl + \gliuaxxxspacing + 0.0 }
\pgfmathsetmacro{\gliuaxxxn}{\gliuaxxxm + \gliuaxxxspacing + 0.0 }
\pgfmathsetmacro{\gliuaxxxo}{\gliuaxxxn + \gliuaxxxspacing + 0.0 }
\pgfmathsetmacro{\gliuaxxxp}{\gliuaxxxo + \gliuaxxxspacing + 0.0 }
\pgfmathsetmacro{\gliuaxxxq}{\gliuaxxxp + \gliuaxxxspacing + 0.0 }
\pgfmathsetmacro{\gliuaxxxr}{\gliuaxxxq + \gliuaxxxspacing + 0.0 }
\pgfmathsetmacro{\gliuaxxxs}{\gliuaxxxr + \gliuaxxxspacing + 0.0 }
\pgfmathsetmacro{\gliuaxxxt}{\gliuaxxxs + \gliuaxxxspacing + 0.0 }
\pgfmathsetmacro{\gliuaxxxu}{\gliuaxxxt + \gliuaxxxspacing + 0.0 }
\pgfmathsetmacro{\gliuaxxxv}{\gliuaxxxu + \gliuaxxxspacing + 0.0 }
\pgfmathsetmacro{\gliuaxxxw}{\gliuaxxxv + \gliuaxxxspacing + 0.0 }
\pgfmathsetmacro{\gliuaxxxx}{\gliuaxxxw + \gliuaxxxspacing + 0.0 }
\pgfmathsetmacro{\gliuaxxxy}{\gliuaxxxx + \gliuaxxxspacing + 0.0 }
\pgfmathsetmacro{\gliuaxxxz}{\gliuaxxxy + \gliuaxxxspacing + 0.0 }

\pgfmathsetmacro{\gliuayyyb}{\gliuayyya + \gliuayyyspacing + 0.0 }
\pgfmathsetmacro{\gliuayyyc}{\gliuayyyb + \gliuayyyspacing + 0.0 }
\pgfmathsetmacro{\gliuayyyd}{\gliuayyyc + \gliuayyyspacing + 0.0 }
\pgfmathsetmacro{\gliuayyye}{\gliuayyyd + \gliuayyyspacing + 0.0 }
\pgfmathsetmacro{\gliuayyyf}{\gliuayyye + \gliuayyyspacing + 0.0 }
\pgfmathsetmacro{\gliuayyyg}{\gliuayyyf + \gliuayyyspacing + 0.0 }
\pgfmathsetmacro{\gliuayyyh}{\gliuayyyg + \gliuayyyspacing + 0.0 }
\pgfmathsetmacro{\gliuayyyi}{\gliuayyyh + \gliuayyyspacing + 0.0 }
\pgfmathsetmacro{\gliuayyyj}{\gliuayyyi + \gliuayyyspacing + 0.0 }
\pgfmathsetmacro{\gliuayyyk}{\gliuayyyj + \gliuayyyspacing + 0.0 }
\pgfmathsetmacro{\gliuayyyl}{\gliuayyyk + \gliuayyyspacing + 0.0 }
\pgfmathsetmacro{\gliuayyym}{\gliuayyyl + \gliuayyyspacing + 0.0 }
\pgfmathsetmacro{\gliuayyyn}{\gliuayyym + \gliuayyyspacing + 0.0 }
\pgfmathsetmacro{\gliuayyyo}{\gliuayyyn + \gliuayyyspacing + 0.0 }
\pgfmathsetmacro{\gliuayyyp}{\gliuayyyo + \gliuayyyspacing + 0.0 }
\pgfmathsetmacro{\gliuayyyq}{\gliuayyyp + \gliuayyyspacing + 0.0 }
\pgfmathsetmacro{\gliuayyyr}{\gliuayyyq + \gliuayyyspacing + 0.0 }
\pgfmathsetmacro{\gliuayyys}{\gliuayyyr + \gliuayyyspacing + 0.0 }
\pgfmathsetmacro{\gliuayyyt}{\gliuayyys + \gliuayyyspacing + 0.0 }
\pgfmathsetmacro{\gliuayyyu}{\gliuayyyt + \gliuayyyspacing + 0.0 }
\pgfmathsetmacro{\gliuayyyv}{\gliuayyyu + \gliuayyyspacing + 0.0 }
\pgfmathsetmacro{\gliuayyyw}{\gliuayyyv + \gliuayyyspacing + 0.0 }
\pgfmathsetmacro{\gliuayyyx}{\gliuayyyw + \gliuayyyspacing + 0.0 }
\pgfmathsetmacro{\gliuayyyy}{\gliuayyyx + \gliuayyyspacing + 0.0 }
\pgfmathsetmacro{\gliuayyyz}{\gliuayyyy + \gliuayyyspacing + 0.0 }

\coordinate (gliuapppaa) at (\gliuaxxxa, \gliuayyya);
\coordinate (gliuapppab) at (\gliuaxxxa, \gliuayyyb);
\coordinate (gliuapppac) at (\gliuaxxxa, \gliuayyyc);
\coordinate (gliuapppad) at (\gliuaxxxa, \gliuayyyd);
\coordinate (gliuapppae) at (\gliuaxxxa, \gliuayyye);
\coordinate (gliuapppaf) at (\gliuaxxxa, \gliuayyyf);
\coordinate (gliuapppag) at (\gliuaxxxa, \gliuayyyg);
\coordinate (gliuapppah) at (\gliuaxxxa, \gliuayyyh);
\coordinate (gliuapppai) at (\gliuaxxxa, \gliuayyyi);
\coordinate (gliuapppaj) at (\gliuaxxxa, \gliuayyyj);
\coordinate (gliuapppak) at (\gliuaxxxa, \gliuayyyk);
\coordinate (gliuapppal) at (\gliuaxxxa, \gliuayyyl);
\coordinate (gliuapppam) at (\gliuaxxxa, \gliuayyym);
\coordinate (gliuapppan) at (\gliuaxxxa, \gliuayyyn);
\coordinate (gliuapppao) at (\gliuaxxxa, \gliuayyyo);
\coordinate (gliuapppap) at (\gliuaxxxa, \gliuayyyp);
\coordinate (gliuapppaq) at (\gliuaxxxa, \gliuayyyq);
\coordinate (gliuapppar) at (\gliuaxxxa, \gliuayyyr);
\coordinate (gliuapppas) at (\gliuaxxxa, \gliuayyys);
\coordinate (gliuapppat) at (\gliuaxxxa, \gliuayyyt);
\coordinate (gliuapppau) at (\gliuaxxxa, \gliuayyyu);
\coordinate (gliuapppav) at (\gliuaxxxa, \gliuayyyv);
\coordinate (gliuapppaw) at (\gliuaxxxa, \gliuayyyw);
\coordinate (gliuapppax) at (\gliuaxxxa, \gliuayyyx);
\coordinate (gliuapppay) at (\gliuaxxxa, \gliuayyyy);
\coordinate (gliuapppaz) at (\gliuaxxxa, \gliuayyyz);
\coordinate (gliuapppba) at (\gliuaxxxb, \gliuayyya);
\coordinate (gliuapppbb) at (\gliuaxxxb, \gliuayyyb);
\coordinate (gliuapppbc) at (\gliuaxxxb, \gliuayyyc);
\coordinate (gliuapppbd) at (\gliuaxxxb, \gliuayyyd);
\coordinate (gliuapppbe) at (\gliuaxxxb, \gliuayyye);
\coordinate (gliuapppbf) at (\gliuaxxxb, \gliuayyyf);
\coordinate (gliuapppbg) at (\gliuaxxxb, \gliuayyyg);
\coordinate (gliuapppbh) at (\gliuaxxxb, \gliuayyyh);
\coordinate (gliuapppbi) at (\gliuaxxxb, \gliuayyyi);
\coordinate (gliuapppbj) at (\gliuaxxxb, \gliuayyyj);
\coordinate (gliuapppbk) at (\gliuaxxxb, \gliuayyyk);
\coordinate (gliuapppbl) at (\gliuaxxxb, \gliuayyyl);
\coordinate (gliuapppbm) at (\gliuaxxxb, \gliuayyym);
\coordinate (gliuapppbn) at (\gliuaxxxb, \gliuayyyn);
\coordinate (gliuapppbo) at (\gliuaxxxb, \gliuayyyo);
\coordinate (gliuapppbp) at (\gliuaxxxb, \gliuayyyp);
\coordinate (gliuapppbq) at (\gliuaxxxb, \gliuayyyq);
\coordinate (gliuapppbr) at (\gliuaxxxb, \gliuayyyr);
\coordinate (gliuapppbs) at (\gliuaxxxb, \gliuayyys);
\coordinate (gliuapppbt) at (\gliuaxxxb, \gliuayyyt);
\coordinate (gliuapppbu) at (\gliuaxxxb, \gliuayyyu);
\coordinate (gliuapppbv) at (\gliuaxxxb, \gliuayyyv);
\coordinate (gliuapppbw) at (\gliuaxxxb, \gliuayyyw);
\coordinate (gliuapppbx) at (\gliuaxxxb, \gliuayyyx);
\coordinate (gliuapppby) at (\gliuaxxxb, \gliuayyyy);
\coordinate (gliuapppbz) at (\gliuaxxxb, \gliuayyyz);
\coordinate (gliuapppca) at (\gliuaxxxc, \gliuayyya);
\coordinate (gliuapppcb) at (\gliuaxxxc, \gliuayyyb);
\coordinate (gliuapppcc) at (\gliuaxxxc, \gliuayyyc);
\coordinate (gliuapppcd) at (\gliuaxxxc, \gliuayyyd);
\coordinate (gliuapppce) at (\gliuaxxxc, \gliuayyye);
\coordinate (gliuapppcf) at (\gliuaxxxc, \gliuayyyf);
\coordinate (gliuapppcg) at (\gliuaxxxc, \gliuayyyg);
\coordinate (gliuapppch) at (\gliuaxxxc, \gliuayyyh);
\coordinate (gliuapppci) at (\gliuaxxxc, \gliuayyyi);
\coordinate (gliuapppcj) at (\gliuaxxxc, \gliuayyyj);
\coordinate (gliuapppck) at (\gliuaxxxc, \gliuayyyk);
\coordinate (gliuapppcl) at (\gliuaxxxc, \gliuayyyl);
\coordinate (gliuapppcm) at (\gliuaxxxc, \gliuayyym);
\coordinate (gliuapppcn) at (\gliuaxxxc, \gliuayyyn);
\coordinate (gliuapppco) at (\gliuaxxxc, \gliuayyyo);
\coordinate (gliuapppcp) at (\gliuaxxxc, \gliuayyyp);
\coordinate (gliuapppcq) at (\gliuaxxxc, \gliuayyyq);
\coordinate (gliuapppcr) at (\gliuaxxxc, \gliuayyyr);
\coordinate (gliuapppcs) at (\gliuaxxxc, \gliuayyys);
\coordinate (gliuapppct) at (\gliuaxxxc, \gliuayyyt);
\coordinate (gliuapppcu) at (\gliuaxxxc, \gliuayyyu);
\coordinate (gliuapppcv) at (\gliuaxxxc, \gliuayyyv);
\coordinate (gliuapppcw) at (\gliuaxxxc, \gliuayyyw);
\coordinate (gliuapppcx) at (\gliuaxxxc, \gliuayyyx);
\coordinate (gliuapppcy) at (\gliuaxxxc, \gliuayyyy);
\coordinate (gliuapppcz) at (\gliuaxxxc, \gliuayyyz);
\coordinate (gliuapppda) at (\gliuaxxxd, \gliuayyya);
\coordinate (gliuapppdb) at (\gliuaxxxd, \gliuayyyb);
\coordinate (gliuapppdc) at (\gliuaxxxd, \gliuayyyc);
\coordinate (gliuapppdd) at (\gliuaxxxd, \gliuayyyd);
\coordinate (gliuapppde) at (\gliuaxxxd, \gliuayyye);
\coordinate (gliuapppdf) at (\gliuaxxxd, \gliuayyyf);
\coordinate (gliuapppdg) at (\gliuaxxxd, \gliuayyyg);
\coordinate (gliuapppdh) at (\gliuaxxxd, \gliuayyyh);
\coordinate (gliuapppdi) at (\gliuaxxxd, \gliuayyyi);
\coordinate (gliuapppdj) at (\gliuaxxxd, \gliuayyyj);
\coordinate (gliuapppdk) at (\gliuaxxxd, \gliuayyyk);
\coordinate (gliuapppdl) at (\gliuaxxxd, \gliuayyyl);
\coordinate (gliuapppdm) at (\gliuaxxxd, \gliuayyym);
\coordinate (gliuapppdn) at (\gliuaxxxd, \gliuayyyn);
\coordinate (gliuapppdo) at (\gliuaxxxd, \gliuayyyo);
\coordinate (gliuapppdp) at (\gliuaxxxd, \gliuayyyp);
\coordinate (gliuapppdq) at (\gliuaxxxd, \gliuayyyq);
\coordinate (gliuapppdr) at (\gliuaxxxd, \gliuayyyr);
\coordinate (gliuapppds) at (\gliuaxxxd, \gliuayyys);
\coordinate (gliuapppdt) at (\gliuaxxxd, \gliuayyyt);
\coordinate (gliuapppdu) at (\gliuaxxxd, \gliuayyyu);
\coordinate (gliuapppdv) at (\gliuaxxxd, \gliuayyyv);
\coordinate (gliuapppdw) at (\gliuaxxxd, \gliuayyyw);
\coordinate (gliuapppdx) at (\gliuaxxxd, \gliuayyyx);
\coordinate (gliuapppdy) at (\gliuaxxxd, \gliuayyyy);
\coordinate (gliuapppdz) at (\gliuaxxxd, \gliuayyyz);
\coordinate (gliuapppea) at (\gliuaxxxe, \gliuayyya);
\coordinate (gliuapppeb) at (\gliuaxxxe, \gliuayyyb);
\coordinate (gliuapppec) at (\gliuaxxxe, \gliuayyyc);
\coordinate (gliuappped) at (\gliuaxxxe, \gliuayyyd);
\coordinate (gliuapppee) at (\gliuaxxxe, \gliuayyye);
\coordinate (gliuapppef) at (\gliuaxxxe, \gliuayyyf);
\coordinate (gliuapppeg) at (\gliuaxxxe, \gliuayyyg);
\coordinate (gliuapppeh) at (\gliuaxxxe, \gliuayyyh);
\coordinate (gliuapppei) at (\gliuaxxxe, \gliuayyyi);
\coordinate (gliuapppej) at (\gliuaxxxe, \gliuayyyj);
\coordinate (gliuapppek) at (\gliuaxxxe, \gliuayyyk);
\coordinate (gliuapppel) at (\gliuaxxxe, \gliuayyyl);
\coordinate (gliuapppem) at (\gliuaxxxe, \gliuayyym);
\coordinate (gliuapppen) at (\gliuaxxxe, \gliuayyyn);
\coordinate (gliuapppeo) at (\gliuaxxxe, \gliuayyyo);
\coordinate (gliuapppep) at (\gliuaxxxe, \gliuayyyp);
\coordinate (gliuapppeq) at (\gliuaxxxe, \gliuayyyq);
\coordinate (gliuappper) at (\gliuaxxxe, \gliuayyyr);
\coordinate (gliuapppes) at (\gliuaxxxe, \gliuayyys);
\coordinate (gliuapppet) at (\gliuaxxxe, \gliuayyyt);
\coordinate (gliuapppeu) at (\gliuaxxxe, \gliuayyyu);
\coordinate (gliuapppev) at (\gliuaxxxe, \gliuayyyv);
\coordinate (gliuapppew) at (\gliuaxxxe, \gliuayyyw);
\coordinate (gliuapppex) at (\gliuaxxxe, \gliuayyyx);
\coordinate (gliuapppey) at (\gliuaxxxe, \gliuayyyy);
\coordinate (gliuapppez) at (\gliuaxxxe, \gliuayyyz);
\coordinate (gliuapppfa) at (\gliuaxxxf, \gliuayyya);
\coordinate (gliuapppfb) at (\gliuaxxxf, \gliuayyyb);
\coordinate (gliuapppfc) at (\gliuaxxxf, \gliuayyyc);
\coordinate (gliuapppfd) at (\gliuaxxxf, \gliuayyyd);
\coordinate (gliuapppfe) at (\gliuaxxxf, \gliuayyye);
\coordinate (gliuapppff) at (\gliuaxxxf, \gliuayyyf);
\coordinate (gliuapppfg) at (\gliuaxxxf, \gliuayyyg);
\coordinate (gliuapppfh) at (\gliuaxxxf, \gliuayyyh);
\coordinate (gliuapppfi) at (\gliuaxxxf, \gliuayyyi);
\coordinate (gliuapppfj) at (\gliuaxxxf, \gliuayyyj);
\coordinate (gliuapppfk) at (\gliuaxxxf, \gliuayyyk);
\coordinate (gliuapppfl) at (\gliuaxxxf, \gliuayyyl);
\coordinate (gliuapppfm) at (\gliuaxxxf, \gliuayyym);
\coordinate (gliuapppfn) at (\gliuaxxxf, \gliuayyyn);
\coordinate (gliuapppfo) at (\gliuaxxxf, \gliuayyyo);
\coordinate (gliuapppfp) at (\gliuaxxxf, \gliuayyyp);
\coordinate (gliuapppfq) at (\gliuaxxxf, \gliuayyyq);
\coordinate (gliuapppfr) at (\gliuaxxxf, \gliuayyyr);
\coordinate (gliuapppfs) at (\gliuaxxxf, \gliuayyys);
\coordinate (gliuapppft) at (\gliuaxxxf, \gliuayyyt);
\coordinate (gliuapppfu) at (\gliuaxxxf, \gliuayyyu);
\coordinate (gliuapppfv) at (\gliuaxxxf, \gliuayyyv);
\coordinate (gliuapppfw) at (\gliuaxxxf, \gliuayyyw);
\coordinate (gliuapppfx) at (\gliuaxxxf, \gliuayyyx);
\coordinate (gliuapppfy) at (\gliuaxxxf, \gliuayyyy);
\coordinate (gliuapppfz) at (\gliuaxxxf, \gliuayyyz);
\coordinate (gliuapppga) at (\gliuaxxxg, \gliuayyya);
\coordinate (gliuapppgb) at (\gliuaxxxg, \gliuayyyb);
\coordinate (gliuapppgc) at (\gliuaxxxg, \gliuayyyc);
\coordinate (gliuapppgd) at (\gliuaxxxg, \gliuayyyd);
\coordinate (gliuapppge) at (\gliuaxxxg, \gliuayyye);
\coordinate (gliuapppgf) at (\gliuaxxxg, \gliuayyyf);
\coordinate (gliuapppgg) at (\gliuaxxxg, \gliuayyyg);
\coordinate (gliuapppgh) at (\gliuaxxxg, \gliuayyyh);
\coordinate (gliuapppgi) at (\gliuaxxxg, \gliuayyyi);
\coordinate (gliuapppgj) at (\gliuaxxxg, \gliuayyyj);
\coordinate (gliuapppgk) at (\gliuaxxxg, \gliuayyyk);
\coordinate (gliuapppgl) at (\gliuaxxxg, \gliuayyyl);
\coordinate (gliuapppgm) at (\gliuaxxxg, \gliuayyym);
\coordinate (gliuapppgn) at (\gliuaxxxg, \gliuayyyn);
\coordinate (gliuapppgo) at (\gliuaxxxg, \gliuayyyo);
\coordinate (gliuapppgp) at (\gliuaxxxg, \gliuayyyp);
\coordinate (gliuapppgq) at (\gliuaxxxg, \gliuayyyq);
\coordinate (gliuapppgr) at (\gliuaxxxg, \gliuayyyr);
\coordinate (gliuapppgs) at (\gliuaxxxg, \gliuayyys);
\coordinate (gliuapppgt) at (\gliuaxxxg, \gliuayyyt);
\coordinate (gliuapppgu) at (\gliuaxxxg, \gliuayyyu);
\coordinate (gliuapppgv) at (\gliuaxxxg, \gliuayyyv);
\coordinate (gliuapppgw) at (\gliuaxxxg, \gliuayyyw);
\coordinate (gliuapppgx) at (\gliuaxxxg, \gliuayyyx);
\coordinate (gliuapppgy) at (\gliuaxxxg, \gliuayyyy);
\coordinate (gliuapppgz) at (\gliuaxxxg, \gliuayyyz);
\coordinate (gliuapppha) at (\gliuaxxxh, \gliuayyya);
\coordinate (gliuappphb) at (\gliuaxxxh, \gliuayyyb);
\coordinate (gliuappphc) at (\gliuaxxxh, \gliuayyyc);
\coordinate (gliuappphd) at (\gliuaxxxh, \gliuayyyd);
\coordinate (gliuappphe) at (\gliuaxxxh, \gliuayyye);
\coordinate (gliuappphf) at (\gliuaxxxh, \gliuayyyf);
\coordinate (gliuappphg) at (\gliuaxxxh, \gliuayyyg);
\coordinate (gliuappphh) at (\gliuaxxxh, \gliuayyyh);
\coordinate (gliuappphi) at (\gliuaxxxh, \gliuayyyi);
\coordinate (gliuappphj) at (\gliuaxxxh, \gliuayyyj);
\coordinate (gliuappphk) at (\gliuaxxxh, \gliuayyyk);
\coordinate (gliuappphl) at (\gliuaxxxh, \gliuayyyl);
\coordinate (gliuappphm) at (\gliuaxxxh, \gliuayyym);
\coordinate (gliuappphn) at (\gliuaxxxh, \gliuayyyn);
\coordinate (gliuapppho) at (\gliuaxxxh, \gliuayyyo);
\coordinate (gliuappphp) at (\gliuaxxxh, \gliuayyyp);
\coordinate (gliuappphq) at (\gliuaxxxh, \gliuayyyq);
\coordinate (gliuappphr) at (\gliuaxxxh, \gliuayyyr);
\coordinate (gliuappphs) at (\gliuaxxxh, \gliuayyys);
\coordinate (gliuapppht) at (\gliuaxxxh, \gliuayyyt);
\coordinate (gliuappphu) at (\gliuaxxxh, \gliuayyyu);
\coordinate (gliuappphv) at (\gliuaxxxh, \gliuayyyv);
\coordinate (gliuappphw) at (\gliuaxxxh, \gliuayyyw);
\coordinate (gliuappphx) at (\gliuaxxxh, \gliuayyyx);
\coordinate (gliuappphy) at (\gliuaxxxh, \gliuayyyy);
\coordinate (gliuappphz) at (\gliuaxxxh, \gliuayyyz);
\coordinate (gliuapppia) at (\gliuaxxxi, \gliuayyya);
\coordinate (gliuapppib) at (\gliuaxxxi, \gliuayyyb);
\coordinate (gliuapppic) at (\gliuaxxxi, \gliuayyyc);
\coordinate (gliuapppid) at (\gliuaxxxi, \gliuayyyd);
\coordinate (gliuapppie) at (\gliuaxxxi, \gliuayyye);
\coordinate (gliuapppif) at (\gliuaxxxi, \gliuayyyf);
\coordinate (gliuapppig) at (\gliuaxxxi, \gliuayyyg);
\coordinate (gliuapppih) at (\gliuaxxxi, \gliuayyyh);
\coordinate (gliuapppii) at (\gliuaxxxi, \gliuayyyi);
\coordinate (gliuapppij) at (\gliuaxxxi, \gliuayyyj);
\coordinate (gliuapppik) at (\gliuaxxxi, \gliuayyyk);
\coordinate (gliuapppil) at (\gliuaxxxi, \gliuayyyl);
\coordinate (gliuapppim) at (\gliuaxxxi, \gliuayyym);
\coordinate (gliuapppin) at (\gliuaxxxi, \gliuayyyn);
\coordinate (gliuapppio) at (\gliuaxxxi, \gliuayyyo);
\coordinate (gliuapppip) at (\gliuaxxxi, \gliuayyyp);
\coordinate (gliuapppiq) at (\gliuaxxxi, \gliuayyyq);
\coordinate (gliuapppir) at (\gliuaxxxi, \gliuayyyr);
\coordinate (gliuapppis) at (\gliuaxxxi, \gliuayyys);
\coordinate (gliuapppit) at (\gliuaxxxi, \gliuayyyt);
\coordinate (gliuapppiu) at (\gliuaxxxi, \gliuayyyu);
\coordinate (gliuapppiv) at (\gliuaxxxi, \gliuayyyv);
\coordinate (gliuapppiw) at (\gliuaxxxi, \gliuayyyw);
\coordinate (gliuapppix) at (\gliuaxxxi, \gliuayyyx);
\coordinate (gliuapppiy) at (\gliuaxxxi, \gliuayyyy);
\coordinate (gliuapppiz) at (\gliuaxxxi, \gliuayyyz);
\coordinate (gliuapppja) at (\gliuaxxxj, \gliuayyya);
\coordinate (gliuapppjb) at (\gliuaxxxj, \gliuayyyb);
\coordinate (gliuapppjc) at (\gliuaxxxj, \gliuayyyc);
\coordinate (gliuapppjd) at (\gliuaxxxj, \gliuayyyd);
\coordinate (gliuapppje) at (\gliuaxxxj, \gliuayyye);
\coordinate (gliuapppjf) at (\gliuaxxxj, \gliuayyyf);
\coordinate (gliuapppjg) at (\gliuaxxxj, \gliuayyyg);
\coordinate (gliuapppjh) at (\gliuaxxxj, \gliuayyyh);
\coordinate (gliuapppji) at (\gliuaxxxj, \gliuayyyi);
\coordinate (gliuapppjj) at (\gliuaxxxj, \gliuayyyj);
\coordinate (gliuapppjk) at (\gliuaxxxj, \gliuayyyk);
\coordinate (gliuapppjl) at (\gliuaxxxj, \gliuayyyl);
\coordinate (gliuapppjm) at (\gliuaxxxj, \gliuayyym);
\coordinate (gliuapppjn) at (\gliuaxxxj, \gliuayyyn);
\coordinate (gliuapppjo) at (\gliuaxxxj, \gliuayyyo);
\coordinate (gliuapppjp) at (\gliuaxxxj, \gliuayyyp);
\coordinate (gliuapppjq) at (\gliuaxxxj, \gliuayyyq);
\coordinate (gliuapppjr) at (\gliuaxxxj, \gliuayyyr);
\coordinate (gliuapppjs) at (\gliuaxxxj, \gliuayyys);
\coordinate (gliuapppjt) at (\gliuaxxxj, \gliuayyyt);
\coordinate (gliuapppju) at (\gliuaxxxj, \gliuayyyu);
\coordinate (gliuapppjv) at (\gliuaxxxj, \gliuayyyv);
\coordinate (gliuapppjw) at (\gliuaxxxj, \gliuayyyw);
\coordinate (gliuapppjx) at (\gliuaxxxj, \gliuayyyx);
\coordinate (gliuapppjy) at (\gliuaxxxj, \gliuayyyy);
\coordinate (gliuapppjz) at (\gliuaxxxj, \gliuayyyz);
\coordinate (gliuapppka) at (\gliuaxxxk, \gliuayyya);
\coordinate (gliuapppkb) at (\gliuaxxxk, \gliuayyyb);
\coordinate (gliuapppkc) at (\gliuaxxxk, \gliuayyyc);
\coordinate (gliuapppkd) at (\gliuaxxxk, \gliuayyyd);
\coordinate (gliuapppke) at (\gliuaxxxk, \gliuayyye);
\coordinate (gliuapppkf) at (\gliuaxxxk, \gliuayyyf);
\coordinate (gliuapppkg) at (\gliuaxxxk, \gliuayyyg);
\coordinate (gliuapppkh) at (\gliuaxxxk, \gliuayyyh);
\coordinate (gliuapppki) at (\gliuaxxxk, \gliuayyyi);
\coordinate (gliuapppkj) at (\gliuaxxxk, \gliuayyyj);
\coordinate (gliuapppkk) at (\gliuaxxxk, \gliuayyyk);
\coordinate (gliuapppkl) at (\gliuaxxxk, \gliuayyyl);
\coordinate (gliuapppkm) at (\gliuaxxxk, \gliuayyym);
\coordinate (gliuapppkn) at (\gliuaxxxk, \gliuayyyn);
\coordinate (gliuapppko) at (\gliuaxxxk, \gliuayyyo);
\coordinate (gliuapppkp) at (\gliuaxxxk, \gliuayyyp);
\coordinate (gliuapppkq) at (\gliuaxxxk, \gliuayyyq);
\coordinate (gliuapppkr) at (\gliuaxxxk, \gliuayyyr);
\coordinate (gliuapppks) at (\gliuaxxxk, \gliuayyys);
\coordinate (gliuapppkt) at (\gliuaxxxk, \gliuayyyt);
\coordinate (gliuapppku) at (\gliuaxxxk, \gliuayyyu);
\coordinate (gliuapppkv) at (\gliuaxxxk, \gliuayyyv);
\coordinate (gliuapppkw) at (\gliuaxxxk, \gliuayyyw);
\coordinate (gliuapppkx) at (\gliuaxxxk, \gliuayyyx);
\coordinate (gliuapppky) at (\gliuaxxxk, \gliuayyyy);
\coordinate (gliuapppkz) at (\gliuaxxxk, \gliuayyyz);
\coordinate (gliuapppla) at (\gliuaxxxl, \gliuayyya);
\coordinate (gliuappplb) at (\gliuaxxxl, \gliuayyyb);
\coordinate (gliuappplc) at (\gliuaxxxl, \gliuayyyc);
\coordinate (gliuapppld) at (\gliuaxxxl, \gliuayyyd);
\coordinate (gliuappple) at (\gliuaxxxl, \gliuayyye);
\coordinate (gliuappplf) at (\gliuaxxxl, \gliuayyyf);
\coordinate (gliuappplg) at (\gliuaxxxl, \gliuayyyg);
\coordinate (gliuappplh) at (\gliuaxxxl, \gliuayyyh);
\coordinate (gliuapppli) at (\gliuaxxxl, \gliuayyyi);
\coordinate (gliuappplj) at (\gliuaxxxl, \gliuayyyj);
\coordinate (gliuappplk) at (\gliuaxxxl, \gliuayyyk);
\coordinate (gliuapppll) at (\gliuaxxxl, \gliuayyyl);
\coordinate (gliuappplm) at (\gliuaxxxl, \gliuayyym);
\coordinate (gliuapppln) at (\gliuaxxxl, \gliuayyyn);
\coordinate (gliuappplo) at (\gliuaxxxl, \gliuayyyo);
\coordinate (gliuappplp) at (\gliuaxxxl, \gliuayyyp);
\coordinate (gliuappplq) at (\gliuaxxxl, \gliuayyyq);
\coordinate (gliuappplr) at (\gliuaxxxl, \gliuayyyr);
\coordinate (gliuapppls) at (\gliuaxxxl, \gliuayyys);
\coordinate (gliuappplt) at (\gliuaxxxl, \gliuayyyt);
\coordinate (gliuappplu) at (\gliuaxxxl, \gliuayyyu);
\coordinate (gliuappplv) at (\gliuaxxxl, \gliuayyyv);
\coordinate (gliuappplw) at (\gliuaxxxl, \gliuayyyw);
\coordinate (gliuappplx) at (\gliuaxxxl, \gliuayyyx);
\coordinate (gliuappply) at (\gliuaxxxl, \gliuayyyy);
\coordinate (gliuappplz) at (\gliuaxxxl, \gliuayyyz);
\coordinate (gliuapppma) at (\gliuaxxxm, \gliuayyya);
\coordinate (gliuapppmb) at (\gliuaxxxm, \gliuayyyb);
\coordinate (gliuapppmc) at (\gliuaxxxm, \gliuayyyc);
\coordinate (gliuapppmd) at (\gliuaxxxm, \gliuayyyd);
\coordinate (gliuapppme) at (\gliuaxxxm, \gliuayyye);
\coordinate (gliuapppmf) at (\gliuaxxxm, \gliuayyyf);
\coordinate (gliuapppmg) at (\gliuaxxxm, \gliuayyyg);
\coordinate (gliuapppmh) at (\gliuaxxxm, \gliuayyyh);
\coordinate (gliuapppmi) at (\gliuaxxxm, \gliuayyyi);
\coordinate (gliuapppmj) at (\gliuaxxxm, \gliuayyyj);
\coordinate (gliuapppmk) at (\gliuaxxxm, \gliuayyyk);
\coordinate (gliuapppml) at (\gliuaxxxm, \gliuayyyl);
\coordinate (gliuapppmm) at (\gliuaxxxm, \gliuayyym);
\coordinate (gliuapppmn) at (\gliuaxxxm, \gliuayyyn);
\coordinate (gliuapppmo) at (\gliuaxxxm, \gliuayyyo);
\coordinate (gliuapppmp) at (\gliuaxxxm, \gliuayyyp);
\coordinate (gliuapppmq) at (\gliuaxxxm, \gliuayyyq);
\coordinate (gliuapppmr) at (\gliuaxxxm, \gliuayyyr);
\coordinate (gliuapppms) at (\gliuaxxxm, \gliuayyys);
\coordinate (gliuapppmt) at (\gliuaxxxm, \gliuayyyt);
\coordinate (gliuapppmu) at (\gliuaxxxm, \gliuayyyu);
\coordinate (gliuapppmv) at (\gliuaxxxm, \gliuayyyv);
\coordinate (gliuapppmw) at (\gliuaxxxm, \gliuayyyw);
\coordinate (gliuapppmx) at (\gliuaxxxm, \gliuayyyx);
\coordinate (gliuapppmy) at (\gliuaxxxm, \gliuayyyy);
\coordinate (gliuapppmz) at (\gliuaxxxm, \gliuayyyz);
\coordinate (gliuapppna) at (\gliuaxxxn, \gliuayyya);
\coordinate (gliuapppnb) at (\gliuaxxxn, \gliuayyyb);
\coordinate (gliuapppnc) at (\gliuaxxxn, \gliuayyyc);
\coordinate (gliuapppnd) at (\gliuaxxxn, \gliuayyyd);
\coordinate (gliuapppne) at (\gliuaxxxn, \gliuayyye);
\coordinate (gliuapppnf) at (\gliuaxxxn, \gliuayyyf);
\coordinate (gliuapppng) at (\gliuaxxxn, \gliuayyyg);
\coordinate (gliuapppnh) at (\gliuaxxxn, \gliuayyyh);
\coordinate (gliuapppni) at (\gliuaxxxn, \gliuayyyi);
\coordinate (gliuapppnj) at (\gliuaxxxn, \gliuayyyj);
\coordinate (gliuapppnk) at (\gliuaxxxn, \gliuayyyk);
\coordinate (gliuapppnl) at (\gliuaxxxn, \gliuayyyl);
\coordinate (gliuapppnm) at (\gliuaxxxn, \gliuayyym);
\coordinate (gliuapppnn) at (\gliuaxxxn, \gliuayyyn);
\coordinate (gliuapppno) at (\gliuaxxxn, \gliuayyyo);
\coordinate (gliuapppnp) at (\gliuaxxxn, \gliuayyyp);
\coordinate (gliuapppnq) at (\gliuaxxxn, \gliuayyyq);
\coordinate (gliuapppnr) at (\gliuaxxxn, \gliuayyyr);
\coordinate (gliuapppns) at (\gliuaxxxn, \gliuayyys);
\coordinate (gliuapppnt) at (\gliuaxxxn, \gliuayyyt);
\coordinate (gliuapppnu) at (\gliuaxxxn, \gliuayyyu);
\coordinate (gliuapppnv) at (\gliuaxxxn, \gliuayyyv);
\coordinate (gliuapppnw) at (\gliuaxxxn, \gliuayyyw);
\coordinate (gliuapppnx) at (\gliuaxxxn, \gliuayyyx);
\coordinate (gliuapppny) at (\gliuaxxxn, \gliuayyyy);
\coordinate (gliuapppnz) at (\gliuaxxxn, \gliuayyyz);
\coordinate (gliuapppoa) at (\gliuaxxxo, \gliuayyya);
\coordinate (gliuapppob) at (\gliuaxxxo, \gliuayyyb);
\coordinate (gliuapppoc) at (\gliuaxxxo, \gliuayyyc);
\coordinate (gliuapppod) at (\gliuaxxxo, \gliuayyyd);
\coordinate (gliuapppoe) at (\gliuaxxxo, \gliuayyye);
\coordinate (gliuapppof) at (\gliuaxxxo, \gliuayyyf);
\coordinate (gliuapppog) at (\gliuaxxxo, \gliuayyyg);
\coordinate (gliuapppoh) at (\gliuaxxxo, \gliuayyyh);
\coordinate (gliuapppoi) at (\gliuaxxxo, \gliuayyyi);
\coordinate (gliuapppoj) at (\gliuaxxxo, \gliuayyyj);
\coordinate (gliuapppok) at (\gliuaxxxo, \gliuayyyk);
\coordinate (gliuapppol) at (\gliuaxxxo, \gliuayyyl);
\coordinate (gliuapppom) at (\gliuaxxxo, \gliuayyym);
\coordinate (gliuapppon) at (\gliuaxxxo, \gliuayyyn);
\coordinate (gliuapppoo) at (\gliuaxxxo, \gliuayyyo);
\coordinate (gliuapppop) at (\gliuaxxxo, \gliuayyyp);
\coordinate (gliuapppoq) at (\gliuaxxxo, \gliuayyyq);
\coordinate (gliuapppor) at (\gliuaxxxo, \gliuayyyr);
\coordinate (gliuapppos) at (\gliuaxxxo, \gliuayyys);
\coordinate (gliuapppot) at (\gliuaxxxo, \gliuayyyt);
\coordinate (gliuapppou) at (\gliuaxxxo, \gliuayyyu);
\coordinate (gliuapppov) at (\gliuaxxxo, \gliuayyyv);
\coordinate (gliuapppow) at (\gliuaxxxo, \gliuayyyw);
\coordinate (gliuapppox) at (\gliuaxxxo, \gliuayyyx);
\coordinate (gliuapppoy) at (\gliuaxxxo, \gliuayyyy);
\coordinate (gliuapppoz) at (\gliuaxxxo, \gliuayyyz);
\coordinate (gliuappppa) at (\gliuaxxxp, \gliuayyya);
\coordinate (gliuappppb) at (\gliuaxxxp, \gliuayyyb);
\coordinate (gliuappppc) at (\gliuaxxxp, \gliuayyyc);
\coordinate (gliuappppd) at (\gliuaxxxp, \gliuayyyd);
\coordinate (gliuappppe) at (\gliuaxxxp, \gliuayyye);
\coordinate (gliuappppf) at (\gliuaxxxp, \gliuayyyf);
\coordinate (gliuappppg) at (\gliuaxxxp, \gliuayyyg);
\coordinate (gliuapppph) at (\gliuaxxxp, \gliuayyyh);
\coordinate (gliuappppi) at (\gliuaxxxp, \gliuayyyi);
\coordinate (gliuappppj) at (\gliuaxxxp, \gliuayyyj);
\coordinate (gliuappppk) at (\gliuaxxxp, \gliuayyyk);
\coordinate (gliuappppl) at (\gliuaxxxp, \gliuayyyl);
\coordinate (gliuappppm) at (\gliuaxxxp, \gliuayyym);
\coordinate (gliuappppn) at (\gliuaxxxp, \gliuayyyn);
\coordinate (gliuappppo) at (\gliuaxxxp, \gliuayyyo);
\coordinate (gliuappppp) at (\gliuaxxxp, \gliuayyyp);
\coordinate (gliuappppq) at (\gliuaxxxp, \gliuayyyq);
\coordinate (gliuappppr) at (\gliuaxxxp, \gliuayyyr);
\coordinate (gliuapppps) at (\gliuaxxxp, \gliuayyys);
\coordinate (gliuappppt) at (\gliuaxxxp, \gliuayyyt);
\coordinate (gliuappppu) at (\gliuaxxxp, \gliuayyyu);
\coordinate (gliuappppv) at (\gliuaxxxp, \gliuayyyv);
\coordinate (gliuappppw) at (\gliuaxxxp, \gliuayyyw);
\coordinate (gliuappppx) at (\gliuaxxxp, \gliuayyyx);
\coordinate (gliuappppy) at (\gliuaxxxp, \gliuayyyy);
\coordinate (gliuappppz) at (\gliuaxxxp, \gliuayyyz);
\coordinate (gliuapppqa) at (\gliuaxxxq, \gliuayyya);
\coordinate (gliuapppqb) at (\gliuaxxxq, \gliuayyyb);
\coordinate (gliuapppqc) at (\gliuaxxxq, \gliuayyyc);
\coordinate (gliuapppqd) at (\gliuaxxxq, \gliuayyyd);
\coordinate (gliuapppqe) at (\gliuaxxxq, \gliuayyye);
\coordinate (gliuapppqf) at (\gliuaxxxq, \gliuayyyf);
\coordinate (gliuapppqg) at (\gliuaxxxq, \gliuayyyg);
\coordinate (gliuapppqh) at (\gliuaxxxq, \gliuayyyh);
\coordinate (gliuapppqi) at (\gliuaxxxq, \gliuayyyi);
\coordinate (gliuapppqj) at (\gliuaxxxq, \gliuayyyj);
\coordinate (gliuapppqk) at (\gliuaxxxq, \gliuayyyk);
\coordinate (gliuapppql) at (\gliuaxxxq, \gliuayyyl);
\coordinate (gliuapppqm) at (\gliuaxxxq, \gliuayyym);
\coordinate (gliuapppqn) at (\gliuaxxxq, \gliuayyyn);
\coordinate (gliuapppqo) at (\gliuaxxxq, \gliuayyyo);
\coordinate (gliuapppqp) at (\gliuaxxxq, \gliuayyyp);
\coordinate (gliuapppqq) at (\gliuaxxxq, \gliuayyyq);
\coordinate (gliuapppqr) at (\gliuaxxxq, \gliuayyyr);
\coordinate (gliuapppqs) at (\gliuaxxxq, \gliuayyys);
\coordinate (gliuapppqt) at (\gliuaxxxq, \gliuayyyt);
\coordinate (gliuapppqu) at (\gliuaxxxq, \gliuayyyu);
\coordinate (gliuapppqv) at (\gliuaxxxq, \gliuayyyv);
\coordinate (gliuapppqw) at (\gliuaxxxq, \gliuayyyw);
\coordinate (gliuapppqx) at (\gliuaxxxq, \gliuayyyx);
\coordinate (gliuapppqy) at (\gliuaxxxq, \gliuayyyy);
\coordinate (gliuapppqz) at (\gliuaxxxq, \gliuayyyz);
\coordinate (gliuapppra) at (\gliuaxxxr, \gliuayyya);
\coordinate (gliuappprb) at (\gliuaxxxr, \gliuayyyb);
\coordinate (gliuappprc) at (\gliuaxxxr, \gliuayyyc);
\coordinate (gliuappprd) at (\gliuaxxxr, \gliuayyyd);
\coordinate (gliuapppre) at (\gliuaxxxr, \gliuayyye);
\coordinate (gliuappprf) at (\gliuaxxxr, \gliuayyyf);
\coordinate (gliuappprg) at (\gliuaxxxr, \gliuayyyg);
\coordinate (gliuappprh) at (\gliuaxxxr, \gliuayyyh);
\coordinate (gliuapppri) at (\gliuaxxxr, \gliuayyyi);
\coordinate (gliuappprj) at (\gliuaxxxr, \gliuayyyj);
\coordinate (gliuappprk) at (\gliuaxxxr, \gliuayyyk);
\coordinate (gliuappprl) at (\gliuaxxxr, \gliuayyyl);
\coordinate (gliuappprm) at (\gliuaxxxr, \gliuayyym);
\coordinate (gliuappprn) at (\gliuaxxxr, \gliuayyyn);
\coordinate (gliuapppro) at (\gliuaxxxr, \gliuayyyo);
\coordinate (gliuappprp) at (\gliuaxxxr, \gliuayyyp);
\coordinate (gliuappprq) at (\gliuaxxxr, \gliuayyyq);
\coordinate (gliuappprr) at (\gliuaxxxr, \gliuayyyr);
\coordinate (gliuappprs) at (\gliuaxxxr, \gliuayyys);
\coordinate (gliuappprt) at (\gliuaxxxr, \gliuayyyt);
\coordinate (gliuapppru) at (\gliuaxxxr, \gliuayyyu);
\coordinate (gliuappprv) at (\gliuaxxxr, \gliuayyyv);
\coordinate (gliuappprw) at (\gliuaxxxr, \gliuayyyw);
\coordinate (gliuappprx) at (\gliuaxxxr, \gliuayyyx);
\coordinate (gliuapppry) at (\gliuaxxxr, \gliuayyyy);
\coordinate (gliuappprz) at (\gliuaxxxr, \gliuayyyz);
\coordinate (gliuapppsa) at (\gliuaxxxs, \gliuayyya);
\coordinate (gliuapppsb) at (\gliuaxxxs, \gliuayyyb);
\coordinate (gliuapppsc) at (\gliuaxxxs, \gliuayyyc);
\coordinate (gliuapppsd) at (\gliuaxxxs, \gliuayyyd);
\coordinate (gliuapppse) at (\gliuaxxxs, \gliuayyye);
\coordinate (gliuapppsf) at (\gliuaxxxs, \gliuayyyf);
\coordinate (gliuapppsg) at (\gliuaxxxs, \gliuayyyg);
\coordinate (gliuapppsh) at (\gliuaxxxs, \gliuayyyh);
\coordinate (gliuapppsi) at (\gliuaxxxs, \gliuayyyi);
\coordinate (gliuapppsj) at (\gliuaxxxs, \gliuayyyj);
\coordinate (gliuapppsk) at (\gliuaxxxs, \gliuayyyk);
\coordinate (gliuapppsl) at (\gliuaxxxs, \gliuayyyl);
\coordinate (gliuapppsm) at (\gliuaxxxs, \gliuayyym);
\coordinate (gliuapppsn) at (\gliuaxxxs, \gliuayyyn);
\coordinate (gliuapppso) at (\gliuaxxxs, \gliuayyyo);
\coordinate (gliuapppsp) at (\gliuaxxxs, \gliuayyyp);
\coordinate (gliuapppsq) at (\gliuaxxxs, \gliuayyyq);
\coordinate (gliuapppsr) at (\gliuaxxxs, \gliuayyyr);
\coordinate (gliuapppss) at (\gliuaxxxs, \gliuayyys);
\coordinate (gliuapppst) at (\gliuaxxxs, \gliuayyyt);
\coordinate (gliuapppsu) at (\gliuaxxxs, \gliuayyyu);
\coordinate (gliuapppsv) at (\gliuaxxxs, \gliuayyyv);
\coordinate (gliuapppsw) at (\gliuaxxxs, \gliuayyyw);
\coordinate (gliuapppsx) at (\gliuaxxxs, \gliuayyyx);
\coordinate (gliuapppsy) at (\gliuaxxxs, \gliuayyyy);
\coordinate (gliuapppsz) at (\gliuaxxxs, \gliuayyyz);
\coordinate (gliuapppta) at (\gliuaxxxt, \gliuayyya);
\coordinate (gliuappptb) at (\gliuaxxxt, \gliuayyyb);
\coordinate (gliuappptc) at (\gliuaxxxt, \gliuayyyc);
\coordinate (gliuappptd) at (\gliuaxxxt, \gliuayyyd);
\coordinate (gliuapppte) at (\gliuaxxxt, \gliuayyye);
\coordinate (gliuappptf) at (\gliuaxxxt, \gliuayyyf);
\coordinate (gliuappptg) at (\gliuaxxxt, \gliuayyyg);
\coordinate (gliuapppth) at (\gliuaxxxt, \gliuayyyh);
\coordinate (gliuapppti) at (\gliuaxxxt, \gliuayyyi);
\coordinate (gliuappptj) at (\gliuaxxxt, \gliuayyyj);
\coordinate (gliuappptk) at (\gliuaxxxt, \gliuayyyk);
\coordinate (gliuappptl) at (\gliuaxxxt, \gliuayyyl);
\coordinate (gliuappptm) at (\gliuaxxxt, \gliuayyym);
\coordinate (gliuappptn) at (\gliuaxxxt, \gliuayyyn);
\coordinate (gliuapppto) at (\gliuaxxxt, \gliuayyyo);
\coordinate (gliuappptp) at (\gliuaxxxt, \gliuayyyp);
\coordinate (gliuappptq) at (\gliuaxxxt, \gliuayyyq);
\coordinate (gliuappptr) at (\gliuaxxxt, \gliuayyyr);
\coordinate (gliuapppts) at (\gliuaxxxt, \gliuayyys);
\coordinate (gliuappptt) at (\gliuaxxxt, \gliuayyyt);
\coordinate (gliuappptu) at (\gliuaxxxt, \gliuayyyu);
\coordinate (gliuappptv) at (\gliuaxxxt, \gliuayyyv);
\coordinate (gliuappptw) at (\gliuaxxxt, \gliuayyyw);
\coordinate (gliuappptx) at (\gliuaxxxt, \gliuayyyx);
\coordinate (gliuapppty) at (\gliuaxxxt, \gliuayyyy);
\coordinate (gliuappptz) at (\gliuaxxxt, \gliuayyyz);
\coordinate (gliuapppua) at (\gliuaxxxu, \gliuayyya);
\coordinate (gliuapppub) at (\gliuaxxxu, \gliuayyyb);
\coordinate (gliuapppuc) at (\gliuaxxxu, \gliuayyyc);
\coordinate (gliuapppud) at (\gliuaxxxu, \gliuayyyd);
\coordinate (gliuapppue) at (\gliuaxxxu, \gliuayyye);
\coordinate (gliuapppuf) at (\gliuaxxxu, \gliuayyyf);
\coordinate (gliuapppug) at (\gliuaxxxu, \gliuayyyg);
\coordinate (gliuapppuh) at (\gliuaxxxu, \gliuayyyh);
\coordinate (gliuapppui) at (\gliuaxxxu, \gliuayyyi);
\coordinate (gliuapppuj) at (\gliuaxxxu, \gliuayyyj);
\coordinate (gliuapppuk) at (\gliuaxxxu, \gliuayyyk);
\coordinate (gliuapppul) at (\gliuaxxxu, \gliuayyyl);
\coordinate (gliuapppum) at (\gliuaxxxu, \gliuayyym);
\coordinate (gliuapppun) at (\gliuaxxxu, \gliuayyyn);
\coordinate (gliuapppuo) at (\gliuaxxxu, \gliuayyyo);
\coordinate (gliuapppup) at (\gliuaxxxu, \gliuayyyp);
\coordinate (gliuapppuq) at (\gliuaxxxu, \gliuayyyq);
\coordinate (gliuapppur) at (\gliuaxxxu, \gliuayyyr);
\coordinate (gliuapppus) at (\gliuaxxxu, \gliuayyys);
\coordinate (gliuappput) at (\gliuaxxxu, \gliuayyyt);
\coordinate (gliuapppuu) at (\gliuaxxxu, \gliuayyyu);
\coordinate (gliuapppuv) at (\gliuaxxxu, \gliuayyyv);
\coordinate (gliuapppuw) at (\gliuaxxxu, \gliuayyyw);
\coordinate (gliuapppux) at (\gliuaxxxu, \gliuayyyx);
\coordinate (gliuapppuy) at (\gliuaxxxu, \gliuayyyy);
\coordinate (gliuapppuz) at (\gliuaxxxu, \gliuayyyz);
\coordinate (gliuapppva) at (\gliuaxxxv, \gliuayyya);
\coordinate (gliuapppvb) at (\gliuaxxxv, \gliuayyyb);
\coordinate (gliuapppvc) at (\gliuaxxxv, \gliuayyyc);
\coordinate (gliuapppvd) at (\gliuaxxxv, \gliuayyyd);
\coordinate (gliuapppve) at (\gliuaxxxv, \gliuayyye);
\coordinate (gliuapppvf) at (\gliuaxxxv, \gliuayyyf);
\coordinate (gliuapppvg) at (\gliuaxxxv, \gliuayyyg);
\coordinate (gliuapppvh) at (\gliuaxxxv, \gliuayyyh);
\coordinate (gliuapppvi) at (\gliuaxxxv, \gliuayyyi);
\coordinate (gliuapppvj) at (\gliuaxxxv, \gliuayyyj);
\coordinate (gliuapppvk) at (\gliuaxxxv, \gliuayyyk);
\coordinate (gliuapppvl) at (\gliuaxxxv, \gliuayyyl);
\coordinate (gliuapppvm) at (\gliuaxxxv, \gliuayyym);
\coordinate (gliuapppvn) at (\gliuaxxxv, \gliuayyyn);
\coordinate (gliuapppvo) at (\gliuaxxxv, \gliuayyyo);
\coordinate (gliuapppvp) at (\gliuaxxxv, \gliuayyyp);
\coordinate (gliuapppvq) at (\gliuaxxxv, \gliuayyyq);
\coordinate (gliuapppvr) at (\gliuaxxxv, \gliuayyyr);
\coordinate (gliuapppvs) at (\gliuaxxxv, \gliuayyys);
\coordinate (gliuapppvt) at (\gliuaxxxv, \gliuayyyt);
\coordinate (gliuapppvu) at (\gliuaxxxv, \gliuayyyu);
\coordinate (gliuapppvv) at (\gliuaxxxv, \gliuayyyv);
\coordinate (gliuapppvw) at (\gliuaxxxv, \gliuayyyw);
\coordinate (gliuapppvx) at (\gliuaxxxv, \gliuayyyx);
\coordinate (gliuapppvy) at (\gliuaxxxv, \gliuayyyy);
\coordinate (gliuapppvz) at (\gliuaxxxv, \gliuayyyz);
\coordinate (gliuapppwa) at (\gliuaxxxw, \gliuayyya);
\coordinate (gliuapppwb) at (\gliuaxxxw, \gliuayyyb);
\coordinate (gliuapppwc) at (\gliuaxxxw, \gliuayyyc);
\coordinate (gliuapppwd) at (\gliuaxxxw, \gliuayyyd);
\coordinate (gliuapppwe) at (\gliuaxxxw, \gliuayyye);
\coordinate (gliuapppwf) at (\gliuaxxxw, \gliuayyyf);
\coordinate (gliuapppwg) at (\gliuaxxxw, \gliuayyyg);
\coordinate (gliuapppwh) at (\gliuaxxxw, \gliuayyyh);
\coordinate (gliuapppwi) at (\gliuaxxxw, \gliuayyyi);
\coordinate (gliuapppwj) at (\gliuaxxxw, \gliuayyyj);
\coordinate (gliuapppwk) at (\gliuaxxxw, \gliuayyyk);
\coordinate (gliuapppwl) at (\gliuaxxxw, \gliuayyyl);
\coordinate (gliuapppwm) at (\gliuaxxxw, \gliuayyym);
\coordinate (gliuapppwn) at (\gliuaxxxw, \gliuayyyn);
\coordinate (gliuapppwo) at (\gliuaxxxw, \gliuayyyo);
\coordinate (gliuapppwp) at (\gliuaxxxw, \gliuayyyp);
\coordinate (gliuapppwq) at (\gliuaxxxw, \gliuayyyq);
\coordinate (gliuapppwr) at (\gliuaxxxw, \gliuayyyr);
\coordinate (gliuapppws) at (\gliuaxxxw, \gliuayyys);
\coordinate (gliuapppwt) at (\gliuaxxxw, \gliuayyyt);
\coordinate (gliuapppwu) at (\gliuaxxxw, \gliuayyyu);
\coordinate (gliuapppwv) at (\gliuaxxxw, \gliuayyyv);
\coordinate (gliuapppww) at (\gliuaxxxw, \gliuayyyw);
\coordinate (gliuapppwx) at (\gliuaxxxw, \gliuayyyx);
\coordinate (gliuapppwy) at (\gliuaxxxw, \gliuayyyy);
\coordinate (gliuapppwz) at (\gliuaxxxw, \gliuayyyz);
\coordinate (gliuapppxa) at (\gliuaxxxx, \gliuayyya);
\coordinate (gliuapppxb) at (\gliuaxxxx, \gliuayyyb);
\coordinate (gliuapppxc) at (\gliuaxxxx, \gliuayyyc);
\coordinate (gliuapppxd) at (\gliuaxxxx, \gliuayyyd);
\coordinate (gliuapppxe) at (\gliuaxxxx, \gliuayyye);
\coordinate (gliuapppxf) at (\gliuaxxxx, \gliuayyyf);
\coordinate (gliuapppxg) at (\gliuaxxxx, \gliuayyyg);
\coordinate (gliuapppxh) at (\gliuaxxxx, \gliuayyyh);
\coordinate (gliuapppxi) at (\gliuaxxxx, \gliuayyyi);
\coordinate (gliuapppxj) at (\gliuaxxxx, \gliuayyyj);
\coordinate (gliuapppxk) at (\gliuaxxxx, \gliuayyyk);
\coordinate (gliuapppxl) at (\gliuaxxxx, \gliuayyyl);
\coordinate (gliuapppxm) at (\gliuaxxxx, \gliuayyym);
\coordinate (gliuapppxn) at (\gliuaxxxx, \gliuayyyn);
\coordinate (gliuapppxo) at (\gliuaxxxx, \gliuayyyo);
\coordinate (gliuapppxp) at (\gliuaxxxx, \gliuayyyp);
\coordinate (gliuapppxq) at (\gliuaxxxx, \gliuayyyq);
\coordinate (gliuapppxr) at (\gliuaxxxx, \gliuayyyr);
\coordinate (gliuapppxs) at (\gliuaxxxx, \gliuayyys);
\coordinate (gliuapppxt) at (\gliuaxxxx, \gliuayyyt);
\coordinate (gliuapppxu) at (\gliuaxxxx, \gliuayyyu);
\coordinate (gliuapppxv) at (\gliuaxxxx, \gliuayyyv);
\coordinate (gliuapppxw) at (\gliuaxxxx, \gliuayyyw);
\coordinate (gliuapppxx) at (\gliuaxxxx, \gliuayyyx);
\coordinate (gliuapppxy) at (\gliuaxxxx, \gliuayyyy);
\coordinate (gliuapppxz) at (\gliuaxxxx, \gliuayyyz);
\coordinate (gliuapppya) at (\gliuaxxxy, \gliuayyya);
\coordinate (gliuapppyb) at (\gliuaxxxy, \gliuayyyb);
\coordinate (gliuapppyc) at (\gliuaxxxy, \gliuayyyc);
\coordinate (gliuapppyd) at (\gliuaxxxy, \gliuayyyd);
\coordinate (gliuapppye) at (\gliuaxxxy, \gliuayyye);
\coordinate (gliuapppyf) at (\gliuaxxxy, \gliuayyyf);
\coordinate (gliuapppyg) at (\gliuaxxxy, \gliuayyyg);
\coordinate (gliuapppyh) at (\gliuaxxxy, \gliuayyyh);
\coordinate (gliuapppyi) at (\gliuaxxxy, \gliuayyyi);
\coordinate (gliuapppyj) at (\gliuaxxxy, \gliuayyyj);
\coordinate (gliuapppyk) at (\gliuaxxxy, \gliuayyyk);
\coordinate (gliuapppyl) at (\gliuaxxxy, \gliuayyyl);
\coordinate (gliuapppym) at (\gliuaxxxy, \gliuayyym);
\coordinate (gliuapppyn) at (\gliuaxxxy, \gliuayyyn);
\coordinate (gliuapppyo) at (\gliuaxxxy, \gliuayyyo);
\coordinate (gliuapppyp) at (\gliuaxxxy, \gliuayyyp);
\coordinate (gliuapppyq) at (\gliuaxxxy, \gliuayyyq);
\coordinate (gliuapppyr) at (\gliuaxxxy, \gliuayyyr);
\coordinate (gliuapppys) at (\gliuaxxxy, \gliuayyys);
\coordinate (gliuapppyt) at (\gliuaxxxy, \gliuayyyt);
\coordinate (gliuapppyu) at (\gliuaxxxy, \gliuayyyu);
\coordinate (gliuapppyv) at (\gliuaxxxy, \gliuayyyv);
\coordinate (gliuapppyw) at (\gliuaxxxy, \gliuayyyw);
\coordinate (gliuapppyx) at (\gliuaxxxy, \gliuayyyx);
\coordinate (gliuapppyy) at (\gliuaxxxy, \gliuayyyy);
\coordinate (gliuapppyz) at (\gliuaxxxy, \gliuayyyz);
\coordinate (gliuapppza) at (\gliuaxxxz, \gliuayyya);
\coordinate (gliuapppzb) at (\gliuaxxxz, \gliuayyyb);
\coordinate (gliuapppzc) at (\gliuaxxxz, \gliuayyyc);
\coordinate (gliuapppzd) at (\gliuaxxxz, \gliuayyyd);
\coordinate (gliuapppze) at (\gliuaxxxz, \gliuayyye);
\coordinate (gliuapppzf) at (\gliuaxxxz, \gliuayyyf);
\coordinate (gliuapppzg) at (\gliuaxxxz, \gliuayyyg);
\coordinate (gliuapppzh) at (\gliuaxxxz, \gliuayyyh);
\coordinate (gliuapppzi) at (\gliuaxxxz, \gliuayyyi);
\coordinate (gliuapppzj) at (\gliuaxxxz, \gliuayyyj);
\coordinate (gliuapppzk) at (\gliuaxxxz, \gliuayyyk);
\coordinate (gliuapppzl) at (\gliuaxxxz, \gliuayyyl);
\coordinate (gliuapppzm) at (\gliuaxxxz, \gliuayyym);
\coordinate (gliuapppzn) at (\gliuaxxxz, \gliuayyyn);
\coordinate (gliuapppzo) at (\gliuaxxxz, \gliuayyyo);
\coordinate (gliuapppzp) at (\gliuaxxxz, \gliuayyyp);
\coordinate (gliuapppzq) at (\gliuaxxxz, \gliuayyyq);
\coordinate (gliuapppzr) at (\gliuaxxxz, \gliuayyyr);
\coordinate (gliuapppzs) at (\gliuaxxxz, \gliuayyys);
\coordinate (gliuapppzt) at (\gliuaxxxz, \gliuayyyt);
\coordinate (gliuapppzu) at (\gliuaxxxz, \gliuayyyu);
\coordinate (gliuapppzv) at (\gliuaxxxz, \gliuayyyv);
\coordinate (gliuapppzw) at (\gliuaxxxz, \gliuayyyw);
\coordinate (gliuapppzx) at (\gliuaxxxz, \gliuayyyx);
\coordinate (gliuapppzy) at (\gliuaxxxz, \gliuayyyy);
\coordinate (gliuapppzz) at (\gliuaxxxz, \gliuayyyz);

%\gangprintcoordinateat{(0,0)}{The last coordinate values: }{($(gliuapppzz)$)}; 



\pgfmathsetmacro{\totalgliubxxx}{12}
\pgfmathsetmacro{\totalgliubyyy}{12}
\pgfmathsetmacro{\gliubxxxspacing}{1}
\pgfmathsetmacro{\gliubyyyspacing}{1}
\pgfmathsetmacro{\gliubxxxa}{\gangliuxxxj}
\pgfmathsetmacro{\gliubyyya}{\gangliuyyyj}

\pgfmathsetmacro{\gliubxxxb}{\gliubxxxa + \gliubxxxspacing + 0.0 }
\pgfmathsetmacro{\gliubxxxc}{\gliubxxxb + \gliubxxxspacing + 0.0 }
\pgfmathsetmacro{\gliubxxxd}{\gliubxxxc + \gliubxxxspacing + 0.0 }
\pgfmathsetmacro{\gliubxxxe}{\gliubxxxd + \gliubxxxspacing + 0.0 }
\pgfmathsetmacro{\gliubxxxf}{\gliubxxxe + \gliubxxxspacing + 0.0 }
\pgfmathsetmacro{\gliubxxxg}{\gliubxxxf + \gliubxxxspacing + 0.0 }
\pgfmathsetmacro{\gliubxxxh}{\gliubxxxg + \gliubxxxspacing + 0.0 }
\pgfmathsetmacro{\gliubxxxi}{\gliubxxxh + \gliubxxxspacing + 0.0 }
\pgfmathsetmacro{\gliubxxxj}{\gliubxxxi + \gliubxxxspacing + 0.0 }
\pgfmathsetmacro{\gliubxxxk}{\gliubxxxj + \gliubxxxspacing + 0.0 }
\pgfmathsetmacro{\gliubxxxl}{\gliubxxxk + \gliubxxxspacing + 0.0 }

\pgfmathsetmacro{\gliubyyyb}{\gliubyyya + \gliubyyyspacing + 0.0 }
\pgfmathsetmacro{\gliubyyyc}{\gliubyyyb + \gliubyyyspacing + 0.0 }
\pgfmathsetmacro{\gliubyyyd}{\gliubyyyc + \gliubyyyspacing + 0.0 }
\pgfmathsetmacro{\gliubyyye}{\gliubyyyd + \gliubyyyspacing + 0.0 }
\pgfmathsetmacro{\gliubyyyf}{\gliubyyye + \gliubyyyspacing + 0.0 }
\pgfmathsetmacro{\gliubyyyg}{\gliubyyyf + \gliubyyyspacing + 0.0 }
\pgfmathsetmacro{\gliubyyyh}{\gliubyyyg + \gliubyyyspacing + 0.0 }
\pgfmathsetmacro{\gliubyyyi}{\gliubyyyh + \gliubyyyspacing + 0.0 }
\pgfmathsetmacro{\gliubyyyj}{\gliubyyyi + \gliubyyyspacing + 0.0 }
\pgfmathsetmacro{\gliubyyyk}{\gliubyyyj + \gliubyyyspacing + 0.0 }
\pgfmathsetmacro{\gliubyyyl}{\gliubyyyk + \gliubyyyspacing + 0.0 }

\coordinate (gliubpppaa) at (\gliubxxxa, \gliubyyya);
\coordinate (gliubpppab) at (\gliubxxxa, \gliubyyyb);
\coordinate (gliubpppac) at (\gliubxxxa, \gliubyyyc);
\coordinate (gliubpppad) at (\gliubxxxa, \gliubyyyd);
\coordinate (gliubpppae) at (\gliubxxxa, \gliubyyye);
\coordinate (gliubpppaf) at (\gliubxxxa, \gliubyyyf);
\coordinate (gliubpppag) at (\gliubxxxa, \gliubyyyg);
\coordinate (gliubpppah) at (\gliubxxxa, \gliubyyyh);
\coordinate (gliubpppai) at (\gliubxxxa, \gliubyyyi);
\coordinate (gliubpppaj) at (\gliubxxxa, \gliubyyyj);
\coordinate (gliubpppak) at (\gliubxxxa, \gliubyyyk);
\coordinate (gliubpppal) at (\gliubxxxa, \gliubyyyl);
\coordinate (gliubpppba) at (\gliubxxxb, \gliubyyya);
\coordinate (gliubpppbb) at (\gliubxxxb, \gliubyyyb);
\coordinate (gliubpppbc) at (\gliubxxxb, \gliubyyyc);
\coordinate (gliubpppbd) at (\gliubxxxb, \gliubyyyd);
\coordinate (gliubpppbe) at (\gliubxxxb, \gliubyyye);
\coordinate (gliubpppbf) at (\gliubxxxb, \gliubyyyf);
\coordinate (gliubpppbg) at (\gliubxxxb, \gliubyyyg);
\coordinate (gliubpppbh) at (\gliubxxxb, \gliubyyyh);
\coordinate (gliubpppbi) at (\gliubxxxb, \gliubyyyi);
\coordinate (gliubpppbj) at (\gliubxxxb, \gliubyyyj);
\coordinate (gliubpppbk) at (\gliubxxxb, \gliubyyyk);
\coordinate (gliubpppbl) at (\gliubxxxb, \gliubyyyl);
\coordinate (gliubpppca) at (\gliubxxxc, \gliubyyya);
\coordinate (gliubpppcb) at (\gliubxxxc, \gliubyyyb);
\coordinate (gliubpppcc) at (\gliubxxxc, \gliubyyyc);
\coordinate (gliubpppcd) at (\gliubxxxc, \gliubyyyd);
\coordinate (gliubpppce) at (\gliubxxxc, \gliubyyye);
\coordinate (gliubpppcf) at (\gliubxxxc, \gliubyyyf);
\coordinate (gliubpppcg) at (\gliubxxxc, \gliubyyyg);
\coordinate (gliubpppch) at (\gliubxxxc, \gliubyyyh);
\coordinate (gliubpppci) at (\gliubxxxc, \gliubyyyi);
\coordinate (gliubpppcj) at (\gliubxxxc, \gliubyyyj);
\coordinate (gliubpppck) at (\gliubxxxc, \gliubyyyk);
\coordinate (gliubpppcl) at (\gliubxxxc, \gliubyyyl);
\coordinate (gliubpppda) at (\gliubxxxd, \gliubyyya);
\coordinate (gliubpppdb) at (\gliubxxxd, \gliubyyyb);
\coordinate (gliubpppdc) at (\gliubxxxd, \gliubyyyc);
\coordinate (gliubpppdd) at (\gliubxxxd, \gliubyyyd);
\coordinate (gliubpppde) at (\gliubxxxd, \gliubyyye);
\coordinate (gliubpppdf) at (\gliubxxxd, \gliubyyyf);
\coordinate (gliubpppdg) at (\gliubxxxd, \gliubyyyg);
\coordinate (gliubpppdh) at (\gliubxxxd, \gliubyyyh);
\coordinate (gliubpppdi) at (\gliubxxxd, \gliubyyyi);
\coordinate (gliubpppdj) at (\gliubxxxd, \gliubyyyj);
\coordinate (gliubpppdk) at (\gliubxxxd, \gliubyyyk);
\coordinate (gliubpppdl) at (\gliubxxxd, \gliubyyyl);
\coordinate (gliubpppea) at (\gliubxxxe, \gliubyyya);
\coordinate (gliubpppeb) at (\gliubxxxe, \gliubyyyb);
\coordinate (gliubpppec) at (\gliubxxxe, \gliubyyyc);
\coordinate (gliubppped) at (\gliubxxxe, \gliubyyyd);
\coordinate (gliubpppee) at (\gliubxxxe, \gliubyyye);
\coordinate (gliubpppef) at (\gliubxxxe, \gliubyyyf);
\coordinate (gliubpppeg) at (\gliubxxxe, \gliubyyyg);
\coordinate (gliubpppeh) at (\gliubxxxe, \gliubyyyh);
\coordinate (gliubpppei) at (\gliubxxxe, \gliubyyyi);
\coordinate (gliubpppej) at (\gliubxxxe, \gliubyyyj);
\coordinate (gliubpppek) at (\gliubxxxe, \gliubyyyk);
\coordinate (gliubpppel) at (\gliubxxxe, \gliubyyyl);
\coordinate (gliubpppfa) at (\gliubxxxf, \gliubyyya);
\coordinate (gliubpppfb) at (\gliubxxxf, \gliubyyyb);
\coordinate (gliubpppfc) at (\gliubxxxf, \gliubyyyc);
\coordinate (gliubpppfd) at (\gliubxxxf, \gliubyyyd);
\coordinate (gliubpppfe) at (\gliubxxxf, \gliubyyye);
\coordinate (gliubpppff) at (\gliubxxxf, \gliubyyyf);
\coordinate (gliubpppfg) at (\gliubxxxf, \gliubyyyg);
\coordinate (gliubpppfh) at (\gliubxxxf, \gliubyyyh);
\coordinate (gliubpppfi) at (\gliubxxxf, \gliubyyyi);
\coordinate (gliubpppfj) at (\gliubxxxf, \gliubyyyj);
\coordinate (gliubpppfk) at (\gliubxxxf, \gliubyyyk);
\coordinate (gliubpppfl) at (\gliubxxxf, \gliubyyyl);
\coordinate (gliubpppga) at (\gliubxxxg, \gliubyyya);
\coordinate (gliubpppgb) at (\gliubxxxg, \gliubyyyb);
\coordinate (gliubpppgc) at (\gliubxxxg, \gliubyyyc);
\coordinate (gliubpppgd) at (\gliubxxxg, \gliubyyyd);
\coordinate (gliubpppge) at (\gliubxxxg, \gliubyyye);
\coordinate (gliubpppgf) at (\gliubxxxg, \gliubyyyf);
\coordinate (gliubpppgg) at (\gliubxxxg, \gliubyyyg);
\coordinate (gliubpppgh) at (\gliubxxxg, \gliubyyyh);
\coordinate (gliubpppgi) at (\gliubxxxg, \gliubyyyi);
\coordinate (gliubpppgj) at (\gliubxxxg, \gliubyyyj);
\coordinate (gliubpppgk) at (\gliubxxxg, \gliubyyyk);
\coordinate (gliubpppgl) at (\gliubxxxg, \gliubyyyl);
\coordinate (gliubpppha) at (\gliubxxxh, \gliubyyya);
\coordinate (gliubppphb) at (\gliubxxxh, \gliubyyyb);
\coordinate (gliubppphc) at (\gliubxxxh, \gliubyyyc);
\coordinate (gliubppphd) at (\gliubxxxh, \gliubyyyd);
\coordinate (gliubppphe) at (\gliubxxxh, \gliubyyye);
\coordinate (gliubppphf) at (\gliubxxxh, \gliubyyyf);
\coordinate (gliubppphg) at (\gliubxxxh, \gliubyyyg);
\coordinate (gliubppphh) at (\gliubxxxh, \gliubyyyh);
\coordinate (gliubppphi) at (\gliubxxxh, \gliubyyyi);
\coordinate (gliubppphj) at (\gliubxxxh, \gliubyyyj);
\coordinate (gliubppphk) at (\gliubxxxh, \gliubyyyk);
\coordinate (gliubppphl) at (\gliubxxxh, \gliubyyyl);
\coordinate (gliubpppia) at (\gliubxxxi, \gliubyyya);
\coordinate (gliubpppib) at (\gliubxxxi, \gliubyyyb);
\coordinate (gliubpppic) at (\gliubxxxi, \gliubyyyc);
\coordinate (gliubpppid) at (\gliubxxxi, \gliubyyyd);
\coordinate (gliubpppie) at (\gliubxxxi, \gliubyyye);
\coordinate (gliubpppif) at (\gliubxxxi, \gliubyyyf);
\coordinate (gliubpppig) at (\gliubxxxi, \gliubyyyg);
\coordinate (gliubpppih) at (\gliubxxxi, \gliubyyyh);
\coordinate (gliubpppii) at (\gliubxxxi, \gliubyyyi);
\coordinate (gliubpppij) at (\gliubxxxi, \gliubyyyj);
\coordinate (gliubpppik) at (\gliubxxxi, \gliubyyyk);
\coordinate (gliubpppil) at (\gliubxxxi, \gliubyyyl);
\coordinate (gliubpppja) at (\gliubxxxj, \gliubyyya);
\coordinate (gliubpppjb) at (\gliubxxxj, \gliubyyyb);
\coordinate (gliubpppjc) at (\gliubxxxj, \gliubyyyc);
\coordinate (gliubpppjd) at (\gliubxxxj, \gliubyyyd);
\coordinate (gliubpppje) at (\gliubxxxj, \gliubyyye);
\coordinate (gliubpppjf) at (\gliubxxxj, \gliubyyyf);
\coordinate (gliubpppjg) at (\gliubxxxj, \gliubyyyg);
\coordinate (gliubpppjh) at (\gliubxxxj, \gliubyyyh);
\coordinate (gliubpppji) at (\gliubxxxj, \gliubyyyi);
\coordinate (gliubpppjj) at (\gliubxxxj, \gliubyyyj);
\coordinate (gliubpppjk) at (\gliubxxxj, \gliubyyyk);
\coordinate (gliubpppjl) at (\gliubxxxj, \gliubyyyl);
\coordinate (gliubpppka) at (\gliubxxxk, \gliubyyya);
\coordinate (gliubpppkb) at (\gliubxxxk, \gliubyyyb);
\coordinate (gliubpppkc) at (\gliubxxxk, \gliubyyyc);
\coordinate (gliubpppkd) at (\gliubxxxk, \gliubyyyd);
\coordinate (gliubpppke) at (\gliubxxxk, \gliubyyye);
\coordinate (gliubpppkf) at (\gliubxxxk, \gliubyyyf);
\coordinate (gliubpppkg) at (\gliubxxxk, \gliubyyyg);
\coordinate (gliubpppkh) at (\gliubxxxk, \gliubyyyh);
\coordinate (gliubpppki) at (\gliubxxxk, \gliubyyyi);
\coordinate (gliubpppkj) at (\gliubxxxk, \gliubyyyj);
\coordinate (gliubpppkk) at (\gliubxxxk, \gliubyyyk);
\coordinate (gliubpppkl) at (\gliubxxxk, \gliubyyyl);
\coordinate (gliubpppla) at (\gliubxxxl, \gliubyyya);
\coordinate (gliubppplb) at (\gliubxxxl, \gliubyyyb);
\coordinate (gliubppplc) at (\gliubxxxl, \gliubyyyc);
\coordinate (gliubpppld) at (\gliubxxxl, \gliubyyyd);
\coordinate (gliubppple) at (\gliubxxxl, \gliubyyye);
\coordinate (gliubppplf) at (\gliubxxxl, \gliubyyyf);
\coordinate (gliubppplg) at (\gliubxxxl, \gliubyyyg);
\coordinate (gliubppplh) at (\gliubxxxl, \gliubyyyh);
\coordinate (gliubpppli) at (\gliubxxxl, \gliubyyyi);
\coordinate (gliubppplj) at (\gliubxxxl, \gliubyyyj);
\coordinate (gliubppplk) at (\gliubxxxl, \gliubyyyk);
\coordinate (gliubpppll) at (\gliubxxxl, \gliubyyyl);

%\gangprintcoordinateat{(0,0)}{The last coordinate values: }{($(gliubpppll)$)}; 



% Draw related part of the coordinate system with dashed helplines with letters as background, which would help to determine all coordinates. 
\coordinatebackgroundxy{gangliu} {f}{g}{v} {f}{g}{q};

%%%%%% The next line is for circuit 3.
\coordinatebackgroundxy{gliua}{a}{b}{h} {a}{b}{h};

%%%%%% The next line is for circuit 4.
\coordinatebackgroundxy{gliub}{b}{c}{f} {a}{b}{g};


% Draw the Opamp at the coordinate (gangliupppli) and name it as "myopamp".
\draw (gangliupppli) node [op amp] (myopamp) {};

% Retrieve the x- and y-components of the coordinates of the "+", "-", and "out" pins of myopamp, supposing we have no idea about them beforehand. 
\getxyingivenunit{cm}{(myopamp.+)}
                 {\myopamppx}{\myopamppy};
\getxyingivenunit{cm}{(myopamp.-)}
                 {\myopampmx}{\myopampmy};
\getxyingivenunit{cm}{(myopamp.out)}
                 {\myopampox}{\myopampoy};

\draw [-o] (myopamp.out) 
      to [short, xshift=1mm] 
      (\gangliuxxxr, \myopampoy) 
        node [anchor=north, yshift=-1mm] {$V_0$};

\draw [-o] (myopamp.+) 
      to [short, xshift=-1mm] 
      (\gangliuxxxj, \myopamppy) 
      node [anchor=north, yshift=-1mm] {$V_i$};

\draw (myopamp.-) -- 
      (\gangliuxxxj, \myopampmy) 
      to [R, l_=$\hspace{-2mm} R \text{=} 100 K\Omega$] 
      (\gangliuxxxh, \myopampmy) -- 
      (gangliuppphi) node [ground]{};

%      to [R] 
%      to [R = $R \text{=} 100 K\Omega$] 
%      to [R, l_=$R \text{=} 100 K\Omega$] 
%      to [R, l_=$\hspace{-2mm} R \text{=} 100 K\Omega$]
%      to [R, n=resistorl] 
%\node [anchor=south, xshift=-1mm, yshift=1mm] 
%      at (resistorl) {$R \text{=} 100 K\Omega$};

      
\draw (\gangliuxxxj, \myopampmy) -- 
      (gangliupppjj) 
      to [R, l_=$R_F \text{=} 300K \Omega$,
                         label/align=rotate] 
      (gangliupppjm);

%%%%%% Remove the next line for circuit 2.
% \draw (gangliupppjm) -| (gangliupppni);
      

%%%%%% The rest are added for circuit 2.      
%%%%%% The rest are added for circuit 2.      

\draw (gangliuppppi) |- (gangliupppno) --
      (gangliupppnn);

\draw (gangliupppnl) 
      to[american potentiometer, n=mypot, 
           l_=$R_P \text{=} 5 K \Omega$,   
                       label/align=rotate] 
      (gangliupppnn);

\draw (gangliupppnl) -- 
      (gangliupppnk) node [ground]{};


%%%%%% Remove the following line for circuit 3. 
%\draw (mypot.wiper) 
%        node [red, anchor=south east] {$V_A$} -| 
%      (gangliupppjm);


%%%%%% Add the rest lines for circuit 3. 
\getxyingivenunit{cm}{(mypot.wiper)}
                 {\mypotwiperx}{\mypotwipery};

\draw (mypot.wiper) 
        node [red, anchor=south east] {$V_A$} --
      (\gliuaxxxe, \mypotwipery);

\draw (\gliuaxxxe, \mypotwipery)  
      to[variable resistor = $R_D \text{=} 15K \Omega$] 
      (\gliuaxxxb, \mypotwipery) 
      node [red, anchor=south] {$V_B$};

\draw (\gliuaxxxb, \mypotwipery) -- 
      (gliuapppbd)
      to [C = $C_D \text{=} 100 \mu F$] 
      (gliuapppbb) node [ground] {};


%%%%%% Remove the following line for circuit 4.
%\draw (\gliuaxxxb, \mypotwipery) -| (gangliupppjm);


%%%%%% Add the rest lines for circuit 4.

\draw (\gliuaxxxb, \mypotwipery) --
      (\gliubxxxf, \mypotwipery)
      to [C = $C_I \text{=} 200 \mu F$] 
      (\gliubxxxc, \mypotwipery);

\draw (\gliubxxxc, \mypotwipery) -- 
      (gliubpppcd)
      to[variable resistor = $R_I \text{=} 39K \Omega$] 
      (gliubpppcb) node [ground] {};

\draw (\gliubxxxc, \mypotwipery) 
        node [red, anchor=south] {$V_C$} -|
      (gangliupppjm);


% The next line is for moving the circuit to the center a little bit only, but the drawn line is not visible.
\draw [white] (gangliupppdp) -- (gangliupppgp);


\end{circuitikz}



\end{document}
