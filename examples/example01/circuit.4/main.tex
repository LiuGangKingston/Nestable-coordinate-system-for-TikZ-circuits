% This is circuit 4 of example 01 of
% https://github.com/LiuGangKingston/Nestable-coordinate-system-for-TikZ-circuits.git


\documentclass[tikz,border=5mm]{standalone}
\usepackage[siunitx]{circuitikz}
\usetikzlibrary{shapes,arrows,positioning}
\input{afewmorecommandsintikzpiture}


\begin{document}

\ctikzset{
/tikz/circuitikz/bipoles/length=1cm
}



 
 
\begin{circuitikz} [scale=0.8]
 
%%%%%% The next line is for circuit 4.
https://github.com/LiuGangKingston/Nestable-coordinate-system-for-Tikz-circuits.git
https://github.com/LiuGangKingston/Nestable-coordinate-system-for-Tikz-circuits.git


\pgfmathsetmacro{\totalgliubxxx}{26}
\pgfmathsetmacro{\totalgliubyyy}{26}
\pgfmathsetmacro{\gliubxxxspacing}{1}
\pgfmathsetmacro{\gliubyyyspacing}{1}
\pgfmathsetmacro{\gliubxxxa}{-8}
\pgfmathsetmacro{\gliubyyya}{-8}

\pgfmathsetmacro{\gliubxxxb}{\gliubxxxa + \gliubxxxspacing + 0.0 }
\pgfmathsetmacro{\gliubxxxc}{\gliubxxxb + \gliubxxxspacing + 0.0 }
\pgfmathsetmacro{\gliubxxxd}{\gliubxxxc + \gliubxxxspacing + 0.0 }
\pgfmathsetmacro{\gliubxxxe}{\gliubxxxd + \gliubxxxspacing + 0.0 }
\pgfmathsetmacro{\gliubxxxf}{\gliubxxxe + \gliubxxxspacing + 0.0 }
\pgfmathsetmacro{\gliubxxxg}{\gliubxxxf + \gliubxxxspacing + 0.0 }
\pgfmathsetmacro{\gliubxxxh}{\gliubxxxg + \gliubxxxspacing + 0.0 }
\pgfmathsetmacro{\gliubxxxi}{\gliubxxxh + \gliubxxxspacing + 0.0 }
\pgfmathsetmacro{\gliubxxxj}{\gliubxxxi + \gliubxxxspacing + 0.0 }
\pgfmathsetmacro{\gliubxxxk}{\gliubxxxj + \gliubxxxspacing + 0.0 }
\pgfmathsetmacro{\gliubxxxl}{\gliubxxxk + \gliubxxxspacing + 0.0 }
\pgfmathsetmacro{\gliubxxxm}{\gliubxxxl + \gliubxxxspacing + 0.0 }
\pgfmathsetmacro{\gliubxxxn}{\gliubxxxm + \gliubxxxspacing + 0.0 }
\pgfmathsetmacro{\gliubxxxo}{\gliubxxxn + \gliubxxxspacing + 0.0 }
\pgfmathsetmacro{\gliubxxxp}{\gliubxxxo + \gliubxxxspacing + 0.0 }
\pgfmathsetmacro{\gliubxxxq}{\gliubxxxp + \gliubxxxspacing + 0.0 }
\pgfmathsetmacro{\gliubxxxr}{\gliubxxxq + \gliubxxxspacing + 0.0 }
\pgfmathsetmacro{\gliubxxxs}{\gliubxxxr + \gliubxxxspacing + 0.0 }
\pgfmathsetmacro{\gliubxxxt}{\gliubxxxs + \gliubxxxspacing + 0.0 }
\pgfmathsetmacro{\gliubxxxu}{\gliubxxxt + \gliubxxxspacing + 0.0 }
\pgfmathsetmacro{\gliubxxxv}{\gliubxxxu + \gliubxxxspacing + 0.0 }
\pgfmathsetmacro{\gliubxxxw}{\gliubxxxv + \gliubxxxspacing + 0.0 }
\pgfmathsetmacro{\gliubxxxx}{\gliubxxxw + \gliubxxxspacing + 0.0 }
\pgfmathsetmacro{\gliubxxxy}{\gliubxxxx + \gliubxxxspacing + 0.0 }
\pgfmathsetmacro{\gliubxxxz}{\gliubxxxy + \gliubxxxspacing + 0.0 }

\pgfmathsetmacro{\gliubyyyb}{\gliubyyya + \gliubyyyspacing + 0.0 }
\pgfmathsetmacro{\gliubyyyc}{\gliubyyyb + \gliubyyyspacing + 0.0 }
\pgfmathsetmacro{\gliubyyyd}{\gliubyyyc + \gliubyyyspacing + 0.0 }
\pgfmathsetmacro{\gliubyyye}{\gliubyyyd + \gliubyyyspacing + 0.0 }
\pgfmathsetmacro{\gliubyyyf}{\gliubyyye + \gliubyyyspacing + 0.0 }
\pgfmathsetmacro{\gliubyyyg}{\gliubyyyf + \gliubyyyspacing + 0.0 }
\pgfmathsetmacro{\gliubyyyh}{\gliubyyyg + \gliubyyyspacing + 0.0 }
\pgfmathsetmacro{\gliubyyyi}{\gliubyyyh + \gliubyyyspacing + 0.0 }
\pgfmathsetmacro{\gliubyyyj}{\gliubyyyi + \gliubyyyspacing + 0.0 }
\pgfmathsetmacro{\gliubyyyk}{\gliubyyyj + \gliubyyyspacing + 0.0 }
\pgfmathsetmacro{\gliubyyyl}{\gliubyyyk + \gliubyyyspacing + 0.0 }
\pgfmathsetmacro{\gliubyyym}{\gliubyyyl + \gliubyyyspacing + 0.0 }
\pgfmathsetmacro{\gliubyyyn}{\gliubyyym + \gliubyyyspacing + 0.0 }
\pgfmathsetmacro{\gliubyyyo}{\gliubyyyn + \gliubyyyspacing + 0.0 }
\pgfmathsetmacro{\gliubyyyp}{\gliubyyyo + \gliubyyyspacing + 0.0 }
\pgfmathsetmacro{\gliubyyyq}{\gliubyyyp + \gliubyyyspacing + 0.0 }
\pgfmathsetmacro{\gliubyyyr}{\gliubyyyq + \gliubyyyspacing + 0.0 }
\pgfmathsetmacro{\gliubyyys}{\gliubyyyr + \gliubyyyspacing + 0.0 }
\pgfmathsetmacro{\gliubyyyt}{\gliubyyys + \gliubyyyspacing + 0.0 }
\pgfmathsetmacro{\gliubyyyu}{\gliubyyyt + \gliubyyyspacing + 0.0 }
\pgfmathsetmacro{\gliubyyyv}{\gliubyyyu + \gliubyyyspacing + 0.0 }
\pgfmathsetmacro{\gliubyyyw}{\gliubyyyv + \gliubyyyspacing + 0.0 }
\pgfmathsetmacro{\gliubyyyx}{\gliubyyyw + \gliubyyyspacing + 0.0 }
\pgfmathsetmacro{\gliubyyyy}{\gliubyyyx + \gliubyyyspacing + 0.0 }
\pgfmathsetmacro{\gliubyyyz}{\gliubyyyy + \gliubyyyspacing + 0.0 }

\coordinate (gliubpppaa) at (\gliubxxxa, \gliubyyya);
\coordinate (gliubpppab) at (\gliubxxxa, \gliubyyyb);
\coordinate (gliubpppac) at (\gliubxxxa, \gliubyyyc);
\coordinate (gliubpppad) at (\gliubxxxa, \gliubyyyd);
\coordinate (gliubpppae) at (\gliubxxxa, \gliubyyye);
\coordinate (gliubpppaf) at (\gliubxxxa, \gliubyyyf);
\coordinate (gliubpppag) at (\gliubxxxa, \gliubyyyg);
\coordinate (gliubpppah) at (\gliubxxxa, \gliubyyyh);
\coordinate (gliubpppai) at (\gliubxxxa, \gliubyyyi);
\coordinate (gliubpppaj) at (\gliubxxxa, \gliubyyyj);
\coordinate (gliubpppak) at (\gliubxxxa, \gliubyyyk);
\coordinate (gliubpppal) at (\gliubxxxa, \gliubyyyl);
\coordinate (gliubpppam) at (\gliubxxxa, \gliubyyym);
\coordinate (gliubpppan) at (\gliubxxxa, \gliubyyyn);
\coordinate (gliubpppao) at (\gliubxxxa, \gliubyyyo);
\coordinate (gliubpppap) at (\gliubxxxa, \gliubyyyp);
\coordinate (gliubpppaq) at (\gliubxxxa, \gliubyyyq);
\coordinate (gliubpppar) at (\gliubxxxa, \gliubyyyr);
\coordinate (gliubpppas) at (\gliubxxxa, \gliubyyys);
\coordinate (gliubpppat) at (\gliubxxxa, \gliubyyyt);
\coordinate (gliubpppau) at (\gliubxxxa, \gliubyyyu);
\coordinate (gliubpppav) at (\gliubxxxa, \gliubyyyv);
\coordinate (gliubpppaw) at (\gliubxxxa, \gliubyyyw);
\coordinate (gliubpppax) at (\gliubxxxa, \gliubyyyx);
\coordinate (gliubpppay) at (\gliubxxxa, \gliubyyyy);
\coordinate (gliubpppaz) at (\gliubxxxa, \gliubyyyz);
\coordinate (gliubpppba) at (\gliubxxxb, \gliubyyya);
\coordinate (gliubpppbb) at (\gliubxxxb, \gliubyyyb);
\coordinate (gliubpppbc) at (\gliubxxxb, \gliubyyyc);
\coordinate (gliubpppbd) at (\gliubxxxb, \gliubyyyd);
\coordinate (gliubpppbe) at (\gliubxxxb, \gliubyyye);
\coordinate (gliubpppbf) at (\gliubxxxb, \gliubyyyf);
\coordinate (gliubpppbg) at (\gliubxxxb, \gliubyyyg);
\coordinate (gliubpppbh) at (\gliubxxxb, \gliubyyyh);
\coordinate (gliubpppbi) at (\gliubxxxb, \gliubyyyi);
\coordinate (gliubpppbj) at (\gliubxxxb, \gliubyyyj);
\coordinate (gliubpppbk) at (\gliubxxxb, \gliubyyyk);
\coordinate (gliubpppbl) at (\gliubxxxb, \gliubyyyl);
\coordinate (gliubpppbm) at (\gliubxxxb, \gliubyyym);
\coordinate (gliubpppbn) at (\gliubxxxb, \gliubyyyn);
\coordinate (gliubpppbo) at (\gliubxxxb, \gliubyyyo);
\coordinate (gliubpppbp) at (\gliubxxxb, \gliubyyyp);
\coordinate (gliubpppbq) at (\gliubxxxb, \gliubyyyq);
\coordinate (gliubpppbr) at (\gliubxxxb, \gliubyyyr);
\coordinate (gliubpppbs) at (\gliubxxxb, \gliubyyys);
\coordinate (gliubpppbt) at (\gliubxxxb, \gliubyyyt);
\coordinate (gliubpppbu) at (\gliubxxxb, \gliubyyyu);
\coordinate (gliubpppbv) at (\gliubxxxb, \gliubyyyv);
\coordinate (gliubpppbw) at (\gliubxxxb, \gliubyyyw);
\coordinate (gliubpppbx) at (\gliubxxxb, \gliubyyyx);
\coordinate (gliubpppby) at (\gliubxxxb, \gliubyyyy);
\coordinate (gliubpppbz) at (\gliubxxxb, \gliubyyyz);
\coordinate (gliubpppca) at (\gliubxxxc, \gliubyyya);
\coordinate (gliubpppcb) at (\gliubxxxc, \gliubyyyb);
\coordinate (gliubpppcc) at (\gliubxxxc, \gliubyyyc);
\coordinate (gliubpppcd) at (\gliubxxxc, \gliubyyyd);
\coordinate (gliubpppce) at (\gliubxxxc, \gliubyyye);
\coordinate (gliubpppcf) at (\gliubxxxc, \gliubyyyf);
\coordinate (gliubpppcg) at (\gliubxxxc, \gliubyyyg);
\coordinate (gliubpppch) at (\gliubxxxc, \gliubyyyh);
\coordinate (gliubpppci) at (\gliubxxxc, \gliubyyyi);
\coordinate (gliubpppcj) at (\gliubxxxc, \gliubyyyj);
\coordinate (gliubpppck) at (\gliubxxxc, \gliubyyyk);
\coordinate (gliubpppcl) at (\gliubxxxc, \gliubyyyl);
\coordinate (gliubpppcm) at (\gliubxxxc, \gliubyyym);
\coordinate (gliubpppcn) at (\gliubxxxc, \gliubyyyn);
\coordinate (gliubpppco) at (\gliubxxxc, \gliubyyyo);
\coordinate (gliubpppcp) at (\gliubxxxc, \gliubyyyp);
\coordinate (gliubpppcq) at (\gliubxxxc, \gliubyyyq);
\coordinate (gliubpppcr) at (\gliubxxxc, \gliubyyyr);
\coordinate (gliubpppcs) at (\gliubxxxc, \gliubyyys);
\coordinate (gliubpppct) at (\gliubxxxc, \gliubyyyt);
\coordinate (gliubpppcu) at (\gliubxxxc, \gliubyyyu);
\coordinate (gliubpppcv) at (\gliubxxxc, \gliubyyyv);
\coordinate (gliubpppcw) at (\gliubxxxc, \gliubyyyw);
\coordinate (gliubpppcx) at (\gliubxxxc, \gliubyyyx);
\coordinate (gliubpppcy) at (\gliubxxxc, \gliubyyyy);
\coordinate (gliubpppcz) at (\gliubxxxc, \gliubyyyz);
\coordinate (gliubpppda) at (\gliubxxxd, \gliubyyya);
\coordinate (gliubpppdb) at (\gliubxxxd, \gliubyyyb);
\coordinate (gliubpppdc) at (\gliubxxxd, \gliubyyyc);
\coordinate (gliubpppdd) at (\gliubxxxd, \gliubyyyd);
\coordinate (gliubpppde) at (\gliubxxxd, \gliubyyye);
\coordinate (gliubpppdf) at (\gliubxxxd, \gliubyyyf);
\coordinate (gliubpppdg) at (\gliubxxxd, \gliubyyyg);
\coordinate (gliubpppdh) at (\gliubxxxd, \gliubyyyh);
\coordinate (gliubpppdi) at (\gliubxxxd, \gliubyyyi);
\coordinate (gliubpppdj) at (\gliubxxxd, \gliubyyyj);
\coordinate (gliubpppdk) at (\gliubxxxd, \gliubyyyk);
\coordinate (gliubpppdl) at (\gliubxxxd, \gliubyyyl);
\coordinate (gliubpppdm) at (\gliubxxxd, \gliubyyym);
\coordinate (gliubpppdn) at (\gliubxxxd, \gliubyyyn);
\coordinate (gliubpppdo) at (\gliubxxxd, \gliubyyyo);
\coordinate (gliubpppdp) at (\gliubxxxd, \gliubyyyp);
\coordinate (gliubpppdq) at (\gliubxxxd, \gliubyyyq);
\coordinate (gliubpppdr) at (\gliubxxxd, \gliubyyyr);
\coordinate (gliubpppds) at (\gliubxxxd, \gliubyyys);
\coordinate (gliubpppdt) at (\gliubxxxd, \gliubyyyt);
\coordinate (gliubpppdu) at (\gliubxxxd, \gliubyyyu);
\coordinate (gliubpppdv) at (\gliubxxxd, \gliubyyyv);
\coordinate (gliubpppdw) at (\gliubxxxd, \gliubyyyw);
\coordinate (gliubpppdx) at (\gliubxxxd, \gliubyyyx);
\coordinate (gliubpppdy) at (\gliubxxxd, \gliubyyyy);
\coordinate (gliubpppdz) at (\gliubxxxd, \gliubyyyz);
\coordinate (gliubpppea) at (\gliubxxxe, \gliubyyya);
\coordinate (gliubpppeb) at (\gliubxxxe, \gliubyyyb);
\coordinate (gliubpppec) at (\gliubxxxe, \gliubyyyc);
\coordinate (gliubppped) at (\gliubxxxe, \gliubyyyd);
\coordinate (gliubpppee) at (\gliubxxxe, \gliubyyye);
\coordinate (gliubpppef) at (\gliubxxxe, \gliubyyyf);
\coordinate (gliubpppeg) at (\gliubxxxe, \gliubyyyg);
\coordinate (gliubpppeh) at (\gliubxxxe, \gliubyyyh);
\coordinate (gliubpppei) at (\gliubxxxe, \gliubyyyi);
\coordinate (gliubpppej) at (\gliubxxxe, \gliubyyyj);
\coordinate (gliubpppek) at (\gliubxxxe, \gliubyyyk);
\coordinate (gliubpppel) at (\gliubxxxe, \gliubyyyl);
\coordinate (gliubpppem) at (\gliubxxxe, \gliubyyym);
\coordinate (gliubpppen) at (\gliubxxxe, \gliubyyyn);
\coordinate (gliubpppeo) at (\gliubxxxe, \gliubyyyo);
\coordinate (gliubpppep) at (\gliubxxxe, \gliubyyyp);
\coordinate (gliubpppeq) at (\gliubxxxe, \gliubyyyq);
\coordinate (gliubppper) at (\gliubxxxe, \gliubyyyr);
\coordinate (gliubpppes) at (\gliubxxxe, \gliubyyys);
\coordinate (gliubpppet) at (\gliubxxxe, \gliubyyyt);
\coordinate (gliubpppeu) at (\gliubxxxe, \gliubyyyu);
\coordinate (gliubpppev) at (\gliubxxxe, \gliubyyyv);
\coordinate (gliubpppew) at (\gliubxxxe, \gliubyyyw);
\coordinate (gliubpppex) at (\gliubxxxe, \gliubyyyx);
\coordinate (gliubpppey) at (\gliubxxxe, \gliubyyyy);
\coordinate (gliubpppez) at (\gliubxxxe, \gliubyyyz);
\coordinate (gliubpppfa) at (\gliubxxxf, \gliubyyya);
\coordinate (gliubpppfb) at (\gliubxxxf, \gliubyyyb);
\coordinate (gliubpppfc) at (\gliubxxxf, \gliubyyyc);
\coordinate (gliubpppfd) at (\gliubxxxf, \gliubyyyd);
\coordinate (gliubpppfe) at (\gliubxxxf, \gliubyyye);
\coordinate (gliubpppff) at (\gliubxxxf, \gliubyyyf);
\coordinate (gliubpppfg) at (\gliubxxxf, \gliubyyyg);
\coordinate (gliubpppfh) at (\gliubxxxf, \gliubyyyh);
\coordinate (gliubpppfi) at (\gliubxxxf, \gliubyyyi);
\coordinate (gliubpppfj) at (\gliubxxxf, \gliubyyyj);
\coordinate (gliubpppfk) at (\gliubxxxf, \gliubyyyk);
\coordinate (gliubpppfl) at (\gliubxxxf, \gliubyyyl);
\coordinate (gliubpppfm) at (\gliubxxxf, \gliubyyym);
\coordinate (gliubpppfn) at (\gliubxxxf, \gliubyyyn);
\coordinate (gliubpppfo) at (\gliubxxxf, \gliubyyyo);
\coordinate (gliubpppfp) at (\gliubxxxf, \gliubyyyp);
\coordinate (gliubpppfq) at (\gliubxxxf, \gliubyyyq);
\coordinate (gliubpppfr) at (\gliubxxxf, \gliubyyyr);
\coordinate (gliubpppfs) at (\gliubxxxf, \gliubyyys);
\coordinate (gliubpppft) at (\gliubxxxf, \gliubyyyt);
\coordinate (gliubpppfu) at (\gliubxxxf, \gliubyyyu);
\coordinate (gliubpppfv) at (\gliubxxxf, \gliubyyyv);
\coordinate (gliubpppfw) at (\gliubxxxf, \gliubyyyw);
\coordinate (gliubpppfx) at (\gliubxxxf, \gliubyyyx);
\coordinate (gliubpppfy) at (\gliubxxxf, \gliubyyyy);
\coordinate (gliubpppfz) at (\gliubxxxf, \gliubyyyz);
\coordinate (gliubpppga) at (\gliubxxxg, \gliubyyya);
\coordinate (gliubpppgb) at (\gliubxxxg, \gliubyyyb);
\coordinate (gliubpppgc) at (\gliubxxxg, \gliubyyyc);
\coordinate (gliubpppgd) at (\gliubxxxg, \gliubyyyd);
\coordinate (gliubpppge) at (\gliubxxxg, \gliubyyye);
\coordinate (gliubpppgf) at (\gliubxxxg, \gliubyyyf);
\coordinate (gliubpppgg) at (\gliubxxxg, \gliubyyyg);
\coordinate (gliubpppgh) at (\gliubxxxg, \gliubyyyh);
\coordinate (gliubpppgi) at (\gliubxxxg, \gliubyyyi);
\coordinate (gliubpppgj) at (\gliubxxxg, \gliubyyyj);
\coordinate (gliubpppgk) at (\gliubxxxg, \gliubyyyk);
\coordinate (gliubpppgl) at (\gliubxxxg, \gliubyyyl);
\coordinate (gliubpppgm) at (\gliubxxxg, \gliubyyym);
\coordinate (gliubpppgn) at (\gliubxxxg, \gliubyyyn);
\coordinate (gliubpppgo) at (\gliubxxxg, \gliubyyyo);
\coordinate (gliubpppgp) at (\gliubxxxg, \gliubyyyp);
\coordinate (gliubpppgq) at (\gliubxxxg, \gliubyyyq);
\coordinate (gliubpppgr) at (\gliubxxxg, \gliubyyyr);
\coordinate (gliubpppgs) at (\gliubxxxg, \gliubyyys);
\coordinate (gliubpppgt) at (\gliubxxxg, \gliubyyyt);
\coordinate (gliubpppgu) at (\gliubxxxg, \gliubyyyu);
\coordinate (gliubpppgv) at (\gliubxxxg, \gliubyyyv);
\coordinate (gliubpppgw) at (\gliubxxxg, \gliubyyyw);
\coordinate (gliubpppgx) at (\gliubxxxg, \gliubyyyx);
\coordinate (gliubpppgy) at (\gliubxxxg, \gliubyyyy);
\coordinate (gliubpppgz) at (\gliubxxxg, \gliubyyyz);
\coordinate (gliubpppha) at (\gliubxxxh, \gliubyyya);
\coordinate (gliubppphb) at (\gliubxxxh, \gliubyyyb);
\coordinate (gliubppphc) at (\gliubxxxh, \gliubyyyc);
\coordinate (gliubppphd) at (\gliubxxxh, \gliubyyyd);
\coordinate (gliubppphe) at (\gliubxxxh, \gliubyyye);
\coordinate (gliubppphf) at (\gliubxxxh, \gliubyyyf);
\coordinate (gliubppphg) at (\gliubxxxh, \gliubyyyg);
\coordinate (gliubppphh) at (\gliubxxxh, \gliubyyyh);
\coordinate (gliubppphi) at (\gliubxxxh, \gliubyyyi);
\coordinate (gliubppphj) at (\gliubxxxh, \gliubyyyj);
\coordinate (gliubppphk) at (\gliubxxxh, \gliubyyyk);
\coordinate (gliubppphl) at (\gliubxxxh, \gliubyyyl);
\coordinate (gliubppphm) at (\gliubxxxh, \gliubyyym);
\coordinate (gliubppphn) at (\gliubxxxh, \gliubyyyn);
\coordinate (gliubpppho) at (\gliubxxxh, \gliubyyyo);
\coordinate (gliubppphp) at (\gliubxxxh, \gliubyyyp);
\coordinate (gliubppphq) at (\gliubxxxh, \gliubyyyq);
\coordinate (gliubppphr) at (\gliubxxxh, \gliubyyyr);
\coordinate (gliubppphs) at (\gliubxxxh, \gliubyyys);
\coordinate (gliubpppht) at (\gliubxxxh, \gliubyyyt);
\coordinate (gliubppphu) at (\gliubxxxh, \gliubyyyu);
\coordinate (gliubppphv) at (\gliubxxxh, \gliubyyyv);
\coordinate (gliubppphw) at (\gliubxxxh, \gliubyyyw);
\coordinate (gliubppphx) at (\gliubxxxh, \gliubyyyx);
\coordinate (gliubppphy) at (\gliubxxxh, \gliubyyyy);
\coordinate (gliubppphz) at (\gliubxxxh, \gliubyyyz);
\coordinate (gliubpppia) at (\gliubxxxi, \gliubyyya);
\coordinate (gliubpppib) at (\gliubxxxi, \gliubyyyb);
\coordinate (gliubpppic) at (\gliubxxxi, \gliubyyyc);
\coordinate (gliubpppid) at (\gliubxxxi, \gliubyyyd);
\coordinate (gliubpppie) at (\gliubxxxi, \gliubyyye);
\coordinate (gliubpppif) at (\gliubxxxi, \gliubyyyf);
\coordinate (gliubpppig) at (\gliubxxxi, \gliubyyyg);
\coordinate (gliubpppih) at (\gliubxxxi, \gliubyyyh);
\coordinate (gliubpppii) at (\gliubxxxi, \gliubyyyi);
\coordinate (gliubpppij) at (\gliubxxxi, \gliubyyyj);
\coordinate (gliubpppik) at (\gliubxxxi, \gliubyyyk);
\coordinate (gliubpppil) at (\gliubxxxi, \gliubyyyl);
\coordinate (gliubpppim) at (\gliubxxxi, \gliubyyym);
\coordinate (gliubpppin) at (\gliubxxxi, \gliubyyyn);
\coordinate (gliubpppio) at (\gliubxxxi, \gliubyyyo);
\coordinate (gliubpppip) at (\gliubxxxi, \gliubyyyp);
\coordinate (gliubpppiq) at (\gliubxxxi, \gliubyyyq);
\coordinate (gliubpppir) at (\gliubxxxi, \gliubyyyr);
\coordinate (gliubpppis) at (\gliubxxxi, \gliubyyys);
\coordinate (gliubpppit) at (\gliubxxxi, \gliubyyyt);
\coordinate (gliubpppiu) at (\gliubxxxi, \gliubyyyu);
\coordinate (gliubpppiv) at (\gliubxxxi, \gliubyyyv);
\coordinate (gliubpppiw) at (\gliubxxxi, \gliubyyyw);
\coordinate (gliubpppix) at (\gliubxxxi, \gliubyyyx);
\coordinate (gliubpppiy) at (\gliubxxxi, \gliubyyyy);
\coordinate (gliubpppiz) at (\gliubxxxi, \gliubyyyz);
\coordinate (gliubpppja) at (\gliubxxxj, \gliubyyya);
\coordinate (gliubpppjb) at (\gliubxxxj, \gliubyyyb);
\coordinate (gliubpppjc) at (\gliubxxxj, \gliubyyyc);
\coordinate (gliubpppjd) at (\gliubxxxj, \gliubyyyd);
\coordinate (gliubpppje) at (\gliubxxxj, \gliubyyye);
\coordinate (gliubpppjf) at (\gliubxxxj, \gliubyyyf);
\coordinate (gliubpppjg) at (\gliubxxxj, \gliubyyyg);
\coordinate (gliubpppjh) at (\gliubxxxj, \gliubyyyh);
\coordinate (gliubpppji) at (\gliubxxxj, \gliubyyyi);
\coordinate (gliubpppjj) at (\gliubxxxj, \gliubyyyj);
\coordinate (gliubpppjk) at (\gliubxxxj, \gliubyyyk);
\coordinate (gliubpppjl) at (\gliubxxxj, \gliubyyyl);
\coordinate (gliubpppjm) at (\gliubxxxj, \gliubyyym);
\coordinate (gliubpppjn) at (\gliubxxxj, \gliubyyyn);
\coordinate (gliubpppjo) at (\gliubxxxj, \gliubyyyo);
\coordinate (gliubpppjp) at (\gliubxxxj, \gliubyyyp);
\coordinate (gliubpppjq) at (\gliubxxxj, \gliubyyyq);
\coordinate (gliubpppjr) at (\gliubxxxj, \gliubyyyr);
\coordinate (gliubpppjs) at (\gliubxxxj, \gliubyyys);
\coordinate (gliubpppjt) at (\gliubxxxj, \gliubyyyt);
\coordinate (gliubpppju) at (\gliubxxxj, \gliubyyyu);
\coordinate (gliubpppjv) at (\gliubxxxj, \gliubyyyv);
\coordinate (gliubpppjw) at (\gliubxxxj, \gliubyyyw);
\coordinate (gliubpppjx) at (\gliubxxxj, \gliubyyyx);
\coordinate (gliubpppjy) at (\gliubxxxj, \gliubyyyy);
\coordinate (gliubpppjz) at (\gliubxxxj, \gliubyyyz);
\coordinate (gliubpppka) at (\gliubxxxk, \gliubyyya);
\coordinate (gliubpppkb) at (\gliubxxxk, \gliubyyyb);
\coordinate (gliubpppkc) at (\gliubxxxk, \gliubyyyc);
\coordinate (gliubpppkd) at (\gliubxxxk, \gliubyyyd);
\coordinate (gliubpppke) at (\gliubxxxk, \gliubyyye);
\coordinate (gliubpppkf) at (\gliubxxxk, \gliubyyyf);
\coordinate (gliubpppkg) at (\gliubxxxk, \gliubyyyg);
\coordinate (gliubpppkh) at (\gliubxxxk, \gliubyyyh);
\coordinate (gliubpppki) at (\gliubxxxk, \gliubyyyi);
\coordinate (gliubpppkj) at (\gliubxxxk, \gliubyyyj);
\coordinate (gliubpppkk) at (\gliubxxxk, \gliubyyyk);
\coordinate (gliubpppkl) at (\gliubxxxk, \gliubyyyl);
\coordinate (gliubpppkm) at (\gliubxxxk, \gliubyyym);
\coordinate (gliubpppkn) at (\gliubxxxk, \gliubyyyn);
\coordinate (gliubpppko) at (\gliubxxxk, \gliubyyyo);
\coordinate (gliubpppkp) at (\gliubxxxk, \gliubyyyp);
\coordinate (gliubpppkq) at (\gliubxxxk, \gliubyyyq);
\coordinate (gliubpppkr) at (\gliubxxxk, \gliubyyyr);
\coordinate (gliubpppks) at (\gliubxxxk, \gliubyyys);
\coordinate (gliubpppkt) at (\gliubxxxk, \gliubyyyt);
\coordinate (gliubpppku) at (\gliubxxxk, \gliubyyyu);
\coordinate (gliubpppkv) at (\gliubxxxk, \gliubyyyv);
\coordinate (gliubpppkw) at (\gliubxxxk, \gliubyyyw);
\coordinate (gliubpppkx) at (\gliubxxxk, \gliubyyyx);
\coordinate (gliubpppky) at (\gliubxxxk, \gliubyyyy);
\coordinate (gliubpppkz) at (\gliubxxxk, \gliubyyyz);
\coordinate (gliubpppla) at (\gliubxxxl, \gliubyyya);
\coordinate (gliubppplb) at (\gliubxxxl, \gliubyyyb);
\coordinate (gliubppplc) at (\gliubxxxl, \gliubyyyc);
\coordinate (gliubpppld) at (\gliubxxxl, \gliubyyyd);
\coordinate (gliubppple) at (\gliubxxxl, \gliubyyye);
\coordinate (gliubppplf) at (\gliubxxxl, \gliubyyyf);
\coordinate (gliubppplg) at (\gliubxxxl, \gliubyyyg);
\coordinate (gliubppplh) at (\gliubxxxl, \gliubyyyh);
\coordinate (gliubpppli) at (\gliubxxxl, \gliubyyyi);
\coordinate (gliubppplj) at (\gliubxxxl, \gliubyyyj);
\coordinate (gliubppplk) at (\gliubxxxl, \gliubyyyk);
\coordinate (gliubpppll) at (\gliubxxxl, \gliubyyyl);
\coordinate (gliubppplm) at (\gliubxxxl, \gliubyyym);
\coordinate (gliubpppln) at (\gliubxxxl, \gliubyyyn);
\coordinate (gliubppplo) at (\gliubxxxl, \gliubyyyo);
\coordinate (gliubppplp) at (\gliubxxxl, \gliubyyyp);
\coordinate (gliubppplq) at (\gliubxxxl, \gliubyyyq);
\coordinate (gliubppplr) at (\gliubxxxl, \gliubyyyr);
\coordinate (gliubpppls) at (\gliubxxxl, \gliubyyys);
\coordinate (gliubppplt) at (\gliubxxxl, \gliubyyyt);
\coordinate (gliubppplu) at (\gliubxxxl, \gliubyyyu);
\coordinate (gliubppplv) at (\gliubxxxl, \gliubyyyv);
\coordinate (gliubppplw) at (\gliubxxxl, \gliubyyyw);
\coordinate (gliubppplx) at (\gliubxxxl, \gliubyyyx);
\coordinate (gliubppply) at (\gliubxxxl, \gliubyyyy);
\coordinate (gliubppplz) at (\gliubxxxl, \gliubyyyz);
\coordinate (gliubpppma) at (\gliubxxxm, \gliubyyya);
\coordinate (gliubpppmb) at (\gliubxxxm, \gliubyyyb);
\coordinate (gliubpppmc) at (\gliubxxxm, \gliubyyyc);
\coordinate (gliubpppmd) at (\gliubxxxm, \gliubyyyd);
\coordinate (gliubpppme) at (\gliubxxxm, \gliubyyye);
\coordinate (gliubpppmf) at (\gliubxxxm, \gliubyyyf);
\coordinate (gliubpppmg) at (\gliubxxxm, \gliubyyyg);
\coordinate (gliubpppmh) at (\gliubxxxm, \gliubyyyh);
\coordinate (gliubpppmi) at (\gliubxxxm, \gliubyyyi);
\coordinate (gliubpppmj) at (\gliubxxxm, \gliubyyyj);
\coordinate (gliubpppmk) at (\gliubxxxm, \gliubyyyk);
\coordinate (gliubpppml) at (\gliubxxxm, \gliubyyyl);
\coordinate (gliubpppmm) at (\gliubxxxm, \gliubyyym);
\coordinate (gliubpppmn) at (\gliubxxxm, \gliubyyyn);
\coordinate (gliubpppmo) at (\gliubxxxm, \gliubyyyo);
\coordinate (gliubpppmp) at (\gliubxxxm, \gliubyyyp);
\coordinate (gliubpppmq) at (\gliubxxxm, \gliubyyyq);
\coordinate (gliubpppmr) at (\gliubxxxm, \gliubyyyr);
\coordinate (gliubpppms) at (\gliubxxxm, \gliubyyys);
\coordinate (gliubpppmt) at (\gliubxxxm, \gliubyyyt);
\coordinate (gliubpppmu) at (\gliubxxxm, \gliubyyyu);
\coordinate (gliubpppmv) at (\gliubxxxm, \gliubyyyv);
\coordinate (gliubpppmw) at (\gliubxxxm, \gliubyyyw);
\coordinate (gliubpppmx) at (\gliubxxxm, \gliubyyyx);
\coordinate (gliubpppmy) at (\gliubxxxm, \gliubyyyy);
\coordinate (gliubpppmz) at (\gliubxxxm, \gliubyyyz);
\coordinate (gliubpppna) at (\gliubxxxn, \gliubyyya);
\coordinate (gliubpppnb) at (\gliubxxxn, \gliubyyyb);
\coordinate (gliubpppnc) at (\gliubxxxn, \gliubyyyc);
\coordinate (gliubpppnd) at (\gliubxxxn, \gliubyyyd);
\coordinate (gliubpppne) at (\gliubxxxn, \gliubyyye);
\coordinate (gliubpppnf) at (\gliubxxxn, \gliubyyyf);
\coordinate (gliubpppng) at (\gliubxxxn, \gliubyyyg);
\coordinate (gliubpppnh) at (\gliubxxxn, \gliubyyyh);
\coordinate (gliubpppni) at (\gliubxxxn, \gliubyyyi);
\coordinate (gliubpppnj) at (\gliubxxxn, \gliubyyyj);
\coordinate (gliubpppnk) at (\gliubxxxn, \gliubyyyk);
\coordinate (gliubpppnl) at (\gliubxxxn, \gliubyyyl);
\coordinate (gliubpppnm) at (\gliubxxxn, \gliubyyym);
\coordinate (gliubpppnn) at (\gliubxxxn, \gliubyyyn);
\coordinate (gliubpppno) at (\gliubxxxn, \gliubyyyo);
\coordinate (gliubpppnp) at (\gliubxxxn, \gliubyyyp);
\coordinate (gliubpppnq) at (\gliubxxxn, \gliubyyyq);
\coordinate (gliubpppnr) at (\gliubxxxn, \gliubyyyr);
\coordinate (gliubpppns) at (\gliubxxxn, \gliubyyys);
\coordinate (gliubpppnt) at (\gliubxxxn, \gliubyyyt);
\coordinate (gliubpppnu) at (\gliubxxxn, \gliubyyyu);
\coordinate (gliubpppnv) at (\gliubxxxn, \gliubyyyv);
\coordinate (gliubpppnw) at (\gliubxxxn, \gliubyyyw);
\coordinate (gliubpppnx) at (\gliubxxxn, \gliubyyyx);
\coordinate (gliubpppny) at (\gliubxxxn, \gliubyyyy);
\coordinate (gliubpppnz) at (\gliubxxxn, \gliubyyyz);
\coordinate (gliubpppoa) at (\gliubxxxo, \gliubyyya);
\coordinate (gliubpppob) at (\gliubxxxo, \gliubyyyb);
\coordinate (gliubpppoc) at (\gliubxxxo, \gliubyyyc);
\coordinate (gliubpppod) at (\gliubxxxo, \gliubyyyd);
\coordinate (gliubpppoe) at (\gliubxxxo, \gliubyyye);
\coordinate (gliubpppof) at (\gliubxxxo, \gliubyyyf);
\coordinate (gliubpppog) at (\gliubxxxo, \gliubyyyg);
\coordinate (gliubpppoh) at (\gliubxxxo, \gliubyyyh);
\coordinate (gliubpppoi) at (\gliubxxxo, \gliubyyyi);
\coordinate (gliubpppoj) at (\gliubxxxo, \gliubyyyj);
\coordinate (gliubpppok) at (\gliubxxxo, \gliubyyyk);
\coordinate (gliubpppol) at (\gliubxxxo, \gliubyyyl);
\coordinate (gliubpppom) at (\gliubxxxo, \gliubyyym);
\coordinate (gliubpppon) at (\gliubxxxo, \gliubyyyn);
\coordinate (gliubpppoo) at (\gliubxxxo, \gliubyyyo);
\coordinate (gliubpppop) at (\gliubxxxo, \gliubyyyp);
\coordinate (gliubpppoq) at (\gliubxxxo, \gliubyyyq);
\coordinate (gliubpppor) at (\gliubxxxo, \gliubyyyr);
\coordinate (gliubpppos) at (\gliubxxxo, \gliubyyys);
\coordinate (gliubpppot) at (\gliubxxxo, \gliubyyyt);
\coordinate (gliubpppou) at (\gliubxxxo, \gliubyyyu);
\coordinate (gliubpppov) at (\gliubxxxo, \gliubyyyv);
\coordinate (gliubpppow) at (\gliubxxxo, \gliubyyyw);
\coordinate (gliubpppox) at (\gliubxxxo, \gliubyyyx);
\coordinate (gliubpppoy) at (\gliubxxxo, \gliubyyyy);
\coordinate (gliubpppoz) at (\gliubxxxo, \gliubyyyz);
\coordinate (gliubppppa) at (\gliubxxxp, \gliubyyya);
\coordinate (gliubppppb) at (\gliubxxxp, \gliubyyyb);
\coordinate (gliubppppc) at (\gliubxxxp, \gliubyyyc);
\coordinate (gliubppppd) at (\gliubxxxp, \gliubyyyd);
\coordinate (gliubppppe) at (\gliubxxxp, \gliubyyye);
\coordinate (gliubppppf) at (\gliubxxxp, \gliubyyyf);
\coordinate (gliubppppg) at (\gliubxxxp, \gliubyyyg);
\coordinate (gliubpppph) at (\gliubxxxp, \gliubyyyh);
\coordinate (gliubppppi) at (\gliubxxxp, \gliubyyyi);
\coordinate (gliubppppj) at (\gliubxxxp, \gliubyyyj);
\coordinate (gliubppppk) at (\gliubxxxp, \gliubyyyk);
\coordinate (gliubppppl) at (\gliubxxxp, \gliubyyyl);
\coordinate (gliubppppm) at (\gliubxxxp, \gliubyyym);
\coordinate (gliubppppn) at (\gliubxxxp, \gliubyyyn);
\coordinate (gliubppppo) at (\gliubxxxp, \gliubyyyo);
\coordinate (gliubppppp) at (\gliubxxxp, \gliubyyyp);
\coordinate (gliubppppq) at (\gliubxxxp, \gliubyyyq);
\coordinate (gliubppppr) at (\gliubxxxp, \gliubyyyr);
\coordinate (gliubpppps) at (\gliubxxxp, \gliubyyys);
\coordinate (gliubppppt) at (\gliubxxxp, \gliubyyyt);
\coordinate (gliubppppu) at (\gliubxxxp, \gliubyyyu);
\coordinate (gliubppppv) at (\gliubxxxp, \gliubyyyv);
\coordinate (gliubppppw) at (\gliubxxxp, \gliubyyyw);
\coordinate (gliubppppx) at (\gliubxxxp, \gliubyyyx);
\coordinate (gliubppppy) at (\gliubxxxp, \gliubyyyy);
\coordinate (gliubppppz) at (\gliubxxxp, \gliubyyyz);
\coordinate (gliubpppqa) at (\gliubxxxq, \gliubyyya);
\coordinate (gliubpppqb) at (\gliubxxxq, \gliubyyyb);
\coordinate (gliubpppqc) at (\gliubxxxq, \gliubyyyc);
\coordinate (gliubpppqd) at (\gliubxxxq, \gliubyyyd);
\coordinate (gliubpppqe) at (\gliubxxxq, \gliubyyye);
\coordinate (gliubpppqf) at (\gliubxxxq, \gliubyyyf);
\coordinate (gliubpppqg) at (\gliubxxxq, \gliubyyyg);
\coordinate (gliubpppqh) at (\gliubxxxq, \gliubyyyh);
\coordinate (gliubpppqi) at (\gliubxxxq, \gliubyyyi);
\coordinate (gliubpppqj) at (\gliubxxxq, \gliubyyyj);
\coordinate (gliubpppqk) at (\gliubxxxq, \gliubyyyk);
\coordinate (gliubpppql) at (\gliubxxxq, \gliubyyyl);
\coordinate (gliubpppqm) at (\gliubxxxq, \gliubyyym);
\coordinate (gliubpppqn) at (\gliubxxxq, \gliubyyyn);
\coordinate (gliubpppqo) at (\gliubxxxq, \gliubyyyo);
\coordinate (gliubpppqp) at (\gliubxxxq, \gliubyyyp);
\coordinate (gliubpppqq) at (\gliubxxxq, \gliubyyyq);
\coordinate (gliubpppqr) at (\gliubxxxq, \gliubyyyr);
\coordinate (gliubpppqs) at (\gliubxxxq, \gliubyyys);
\coordinate (gliubpppqt) at (\gliubxxxq, \gliubyyyt);
\coordinate (gliubpppqu) at (\gliubxxxq, \gliubyyyu);
\coordinate (gliubpppqv) at (\gliubxxxq, \gliubyyyv);
\coordinate (gliubpppqw) at (\gliubxxxq, \gliubyyyw);
\coordinate (gliubpppqx) at (\gliubxxxq, \gliubyyyx);
\coordinate (gliubpppqy) at (\gliubxxxq, \gliubyyyy);
\coordinate (gliubpppqz) at (\gliubxxxq, \gliubyyyz);
\coordinate (gliubpppra) at (\gliubxxxr, \gliubyyya);
\coordinate (gliubppprb) at (\gliubxxxr, \gliubyyyb);
\coordinate (gliubppprc) at (\gliubxxxr, \gliubyyyc);
\coordinate (gliubppprd) at (\gliubxxxr, \gliubyyyd);
\coordinate (gliubpppre) at (\gliubxxxr, \gliubyyye);
\coordinate (gliubppprf) at (\gliubxxxr, \gliubyyyf);
\coordinate (gliubppprg) at (\gliubxxxr, \gliubyyyg);
\coordinate (gliubppprh) at (\gliubxxxr, \gliubyyyh);
\coordinate (gliubpppri) at (\gliubxxxr, \gliubyyyi);
\coordinate (gliubppprj) at (\gliubxxxr, \gliubyyyj);
\coordinate (gliubppprk) at (\gliubxxxr, \gliubyyyk);
\coordinate (gliubppprl) at (\gliubxxxr, \gliubyyyl);
\coordinate (gliubppprm) at (\gliubxxxr, \gliubyyym);
\coordinate (gliubppprn) at (\gliubxxxr, \gliubyyyn);
\coordinate (gliubpppro) at (\gliubxxxr, \gliubyyyo);
\coordinate (gliubppprp) at (\gliubxxxr, \gliubyyyp);
\coordinate (gliubppprq) at (\gliubxxxr, \gliubyyyq);
\coordinate (gliubppprr) at (\gliubxxxr, \gliubyyyr);
\coordinate (gliubppprs) at (\gliubxxxr, \gliubyyys);
\coordinate (gliubppprt) at (\gliubxxxr, \gliubyyyt);
\coordinate (gliubpppru) at (\gliubxxxr, \gliubyyyu);
\coordinate (gliubppprv) at (\gliubxxxr, \gliubyyyv);
\coordinate (gliubppprw) at (\gliubxxxr, \gliubyyyw);
\coordinate (gliubppprx) at (\gliubxxxr, \gliubyyyx);
\coordinate (gliubpppry) at (\gliubxxxr, \gliubyyyy);
\coordinate (gliubppprz) at (\gliubxxxr, \gliubyyyz);
\coordinate (gliubpppsa) at (\gliubxxxs, \gliubyyya);
\coordinate (gliubpppsb) at (\gliubxxxs, \gliubyyyb);
\coordinate (gliubpppsc) at (\gliubxxxs, \gliubyyyc);
\coordinate (gliubpppsd) at (\gliubxxxs, \gliubyyyd);
\coordinate (gliubpppse) at (\gliubxxxs, \gliubyyye);
\coordinate (gliubpppsf) at (\gliubxxxs, \gliubyyyf);
\coordinate (gliubpppsg) at (\gliubxxxs, \gliubyyyg);
\coordinate (gliubpppsh) at (\gliubxxxs, \gliubyyyh);
\coordinate (gliubpppsi) at (\gliubxxxs, \gliubyyyi);
\coordinate (gliubpppsj) at (\gliubxxxs, \gliubyyyj);
\coordinate (gliubpppsk) at (\gliubxxxs, \gliubyyyk);
\coordinate (gliubpppsl) at (\gliubxxxs, \gliubyyyl);
\coordinate (gliubpppsm) at (\gliubxxxs, \gliubyyym);
\coordinate (gliubpppsn) at (\gliubxxxs, \gliubyyyn);
\coordinate (gliubpppso) at (\gliubxxxs, \gliubyyyo);
\coordinate (gliubpppsp) at (\gliubxxxs, \gliubyyyp);
\coordinate (gliubpppsq) at (\gliubxxxs, \gliubyyyq);
\coordinate (gliubpppsr) at (\gliubxxxs, \gliubyyyr);
\coordinate (gliubpppss) at (\gliubxxxs, \gliubyyys);
\coordinate (gliubpppst) at (\gliubxxxs, \gliubyyyt);
\coordinate (gliubpppsu) at (\gliubxxxs, \gliubyyyu);
\coordinate (gliubpppsv) at (\gliubxxxs, \gliubyyyv);
\coordinate (gliubpppsw) at (\gliubxxxs, \gliubyyyw);
\coordinate (gliubpppsx) at (\gliubxxxs, \gliubyyyx);
\coordinate (gliubpppsy) at (\gliubxxxs, \gliubyyyy);
\coordinate (gliubpppsz) at (\gliubxxxs, \gliubyyyz);
\coordinate (gliubpppta) at (\gliubxxxt, \gliubyyya);
\coordinate (gliubppptb) at (\gliubxxxt, \gliubyyyb);
\coordinate (gliubppptc) at (\gliubxxxt, \gliubyyyc);
\coordinate (gliubppptd) at (\gliubxxxt, \gliubyyyd);
\coordinate (gliubpppte) at (\gliubxxxt, \gliubyyye);
\coordinate (gliubppptf) at (\gliubxxxt, \gliubyyyf);
\coordinate (gliubppptg) at (\gliubxxxt, \gliubyyyg);
\coordinate (gliubpppth) at (\gliubxxxt, \gliubyyyh);
\coordinate (gliubpppti) at (\gliubxxxt, \gliubyyyi);
\coordinate (gliubppptj) at (\gliubxxxt, \gliubyyyj);
\coordinate (gliubppptk) at (\gliubxxxt, \gliubyyyk);
\coordinate (gliubppptl) at (\gliubxxxt, \gliubyyyl);
\coordinate (gliubppptm) at (\gliubxxxt, \gliubyyym);
\coordinate (gliubppptn) at (\gliubxxxt, \gliubyyyn);
\coordinate (gliubpppto) at (\gliubxxxt, \gliubyyyo);
\coordinate (gliubppptp) at (\gliubxxxt, \gliubyyyp);
\coordinate (gliubppptq) at (\gliubxxxt, \gliubyyyq);
\coordinate (gliubppptr) at (\gliubxxxt, \gliubyyyr);
\coordinate (gliubpppts) at (\gliubxxxt, \gliubyyys);
\coordinate (gliubppptt) at (\gliubxxxt, \gliubyyyt);
\coordinate (gliubppptu) at (\gliubxxxt, \gliubyyyu);
\coordinate (gliubppptv) at (\gliubxxxt, \gliubyyyv);
\coordinate (gliubppptw) at (\gliubxxxt, \gliubyyyw);
\coordinate (gliubppptx) at (\gliubxxxt, \gliubyyyx);
\coordinate (gliubpppty) at (\gliubxxxt, \gliubyyyy);
\coordinate (gliubppptz) at (\gliubxxxt, \gliubyyyz);
\coordinate (gliubpppua) at (\gliubxxxu, \gliubyyya);
\coordinate (gliubpppub) at (\gliubxxxu, \gliubyyyb);
\coordinate (gliubpppuc) at (\gliubxxxu, \gliubyyyc);
\coordinate (gliubpppud) at (\gliubxxxu, \gliubyyyd);
\coordinate (gliubpppue) at (\gliubxxxu, \gliubyyye);
\coordinate (gliubpppuf) at (\gliubxxxu, \gliubyyyf);
\coordinate (gliubpppug) at (\gliubxxxu, \gliubyyyg);
\coordinate (gliubpppuh) at (\gliubxxxu, \gliubyyyh);
\coordinate (gliubpppui) at (\gliubxxxu, \gliubyyyi);
\coordinate (gliubpppuj) at (\gliubxxxu, \gliubyyyj);
\coordinate (gliubpppuk) at (\gliubxxxu, \gliubyyyk);
\coordinate (gliubpppul) at (\gliubxxxu, \gliubyyyl);
\coordinate (gliubpppum) at (\gliubxxxu, \gliubyyym);
\coordinate (gliubpppun) at (\gliubxxxu, \gliubyyyn);
\coordinate (gliubpppuo) at (\gliubxxxu, \gliubyyyo);
\coordinate (gliubpppup) at (\gliubxxxu, \gliubyyyp);
\coordinate (gliubpppuq) at (\gliubxxxu, \gliubyyyq);
\coordinate (gliubpppur) at (\gliubxxxu, \gliubyyyr);
\coordinate (gliubpppus) at (\gliubxxxu, \gliubyyys);
\coordinate (gliubppput) at (\gliubxxxu, \gliubyyyt);
\coordinate (gliubpppuu) at (\gliubxxxu, \gliubyyyu);
\coordinate (gliubpppuv) at (\gliubxxxu, \gliubyyyv);
\coordinate (gliubpppuw) at (\gliubxxxu, \gliubyyyw);
\coordinate (gliubpppux) at (\gliubxxxu, \gliubyyyx);
\coordinate (gliubpppuy) at (\gliubxxxu, \gliubyyyy);
\coordinate (gliubpppuz) at (\gliubxxxu, \gliubyyyz);
\coordinate (gliubpppva) at (\gliubxxxv, \gliubyyya);
\coordinate (gliubpppvb) at (\gliubxxxv, \gliubyyyb);
\coordinate (gliubpppvc) at (\gliubxxxv, \gliubyyyc);
\coordinate (gliubpppvd) at (\gliubxxxv, \gliubyyyd);
\coordinate (gliubpppve) at (\gliubxxxv, \gliubyyye);
\coordinate (gliubpppvf) at (\gliubxxxv, \gliubyyyf);
\coordinate (gliubpppvg) at (\gliubxxxv, \gliubyyyg);
\coordinate (gliubpppvh) at (\gliubxxxv, \gliubyyyh);
\coordinate (gliubpppvi) at (\gliubxxxv, \gliubyyyi);
\coordinate (gliubpppvj) at (\gliubxxxv, \gliubyyyj);
\coordinate (gliubpppvk) at (\gliubxxxv, \gliubyyyk);
\coordinate (gliubpppvl) at (\gliubxxxv, \gliubyyyl);
\coordinate (gliubpppvm) at (\gliubxxxv, \gliubyyym);
\coordinate (gliubpppvn) at (\gliubxxxv, \gliubyyyn);
\coordinate (gliubpppvo) at (\gliubxxxv, \gliubyyyo);
\coordinate (gliubpppvp) at (\gliubxxxv, \gliubyyyp);
\coordinate (gliubpppvq) at (\gliubxxxv, \gliubyyyq);
\coordinate (gliubpppvr) at (\gliubxxxv, \gliubyyyr);
\coordinate (gliubpppvs) at (\gliubxxxv, \gliubyyys);
\coordinate (gliubpppvt) at (\gliubxxxv, \gliubyyyt);
\coordinate (gliubpppvu) at (\gliubxxxv, \gliubyyyu);
\coordinate (gliubpppvv) at (\gliubxxxv, \gliubyyyv);
\coordinate (gliubpppvw) at (\gliubxxxv, \gliubyyyw);
\coordinate (gliubpppvx) at (\gliubxxxv, \gliubyyyx);
\coordinate (gliubpppvy) at (\gliubxxxv, \gliubyyyy);
\coordinate (gliubpppvz) at (\gliubxxxv, \gliubyyyz);
\coordinate (gliubpppwa) at (\gliubxxxw, \gliubyyya);
\coordinate (gliubpppwb) at (\gliubxxxw, \gliubyyyb);
\coordinate (gliubpppwc) at (\gliubxxxw, \gliubyyyc);
\coordinate (gliubpppwd) at (\gliubxxxw, \gliubyyyd);
\coordinate (gliubpppwe) at (\gliubxxxw, \gliubyyye);
\coordinate (gliubpppwf) at (\gliubxxxw, \gliubyyyf);
\coordinate (gliubpppwg) at (\gliubxxxw, \gliubyyyg);
\coordinate (gliubpppwh) at (\gliubxxxw, \gliubyyyh);
\coordinate (gliubpppwi) at (\gliubxxxw, \gliubyyyi);
\coordinate (gliubpppwj) at (\gliubxxxw, \gliubyyyj);
\coordinate (gliubpppwk) at (\gliubxxxw, \gliubyyyk);
\coordinate (gliubpppwl) at (\gliubxxxw, \gliubyyyl);
\coordinate (gliubpppwm) at (\gliubxxxw, \gliubyyym);
\coordinate (gliubpppwn) at (\gliubxxxw, \gliubyyyn);
\coordinate (gliubpppwo) at (\gliubxxxw, \gliubyyyo);
\coordinate (gliubpppwp) at (\gliubxxxw, \gliubyyyp);
\coordinate (gliubpppwq) at (\gliubxxxw, \gliubyyyq);
\coordinate (gliubpppwr) at (\gliubxxxw, \gliubyyyr);
\coordinate (gliubpppws) at (\gliubxxxw, \gliubyyys);
\coordinate (gliubpppwt) at (\gliubxxxw, \gliubyyyt);
\coordinate (gliubpppwu) at (\gliubxxxw, \gliubyyyu);
\coordinate (gliubpppwv) at (\gliubxxxw, \gliubyyyv);
\coordinate (gliubpppww) at (\gliubxxxw, \gliubyyyw);
\coordinate (gliubpppwx) at (\gliubxxxw, \gliubyyyx);
\coordinate (gliubpppwy) at (\gliubxxxw, \gliubyyyy);
\coordinate (gliubpppwz) at (\gliubxxxw, \gliubyyyz);
\coordinate (gliubpppxa) at (\gliubxxxx, \gliubyyya);
\coordinate (gliubpppxb) at (\gliubxxxx, \gliubyyyb);
\coordinate (gliubpppxc) at (\gliubxxxx, \gliubyyyc);
\coordinate (gliubpppxd) at (\gliubxxxx, \gliubyyyd);
\coordinate (gliubpppxe) at (\gliubxxxx, \gliubyyye);
\coordinate (gliubpppxf) at (\gliubxxxx, \gliubyyyf);
\coordinate (gliubpppxg) at (\gliubxxxx, \gliubyyyg);
\coordinate (gliubpppxh) at (\gliubxxxx, \gliubyyyh);
\coordinate (gliubpppxi) at (\gliubxxxx, \gliubyyyi);
\coordinate (gliubpppxj) at (\gliubxxxx, \gliubyyyj);
\coordinate (gliubpppxk) at (\gliubxxxx, \gliubyyyk);
\coordinate (gliubpppxl) at (\gliubxxxx, \gliubyyyl);
\coordinate (gliubpppxm) at (\gliubxxxx, \gliubyyym);
\coordinate (gliubpppxn) at (\gliubxxxx, \gliubyyyn);
\coordinate (gliubpppxo) at (\gliubxxxx, \gliubyyyo);
\coordinate (gliubpppxp) at (\gliubxxxx, \gliubyyyp);
\coordinate (gliubpppxq) at (\gliubxxxx, \gliubyyyq);
\coordinate (gliubpppxr) at (\gliubxxxx, \gliubyyyr);
\coordinate (gliubpppxs) at (\gliubxxxx, \gliubyyys);
\coordinate (gliubpppxt) at (\gliubxxxx, \gliubyyyt);
\coordinate (gliubpppxu) at (\gliubxxxx, \gliubyyyu);
\coordinate (gliubpppxv) at (\gliubxxxx, \gliubyyyv);
\coordinate (gliubpppxw) at (\gliubxxxx, \gliubyyyw);
\coordinate (gliubpppxx) at (\gliubxxxx, \gliubyyyx);
\coordinate (gliubpppxy) at (\gliubxxxx, \gliubyyyy);
\coordinate (gliubpppxz) at (\gliubxxxx, \gliubyyyz);
\coordinate (gliubpppya) at (\gliubxxxy, \gliubyyya);
\coordinate (gliubpppyb) at (\gliubxxxy, \gliubyyyb);
\coordinate (gliubpppyc) at (\gliubxxxy, \gliubyyyc);
\coordinate (gliubpppyd) at (\gliubxxxy, \gliubyyyd);
\coordinate (gliubpppye) at (\gliubxxxy, \gliubyyye);
\coordinate (gliubpppyf) at (\gliubxxxy, \gliubyyyf);
\coordinate (gliubpppyg) at (\gliubxxxy, \gliubyyyg);
\coordinate (gliubpppyh) at (\gliubxxxy, \gliubyyyh);
\coordinate (gliubpppyi) at (\gliubxxxy, \gliubyyyi);
\coordinate (gliubpppyj) at (\gliubxxxy, \gliubyyyj);
\coordinate (gliubpppyk) at (\gliubxxxy, \gliubyyyk);
\coordinate (gliubpppyl) at (\gliubxxxy, \gliubyyyl);
\coordinate (gliubpppym) at (\gliubxxxy, \gliubyyym);
\coordinate (gliubpppyn) at (\gliubxxxy, \gliubyyyn);
\coordinate (gliubpppyo) at (\gliubxxxy, \gliubyyyo);
\coordinate (gliubpppyp) at (\gliubxxxy, \gliubyyyp);
\coordinate (gliubpppyq) at (\gliubxxxy, \gliubyyyq);
\coordinate (gliubpppyr) at (\gliubxxxy, \gliubyyyr);
\coordinate (gliubpppys) at (\gliubxxxy, \gliubyyys);
\coordinate (gliubpppyt) at (\gliubxxxy, \gliubyyyt);
\coordinate (gliubpppyu) at (\gliubxxxy, \gliubyyyu);
\coordinate (gliubpppyv) at (\gliubxxxy, \gliubyyyv);
\coordinate (gliubpppyw) at (\gliubxxxy, \gliubyyyw);
\coordinate (gliubpppyx) at (\gliubxxxy, \gliubyyyx);
\coordinate (gliubpppyy) at (\gliubxxxy, \gliubyyyy);
\coordinate (gliubpppyz) at (\gliubxxxy, \gliubyyyz);
\coordinate (gliubpppza) at (\gliubxxxz, \gliubyyya);
\coordinate (gliubpppzb) at (\gliubxxxz, \gliubyyyb);
\coordinate (gliubpppzc) at (\gliubxxxz, \gliubyyyc);
\coordinate (gliubpppzd) at (\gliubxxxz, \gliubyyyd);
\coordinate (gliubpppze) at (\gliubxxxz, \gliubyyye);
\coordinate (gliubpppzf) at (\gliubxxxz, \gliubyyyf);
\coordinate (gliubpppzg) at (\gliubxxxz, \gliubyyyg);
\coordinate (gliubpppzh) at (\gliubxxxz, \gliubyyyh);
\coordinate (gliubpppzi) at (\gliubxxxz, \gliubyyyi);
\coordinate (gliubpppzj) at (\gliubxxxz, \gliubyyyj);
\coordinate (gliubpppzk) at (\gliubxxxz, \gliubyyyk);
\coordinate (gliubpppzl) at (\gliubxxxz, \gliubyyyl);
\coordinate (gliubpppzm) at (\gliubxxxz, \gliubyyym);
\coordinate (gliubpppzn) at (\gliubxxxz, \gliubyyyn);
\coordinate (gliubpppzo) at (\gliubxxxz, \gliubyyyo);
\coordinate (gliubpppzp) at (\gliubxxxz, \gliubyyyp);
\coordinate (gliubpppzq) at (\gliubxxxz, \gliubyyyq);
\coordinate (gliubpppzr) at (\gliubxxxz, \gliubyyyr);
\coordinate (gliubpppzs) at (\gliubxxxz, \gliubyyys);
\coordinate (gliubpppzt) at (\gliubxxxz, \gliubyyyt);
\coordinate (gliubpppzu) at (\gliubxxxz, \gliubyyyu);
\coordinate (gliubpppzv) at (\gliubxxxz, \gliubyyyv);
\coordinate (gliubpppzw) at (\gliubxxxz, \gliubyyyw);
\coordinate (gliubpppzx) at (\gliubxxxz, \gliubyyyx);
\coordinate (gliubpppzy) at (\gliubxxxz, \gliubyyyy);
\coordinate (gliubpppzz) at (\gliubxxxz, \gliubyyyz);

%\gangprintcoordinateat{(0,0)}{The last coordinate values: }{($(gliubpppzz)$)}; 



% Draw related part of the coordinate system with dashed helplines with letters as background, which would help to determine all coordinates. 
\coordinatebackgroundxy{gangliu} {f}{g}{v} {f}{g}{q};

%%%%%% The next line is for circuit 3.
\coordinatebackgroundxy{gliua}{a}{b}{h} {a}{b}{h};

%%%%%% The next line is for circuit 4.
\coordinatebackgroundxy{gliub}{b}{c}{f} {a}{b}{g};


% Draw the Opamp at the coordinate (gangliupppli) and name it as "myopamp".
\draw (gangliupppli) node [op amp] (myopamp) {};

% Retrieve the x- and y-components of the coordinates of the "+", "-", and "out" pins of myopamp, supposing we have no idea about them beforehand. 
\getxyingivenunit{cm}{(myopamp.+)}
                 {\myopamppx}{\myopamppy};
\getxyingivenunit{cm}{(myopamp.-)}
                 {\myopampmx}{\myopampmy};
\getxyingivenunit{cm}{(myopamp.out)}
                 {\myopampox}{\myopampoy};

\draw [-o] (myopamp.out) 
      to [short, xshift=1mm] 
      (\gangliuxxxr, \myopampoy) 
        node [anchor=north, yshift=-1mm] {$V_0$};

\draw [-o] (myopamp.+) 
      to [short, xshift=-1mm] 
      (\gangliuxxxj, \myopamppy) 
      node [anchor=north, yshift=-1mm] {$V_i$};

\draw (myopamp.-) -- 
      (\gangliuxxxj, \myopampmy) 
      to [R, l_=$\hspace{-2mm} R \text{=} 100 K\Omega$] 
      (\gangliuxxxh, \myopampmy) -- 
      (gangliuppphi) node [ground]{};

%      to [R] 
%      to [R = $R \text{=} 100 K\Omega$] 
%      to [R, l_=$R \text{=} 100 K\Omega$] 
%      to [R, l_=$\hspace{-2mm} R \text{=} 100 K\Omega$]
%      to [R, n=resistorl] 
%\node [anchor=south, xshift=-1mm, yshift=1mm] 
%      at (resistorl) {$R \text{=} 100 K\Omega$};

      
\draw (\gangliuxxxj, \myopampmy) -- 
      (gangliupppjj) 
      to [R, l_=$R_F \text{=} 300K \Omega$,
                         label/align=rotate] 
      (gangliupppjm);

%%%%%% Remove the next line for circuit 2.
% \draw (gangliupppjm) -| (gangliupppni);
      

%%%%%% The rest are added for circuit 2.      
%%%%%% The rest are added for circuit 2.      

\draw (gangliuppppi) |- (gangliupppno) --
      (gangliupppnn);

\draw (gangliupppnl) 
      to[american potentiometer, n=mypot, 
           l_=$R_P \text{=} 5 K \Omega$,   
                       label/align=rotate] 
      (gangliupppnn);

\draw (gangliupppnl) -- 
      (gangliupppnk) node [ground]{};


%%%%%% Remove the following line for circuit 3. 
%\draw (mypot.wiper) 
%        node [red, anchor=south east] {$V_A$} -| 
%      (gangliupppjm);


%%%%%% Add the rest lines for circuit 3. 
\getxyingivenunit{cm}{(mypot.wiper)}
                 {\mypotwiperx}{\mypotwipery};

\draw (mypot.wiper) 
        node [red, anchor=south east] {$V_A$} --
      (\gliuaxxxe, \mypotwipery);

\draw (\gliuaxxxe, \mypotwipery)  
      to[variable resistor = $R_D \text{=} 15K \Omega$] 
      (\gangliuxxxl, \mypotwipery) 
      node [red, anchor=south] {$V_B$};

\draw (\gangliuxxxl, \mypotwipery) -- 
      (gangliupppll)
      to [C = $C_D \text{=} 100 \mu F$] 
      (gangliuppplk) node [ground] {};


%%%%%% Remove the following line for circuit 4.
%\draw (\gangliuxxxl, \mypotwipery) -| (gangliupppjm);


%%%%%% Add the rest lines for circuit 4.

\draw (\gangliuxxxl, \mypotwipery) --
      (\gangliuxxxk, \mypotwipery)
      to [C = $C_I \text{=} 200 \mu F$] 
      (\gliubxxxc, \mypotwipery);

\draw (\gliubxxxc, \mypotwipery) -- 
      (\gliubxxxc, \gangliuyyyl)
      to[variable resistor = $R_I \text{=} 39K \Omega$] 
      (\gliubxxxc, \gangliuyyyk) node [ground] {};

\draw (\gliubxxxc, \mypotwipery) 
        node [red, anchor=south] {$V_C$} -|
      (gangliupppjm);



\end{circuitikz}



\end{document}
