\documentclass[tikz,border=5mm]{standalone}
\usepackage[siunitx]{circuitikz}
\usetikzlibrary{shapes,arrows,positioning}
%   This is an accessory  file for 
%   https://github.com/LiuGangKingston/Nestable-coordinate-system-for-Tikz-circuits.git
%            Version 1.0
%   free for non-commercial use.
%   Please send us emails for any problems/suggestions/comments.
%   Please be advised that none of us accept any responsibility
%   for any consequences arising out of the usage of this
%   software, especially for damage.
%   For usage, please refer to the README file and the following lines.
%   This code was written by
%        Gang Liu (gl.cell@outlook)
%                 (http://orcid.org/0000-0003-1575-9290)
%          and
%        Shiwei Huang (huang937@gmail.com)
%   Copyright (c) 2021
%
%
%  The following command is to get the x-component and y-component 
%  of a coordinate. The command is
%  \getxyofcoordinate{the coordinate}{x-component}{y-component};
\newcommand{\getxyofcoordinate}[3]{%
\coordinate (tempcoord) at ($#1$);
\path (tempcoord) node {};
\pgfgetlastxy{\tempx}{\tempy};
\pgfmathsetmacro{#2}{\tempx}
\pgfmathsetmacro{#3}{\tempy}
}


%  The following command is the same as above but for given unit.
%  The command is
%  \getxyingivenunit{the unit like cm}{the coordinate}{x-component}{y-component};
\newcommand{\getxyingivenunit}[4]{%
\coordinate (tempcoord) at (1#1,1#1);
\path (tempcoord) node {};
\pgfgetlastxy{\tempxunit}{\tempyunit};
\coordinate (tempcoord) at ($#2$);
\path (tempcoord) node {};
\pgfgetlastxy{\tempx}{\tempy};
\pgfmathsetmacro{#3}{\tempx/\tempxunit}
\pgfmathsetmacro{#4}{\tempy/\tempyunit}
}


%  The following command is to print the value of a coordinate with some words at the first coordinate postion 
%  The command is
%  \printcoordinateat{the first coordinate}{the words}{the coordinate};
\newcommand{\printcoordinateat}[3]{%
\getxyingivenunit{cm}{#3}{\tempxx}{\tempyy}
\node at #1 {#2 ($\tempxx$, $\tempyy$).};
}


%  The following command is to print a keyworded coordinate system as a background.
%  The command is
%  \coordinatebackground{the KEYWORD}
%                                            {the first letter in both x and y directions}
%                                       {the second letter in both x and y directions}
%                                             {the last letter in both x and y directions};
\newcommand{\coordinatebackground}[4]{
\pgfmathsetmacro{\colourpercent}{30}
\foreach \i in {#2,#3,...,#4} 
{\node [black!\colourpercent] at (#1ppp\i\i) {\i};}
\foreach \i in {#2,#4} 
{\node [white] at (#1ppp\i\i) {\i};}
\coordinatebackgroundxy{#1}{#2}{#3}{#4}{#2}{#3}{#4};
}


%  The following command is to print a keyworded coordinate system as a background.
%  The command is
%  \coordinatebackgroundxy{the KEYWORD}
%                                                {the first letter in the x direction}
%                                           {the second letter in the x direction}
%                                                 {the last letter in the x direction}
%                                                {the first letter in the y direction}
%                                           {the second letter in the y direction}
%                                                 {the last letter in the y direction};
\newcommand{\coordinatebackgroundxy}[7]{
\pgfmathsetmacro{\bordercolourpercent}{60}
\pgfmathsetmacro{\colourpercent}{30}

\foreach \i in {#2,#3,...,#4} 
\foreach \j in {#5} 
\foreach \k in {#7} 
{\draw [dashed,black!\colourpercent] (#1ppp\i\j) -- (#1ppp\i\k);}

\foreach \i in {#5,#6,...,#7} 
\foreach \j in {#2} 
\foreach \k in {#4} 
{\draw [dashed,black!\colourpercent] (#1ppp\j\i) -- (#1ppp\k\i);}

\foreach \i in {#2,#4} 
\foreach \j in {#5} 
\foreach \k in {#7} 
{\draw [dashed,black!\bordercolourpercent] (#1ppp\i\j) -- (#1ppp\i\k);}

\foreach \i in {#5,#7} 
\foreach \j in {#2} 
\foreach \k in {#4} 
{\draw [dashed,black!\bordercolourpercent] (#1ppp\j\i) -- (#1ppp\k\i);}

\foreach \i in {#2,#3,...,#4} 
\foreach \j in {#5} 
\foreach \k in {#7} 
{
\node [black!\bordercolourpercent] at ($(#1ppp\i\j) + (0,-.2)$) {\i};
\node [black!\bordercolourpercent] at ($(#1ppp\i\k) + (0,.2)$) {\i};
}

\foreach \i in {#5,#6,...,#7} 
\foreach \j in {#2} 
\foreach \k in {#4} 
{
\node [black!\bordercolourpercent] at ($(#1ppp\k\i) + (.2,0)$) {\i};
\node [black!\bordercolourpercent] at ($(#1ppp\j\i) + (-.2,0)$) {\i};
}

}










\begin{document}

\ctikzset{
/tikz/circuitikz/bipoles/length=1cm
}



 
 
\begin{circuitikz} [scale=0.8]
 
%%%%%% The next line is for fig. 4.
https://github.com/LiuGangKingston/Nestable-coordinate-system-for-Tikz-circuits.git
https://github.com/LiuGangKingston/Nestable-coordinate-system-for-Tikz-circuits.git


\pgfmathsetmacro{\totalganglbxxx}{26}
\pgfmathsetmacro{\totalganglbyyy}{26}
\pgfmathsetmacro{\ganglbxxxspacing}{1}
\pgfmathsetmacro{\ganglbyyyspacing}{1}
\pgfmathsetmacro{\ganglbxxxa}{-8}
\pgfmathsetmacro{\ganglbyyya}{-8}

\pgfmathsetmacro{\ganglbxxxb}{\ganglbxxxa + \ganglbxxxspacing + 0.0 }
\pgfmathsetmacro{\ganglbxxxc}{\ganglbxxxb + \ganglbxxxspacing + 0.0 }
\pgfmathsetmacro{\ganglbxxxd}{\ganglbxxxc + \ganglbxxxspacing + 0.0 }
\pgfmathsetmacro{\ganglbxxxe}{\ganglbxxxd + \ganglbxxxspacing + 0.0 }
\pgfmathsetmacro{\ganglbxxxf}{\ganglbxxxe + \ganglbxxxspacing + 0.0 }
\pgfmathsetmacro{\ganglbxxxg}{\ganglbxxxf + \ganglbxxxspacing + 0.0 }
\pgfmathsetmacro{\ganglbxxxh}{\ganglbxxxg + \ganglbxxxspacing + 0.0 }
\pgfmathsetmacro{\ganglbxxxi}{\ganglbxxxh + \ganglbxxxspacing + 0.0 }
\pgfmathsetmacro{\ganglbxxxj}{\ganglbxxxi + \ganglbxxxspacing + 0.0 }
\pgfmathsetmacro{\ganglbxxxk}{\ganglbxxxj + \ganglbxxxspacing + 0.0 }
\pgfmathsetmacro{\ganglbxxxl}{\ganglbxxxk + \ganglbxxxspacing + 0.0 }
\pgfmathsetmacro{\ganglbxxxm}{\ganglbxxxl + \ganglbxxxspacing + 0.0 }
\pgfmathsetmacro{\ganglbxxxn}{\ganglbxxxm + \ganglbxxxspacing + 0.0 }
\pgfmathsetmacro{\ganglbxxxo}{\ganglbxxxn + \ganglbxxxspacing + 0.0 }
\pgfmathsetmacro{\ganglbxxxp}{\ganglbxxxo + \ganglbxxxspacing + 0.0 }
\pgfmathsetmacro{\ganglbxxxq}{\ganglbxxxp + \ganglbxxxspacing + 0.0 }
\pgfmathsetmacro{\ganglbxxxr}{\ganglbxxxq + \ganglbxxxspacing + 0.0 }
\pgfmathsetmacro{\ganglbxxxs}{\ganglbxxxr + \ganglbxxxspacing + 0.0 }
\pgfmathsetmacro{\ganglbxxxt}{\ganglbxxxs + \ganglbxxxspacing + 0.0 }
\pgfmathsetmacro{\ganglbxxxu}{\ganglbxxxt + \ganglbxxxspacing + 0.0 }
\pgfmathsetmacro{\ganglbxxxv}{\ganglbxxxu + \ganglbxxxspacing + 0.0 }
\pgfmathsetmacro{\ganglbxxxw}{\ganglbxxxv + \ganglbxxxspacing + 0.0 }
\pgfmathsetmacro{\ganglbxxxx}{\ganglbxxxw + \ganglbxxxspacing + 0.0 }
\pgfmathsetmacro{\ganglbxxxy}{\ganglbxxxx + \ganglbxxxspacing + 0.0 }
\pgfmathsetmacro{\ganglbxxxz}{\ganglbxxxy + \ganglbxxxspacing + 0.0 }

\pgfmathsetmacro{\ganglbyyyb}{\ganglbyyya + \ganglbyyyspacing + 0.0 }
\pgfmathsetmacro{\ganglbyyyc}{\ganglbyyyb + \ganglbyyyspacing + 0.0 }
\pgfmathsetmacro{\ganglbyyyd}{\ganglbyyyc + \ganglbyyyspacing + 0.0 }
\pgfmathsetmacro{\ganglbyyye}{\ganglbyyyd + \ganglbyyyspacing + 0.0 }
\pgfmathsetmacro{\ganglbyyyf}{\ganglbyyye + \ganglbyyyspacing + 0.0 }
\pgfmathsetmacro{\ganglbyyyg}{\ganglbyyyf + \ganglbyyyspacing + 0.0 }
\pgfmathsetmacro{\ganglbyyyh}{\ganglbyyyg + \ganglbyyyspacing + 0.0 }
\pgfmathsetmacro{\ganglbyyyi}{\ganglbyyyh + \ganglbyyyspacing + 0.0 }
\pgfmathsetmacro{\ganglbyyyj}{\ganglbyyyi + \ganglbyyyspacing + 0.0 }
\pgfmathsetmacro{\ganglbyyyk}{\ganglbyyyj + \ganglbyyyspacing + 0.0 }
\pgfmathsetmacro{\ganglbyyyl}{\ganglbyyyk + \ganglbyyyspacing + 0.0 }
\pgfmathsetmacro{\ganglbyyym}{\ganglbyyyl + \ganglbyyyspacing + 0.0 }
\pgfmathsetmacro{\ganglbyyyn}{\ganglbyyym + \ganglbyyyspacing + 0.0 }
\pgfmathsetmacro{\ganglbyyyo}{\ganglbyyyn + \ganglbyyyspacing + 0.0 }
\pgfmathsetmacro{\ganglbyyyp}{\ganglbyyyo + \ganglbyyyspacing + 0.0 }
\pgfmathsetmacro{\ganglbyyyq}{\ganglbyyyp + \ganglbyyyspacing + 0.0 }
\pgfmathsetmacro{\ganglbyyyr}{\ganglbyyyq + \ganglbyyyspacing + 0.0 }
\pgfmathsetmacro{\ganglbyyys}{\ganglbyyyr + \ganglbyyyspacing + 0.0 }
\pgfmathsetmacro{\ganglbyyyt}{\ganglbyyys + \ganglbyyyspacing + 0.0 }
\pgfmathsetmacro{\ganglbyyyu}{\ganglbyyyt + \ganglbyyyspacing + 0.0 }
\pgfmathsetmacro{\ganglbyyyv}{\ganglbyyyu + \ganglbyyyspacing + 0.0 }
\pgfmathsetmacro{\ganglbyyyw}{\ganglbyyyv + \ganglbyyyspacing + 0.0 }
\pgfmathsetmacro{\ganglbyyyx}{\ganglbyyyw + \ganglbyyyspacing + 0.0 }
\pgfmathsetmacro{\ganglbyyyy}{\ganglbyyyx + \ganglbyyyspacing + 0.0 }
\pgfmathsetmacro{\ganglbyyyz}{\ganglbyyyy + \ganglbyyyspacing + 0.0 }

\coordinate (ganglbpppaa) at (\ganglbxxxa, \ganglbyyya);
\coordinate (ganglbpppab) at (\ganglbxxxa, \ganglbyyyb);
\coordinate (ganglbpppac) at (\ganglbxxxa, \ganglbyyyc);
\coordinate (ganglbpppad) at (\ganglbxxxa, \ganglbyyyd);
\coordinate (ganglbpppae) at (\ganglbxxxa, \ganglbyyye);
\coordinate (ganglbpppaf) at (\ganglbxxxa, \ganglbyyyf);
\coordinate (ganglbpppag) at (\ganglbxxxa, \ganglbyyyg);
\coordinate (ganglbpppah) at (\ganglbxxxa, \ganglbyyyh);
\coordinate (ganglbpppai) at (\ganglbxxxa, \ganglbyyyi);
\coordinate (ganglbpppaj) at (\ganglbxxxa, \ganglbyyyj);
\coordinate (ganglbpppak) at (\ganglbxxxa, \ganglbyyyk);
\coordinate (ganglbpppal) at (\ganglbxxxa, \ganglbyyyl);
\coordinate (ganglbpppam) at (\ganglbxxxa, \ganglbyyym);
\coordinate (ganglbpppan) at (\ganglbxxxa, \ganglbyyyn);
\coordinate (ganglbpppao) at (\ganglbxxxa, \ganglbyyyo);
\coordinate (ganglbpppap) at (\ganglbxxxa, \ganglbyyyp);
\coordinate (ganglbpppaq) at (\ganglbxxxa, \ganglbyyyq);
\coordinate (ganglbpppar) at (\ganglbxxxa, \ganglbyyyr);
\coordinate (ganglbpppas) at (\ganglbxxxa, \ganglbyyys);
\coordinate (ganglbpppat) at (\ganglbxxxa, \ganglbyyyt);
\coordinate (ganglbpppau) at (\ganglbxxxa, \ganglbyyyu);
\coordinate (ganglbpppav) at (\ganglbxxxa, \ganglbyyyv);
\coordinate (ganglbpppaw) at (\ganglbxxxa, \ganglbyyyw);
\coordinate (ganglbpppax) at (\ganglbxxxa, \ganglbyyyx);
\coordinate (ganglbpppay) at (\ganglbxxxa, \ganglbyyyy);
\coordinate (ganglbpppaz) at (\ganglbxxxa, \ganglbyyyz);
\coordinate (ganglbpppba) at (\ganglbxxxb, \ganglbyyya);
\coordinate (ganglbpppbb) at (\ganglbxxxb, \ganglbyyyb);
\coordinate (ganglbpppbc) at (\ganglbxxxb, \ganglbyyyc);
\coordinate (ganglbpppbd) at (\ganglbxxxb, \ganglbyyyd);
\coordinate (ganglbpppbe) at (\ganglbxxxb, \ganglbyyye);
\coordinate (ganglbpppbf) at (\ganglbxxxb, \ganglbyyyf);
\coordinate (ganglbpppbg) at (\ganglbxxxb, \ganglbyyyg);
\coordinate (ganglbpppbh) at (\ganglbxxxb, \ganglbyyyh);
\coordinate (ganglbpppbi) at (\ganglbxxxb, \ganglbyyyi);
\coordinate (ganglbpppbj) at (\ganglbxxxb, \ganglbyyyj);
\coordinate (ganglbpppbk) at (\ganglbxxxb, \ganglbyyyk);
\coordinate (ganglbpppbl) at (\ganglbxxxb, \ganglbyyyl);
\coordinate (ganglbpppbm) at (\ganglbxxxb, \ganglbyyym);
\coordinate (ganglbpppbn) at (\ganglbxxxb, \ganglbyyyn);
\coordinate (ganglbpppbo) at (\ganglbxxxb, \ganglbyyyo);
\coordinate (ganglbpppbp) at (\ganglbxxxb, \ganglbyyyp);
\coordinate (ganglbpppbq) at (\ganglbxxxb, \ganglbyyyq);
\coordinate (ganglbpppbr) at (\ganglbxxxb, \ganglbyyyr);
\coordinate (ganglbpppbs) at (\ganglbxxxb, \ganglbyyys);
\coordinate (ganglbpppbt) at (\ganglbxxxb, \ganglbyyyt);
\coordinate (ganglbpppbu) at (\ganglbxxxb, \ganglbyyyu);
\coordinate (ganglbpppbv) at (\ganglbxxxb, \ganglbyyyv);
\coordinate (ganglbpppbw) at (\ganglbxxxb, \ganglbyyyw);
\coordinate (ganglbpppbx) at (\ganglbxxxb, \ganglbyyyx);
\coordinate (ganglbpppby) at (\ganglbxxxb, \ganglbyyyy);
\coordinate (ganglbpppbz) at (\ganglbxxxb, \ganglbyyyz);
\coordinate (ganglbpppca) at (\ganglbxxxc, \ganglbyyya);
\coordinate (ganglbpppcb) at (\ganglbxxxc, \ganglbyyyb);
\coordinate (ganglbpppcc) at (\ganglbxxxc, \ganglbyyyc);
\coordinate (ganglbpppcd) at (\ganglbxxxc, \ganglbyyyd);
\coordinate (ganglbpppce) at (\ganglbxxxc, \ganglbyyye);
\coordinate (ganglbpppcf) at (\ganglbxxxc, \ganglbyyyf);
\coordinate (ganglbpppcg) at (\ganglbxxxc, \ganglbyyyg);
\coordinate (ganglbpppch) at (\ganglbxxxc, \ganglbyyyh);
\coordinate (ganglbpppci) at (\ganglbxxxc, \ganglbyyyi);
\coordinate (ganglbpppcj) at (\ganglbxxxc, \ganglbyyyj);
\coordinate (ganglbpppck) at (\ganglbxxxc, \ganglbyyyk);
\coordinate (ganglbpppcl) at (\ganglbxxxc, \ganglbyyyl);
\coordinate (ganglbpppcm) at (\ganglbxxxc, \ganglbyyym);
\coordinate (ganglbpppcn) at (\ganglbxxxc, \ganglbyyyn);
\coordinate (ganglbpppco) at (\ganglbxxxc, \ganglbyyyo);
\coordinate (ganglbpppcp) at (\ganglbxxxc, \ganglbyyyp);
\coordinate (ganglbpppcq) at (\ganglbxxxc, \ganglbyyyq);
\coordinate (ganglbpppcr) at (\ganglbxxxc, \ganglbyyyr);
\coordinate (ganglbpppcs) at (\ganglbxxxc, \ganglbyyys);
\coordinate (ganglbpppct) at (\ganglbxxxc, \ganglbyyyt);
\coordinate (ganglbpppcu) at (\ganglbxxxc, \ganglbyyyu);
\coordinate (ganglbpppcv) at (\ganglbxxxc, \ganglbyyyv);
\coordinate (ganglbpppcw) at (\ganglbxxxc, \ganglbyyyw);
\coordinate (ganglbpppcx) at (\ganglbxxxc, \ganglbyyyx);
\coordinate (ganglbpppcy) at (\ganglbxxxc, \ganglbyyyy);
\coordinate (ganglbpppcz) at (\ganglbxxxc, \ganglbyyyz);
\coordinate (ganglbpppda) at (\ganglbxxxd, \ganglbyyya);
\coordinate (ganglbpppdb) at (\ganglbxxxd, \ganglbyyyb);
\coordinate (ganglbpppdc) at (\ganglbxxxd, \ganglbyyyc);
\coordinate (ganglbpppdd) at (\ganglbxxxd, \ganglbyyyd);
\coordinate (ganglbpppde) at (\ganglbxxxd, \ganglbyyye);
\coordinate (ganglbpppdf) at (\ganglbxxxd, \ganglbyyyf);
\coordinate (ganglbpppdg) at (\ganglbxxxd, \ganglbyyyg);
\coordinate (ganglbpppdh) at (\ganglbxxxd, \ganglbyyyh);
\coordinate (ganglbpppdi) at (\ganglbxxxd, \ganglbyyyi);
\coordinate (ganglbpppdj) at (\ganglbxxxd, \ganglbyyyj);
\coordinate (ganglbpppdk) at (\ganglbxxxd, \ganglbyyyk);
\coordinate (ganglbpppdl) at (\ganglbxxxd, \ganglbyyyl);
\coordinate (ganglbpppdm) at (\ganglbxxxd, \ganglbyyym);
\coordinate (ganglbpppdn) at (\ganglbxxxd, \ganglbyyyn);
\coordinate (ganglbpppdo) at (\ganglbxxxd, \ganglbyyyo);
\coordinate (ganglbpppdp) at (\ganglbxxxd, \ganglbyyyp);
\coordinate (ganglbpppdq) at (\ganglbxxxd, \ganglbyyyq);
\coordinate (ganglbpppdr) at (\ganglbxxxd, \ganglbyyyr);
\coordinate (ganglbpppds) at (\ganglbxxxd, \ganglbyyys);
\coordinate (ganglbpppdt) at (\ganglbxxxd, \ganglbyyyt);
\coordinate (ganglbpppdu) at (\ganglbxxxd, \ganglbyyyu);
\coordinate (ganglbpppdv) at (\ganglbxxxd, \ganglbyyyv);
\coordinate (ganglbpppdw) at (\ganglbxxxd, \ganglbyyyw);
\coordinate (ganglbpppdx) at (\ganglbxxxd, \ganglbyyyx);
\coordinate (ganglbpppdy) at (\ganglbxxxd, \ganglbyyyy);
\coordinate (ganglbpppdz) at (\ganglbxxxd, \ganglbyyyz);
\coordinate (ganglbpppea) at (\ganglbxxxe, \ganglbyyya);
\coordinate (ganglbpppeb) at (\ganglbxxxe, \ganglbyyyb);
\coordinate (ganglbpppec) at (\ganglbxxxe, \ganglbyyyc);
\coordinate (ganglbppped) at (\ganglbxxxe, \ganglbyyyd);
\coordinate (ganglbpppee) at (\ganglbxxxe, \ganglbyyye);
\coordinate (ganglbpppef) at (\ganglbxxxe, \ganglbyyyf);
\coordinate (ganglbpppeg) at (\ganglbxxxe, \ganglbyyyg);
\coordinate (ganglbpppeh) at (\ganglbxxxe, \ganglbyyyh);
\coordinate (ganglbpppei) at (\ganglbxxxe, \ganglbyyyi);
\coordinate (ganglbpppej) at (\ganglbxxxe, \ganglbyyyj);
\coordinate (ganglbpppek) at (\ganglbxxxe, \ganglbyyyk);
\coordinate (ganglbpppel) at (\ganglbxxxe, \ganglbyyyl);
\coordinate (ganglbpppem) at (\ganglbxxxe, \ganglbyyym);
\coordinate (ganglbpppen) at (\ganglbxxxe, \ganglbyyyn);
\coordinate (ganglbpppeo) at (\ganglbxxxe, \ganglbyyyo);
\coordinate (ganglbpppep) at (\ganglbxxxe, \ganglbyyyp);
\coordinate (ganglbpppeq) at (\ganglbxxxe, \ganglbyyyq);
\coordinate (ganglbppper) at (\ganglbxxxe, \ganglbyyyr);
\coordinate (ganglbpppes) at (\ganglbxxxe, \ganglbyyys);
\coordinate (ganglbpppet) at (\ganglbxxxe, \ganglbyyyt);
\coordinate (ganglbpppeu) at (\ganglbxxxe, \ganglbyyyu);
\coordinate (ganglbpppev) at (\ganglbxxxe, \ganglbyyyv);
\coordinate (ganglbpppew) at (\ganglbxxxe, \ganglbyyyw);
\coordinate (ganglbpppex) at (\ganglbxxxe, \ganglbyyyx);
\coordinate (ganglbpppey) at (\ganglbxxxe, \ganglbyyyy);
\coordinate (ganglbpppez) at (\ganglbxxxe, \ganglbyyyz);
\coordinate (ganglbpppfa) at (\ganglbxxxf, \ganglbyyya);
\coordinate (ganglbpppfb) at (\ganglbxxxf, \ganglbyyyb);
\coordinate (ganglbpppfc) at (\ganglbxxxf, \ganglbyyyc);
\coordinate (ganglbpppfd) at (\ganglbxxxf, \ganglbyyyd);
\coordinate (ganglbpppfe) at (\ganglbxxxf, \ganglbyyye);
\coordinate (ganglbpppff) at (\ganglbxxxf, \ganglbyyyf);
\coordinate (ganglbpppfg) at (\ganglbxxxf, \ganglbyyyg);
\coordinate (ganglbpppfh) at (\ganglbxxxf, \ganglbyyyh);
\coordinate (ganglbpppfi) at (\ganglbxxxf, \ganglbyyyi);
\coordinate (ganglbpppfj) at (\ganglbxxxf, \ganglbyyyj);
\coordinate (ganglbpppfk) at (\ganglbxxxf, \ganglbyyyk);
\coordinate (ganglbpppfl) at (\ganglbxxxf, \ganglbyyyl);
\coordinate (ganglbpppfm) at (\ganglbxxxf, \ganglbyyym);
\coordinate (ganglbpppfn) at (\ganglbxxxf, \ganglbyyyn);
\coordinate (ganglbpppfo) at (\ganglbxxxf, \ganglbyyyo);
\coordinate (ganglbpppfp) at (\ganglbxxxf, \ganglbyyyp);
\coordinate (ganglbpppfq) at (\ganglbxxxf, \ganglbyyyq);
\coordinate (ganglbpppfr) at (\ganglbxxxf, \ganglbyyyr);
\coordinate (ganglbpppfs) at (\ganglbxxxf, \ganglbyyys);
\coordinate (ganglbpppft) at (\ganglbxxxf, \ganglbyyyt);
\coordinate (ganglbpppfu) at (\ganglbxxxf, \ganglbyyyu);
\coordinate (ganglbpppfv) at (\ganglbxxxf, \ganglbyyyv);
\coordinate (ganglbpppfw) at (\ganglbxxxf, \ganglbyyyw);
\coordinate (ganglbpppfx) at (\ganglbxxxf, \ganglbyyyx);
\coordinate (ganglbpppfy) at (\ganglbxxxf, \ganglbyyyy);
\coordinate (ganglbpppfz) at (\ganglbxxxf, \ganglbyyyz);
\coordinate (ganglbpppga) at (\ganglbxxxg, \ganglbyyya);
\coordinate (ganglbpppgb) at (\ganglbxxxg, \ganglbyyyb);
\coordinate (ganglbpppgc) at (\ganglbxxxg, \ganglbyyyc);
\coordinate (ganglbpppgd) at (\ganglbxxxg, \ganglbyyyd);
\coordinate (ganglbpppge) at (\ganglbxxxg, \ganglbyyye);
\coordinate (ganglbpppgf) at (\ganglbxxxg, \ganglbyyyf);
\coordinate (ganglbpppgg) at (\ganglbxxxg, \ganglbyyyg);
\coordinate (ganglbpppgh) at (\ganglbxxxg, \ganglbyyyh);
\coordinate (ganglbpppgi) at (\ganglbxxxg, \ganglbyyyi);
\coordinate (ganglbpppgj) at (\ganglbxxxg, \ganglbyyyj);
\coordinate (ganglbpppgk) at (\ganglbxxxg, \ganglbyyyk);
\coordinate (ganglbpppgl) at (\ganglbxxxg, \ganglbyyyl);
\coordinate (ganglbpppgm) at (\ganglbxxxg, \ganglbyyym);
\coordinate (ganglbpppgn) at (\ganglbxxxg, \ganglbyyyn);
\coordinate (ganglbpppgo) at (\ganglbxxxg, \ganglbyyyo);
\coordinate (ganglbpppgp) at (\ganglbxxxg, \ganglbyyyp);
\coordinate (ganglbpppgq) at (\ganglbxxxg, \ganglbyyyq);
\coordinate (ganglbpppgr) at (\ganglbxxxg, \ganglbyyyr);
\coordinate (ganglbpppgs) at (\ganglbxxxg, \ganglbyyys);
\coordinate (ganglbpppgt) at (\ganglbxxxg, \ganglbyyyt);
\coordinate (ganglbpppgu) at (\ganglbxxxg, \ganglbyyyu);
\coordinate (ganglbpppgv) at (\ganglbxxxg, \ganglbyyyv);
\coordinate (ganglbpppgw) at (\ganglbxxxg, \ganglbyyyw);
\coordinate (ganglbpppgx) at (\ganglbxxxg, \ganglbyyyx);
\coordinate (ganglbpppgy) at (\ganglbxxxg, \ganglbyyyy);
\coordinate (ganglbpppgz) at (\ganglbxxxg, \ganglbyyyz);
\coordinate (ganglbpppha) at (\ganglbxxxh, \ganglbyyya);
\coordinate (ganglbppphb) at (\ganglbxxxh, \ganglbyyyb);
\coordinate (ganglbppphc) at (\ganglbxxxh, \ganglbyyyc);
\coordinate (ganglbppphd) at (\ganglbxxxh, \ganglbyyyd);
\coordinate (ganglbppphe) at (\ganglbxxxh, \ganglbyyye);
\coordinate (ganglbppphf) at (\ganglbxxxh, \ganglbyyyf);
\coordinate (ganglbppphg) at (\ganglbxxxh, \ganglbyyyg);
\coordinate (ganglbppphh) at (\ganglbxxxh, \ganglbyyyh);
\coordinate (ganglbppphi) at (\ganglbxxxh, \ganglbyyyi);
\coordinate (ganglbppphj) at (\ganglbxxxh, \ganglbyyyj);
\coordinate (ganglbppphk) at (\ganglbxxxh, \ganglbyyyk);
\coordinate (ganglbppphl) at (\ganglbxxxh, \ganglbyyyl);
\coordinate (ganglbppphm) at (\ganglbxxxh, \ganglbyyym);
\coordinate (ganglbppphn) at (\ganglbxxxh, \ganglbyyyn);
\coordinate (ganglbpppho) at (\ganglbxxxh, \ganglbyyyo);
\coordinate (ganglbppphp) at (\ganglbxxxh, \ganglbyyyp);
\coordinate (ganglbppphq) at (\ganglbxxxh, \ganglbyyyq);
\coordinate (ganglbppphr) at (\ganglbxxxh, \ganglbyyyr);
\coordinate (ganglbppphs) at (\ganglbxxxh, \ganglbyyys);
\coordinate (ganglbpppht) at (\ganglbxxxh, \ganglbyyyt);
\coordinate (ganglbppphu) at (\ganglbxxxh, \ganglbyyyu);
\coordinate (ganglbppphv) at (\ganglbxxxh, \ganglbyyyv);
\coordinate (ganglbppphw) at (\ganglbxxxh, \ganglbyyyw);
\coordinate (ganglbppphx) at (\ganglbxxxh, \ganglbyyyx);
\coordinate (ganglbppphy) at (\ganglbxxxh, \ganglbyyyy);
\coordinate (ganglbppphz) at (\ganglbxxxh, \ganglbyyyz);
\coordinate (ganglbpppia) at (\ganglbxxxi, \ganglbyyya);
\coordinate (ganglbpppib) at (\ganglbxxxi, \ganglbyyyb);
\coordinate (ganglbpppic) at (\ganglbxxxi, \ganglbyyyc);
\coordinate (ganglbpppid) at (\ganglbxxxi, \ganglbyyyd);
\coordinate (ganglbpppie) at (\ganglbxxxi, \ganglbyyye);
\coordinate (ganglbpppif) at (\ganglbxxxi, \ganglbyyyf);
\coordinate (ganglbpppig) at (\ganglbxxxi, \ganglbyyyg);
\coordinate (ganglbpppih) at (\ganglbxxxi, \ganglbyyyh);
\coordinate (ganglbpppii) at (\ganglbxxxi, \ganglbyyyi);
\coordinate (ganglbpppij) at (\ganglbxxxi, \ganglbyyyj);
\coordinate (ganglbpppik) at (\ganglbxxxi, \ganglbyyyk);
\coordinate (ganglbpppil) at (\ganglbxxxi, \ganglbyyyl);
\coordinate (ganglbpppim) at (\ganglbxxxi, \ganglbyyym);
\coordinate (ganglbpppin) at (\ganglbxxxi, \ganglbyyyn);
\coordinate (ganglbpppio) at (\ganglbxxxi, \ganglbyyyo);
\coordinate (ganglbpppip) at (\ganglbxxxi, \ganglbyyyp);
\coordinate (ganglbpppiq) at (\ganglbxxxi, \ganglbyyyq);
\coordinate (ganglbpppir) at (\ganglbxxxi, \ganglbyyyr);
\coordinate (ganglbpppis) at (\ganglbxxxi, \ganglbyyys);
\coordinate (ganglbpppit) at (\ganglbxxxi, \ganglbyyyt);
\coordinate (ganglbpppiu) at (\ganglbxxxi, \ganglbyyyu);
\coordinate (ganglbpppiv) at (\ganglbxxxi, \ganglbyyyv);
\coordinate (ganglbpppiw) at (\ganglbxxxi, \ganglbyyyw);
\coordinate (ganglbpppix) at (\ganglbxxxi, \ganglbyyyx);
\coordinate (ganglbpppiy) at (\ganglbxxxi, \ganglbyyyy);
\coordinate (ganglbpppiz) at (\ganglbxxxi, \ganglbyyyz);
\coordinate (ganglbpppja) at (\ganglbxxxj, \ganglbyyya);
\coordinate (ganglbpppjb) at (\ganglbxxxj, \ganglbyyyb);
\coordinate (ganglbpppjc) at (\ganglbxxxj, \ganglbyyyc);
\coordinate (ganglbpppjd) at (\ganglbxxxj, \ganglbyyyd);
\coordinate (ganglbpppje) at (\ganglbxxxj, \ganglbyyye);
\coordinate (ganglbpppjf) at (\ganglbxxxj, \ganglbyyyf);
\coordinate (ganglbpppjg) at (\ganglbxxxj, \ganglbyyyg);
\coordinate (ganglbpppjh) at (\ganglbxxxj, \ganglbyyyh);
\coordinate (ganglbpppji) at (\ganglbxxxj, \ganglbyyyi);
\coordinate (ganglbpppjj) at (\ganglbxxxj, \ganglbyyyj);
\coordinate (ganglbpppjk) at (\ganglbxxxj, \ganglbyyyk);
\coordinate (ganglbpppjl) at (\ganglbxxxj, \ganglbyyyl);
\coordinate (ganglbpppjm) at (\ganglbxxxj, \ganglbyyym);
\coordinate (ganglbpppjn) at (\ganglbxxxj, \ganglbyyyn);
\coordinate (ganglbpppjo) at (\ganglbxxxj, \ganglbyyyo);
\coordinate (ganglbpppjp) at (\ganglbxxxj, \ganglbyyyp);
\coordinate (ganglbpppjq) at (\ganglbxxxj, \ganglbyyyq);
\coordinate (ganglbpppjr) at (\ganglbxxxj, \ganglbyyyr);
\coordinate (ganglbpppjs) at (\ganglbxxxj, \ganglbyyys);
\coordinate (ganglbpppjt) at (\ganglbxxxj, \ganglbyyyt);
\coordinate (ganglbpppju) at (\ganglbxxxj, \ganglbyyyu);
\coordinate (ganglbpppjv) at (\ganglbxxxj, \ganglbyyyv);
\coordinate (ganglbpppjw) at (\ganglbxxxj, \ganglbyyyw);
\coordinate (ganglbpppjx) at (\ganglbxxxj, \ganglbyyyx);
\coordinate (ganglbpppjy) at (\ganglbxxxj, \ganglbyyyy);
\coordinate (ganglbpppjz) at (\ganglbxxxj, \ganglbyyyz);
\coordinate (ganglbpppka) at (\ganglbxxxk, \ganglbyyya);
\coordinate (ganglbpppkb) at (\ganglbxxxk, \ganglbyyyb);
\coordinate (ganglbpppkc) at (\ganglbxxxk, \ganglbyyyc);
\coordinate (ganglbpppkd) at (\ganglbxxxk, \ganglbyyyd);
\coordinate (ganglbpppke) at (\ganglbxxxk, \ganglbyyye);
\coordinate (ganglbpppkf) at (\ganglbxxxk, \ganglbyyyf);
\coordinate (ganglbpppkg) at (\ganglbxxxk, \ganglbyyyg);
\coordinate (ganglbpppkh) at (\ganglbxxxk, \ganglbyyyh);
\coordinate (ganglbpppki) at (\ganglbxxxk, \ganglbyyyi);
\coordinate (ganglbpppkj) at (\ganglbxxxk, \ganglbyyyj);
\coordinate (ganglbpppkk) at (\ganglbxxxk, \ganglbyyyk);
\coordinate (ganglbpppkl) at (\ganglbxxxk, \ganglbyyyl);
\coordinate (ganglbpppkm) at (\ganglbxxxk, \ganglbyyym);
\coordinate (ganglbpppkn) at (\ganglbxxxk, \ganglbyyyn);
\coordinate (ganglbpppko) at (\ganglbxxxk, \ganglbyyyo);
\coordinate (ganglbpppkp) at (\ganglbxxxk, \ganglbyyyp);
\coordinate (ganglbpppkq) at (\ganglbxxxk, \ganglbyyyq);
\coordinate (ganglbpppkr) at (\ganglbxxxk, \ganglbyyyr);
\coordinate (ganglbpppks) at (\ganglbxxxk, \ganglbyyys);
\coordinate (ganglbpppkt) at (\ganglbxxxk, \ganglbyyyt);
\coordinate (ganglbpppku) at (\ganglbxxxk, \ganglbyyyu);
\coordinate (ganglbpppkv) at (\ganglbxxxk, \ganglbyyyv);
\coordinate (ganglbpppkw) at (\ganglbxxxk, \ganglbyyyw);
\coordinate (ganglbpppkx) at (\ganglbxxxk, \ganglbyyyx);
\coordinate (ganglbpppky) at (\ganglbxxxk, \ganglbyyyy);
\coordinate (ganglbpppkz) at (\ganglbxxxk, \ganglbyyyz);
\coordinate (ganglbpppla) at (\ganglbxxxl, \ganglbyyya);
\coordinate (ganglbppplb) at (\ganglbxxxl, \ganglbyyyb);
\coordinate (ganglbppplc) at (\ganglbxxxl, \ganglbyyyc);
\coordinate (ganglbpppld) at (\ganglbxxxl, \ganglbyyyd);
\coordinate (ganglbppple) at (\ganglbxxxl, \ganglbyyye);
\coordinate (ganglbppplf) at (\ganglbxxxl, \ganglbyyyf);
\coordinate (ganglbppplg) at (\ganglbxxxl, \ganglbyyyg);
\coordinate (ganglbppplh) at (\ganglbxxxl, \ganglbyyyh);
\coordinate (ganglbpppli) at (\ganglbxxxl, \ganglbyyyi);
\coordinate (ganglbppplj) at (\ganglbxxxl, \ganglbyyyj);
\coordinate (ganglbppplk) at (\ganglbxxxl, \ganglbyyyk);
\coordinate (ganglbpppll) at (\ganglbxxxl, \ganglbyyyl);
\coordinate (ganglbppplm) at (\ganglbxxxl, \ganglbyyym);
\coordinate (ganglbpppln) at (\ganglbxxxl, \ganglbyyyn);
\coordinate (ganglbppplo) at (\ganglbxxxl, \ganglbyyyo);
\coordinate (ganglbppplp) at (\ganglbxxxl, \ganglbyyyp);
\coordinate (ganglbppplq) at (\ganglbxxxl, \ganglbyyyq);
\coordinate (ganglbppplr) at (\ganglbxxxl, \ganglbyyyr);
\coordinate (ganglbpppls) at (\ganglbxxxl, \ganglbyyys);
\coordinate (ganglbppplt) at (\ganglbxxxl, \ganglbyyyt);
\coordinate (ganglbppplu) at (\ganglbxxxl, \ganglbyyyu);
\coordinate (ganglbppplv) at (\ganglbxxxl, \ganglbyyyv);
\coordinate (ganglbppplw) at (\ganglbxxxl, \ganglbyyyw);
\coordinate (ganglbppplx) at (\ganglbxxxl, \ganglbyyyx);
\coordinate (ganglbppply) at (\ganglbxxxl, \ganglbyyyy);
\coordinate (ganglbppplz) at (\ganglbxxxl, \ganglbyyyz);
\coordinate (ganglbpppma) at (\ganglbxxxm, \ganglbyyya);
\coordinate (ganglbpppmb) at (\ganglbxxxm, \ganglbyyyb);
\coordinate (ganglbpppmc) at (\ganglbxxxm, \ganglbyyyc);
\coordinate (ganglbpppmd) at (\ganglbxxxm, \ganglbyyyd);
\coordinate (ganglbpppme) at (\ganglbxxxm, \ganglbyyye);
\coordinate (ganglbpppmf) at (\ganglbxxxm, \ganglbyyyf);
\coordinate (ganglbpppmg) at (\ganglbxxxm, \ganglbyyyg);
\coordinate (ganglbpppmh) at (\ganglbxxxm, \ganglbyyyh);
\coordinate (ganglbpppmi) at (\ganglbxxxm, \ganglbyyyi);
\coordinate (ganglbpppmj) at (\ganglbxxxm, \ganglbyyyj);
\coordinate (ganglbpppmk) at (\ganglbxxxm, \ganglbyyyk);
\coordinate (ganglbpppml) at (\ganglbxxxm, \ganglbyyyl);
\coordinate (ganglbpppmm) at (\ganglbxxxm, \ganglbyyym);
\coordinate (ganglbpppmn) at (\ganglbxxxm, \ganglbyyyn);
\coordinate (ganglbpppmo) at (\ganglbxxxm, \ganglbyyyo);
\coordinate (ganglbpppmp) at (\ganglbxxxm, \ganglbyyyp);
\coordinate (ganglbpppmq) at (\ganglbxxxm, \ganglbyyyq);
\coordinate (ganglbpppmr) at (\ganglbxxxm, \ganglbyyyr);
\coordinate (ganglbpppms) at (\ganglbxxxm, \ganglbyyys);
\coordinate (ganglbpppmt) at (\ganglbxxxm, \ganglbyyyt);
\coordinate (ganglbpppmu) at (\ganglbxxxm, \ganglbyyyu);
\coordinate (ganglbpppmv) at (\ganglbxxxm, \ganglbyyyv);
\coordinate (ganglbpppmw) at (\ganglbxxxm, \ganglbyyyw);
\coordinate (ganglbpppmx) at (\ganglbxxxm, \ganglbyyyx);
\coordinate (ganglbpppmy) at (\ganglbxxxm, \ganglbyyyy);
\coordinate (ganglbpppmz) at (\ganglbxxxm, \ganglbyyyz);
\coordinate (ganglbpppna) at (\ganglbxxxn, \ganglbyyya);
\coordinate (ganglbpppnb) at (\ganglbxxxn, \ganglbyyyb);
\coordinate (ganglbpppnc) at (\ganglbxxxn, \ganglbyyyc);
\coordinate (ganglbpppnd) at (\ganglbxxxn, \ganglbyyyd);
\coordinate (ganglbpppne) at (\ganglbxxxn, \ganglbyyye);
\coordinate (ganglbpppnf) at (\ganglbxxxn, \ganglbyyyf);
\coordinate (ganglbpppng) at (\ganglbxxxn, \ganglbyyyg);
\coordinate (ganglbpppnh) at (\ganglbxxxn, \ganglbyyyh);
\coordinate (ganglbpppni) at (\ganglbxxxn, \ganglbyyyi);
\coordinate (ganglbpppnj) at (\ganglbxxxn, \ganglbyyyj);
\coordinate (ganglbpppnk) at (\ganglbxxxn, \ganglbyyyk);
\coordinate (ganglbpppnl) at (\ganglbxxxn, \ganglbyyyl);
\coordinate (ganglbpppnm) at (\ganglbxxxn, \ganglbyyym);
\coordinate (ganglbpppnn) at (\ganglbxxxn, \ganglbyyyn);
\coordinate (ganglbpppno) at (\ganglbxxxn, \ganglbyyyo);
\coordinate (ganglbpppnp) at (\ganglbxxxn, \ganglbyyyp);
\coordinate (ganglbpppnq) at (\ganglbxxxn, \ganglbyyyq);
\coordinate (ganglbpppnr) at (\ganglbxxxn, \ganglbyyyr);
\coordinate (ganglbpppns) at (\ganglbxxxn, \ganglbyyys);
\coordinate (ganglbpppnt) at (\ganglbxxxn, \ganglbyyyt);
\coordinate (ganglbpppnu) at (\ganglbxxxn, \ganglbyyyu);
\coordinate (ganglbpppnv) at (\ganglbxxxn, \ganglbyyyv);
\coordinate (ganglbpppnw) at (\ganglbxxxn, \ganglbyyyw);
\coordinate (ganglbpppnx) at (\ganglbxxxn, \ganglbyyyx);
\coordinate (ganglbpppny) at (\ganglbxxxn, \ganglbyyyy);
\coordinate (ganglbpppnz) at (\ganglbxxxn, \ganglbyyyz);
\coordinate (ganglbpppoa) at (\ganglbxxxo, \ganglbyyya);
\coordinate (ganglbpppob) at (\ganglbxxxo, \ganglbyyyb);
\coordinate (ganglbpppoc) at (\ganglbxxxo, \ganglbyyyc);
\coordinate (ganglbpppod) at (\ganglbxxxo, \ganglbyyyd);
\coordinate (ganglbpppoe) at (\ganglbxxxo, \ganglbyyye);
\coordinate (ganglbpppof) at (\ganglbxxxo, \ganglbyyyf);
\coordinate (ganglbpppog) at (\ganglbxxxo, \ganglbyyyg);
\coordinate (ganglbpppoh) at (\ganglbxxxo, \ganglbyyyh);
\coordinate (ganglbpppoi) at (\ganglbxxxo, \ganglbyyyi);
\coordinate (ganglbpppoj) at (\ganglbxxxo, \ganglbyyyj);
\coordinate (ganglbpppok) at (\ganglbxxxo, \ganglbyyyk);
\coordinate (ganglbpppol) at (\ganglbxxxo, \ganglbyyyl);
\coordinate (ganglbpppom) at (\ganglbxxxo, \ganglbyyym);
\coordinate (ganglbpppon) at (\ganglbxxxo, \ganglbyyyn);
\coordinate (ganglbpppoo) at (\ganglbxxxo, \ganglbyyyo);
\coordinate (ganglbpppop) at (\ganglbxxxo, \ganglbyyyp);
\coordinate (ganglbpppoq) at (\ganglbxxxo, \ganglbyyyq);
\coordinate (ganglbpppor) at (\ganglbxxxo, \ganglbyyyr);
\coordinate (ganglbpppos) at (\ganglbxxxo, \ganglbyyys);
\coordinate (ganglbpppot) at (\ganglbxxxo, \ganglbyyyt);
\coordinate (ganglbpppou) at (\ganglbxxxo, \ganglbyyyu);
\coordinate (ganglbpppov) at (\ganglbxxxo, \ganglbyyyv);
\coordinate (ganglbpppow) at (\ganglbxxxo, \ganglbyyyw);
\coordinate (ganglbpppox) at (\ganglbxxxo, \ganglbyyyx);
\coordinate (ganglbpppoy) at (\ganglbxxxo, \ganglbyyyy);
\coordinate (ganglbpppoz) at (\ganglbxxxo, \ganglbyyyz);
\coordinate (ganglbppppa) at (\ganglbxxxp, \ganglbyyya);
\coordinate (ganglbppppb) at (\ganglbxxxp, \ganglbyyyb);
\coordinate (ganglbppppc) at (\ganglbxxxp, \ganglbyyyc);
\coordinate (ganglbppppd) at (\ganglbxxxp, \ganglbyyyd);
\coordinate (ganglbppppe) at (\ganglbxxxp, \ganglbyyye);
\coordinate (ganglbppppf) at (\ganglbxxxp, \ganglbyyyf);
\coordinate (ganglbppppg) at (\ganglbxxxp, \ganglbyyyg);
\coordinate (ganglbpppph) at (\ganglbxxxp, \ganglbyyyh);
\coordinate (ganglbppppi) at (\ganglbxxxp, \ganglbyyyi);
\coordinate (ganglbppppj) at (\ganglbxxxp, \ganglbyyyj);
\coordinate (ganglbppppk) at (\ganglbxxxp, \ganglbyyyk);
\coordinate (ganglbppppl) at (\ganglbxxxp, \ganglbyyyl);
\coordinate (ganglbppppm) at (\ganglbxxxp, \ganglbyyym);
\coordinate (ganglbppppn) at (\ganglbxxxp, \ganglbyyyn);
\coordinate (ganglbppppo) at (\ganglbxxxp, \ganglbyyyo);
\coordinate (ganglbppppp) at (\ganglbxxxp, \ganglbyyyp);
\coordinate (ganglbppppq) at (\ganglbxxxp, \ganglbyyyq);
\coordinate (ganglbppppr) at (\ganglbxxxp, \ganglbyyyr);
\coordinate (ganglbpppps) at (\ganglbxxxp, \ganglbyyys);
\coordinate (ganglbppppt) at (\ganglbxxxp, \ganglbyyyt);
\coordinate (ganglbppppu) at (\ganglbxxxp, \ganglbyyyu);
\coordinate (ganglbppppv) at (\ganglbxxxp, \ganglbyyyv);
\coordinate (ganglbppppw) at (\ganglbxxxp, \ganglbyyyw);
\coordinate (ganglbppppx) at (\ganglbxxxp, \ganglbyyyx);
\coordinate (ganglbppppy) at (\ganglbxxxp, \ganglbyyyy);
\coordinate (ganglbppppz) at (\ganglbxxxp, \ganglbyyyz);
\coordinate (ganglbpppqa) at (\ganglbxxxq, \ganglbyyya);
\coordinate (ganglbpppqb) at (\ganglbxxxq, \ganglbyyyb);
\coordinate (ganglbpppqc) at (\ganglbxxxq, \ganglbyyyc);
\coordinate (ganglbpppqd) at (\ganglbxxxq, \ganglbyyyd);
\coordinate (ganglbpppqe) at (\ganglbxxxq, \ganglbyyye);
\coordinate (ganglbpppqf) at (\ganglbxxxq, \ganglbyyyf);
\coordinate (ganglbpppqg) at (\ganglbxxxq, \ganglbyyyg);
\coordinate (ganglbpppqh) at (\ganglbxxxq, \ganglbyyyh);
\coordinate (ganglbpppqi) at (\ganglbxxxq, \ganglbyyyi);
\coordinate (ganglbpppqj) at (\ganglbxxxq, \ganglbyyyj);
\coordinate (ganglbpppqk) at (\ganglbxxxq, \ganglbyyyk);
\coordinate (ganglbpppql) at (\ganglbxxxq, \ganglbyyyl);
\coordinate (ganglbpppqm) at (\ganglbxxxq, \ganglbyyym);
\coordinate (ganglbpppqn) at (\ganglbxxxq, \ganglbyyyn);
\coordinate (ganglbpppqo) at (\ganglbxxxq, \ganglbyyyo);
\coordinate (ganglbpppqp) at (\ganglbxxxq, \ganglbyyyp);
\coordinate (ganglbpppqq) at (\ganglbxxxq, \ganglbyyyq);
\coordinate (ganglbpppqr) at (\ganglbxxxq, \ganglbyyyr);
\coordinate (ganglbpppqs) at (\ganglbxxxq, \ganglbyyys);
\coordinate (ganglbpppqt) at (\ganglbxxxq, \ganglbyyyt);
\coordinate (ganglbpppqu) at (\ganglbxxxq, \ganglbyyyu);
\coordinate (ganglbpppqv) at (\ganglbxxxq, \ganglbyyyv);
\coordinate (ganglbpppqw) at (\ganglbxxxq, \ganglbyyyw);
\coordinate (ganglbpppqx) at (\ganglbxxxq, \ganglbyyyx);
\coordinate (ganglbpppqy) at (\ganglbxxxq, \ganglbyyyy);
\coordinate (ganglbpppqz) at (\ganglbxxxq, \ganglbyyyz);
\coordinate (ganglbpppra) at (\ganglbxxxr, \ganglbyyya);
\coordinate (ganglbppprb) at (\ganglbxxxr, \ganglbyyyb);
\coordinate (ganglbppprc) at (\ganglbxxxr, \ganglbyyyc);
\coordinate (ganglbppprd) at (\ganglbxxxr, \ganglbyyyd);
\coordinate (ganglbpppre) at (\ganglbxxxr, \ganglbyyye);
\coordinate (ganglbppprf) at (\ganglbxxxr, \ganglbyyyf);
\coordinate (ganglbppprg) at (\ganglbxxxr, \ganglbyyyg);
\coordinate (ganglbppprh) at (\ganglbxxxr, \ganglbyyyh);
\coordinate (ganglbpppri) at (\ganglbxxxr, \ganglbyyyi);
\coordinate (ganglbppprj) at (\ganglbxxxr, \ganglbyyyj);
\coordinate (ganglbppprk) at (\ganglbxxxr, \ganglbyyyk);
\coordinate (ganglbppprl) at (\ganglbxxxr, \ganglbyyyl);
\coordinate (ganglbppprm) at (\ganglbxxxr, \ganglbyyym);
\coordinate (ganglbppprn) at (\ganglbxxxr, \ganglbyyyn);
\coordinate (ganglbpppro) at (\ganglbxxxr, \ganglbyyyo);
\coordinate (ganglbppprp) at (\ganglbxxxr, \ganglbyyyp);
\coordinate (ganglbppprq) at (\ganglbxxxr, \ganglbyyyq);
\coordinate (ganglbppprr) at (\ganglbxxxr, \ganglbyyyr);
\coordinate (ganglbppprs) at (\ganglbxxxr, \ganglbyyys);
\coordinate (ganglbppprt) at (\ganglbxxxr, \ganglbyyyt);
\coordinate (ganglbpppru) at (\ganglbxxxr, \ganglbyyyu);
\coordinate (ganglbppprv) at (\ganglbxxxr, \ganglbyyyv);
\coordinate (ganglbppprw) at (\ganglbxxxr, \ganglbyyyw);
\coordinate (ganglbppprx) at (\ganglbxxxr, \ganglbyyyx);
\coordinate (ganglbpppry) at (\ganglbxxxr, \ganglbyyyy);
\coordinate (ganglbppprz) at (\ganglbxxxr, \ganglbyyyz);
\coordinate (ganglbpppsa) at (\ganglbxxxs, \ganglbyyya);
\coordinate (ganglbpppsb) at (\ganglbxxxs, \ganglbyyyb);
\coordinate (ganglbpppsc) at (\ganglbxxxs, \ganglbyyyc);
\coordinate (ganglbpppsd) at (\ganglbxxxs, \ganglbyyyd);
\coordinate (ganglbpppse) at (\ganglbxxxs, \ganglbyyye);
\coordinate (ganglbpppsf) at (\ganglbxxxs, \ganglbyyyf);
\coordinate (ganglbpppsg) at (\ganglbxxxs, \ganglbyyyg);
\coordinate (ganglbpppsh) at (\ganglbxxxs, \ganglbyyyh);
\coordinate (ganglbpppsi) at (\ganglbxxxs, \ganglbyyyi);
\coordinate (ganglbpppsj) at (\ganglbxxxs, \ganglbyyyj);
\coordinate (ganglbpppsk) at (\ganglbxxxs, \ganglbyyyk);
\coordinate (ganglbpppsl) at (\ganglbxxxs, \ganglbyyyl);
\coordinate (ganglbpppsm) at (\ganglbxxxs, \ganglbyyym);
\coordinate (ganglbpppsn) at (\ganglbxxxs, \ganglbyyyn);
\coordinate (ganglbpppso) at (\ganglbxxxs, \ganglbyyyo);
\coordinate (ganglbpppsp) at (\ganglbxxxs, \ganglbyyyp);
\coordinate (ganglbpppsq) at (\ganglbxxxs, \ganglbyyyq);
\coordinate (ganglbpppsr) at (\ganglbxxxs, \ganglbyyyr);
\coordinate (ganglbpppss) at (\ganglbxxxs, \ganglbyyys);
\coordinate (ganglbpppst) at (\ganglbxxxs, \ganglbyyyt);
\coordinate (ganglbpppsu) at (\ganglbxxxs, \ganglbyyyu);
\coordinate (ganglbpppsv) at (\ganglbxxxs, \ganglbyyyv);
\coordinate (ganglbpppsw) at (\ganglbxxxs, \ganglbyyyw);
\coordinate (ganglbpppsx) at (\ganglbxxxs, \ganglbyyyx);
\coordinate (ganglbpppsy) at (\ganglbxxxs, \ganglbyyyy);
\coordinate (ganglbpppsz) at (\ganglbxxxs, \ganglbyyyz);
\coordinate (ganglbpppta) at (\ganglbxxxt, \ganglbyyya);
\coordinate (ganglbppptb) at (\ganglbxxxt, \ganglbyyyb);
\coordinate (ganglbppptc) at (\ganglbxxxt, \ganglbyyyc);
\coordinate (ganglbppptd) at (\ganglbxxxt, \ganglbyyyd);
\coordinate (ganglbpppte) at (\ganglbxxxt, \ganglbyyye);
\coordinate (ganglbppptf) at (\ganglbxxxt, \ganglbyyyf);
\coordinate (ganglbppptg) at (\ganglbxxxt, \ganglbyyyg);
\coordinate (ganglbpppth) at (\ganglbxxxt, \ganglbyyyh);
\coordinate (ganglbpppti) at (\ganglbxxxt, \ganglbyyyi);
\coordinate (ganglbppptj) at (\ganglbxxxt, \ganglbyyyj);
\coordinate (ganglbppptk) at (\ganglbxxxt, \ganglbyyyk);
\coordinate (ganglbppptl) at (\ganglbxxxt, \ganglbyyyl);
\coordinate (ganglbppptm) at (\ganglbxxxt, \ganglbyyym);
\coordinate (ganglbppptn) at (\ganglbxxxt, \ganglbyyyn);
\coordinate (ganglbpppto) at (\ganglbxxxt, \ganglbyyyo);
\coordinate (ganglbppptp) at (\ganglbxxxt, \ganglbyyyp);
\coordinate (ganglbppptq) at (\ganglbxxxt, \ganglbyyyq);
\coordinate (ganglbppptr) at (\ganglbxxxt, \ganglbyyyr);
\coordinate (ganglbpppts) at (\ganglbxxxt, \ganglbyyys);
\coordinate (ganglbppptt) at (\ganglbxxxt, \ganglbyyyt);
\coordinate (ganglbppptu) at (\ganglbxxxt, \ganglbyyyu);
\coordinate (ganglbppptv) at (\ganglbxxxt, \ganglbyyyv);
\coordinate (ganglbppptw) at (\ganglbxxxt, \ganglbyyyw);
\coordinate (ganglbppptx) at (\ganglbxxxt, \ganglbyyyx);
\coordinate (ganglbpppty) at (\ganglbxxxt, \ganglbyyyy);
\coordinate (ganglbppptz) at (\ganglbxxxt, \ganglbyyyz);
\coordinate (ganglbpppua) at (\ganglbxxxu, \ganglbyyya);
\coordinate (ganglbpppub) at (\ganglbxxxu, \ganglbyyyb);
\coordinate (ganglbpppuc) at (\ganglbxxxu, \ganglbyyyc);
\coordinate (ganglbpppud) at (\ganglbxxxu, \ganglbyyyd);
\coordinate (ganglbpppue) at (\ganglbxxxu, \ganglbyyye);
\coordinate (ganglbpppuf) at (\ganglbxxxu, \ganglbyyyf);
\coordinate (ganglbpppug) at (\ganglbxxxu, \ganglbyyyg);
\coordinate (ganglbpppuh) at (\ganglbxxxu, \ganglbyyyh);
\coordinate (ganglbpppui) at (\ganglbxxxu, \ganglbyyyi);
\coordinate (ganglbpppuj) at (\ganglbxxxu, \ganglbyyyj);
\coordinate (ganglbpppuk) at (\ganglbxxxu, \ganglbyyyk);
\coordinate (ganglbpppul) at (\ganglbxxxu, \ganglbyyyl);
\coordinate (ganglbpppum) at (\ganglbxxxu, \ganglbyyym);
\coordinate (ganglbpppun) at (\ganglbxxxu, \ganglbyyyn);
\coordinate (ganglbpppuo) at (\ganglbxxxu, \ganglbyyyo);
\coordinate (ganglbpppup) at (\ganglbxxxu, \ganglbyyyp);
\coordinate (ganglbpppuq) at (\ganglbxxxu, \ganglbyyyq);
\coordinate (ganglbpppur) at (\ganglbxxxu, \ganglbyyyr);
\coordinate (ganglbpppus) at (\ganglbxxxu, \ganglbyyys);
\coordinate (ganglbppput) at (\ganglbxxxu, \ganglbyyyt);
\coordinate (ganglbpppuu) at (\ganglbxxxu, \ganglbyyyu);
\coordinate (ganglbpppuv) at (\ganglbxxxu, \ganglbyyyv);
\coordinate (ganglbpppuw) at (\ganglbxxxu, \ganglbyyyw);
\coordinate (ganglbpppux) at (\ganglbxxxu, \ganglbyyyx);
\coordinate (ganglbpppuy) at (\ganglbxxxu, \ganglbyyyy);
\coordinate (ganglbpppuz) at (\ganglbxxxu, \ganglbyyyz);
\coordinate (ganglbpppva) at (\ganglbxxxv, \ganglbyyya);
\coordinate (ganglbpppvb) at (\ganglbxxxv, \ganglbyyyb);
\coordinate (ganglbpppvc) at (\ganglbxxxv, \ganglbyyyc);
\coordinate (ganglbpppvd) at (\ganglbxxxv, \ganglbyyyd);
\coordinate (ganglbpppve) at (\ganglbxxxv, \ganglbyyye);
\coordinate (ganglbpppvf) at (\ganglbxxxv, \ganglbyyyf);
\coordinate (ganglbpppvg) at (\ganglbxxxv, \ganglbyyyg);
\coordinate (ganglbpppvh) at (\ganglbxxxv, \ganglbyyyh);
\coordinate (ganglbpppvi) at (\ganglbxxxv, \ganglbyyyi);
\coordinate (ganglbpppvj) at (\ganglbxxxv, \ganglbyyyj);
\coordinate (ganglbpppvk) at (\ganglbxxxv, \ganglbyyyk);
\coordinate (ganglbpppvl) at (\ganglbxxxv, \ganglbyyyl);
\coordinate (ganglbpppvm) at (\ganglbxxxv, \ganglbyyym);
\coordinate (ganglbpppvn) at (\ganglbxxxv, \ganglbyyyn);
\coordinate (ganglbpppvo) at (\ganglbxxxv, \ganglbyyyo);
\coordinate (ganglbpppvp) at (\ganglbxxxv, \ganglbyyyp);
\coordinate (ganglbpppvq) at (\ganglbxxxv, \ganglbyyyq);
\coordinate (ganglbpppvr) at (\ganglbxxxv, \ganglbyyyr);
\coordinate (ganglbpppvs) at (\ganglbxxxv, \ganglbyyys);
\coordinate (ganglbpppvt) at (\ganglbxxxv, \ganglbyyyt);
\coordinate (ganglbpppvu) at (\ganglbxxxv, \ganglbyyyu);
\coordinate (ganglbpppvv) at (\ganglbxxxv, \ganglbyyyv);
\coordinate (ganglbpppvw) at (\ganglbxxxv, \ganglbyyyw);
\coordinate (ganglbpppvx) at (\ganglbxxxv, \ganglbyyyx);
\coordinate (ganglbpppvy) at (\ganglbxxxv, \ganglbyyyy);
\coordinate (ganglbpppvz) at (\ganglbxxxv, \ganglbyyyz);
\coordinate (ganglbpppwa) at (\ganglbxxxw, \ganglbyyya);
\coordinate (ganglbpppwb) at (\ganglbxxxw, \ganglbyyyb);
\coordinate (ganglbpppwc) at (\ganglbxxxw, \ganglbyyyc);
\coordinate (ganglbpppwd) at (\ganglbxxxw, \ganglbyyyd);
\coordinate (ganglbpppwe) at (\ganglbxxxw, \ganglbyyye);
\coordinate (ganglbpppwf) at (\ganglbxxxw, \ganglbyyyf);
\coordinate (ganglbpppwg) at (\ganglbxxxw, \ganglbyyyg);
\coordinate (ganglbpppwh) at (\ganglbxxxw, \ganglbyyyh);
\coordinate (ganglbpppwi) at (\ganglbxxxw, \ganglbyyyi);
\coordinate (ganglbpppwj) at (\ganglbxxxw, \ganglbyyyj);
\coordinate (ganglbpppwk) at (\ganglbxxxw, \ganglbyyyk);
\coordinate (ganglbpppwl) at (\ganglbxxxw, \ganglbyyyl);
\coordinate (ganglbpppwm) at (\ganglbxxxw, \ganglbyyym);
\coordinate (ganglbpppwn) at (\ganglbxxxw, \ganglbyyyn);
\coordinate (ganglbpppwo) at (\ganglbxxxw, \ganglbyyyo);
\coordinate (ganglbpppwp) at (\ganglbxxxw, \ganglbyyyp);
\coordinate (ganglbpppwq) at (\ganglbxxxw, \ganglbyyyq);
\coordinate (ganglbpppwr) at (\ganglbxxxw, \ganglbyyyr);
\coordinate (ganglbpppws) at (\ganglbxxxw, \ganglbyyys);
\coordinate (ganglbpppwt) at (\ganglbxxxw, \ganglbyyyt);
\coordinate (ganglbpppwu) at (\ganglbxxxw, \ganglbyyyu);
\coordinate (ganglbpppwv) at (\ganglbxxxw, \ganglbyyyv);
\coordinate (ganglbpppww) at (\ganglbxxxw, \ganglbyyyw);
\coordinate (ganglbpppwx) at (\ganglbxxxw, \ganglbyyyx);
\coordinate (ganglbpppwy) at (\ganglbxxxw, \ganglbyyyy);
\coordinate (ganglbpppwz) at (\ganglbxxxw, \ganglbyyyz);
\coordinate (ganglbpppxa) at (\ganglbxxxx, \ganglbyyya);
\coordinate (ganglbpppxb) at (\ganglbxxxx, \ganglbyyyb);
\coordinate (ganglbpppxc) at (\ganglbxxxx, \ganglbyyyc);
\coordinate (ganglbpppxd) at (\ganglbxxxx, \ganglbyyyd);
\coordinate (ganglbpppxe) at (\ganglbxxxx, \ganglbyyye);
\coordinate (ganglbpppxf) at (\ganglbxxxx, \ganglbyyyf);
\coordinate (ganglbpppxg) at (\ganglbxxxx, \ganglbyyyg);
\coordinate (ganglbpppxh) at (\ganglbxxxx, \ganglbyyyh);
\coordinate (ganglbpppxi) at (\ganglbxxxx, \ganglbyyyi);
\coordinate (ganglbpppxj) at (\ganglbxxxx, \ganglbyyyj);
\coordinate (ganglbpppxk) at (\ganglbxxxx, \ganglbyyyk);
\coordinate (ganglbpppxl) at (\ganglbxxxx, \ganglbyyyl);
\coordinate (ganglbpppxm) at (\ganglbxxxx, \ganglbyyym);
\coordinate (ganglbpppxn) at (\ganglbxxxx, \ganglbyyyn);
\coordinate (ganglbpppxo) at (\ganglbxxxx, \ganglbyyyo);
\coordinate (ganglbpppxp) at (\ganglbxxxx, \ganglbyyyp);
\coordinate (ganglbpppxq) at (\ganglbxxxx, \ganglbyyyq);
\coordinate (ganglbpppxr) at (\ganglbxxxx, \ganglbyyyr);
\coordinate (ganglbpppxs) at (\ganglbxxxx, \ganglbyyys);
\coordinate (ganglbpppxt) at (\ganglbxxxx, \ganglbyyyt);
\coordinate (ganglbpppxu) at (\ganglbxxxx, \ganglbyyyu);
\coordinate (ganglbpppxv) at (\ganglbxxxx, \ganglbyyyv);
\coordinate (ganglbpppxw) at (\ganglbxxxx, \ganglbyyyw);
\coordinate (ganglbpppxx) at (\ganglbxxxx, \ganglbyyyx);
\coordinate (ganglbpppxy) at (\ganglbxxxx, \ganglbyyyy);
\coordinate (ganglbpppxz) at (\ganglbxxxx, \ganglbyyyz);
\coordinate (ganglbpppya) at (\ganglbxxxy, \ganglbyyya);
\coordinate (ganglbpppyb) at (\ganglbxxxy, \ganglbyyyb);
\coordinate (ganglbpppyc) at (\ganglbxxxy, \ganglbyyyc);
\coordinate (ganglbpppyd) at (\ganglbxxxy, \ganglbyyyd);
\coordinate (ganglbpppye) at (\ganglbxxxy, \ganglbyyye);
\coordinate (ganglbpppyf) at (\ganglbxxxy, \ganglbyyyf);
\coordinate (ganglbpppyg) at (\ganglbxxxy, \ganglbyyyg);
\coordinate (ganglbpppyh) at (\ganglbxxxy, \ganglbyyyh);
\coordinate (ganglbpppyi) at (\ganglbxxxy, \ganglbyyyi);
\coordinate (ganglbpppyj) at (\ganglbxxxy, \ganglbyyyj);
\coordinate (ganglbpppyk) at (\ganglbxxxy, \ganglbyyyk);
\coordinate (ganglbpppyl) at (\ganglbxxxy, \ganglbyyyl);
\coordinate (ganglbpppym) at (\ganglbxxxy, \ganglbyyym);
\coordinate (ganglbpppyn) at (\ganglbxxxy, \ganglbyyyn);
\coordinate (ganglbpppyo) at (\ganglbxxxy, \ganglbyyyo);
\coordinate (ganglbpppyp) at (\ganglbxxxy, \ganglbyyyp);
\coordinate (ganglbpppyq) at (\ganglbxxxy, \ganglbyyyq);
\coordinate (ganglbpppyr) at (\ganglbxxxy, \ganglbyyyr);
\coordinate (ganglbpppys) at (\ganglbxxxy, \ganglbyyys);
\coordinate (ganglbpppyt) at (\ganglbxxxy, \ganglbyyyt);
\coordinate (ganglbpppyu) at (\ganglbxxxy, \ganglbyyyu);
\coordinate (ganglbpppyv) at (\ganglbxxxy, \ganglbyyyv);
\coordinate (ganglbpppyw) at (\ganglbxxxy, \ganglbyyyw);
\coordinate (ganglbpppyx) at (\ganglbxxxy, \ganglbyyyx);
\coordinate (ganglbpppyy) at (\ganglbxxxy, \ganglbyyyy);
\coordinate (ganglbpppyz) at (\ganglbxxxy, \ganglbyyyz);
\coordinate (ganglbpppza) at (\ganglbxxxz, \ganglbyyya);
\coordinate (ganglbpppzb) at (\ganglbxxxz, \ganglbyyyb);
\coordinate (ganglbpppzc) at (\ganglbxxxz, \ganglbyyyc);
\coordinate (ganglbpppzd) at (\ganglbxxxz, \ganglbyyyd);
\coordinate (ganglbpppze) at (\ganglbxxxz, \ganglbyyye);
\coordinate (ganglbpppzf) at (\ganglbxxxz, \ganglbyyyf);
\coordinate (ganglbpppzg) at (\ganglbxxxz, \ganglbyyyg);
\coordinate (ganglbpppzh) at (\ganglbxxxz, \ganglbyyyh);
\coordinate (ganglbpppzi) at (\ganglbxxxz, \ganglbyyyi);
\coordinate (ganglbpppzj) at (\ganglbxxxz, \ganglbyyyj);
\coordinate (ganglbpppzk) at (\ganglbxxxz, \ganglbyyyk);
\coordinate (ganglbpppzl) at (\ganglbxxxz, \ganglbyyyl);
\coordinate (ganglbpppzm) at (\ganglbxxxz, \ganglbyyym);
\coordinate (ganglbpppzn) at (\ganglbxxxz, \ganglbyyyn);
\coordinate (ganglbpppzo) at (\ganglbxxxz, \ganglbyyyo);
\coordinate (ganglbpppzp) at (\ganglbxxxz, \ganglbyyyp);
\coordinate (ganglbpppzq) at (\ganglbxxxz, \ganglbyyyq);
\coordinate (ganglbpppzr) at (\ganglbxxxz, \ganglbyyyr);
\coordinate (ganglbpppzs) at (\ganglbxxxz, \ganglbyyys);
\coordinate (ganglbpppzt) at (\ganglbxxxz, \ganglbyyyt);
\coordinate (ganglbpppzu) at (\ganglbxxxz, \ganglbyyyu);
\coordinate (ganglbpppzv) at (\ganglbxxxz, \ganglbyyyv);
\coordinate (ganglbpppzw) at (\ganglbxxxz, \ganglbyyyw);
\coordinate (ganglbpppzx) at (\ganglbxxxz, \ganglbyyyx);
\coordinate (ganglbpppzy) at (\ganglbxxxz, \ganglbyyyy);
\coordinate (ganglbpppzz) at (\ganglbxxxz, \ganglbyyyz);

%\gangprintcoordinateat{(0,0)}{The last coordinate values: }{($(ganglbpppzz)$)}; 



% Draw related part of the coordinate system with dashed helplines with letters as background, which would help to determine all coordinates. 
\coordinatebackgroundxy{gangliu} {f}{g}{v} {f}{g}{q};

%%%%%% The next line is for fig. 3.
\coordinatebackgroundxy{gangla}{a}{b}{h} {a}{b}{h};

%%%%%% The next line is for fig. 4.
\coordinatebackgroundxy{ganglb}{a}{b}{f} {a}{b}{g};

\draw [white] (gangliupppdp) -- (gangliupppgp);


% Draw the Opamp at the coordinate (gangliupppli) and name it as "myopamp".
\draw (gangliupppli) 
      node [op amp] (myopamp) {} ; 

% Retrieve the x- and y-components of the coordinates of the "+", "-", and "out" pins of myopamp, supposing we have no idea about them beforehand. 
\getxyingivenunit{cm}{(myopamp.+)}
                 {\myopamppx}{\myopamppy};
\getxyingivenunit{cm}{(myopamp.-)}
                 {\myopampnx}{\myopampny};
\getxyingivenunit{cm}{(myopamp.out)}
                 {\myopampox}{\myopampoy};

\draw (myopamp.-) -| (gangliupppjj) 
      to [R, l_=$R_F \text{=} 300K \Omega$,
                label/align=rotate] 
      (gangliupppjm) 
%%%%%% removed for fig. 2: -| (gangliupppni)
      ;

\draw [-o] (myopamp.out) 
      to[short, xshift=1mm] 
      (\gangliuxxxq, \myopampoy) node [anchor=north, yshift=-1mm] {$V_0$};

\draw (\gangliuxxxj, \myopampny) -- 
      (\gangliuxxxi, \myopampny) 
      to [R, l_=$R \text{=} 100 K\Omega$]  (\gangliuxxxg, \myopampny) -- (gangliupppgi) node [ground]{};;

\draw [-o] (myopamp.+) 
      to[short, xshift=-1mm] 
      (\gangliuxxxj, \myopamppy) node [anchor=north, yshift=-1mm] {$V_i$};
      
      
      

%%%%%% The rest are added for fig. 2.      
%%%%%% The rest are added for fig. 2.      

\draw (gangliuppppi) |- (gangliupppno) --
      (gangliupppnn);

\draw (gangliupppnk) node [ground]{} --
      (gangliupppnl) 
      to[american potentiometer,n=mypot, 
         l_=$R_P \text{=} 56 K \Omega$, label/align=rotate] 
      (gangliupppnn);

%%%%%% Remove the following line for Fig. 3. 
%\draw (mypot.wiper) 
%      node [red, anchor=south east] {$V_A$} -| 
%      (gangliupppjm);


%%%%%% Add the rest lines for Fig. 3. 
\getxyingivenunit{cm}{(mypot.wiper)}
                 {\mypotwiperx}{\mypotwipery};

\draw (mypot.wiper) 
      node [red, anchor=south east] {$V_A$} -- (\ganglaxxxe, \mypotwipery);

\draw (\ganglaxxxe, \mypotwipery)  
      to[variable american resistor, n=resistora, 
         l=$R_D \text{=} 560K \Omega$] 
      (\ganglaxxxc, \mypotwipery) -- 
      (\ganglaxxxb, \mypotwipery) 
      node [red, anchor=south] {$V_B$};

\draw (\ganglaxxxb, \mypotwipery) -- 
      (ganglapppbd)
      to [C, l=$C_D \text{=} 100 \mu F$] 
      (ganglapppbb) node [ground] {};

%%%%%% Remove the following line for fig. 4.
%\draw (gangliupppjm) -- (\ganglaxxxb, \mypotwipery);


%%%%%% Add the rest lines for fig. 4.

\draw (\ganglbxxxc, \mypotwipery)
      to [C, l_=$C_I \text{=} 200 \mu F$] 
      (\ganglbxxxf, \mypotwipery) --
      (\ganglaxxxb, \mypotwipery);

\draw (\ganglbxxxc, \mypotwipery) -- 
      (ganglbpppcd)
      to[variable american resistor, n=resistorb, 
         l=$R_I \text{=} 4.7M \Omega$] 
      (ganglbpppcb) node [ground] {};
      

\draw (\ganglbxxxc, \mypotwipery) 
        node [red, anchor=south] {$V_C$} --
      (gangliupppjm);


\end{circuitikz}



\end{document}
