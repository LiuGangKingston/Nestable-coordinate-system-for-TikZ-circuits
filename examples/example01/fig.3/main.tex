\documentclass[tikz,border=5mm]{standalone}
\usepackage[siunitx]{circuitikz}
\usetikzlibrary{shapes,arrows,positioning}
%   This is an accessory  file for 
%   https://github.com/LiuGangKingston/Nestable-coordinate-system-for-Tikz-circuits.git
%            Version 1.0
%   free for non-commercial use.
%   Please send us emails for any problems/suggestions/comments.
%   Please be advised that none of us accept any responsibility
%   for any consequences arising out of the usage of this
%   software, especially for damage.
%   For usage, please refer to the README file and the following lines.
%   This code was written by
%        Gang Liu (gl.cell@outlook)
%                 (http://orcid.org/0000-0003-1575-9290)
%          and
%        Shiwei Huang (huang937@gmail.com)
%   Copyright (c) 2021
%
%
%  The following command is to get the x-component and y-component 
%  of a coordinate. The command is
%  \getxyofcoordinate{the coordinate}{x-component}{y-component};
\newcommand{\getxyofcoordinate}[3]{%
\coordinate (tempcoord) at ($#1$);
\path (tempcoord) node {};
\pgfgetlastxy{\tempx}{\tempy};
\pgfmathsetmacro{#2}{\tempx}
\pgfmathsetmacro{#3}{\tempy}
}


%  The following command is the same as above but for given unit.
%  The command is
%  \getxyingivenunit{the unit like cm}{the coordinate}{x-component}{y-component};
\newcommand{\getxyingivenunit}[4]{%
\coordinate (tempcoord) at (1#1,1#1);
\path (tempcoord) node {};
\pgfgetlastxy{\tempxunit}{\tempyunit};
\coordinate (tempcoord) at ($#2$);
\path (tempcoord) node {};
\pgfgetlastxy{\tempx}{\tempy};
\pgfmathsetmacro{#3}{\tempx/\tempxunit}
\pgfmathsetmacro{#4}{\tempy/\tempyunit}
}


%  The following command is to print the value of a coordinate with some words at the first coordinate postion 
%  The command is
%  \printcoordinateat{the first coordinate}{the words}{the coordinate};
\newcommand{\printcoordinateat}[3]{%
\getxyingivenunit{cm}{#3}{\tempxx}{\tempyy}
\node at #1 {#2 ($\tempxx$, $\tempyy$).};
}


%  The following command is to print a keyworded coordinate system as a background.
%  The command is
%  \coordinatebackground{the KEYWORD}
%                                            {the first letter in both x and y directions}
%                                       {the second letter in both x and y directions}
%                                             {the last letter in both x and y directions};
\newcommand{\coordinatebackground}[4]{
\pgfmathsetmacro{\colourpercent}{30}
\foreach \i in {#2,#3,...,#4} 
{\node [black!\colourpercent] at (#1ppp\i\i) {\i};}
\foreach \i in {#2,#4} 
{\node [white] at (#1ppp\i\i) {\i};}
\coordinatebackgroundxy{#1}{#2}{#3}{#4}{#2}{#3}{#4};
}


%  The following command is to print a keyworded coordinate system as a background.
%  The command is
%  \coordinatebackgroundxy{the KEYWORD}
%                                                {the first letter in the x direction}
%                                           {the second letter in the x direction}
%                                                 {the last letter in the x direction}
%                                                {the first letter in the y direction}
%                                           {the second letter in the y direction}
%                                                 {the last letter in the y direction};
\newcommand{\coordinatebackgroundxy}[7]{
\pgfmathsetmacro{\bordercolourpercent}{60}
\pgfmathsetmacro{\colourpercent}{30}

\foreach \i in {#2,#3,...,#4} 
\foreach \j in {#5} 
\foreach \k in {#7} 
{\draw [dashed,black!\colourpercent] (#1ppp\i\j) -- (#1ppp\i\k);}

\foreach \i in {#5,#6,...,#7} 
\foreach \j in {#2} 
\foreach \k in {#4} 
{\draw [dashed,black!\colourpercent] (#1ppp\j\i) -- (#1ppp\k\i);}

\foreach \i in {#2,#4} 
\foreach \j in {#5} 
\foreach \k in {#7} 
{\draw [dashed,black!\bordercolourpercent] (#1ppp\i\j) -- (#1ppp\i\k);}

\foreach \i in {#5,#7} 
\foreach \j in {#2} 
\foreach \k in {#4} 
{\draw [dashed,black!\bordercolourpercent] (#1ppp\j\i) -- (#1ppp\k\i);}

\foreach \i in {#2,#3,...,#4} 
\foreach \j in {#5} 
\foreach \k in {#7} 
{
\node [black!\bordercolourpercent] at ($(#1ppp\i\j) + (0,-.2)$) {\i};
\node [black!\bordercolourpercent] at ($(#1ppp\i\k) + (0,.2)$) {\i};
}

\foreach \i in {#5,#6,...,#7} 
\foreach \j in {#2} 
\foreach \k in {#4} 
{
\node [black!\bordercolourpercent] at ($(#1ppp\k\i) + (.2,0)$) {\i};
\node [black!\bordercolourpercent] at ($(#1ppp\j\i) + (-.2,0)$) {\i};
}

}










\begin{document}

\ctikzset{
/tikz/circuitikz/bipoles/length=1cm
}



 
 
\begin{circuitikz} [scale=0.8]
 
%%%%%% The next line is for fig. 3.
https://github.com/LiuGangKingston/Nestable-coordinate-system-for-Tikz-circuits.git
https://github.com/LiuGangKingston/Nestable-coordinate-system-for-Tikz-circuits.git


% https://github.com/LiuGangKingston/Nestable-coordinate-system-for-Tikz-circuits.git
% https://github.com/LiuGangKingston/Nestable-coordinate-system-for-Tikz-circuits.git


\pgfmathsetmacro{\totalgangliuxxx}{26}
\pgfmathsetmacro{\totalgangliuyyy}{26}
\pgfmathsetmacro{\gangliuxxxspacing}{1}
\pgfmathsetmacro{\gangliuyyyspacing}{1}
\pgfmathsetmacro{\gangliuxxxa}{-8}
\pgfmathsetmacro{\gangliuyyya}{-8}

\pgfmathsetmacro{\gangliuxxxb}{\gangliuxxxa + \gangliuxxxspacing + 0.0 }
\pgfmathsetmacro{\gangliuxxxc}{\gangliuxxxb + \gangliuxxxspacing + 0.0 }
\pgfmathsetmacro{\gangliuxxxd}{\gangliuxxxc + \gangliuxxxspacing + 0.0 }
\pgfmathsetmacro{\gangliuxxxe}{\gangliuxxxd + \gangliuxxxspacing + 0.0 }
\pgfmathsetmacro{\gangliuxxxf}{\gangliuxxxe + \gangliuxxxspacing + 0.0 }
\pgfmathsetmacro{\gangliuxxxg}{\gangliuxxxf + \gangliuxxxspacing + 0.0 }
\pgfmathsetmacro{\gangliuxxxh}{\gangliuxxxg + \gangliuxxxspacing + 0.0 }
\pgfmathsetmacro{\gangliuxxxi}{\gangliuxxxh + \gangliuxxxspacing + 0.0 }
\pgfmathsetmacro{\gangliuxxxj}{\gangliuxxxi + \gangliuxxxspacing + 0.0 }
\pgfmathsetmacro{\gangliuxxxk}{\gangliuxxxj + \gangliuxxxspacing + 0.0 }
\pgfmathsetmacro{\gangliuxxxl}{\gangliuxxxk + \gangliuxxxspacing + 0.0 }
\pgfmathsetmacro{\gangliuxxxm}{\gangliuxxxl + \gangliuxxxspacing + 0.0 }
\pgfmathsetmacro{\gangliuxxxn}{\gangliuxxxm + \gangliuxxxspacing + 0.0 }
\pgfmathsetmacro{\gangliuxxxo}{\gangliuxxxn + \gangliuxxxspacing + 0.0 }
\pgfmathsetmacro{\gangliuxxxp}{\gangliuxxxo + \gangliuxxxspacing + 0.0 }
\pgfmathsetmacro{\gangliuxxxq}{\gangliuxxxp + \gangliuxxxspacing + 0.0 }
\pgfmathsetmacro{\gangliuxxxr}{\gangliuxxxq + \gangliuxxxspacing + 0.0 }
\pgfmathsetmacro{\gangliuxxxs}{\gangliuxxxr + \gangliuxxxspacing + 0.0 }
\pgfmathsetmacro{\gangliuxxxt}{\gangliuxxxs + \gangliuxxxspacing + 0.0 }
\pgfmathsetmacro{\gangliuxxxu}{\gangliuxxxt + \gangliuxxxspacing + 0.0 }
\pgfmathsetmacro{\gangliuxxxv}{\gangliuxxxu + \gangliuxxxspacing + 0.0 }
\pgfmathsetmacro{\gangliuxxxw}{\gangliuxxxv + \gangliuxxxspacing + 0.0 }
\pgfmathsetmacro{\gangliuxxxx}{\gangliuxxxw + \gangliuxxxspacing + 0.0 }
\pgfmathsetmacro{\gangliuxxxy}{\gangliuxxxx + \gangliuxxxspacing + 0.0 }
\pgfmathsetmacro{\gangliuxxxz}{\gangliuxxxy + \gangliuxxxspacing + 0.0 }

\pgfmathsetmacro{\gangliuyyyb}{\gangliuyyya + \gangliuyyyspacing + 0.0 }
\pgfmathsetmacro{\gangliuyyyc}{\gangliuyyyb + \gangliuyyyspacing + 0.0 }
\pgfmathsetmacro{\gangliuyyyd}{\gangliuyyyc + \gangliuyyyspacing + 0.0 }
\pgfmathsetmacro{\gangliuyyye}{\gangliuyyyd + \gangliuyyyspacing + 0.0 }
\pgfmathsetmacro{\gangliuyyyf}{\gangliuyyye + \gangliuyyyspacing + 0.0 }
\pgfmathsetmacro{\gangliuyyyg}{\gangliuyyyf + \gangliuyyyspacing + 0.0 }
\pgfmathsetmacro{\gangliuyyyh}{\gangliuyyyg + \gangliuyyyspacing + 0.0 }
\pgfmathsetmacro{\gangliuyyyi}{\gangliuyyyh + \gangliuyyyspacing + 0.0 }
\pgfmathsetmacro{\gangliuyyyj}{\gangliuyyyi + \gangliuyyyspacing + 0.0 }
\pgfmathsetmacro{\gangliuyyyk}{\gangliuyyyj + \gangliuyyyspacing + 0.0 }
\pgfmathsetmacro{\gangliuyyyl}{\gangliuyyyk + \gangliuyyyspacing + 0.0 }
\pgfmathsetmacro{\gangliuyyym}{\gangliuyyyl + \gangliuyyyspacing + 0.0 }
\pgfmathsetmacro{\gangliuyyyn}{\gangliuyyym + \gangliuyyyspacing + 0.0 }
\pgfmathsetmacro{\gangliuyyyo}{\gangliuyyyn + \gangliuyyyspacing + 0.0 }
\pgfmathsetmacro{\gangliuyyyp}{\gangliuyyyo + \gangliuyyyspacing + 0.0 }
\pgfmathsetmacro{\gangliuyyyq}{\gangliuyyyp + \gangliuyyyspacing + 0.0 }
\pgfmathsetmacro{\gangliuyyyr}{\gangliuyyyq + \gangliuyyyspacing + 0.0 }
\pgfmathsetmacro{\gangliuyyys}{\gangliuyyyr + \gangliuyyyspacing + 0.0 }
\pgfmathsetmacro{\gangliuyyyt}{\gangliuyyys + \gangliuyyyspacing + 0.0 }
\pgfmathsetmacro{\gangliuyyyu}{\gangliuyyyt + \gangliuyyyspacing + 0.0 }
\pgfmathsetmacro{\gangliuyyyv}{\gangliuyyyu + \gangliuyyyspacing + 0.0 }
\pgfmathsetmacro{\gangliuyyyw}{\gangliuyyyv + \gangliuyyyspacing + 0.0 }
\pgfmathsetmacro{\gangliuyyyx}{\gangliuyyyw + \gangliuyyyspacing + 0.0 }
\pgfmathsetmacro{\gangliuyyyy}{\gangliuyyyx + \gangliuyyyspacing + 0.0 }
\pgfmathsetmacro{\gangliuyyyz}{\gangliuyyyy + \gangliuyyyspacing + 0.0 }

\coordinate (gangliupppaa) at (\gangliuxxxa, \gangliuyyya);
\coordinate (gangliupppab) at (\gangliuxxxa, \gangliuyyyb);
\coordinate (gangliupppac) at (\gangliuxxxa, \gangliuyyyc);
\coordinate (gangliupppad) at (\gangliuxxxa, \gangliuyyyd);
\coordinate (gangliupppae) at (\gangliuxxxa, \gangliuyyye);
\coordinate (gangliupppaf) at (\gangliuxxxa, \gangliuyyyf);
\coordinate (gangliupppag) at (\gangliuxxxa, \gangliuyyyg);
\coordinate (gangliupppah) at (\gangliuxxxa, \gangliuyyyh);
\coordinate (gangliupppai) at (\gangliuxxxa, \gangliuyyyi);
\coordinate (gangliupppaj) at (\gangliuxxxa, \gangliuyyyj);
\coordinate (gangliupppak) at (\gangliuxxxa, \gangliuyyyk);
\coordinate (gangliupppal) at (\gangliuxxxa, \gangliuyyyl);
\coordinate (gangliupppam) at (\gangliuxxxa, \gangliuyyym);
\coordinate (gangliupppan) at (\gangliuxxxa, \gangliuyyyn);
\coordinate (gangliupppao) at (\gangliuxxxa, \gangliuyyyo);
\coordinate (gangliupppap) at (\gangliuxxxa, \gangliuyyyp);
\coordinate (gangliupppaq) at (\gangliuxxxa, \gangliuyyyq);
\coordinate (gangliupppar) at (\gangliuxxxa, \gangliuyyyr);
\coordinate (gangliupppas) at (\gangliuxxxa, \gangliuyyys);
\coordinate (gangliupppat) at (\gangliuxxxa, \gangliuyyyt);
\coordinate (gangliupppau) at (\gangliuxxxa, \gangliuyyyu);
\coordinate (gangliupppav) at (\gangliuxxxa, \gangliuyyyv);
\coordinate (gangliupppaw) at (\gangliuxxxa, \gangliuyyyw);
\coordinate (gangliupppax) at (\gangliuxxxa, \gangliuyyyx);
\coordinate (gangliupppay) at (\gangliuxxxa, \gangliuyyyy);
\coordinate (gangliupppaz) at (\gangliuxxxa, \gangliuyyyz);
\coordinate (gangliupppba) at (\gangliuxxxb, \gangliuyyya);
\coordinate (gangliupppbb) at (\gangliuxxxb, \gangliuyyyb);
\coordinate (gangliupppbc) at (\gangliuxxxb, \gangliuyyyc);
\coordinate (gangliupppbd) at (\gangliuxxxb, \gangliuyyyd);
\coordinate (gangliupppbe) at (\gangliuxxxb, \gangliuyyye);
\coordinate (gangliupppbf) at (\gangliuxxxb, \gangliuyyyf);
\coordinate (gangliupppbg) at (\gangliuxxxb, \gangliuyyyg);
\coordinate (gangliupppbh) at (\gangliuxxxb, \gangliuyyyh);
\coordinate (gangliupppbi) at (\gangliuxxxb, \gangliuyyyi);
\coordinate (gangliupppbj) at (\gangliuxxxb, \gangliuyyyj);
\coordinate (gangliupppbk) at (\gangliuxxxb, \gangliuyyyk);
\coordinate (gangliupppbl) at (\gangliuxxxb, \gangliuyyyl);
\coordinate (gangliupppbm) at (\gangliuxxxb, \gangliuyyym);
\coordinate (gangliupppbn) at (\gangliuxxxb, \gangliuyyyn);
\coordinate (gangliupppbo) at (\gangliuxxxb, \gangliuyyyo);
\coordinate (gangliupppbp) at (\gangliuxxxb, \gangliuyyyp);
\coordinate (gangliupppbq) at (\gangliuxxxb, \gangliuyyyq);
\coordinate (gangliupppbr) at (\gangliuxxxb, \gangliuyyyr);
\coordinate (gangliupppbs) at (\gangliuxxxb, \gangliuyyys);
\coordinate (gangliupppbt) at (\gangliuxxxb, \gangliuyyyt);
\coordinate (gangliupppbu) at (\gangliuxxxb, \gangliuyyyu);
\coordinate (gangliupppbv) at (\gangliuxxxb, \gangliuyyyv);
\coordinate (gangliupppbw) at (\gangliuxxxb, \gangliuyyyw);
\coordinate (gangliupppbx) at (\gangliuxxxb, \gangliuyyyx);
\coordinate (gangliupppby) at (\gangliuxxxb, \gangliuyyyy);
\coordinate (gangliupppbz) at (\gangliuxxxb, \gangliuyyyz);
\coordinate (gangliupppca) at (\gangliuxxxc, \gangliuyyya);
\coordinate (gangliupppcb) at (\gangliuxxxc, \gangliuyyyb);
\coordinate (gangliupppcc) at (\gangliuxxxc, \gangliuyyyc);
\coordinate (gangliupppcd) at (\gangliuxxxc, \gangliuyyyd);
\coordinate (gangliupppce) at (\gangliuxxxc, \gangliuyyye);
\coordinate (gangliupppcf) at (\gangliuxxxc, \gangliuyyyf);
\coordinate (gangliupppcg) at (\gangliuxxxc, \gangliuyyyg);
\coordinate (gangliupppch) at (\gangliuxxxc, \gangliuyyyh);
\coordinate (gangliupppci) at (\gangliuxxxc, \gangliuyyyi);
\coordinate (gangliupppcj) at (\gangliuxxxc, \gangliuyyyj);
\coordinate (gangliupppck) at (\gangliuxxxc, \gangliuyyyk);
\coordinate (gangliupppcl) at (\gangliuxxxc, \gangliuyyyl);
\coordinate (gangliupppcm) at (\gangliuxxxc, \gangliuyyym);
\coordinate (gangliupppcn) at (\gangliuxxxc, \gangliuyyyn);
\coordinate (gangliupppco) at (\gangliuxxxc, \gangliuyyyo);
\coordinate (gangliupppcp) at (\gangliuxxxc, \gangliuyyyp);
\coordinate (gangliupppcq) at (\gangliuxxxc, \gangliuyyyq);
\coordinate (gangliupppcr) at (\gangliuxxxc, \gangliuyyyr);
\coordinate (gangliupppcs) at (\gangliuxxxc, \gangliuyyys);
\coordinate (gangliupppct) at (\gangliuxxxc, \gangliuyyyt);
\coordinate (gangliupppcu) at (\gangliuxxxc, \gangliuyyyu);
\coordinate (gangliupppcv) at (\gangliuxxxc, \gangliuyyyv);
\coordinate (gangliupppcw) at (\gangliuxxxc, \gangliuyyyw);
\coordinate (gangliupppcx) at (\gangliuxxxc, \gangliuyyyx);
\coordinate (gangliupppcy) at (\gangliuxxxc, \gangliuyyyy);
\coordinate (gangliupppcz) at (\gangliuxxxc, \gangliuyyyz);
\coordinate (gangliupppda) at (\gangliuxxxd, \gangliuyyya);
\coordinate (gangliupppdb) at (\gangliuxxxd, \gangliuyyyb);
\coordinate (gangliupppdc) at (\gangliuxxxd, \gangliuyyyc);
\coordinate (gangliupppdd) at (\gangliuxxxd, \gangliuyyyd);
\coordinate (gangliupppde) at (\gangliuxxxd, \gangliuyyye);
\coordinate (gangliupppdf) at (\gangliuxxxd, \gangliuyyyf);
\coordinate (gangliupppdg) at (\gangliuxxxd, \gangliuyyyg);
\coordinate (gangliupppdh) at (\gangliuxxxd, \gangliuyyyh);
\coordinate (gangliupppdi) at (\gangliuxxxd, \gangliuyyyi);
\coordinate (gangliupppdj) at (\gangliuxxxd, \gangliuyyyj);
\coordinate (gangliupppdk) at (\gangliuxxxd, \gangliuyyyk);
\coordinate (gangliupppdl) at (\gangliuxxxd, \gangliuyyyl);
\coordinate (gangliupppdm) at (\gangliuxxxd, \gangliuyyym);
\coordinate (gangliupppdn) at (\gangliuxxxd, \gangliuyyyn);
\coordinate (gangliupppdo) at (\gangliuxxxd, \gangliuyyyo);
\coordinate (gangliupppdp) at (\gangliuxxxd, \gangliuyyyp);
\coordinate (gangliupppdq) at (\gangliuxxxd, \gangliuyyyq);
\coordinate (gangliupppdr) at (\gangliuxxxd, \gangliuyyyr);
\coordinate (gangliupppds) at (\gangliuxxxd, \gangliuyyys);
\coordinate (gangliupppdt) at (\gangliuxxxd, \gangliuyyyt);
\coordinate (gangliupppdu) at (\gangliuxxxd, \gangliuyyyu);
\coordinate (gangliupppdv) at (\gangliuxxxd, \gangliuyyyv);
\coordinate (gangliupppdw) at (\gangliuxxxd, \gangliuyyyw);
\coordinate (gangliupppdx) at (\gangliuxxxd, \gangliuyyyx);
\coordinate (gangliupppdy) at (\gangliuxxxd, \gangliuyyyy);
\coordinate (gangliupppdz) at (\gangliuxxxd, \gangliuyyyz);
\coordinate (gangliupppea) at (\gangliuxxxe, \gangliuyyya);
\coordinate (gangliupppeb) at (\gangliuxxxe, \gangliuyyyb);
\coordinate (gangliupppec) at (\gangliuxxxe, \gangliuyyyc);
\coordinate (gangliuppped) at (\gangliuxxxe, \gangliuyyyd);
\coordinate (gangliupppee) at (\gangliuxxxe, \gangliuyyye);
\coordinate (gangliupppef) at (\gangliuxxxe, \gangliuyyyf);
\coordinate (gangliupppeg) at (\gangliuxxxe, \gangliuyyyg);
\coordinate (gangliupppeh) at (\gangliuxxxe, \gangliuyyyh);
\coordinate (gangliupppei) at (\gangliuxxxe, \gangliuyyyi);
\coordinate (gangliupppej) at (\gangliuxxxe, \gangliuyyyj);
\coordinate (gangliupppek) at (\gangliuxxxe, \gangliuyyyk);
\coordinate (gangliupppel) at (\gangliuxxxe, \gangliuyyyl);
\coordinate (gangliupppem) at (\gangliuxxxe, \gangliuyyym);
\coordinate (gangliupppen) at (\gangliuxxxe, \gangliuyyyn);
\coordinate (gangliupppeo) at (\gangliuxxxe, \gangliuyyyo);
\coordinate (gangliupppep) at (\gangliuxxxe, \gangliuyyyp);
\coordinate (gangliupppeq) at (\gangliuxxxe, \gangliuyyyq);
\coordinate (gangliuppper) at (\gangliuxxxe, \gangliuyyyr);
\coordinate (gangliupppes) at (\gangliuxxxe, \gangliuyyys);
\coordinate (gangliupppet) at (\gangliuxxxe, \gangliuyyyt);
\coordinate (gangliupppeu) at (\gangliuxxxe, \gangliuyyyu);
\coordinate (gangliupppev) at (\gangliuxxxe, \gangliuyyyv);
\coordinate (gangliupppew) at (\gangliuxxxe, \gangliuyyyw);
\coordinate (gangliupppex) at (\gangliuxxxe, \gangliuyyyx);
\coordinate (gangliupppey) at (\gangliuxxxe, \gangliuyyyy);
\coordinate (gangliupppez) at (\gangliuxxxe, \gangliuyyyz);
\coordinate (gangliupppfa) at (\gangliuxxxf, \gangliuyyya);
\coordinate (gangliupppfb) at (\gangliuxxxf, \gangliuyyyb);
\coordinate (gangliupppfc) at (\gangliuxxxf, \gangliuyyyc);
\coordinate (gangliupppfd) at (\gangliuxxxf, \gangliuyyyd);
\coordinate (gangliupppfe) at (\gangliuxxxf, \gangliuyyye);
\coordinate (gangliupppff) at (\gangliuxxxf, \gangliuyyyf);
\coordinate (gangliupppfg) at (\gangliuxxxf, \gangliuyyyg);
\coordinate (gangliupppfh) at (\gangliuxxxf, \gangliuyyyh);
\coordinate (gangliupppfi) at (\gangliuxxxf, \gangliuyyyi);
\coordinate (gangliupppfj) at (\gangliuxxxf, \gangliuyyyj);
\coordinate (gangliupppfk) at (\gangliuxxxf, \gangliuyyyk);
\coordinate (gangliupppfl) at (\gangliuxxxf, \gangliuyyyl);
\coordinate (gangliupppfm) at (\gangliuxxxf, \gangliuyyym);
\coordinate (gangliupppfn) at (\gangliuxxxf, \gangliuyyyn);
\coordinate (gangliupppfo) at (\gangliuxxxf, \gangliuyyyo);
\coordinate (gangliupppfp) at (\gangliuxxxf, \gangliuyyyp);
\coordinate (gangliupppfq) at (\gangliuxxxf, \gangliuyyyq);
\coordinate (gangliupppfr) at (\gangliuxxxf, \gangliuyyyr);
\coordinate (gangliupppfs) at (\gangliuxxxf, \gangliuyyys);
\coordinate (gangliupppft) at (\gangliuxxxf, \gangliuyyyt);
\coordinate (gangliupppfu) at (\gangliuxxxf, \gangliuyyyu);
\coordinate (gangliupppfv) at (\gangliuxxxf, \gangliuyyyv);
\coordinate (gangliupppfw) at (\gangliuxxxf, \gangliuyyyw);
\coordinate (gangliupppfx) at (\gangliuxxxf, \gangliuyyyx);
\coordinate (gangliupppfy) at (\gangliuxxxf, \gangliuyyyy);
\coordinate (gangliupppfz) at (\gangliuxxxf, \gangliuyyyz);
\coordinate (gangliupppga) at (\gangliuxxxg, \gangliuyyya);
\coordinate (gangliupppgb) at (\gangliuxxxg, \gangliuyyyb);
\coordinate (gangliupppgc) at (\gangliuxxxg, \gangliuyyyc);
\coordinate (gangliupppgd) at (\gangliuxxxg, \gangliuyyyd);
\coordinate (gangliupppge) at (\gangliuxxxg, \gangliuyyye);
\coordinate (gangliupppgf) at (\gangliuxxxg, \gangliuyyyf);
\coordinate (gangliupppgg) at (\gangliuxxxg, \gangliuyyyg);
\coordinate (gangliupppgh) at (\gangliuxxxg, \gangliuyyyh);
\coordinate (gangliupppgi) at (\gangliuxxxg, \gangliuyyyi);
\coordinate (gangliupppgj) at (\gangliuxxxg, \gangliuyyyj);
\coordinate (gangliupppgk) at (\gangliuxxxg, \gangliuyyyk);
\coordinate (gangliupppgl) at (\gangliuxxxg, \gangliuyyyl);
\coordinate (gangliupppgm) at (\gangliuxxxg, \gangliuyyym);
\coordinate (gangliupppgn) at (\gangliuxxxg, \gangliuyyyn);
\coordinate (gangliupppgo) at (\gangliuxxxg, \gangliuyyyo);
\coordinate (gangliupppgp) at (\gangliuxxxg, \gangliuyyyp);
\coordinate (gangliupppgq) at (\gangliuxxxg, \gangliuyyyq);
\coordinate (gangliupppgr) at (\gangliuxxxg, \gangliuyyyr);
\coordinate (gangliupppgs) at (\gangliuxxxg, \gangliuyyys);
\coordinate (gangliupppgt) at (\gangliuxxxg, \gangliuyyyt);
\coordinate (gangliupppgu) at (\gangliuxxxg, \gangliuyyyu);
\coordinate (gangliupppgv) at (\gangliuxxxg, \gangliuyyyv);
\coordinate (gangliupppgw) at (\gangliuxxxg, \gangliuyyyw);
\coordinate (gangliupppgx) at (\gangliuxxxg, \gangliuyyyx);
\coordinate (gangliupppgy) at (\gangliuxxxg, \gangliuyyyy);
\coordinate (gangliupppgz) at (\gangliuxxxg, \gangliuyyyz);
\coordinate (gangliupppha) at (\gangliuxxxh, \gangliuyyya);
\coordinate (gangliuppphb) at (\gangliuxxxh, \gangliuyyyb);
\coordinate (gangliuppphc) at (\gangliuxxxh, \gangliuyyyc);
\coordinate (gangliuppphd) at (\gangliuxxxh, \gangliuyyyd);
\coordinate (gangliuppphe) at (\gangliuxxxh, \gangliuyyye);
\coordinate (gangliuppphf) at (\gangliuxxxh, \gangliuyyyf);
\coordinate (gangliuppphg) at (\gangliuxxxh, \gangliuyyyg);
\coordinate (gangliuppphh) at (\gangliuxxxh, \gangliuyyyh);
\coordinate (gangliuppphi) at (\gangliuxxxh, \gangliuyyyi);
\coordinate (gangliuppphj) at (\gangliuxxxh, \gangliuyyyj);
\coordinate (gangliuppphk) at (\gangliuxxxh, \gangliuyyyk);
\coordinate (gangliuppphl) at (\gangliuxxxh, \gangliuyyyl);
\coordinate (gangliuppphm) at (\gangliuxxxh, \gangliuyyym);
\coordinate (gangliuppphn) at (\gangliuxxxh, \gangliuyyyn);
\coordinate (gangliupppho) at (\gangliuxxxh, \gangliuyyyo);
\coordinate (gangliuppphp) at (\gangliuxxxh, \gangliuyyyp);
\coordinate (gangliuppphq) at (\gangliuxxxh, \gangliuyyyq);
\coordinate (gangliuppphr) at (\gangliuxxxh, \gangliuyyyr);
\coordinate (gangliuppphs) at (\gangliuxxxh, \gangliuyyys);
\coordinate (gangliupppht) at (\gangliuxxxh, \gangliuyyyt);
\coordinate (gangliuppphu) at (\gangliuxxxh, \gangliuyyyu);
\coordinate (gangliuppphv) at (\gangliuxxxh, \gangliuyyyv);
\coordinate (gangliuppphw) at (\gangliuxxxh, \gangliuyyyw);
\coordinate (gangliuppphx) at (\gangliuxxxh, \gangliuyyyx);
\coordinate (gangliuppphy) at (\gangliuxxxh, \gangliuyyyy);
\coordinate (gangliuppphz) at (\gangliuxxxh, \gangliuyyyz);
\coordinate (gangliupppia) at (\gangliuxxxi, \gangliuyyya);
\coordinate (gangliupppib) at (\gangliuxxxi, \gangliuyyyb);
\coordinate (gangliupppic) at (\gangliuxxxi, \gangliuyyyc);
\coordinate (gangliupppid) at (\gangliuxxxi, \gangliuyyyd);
\coordinate (gangliupppie) at (\gangliuxxxi, \gangliuyyye);
\coordinate (gangliupppif) at (\gangliuxxxi, \gangliuyyyf);
\coordinate (gangliupppig) at (\gangliuxxxi, \gangliuyyyg);
\coordinate (gangliupppih) at (\gangliuxxxi, \gangliuyyyh);
\coordinate (gangliupppii) at (\gangliuxxxi, \gangliuyyyi);
\coordinate (gangliupppij) at (\gangliuxxxi, \gangliuyyyj);
\coordinate (gangliupppik) at (\gangliuxxxi, \gangliuyyyk);
\coordinate (gangliupppil) at (\gangliuxxxi, \gangliuyyyl);
\coordinate (gangliupppim) at (\gangliuxxxi, \gangliuyyym);
\coordinate (gangliupppin) at (\gangliuxxxi, \gangliuyyyn);
\coordinate (gangliupppio) at (\gangliuxxxi, \gangliuyyyo);
\coordinate (gangliupppip) at (\gangliuxxxi, \gangliuyyyp);
\coordinate (gangliupppiq) at (\gangliuxxxi, \gangliuyyyq);
\coordinate (gangliupppir) at (\gangliuxxxi, \gangliuyyyr);
\coordinate (gangliupppis) at (\gangliuxxxi, \gangliuyyys);
\coordinate (gangliupppit) at (\gangliuxxxi, \gangliuyyyt);
\coordinate (gangliupppiu) at (\gangliuxxxi, \gangliuyyyu);
\coordinate (gangliupppiv) at (\gangliuxxxi, \gangliuyyyv);
\coordinate (gangliupppiw) at (\gangliuxxxi, \gangliuyyyw);
\coordinate (gangliupppix) at (\gangliuxxxi, \gangliuyyyx);
\coordinate (gangliupppiy) at (\gangliuxxxi, \gangliuyyyy);
\coordinate (gangliupppiz) at (\gangliuxxxi, \gangliuyyyz);
\coordinate (gangliupppja) at (\gangliuxxxj, \gangliuyyya);
\coordinate (gangliupppjb) at (\gangliuxxxj, \gangliuyyyb);
\coordinate (gangliupppjc) at (\gangliuxxxj, \gangliuyyyc);
\coordinate (gangliupppjd) at (\gangliuxxxj, \gangliuyyyd);
\coordinate (gangliupppje) at (\gangliuxxxj, \gangliuyyye);
\coordinate (gangliupppjf) at (\gangliuxxxj, \gangliuyyyf);
\coordinate (gangliupppjg) at (\gangliuxxxj, \gangliuyyyg);
\coordinate (gangliupppjh) at (\gangliuxxxj, \gangliuyyyh);
\coordinate (gangliupppji) at (\gangliuxxxj, \gangliuyyyi);
\coordinate (gangliupppjj) at (\gangliuxxxj, \gangliuyyyj);
\coordinate (gangliupppjk) at (\gangliuxxxj, \gangliuyyyk);
\coordinate (gangliupppjl) at (\gangliuxxxj, \gangliuyyyl);
\coordinate (gangliupppjm) at (\gangliuxxxj, \gangliuyyym);
\coordinate (gangliupppjn) at (\gangliuxxxj, \gangliuyyyn);
\coordinate (gangliupppjo) at (\gangliuxxxj, \gangliuyyyo);
\coordinate (gangliupppjp) at (\gangliuxxxj, \gangliuyyyp);
\coordinate (gangliupppjq) at (\gangliuxxxj, \gangliuyyyq);
\coordinate (gangliupppjr) at (\gangliuxxxj, \gangliuyyyr);
\coordinate (gangliupppjs) at (\gangliuxxxj, \gangliuyyys);
\coordinate (gangliupppjt) at (\gangliuxxxj, \gangliuyyyt);
\coordinate (gangliupppju) at (\gangliuxxxj, \gangliuyyyu);
\coordinate (gangliupppjv) at (\gangliuxxxj, \gangliuyyyv);
\coordinate (gangliupppjw) at (\gangliuxxxj, \gangliuyyyw);
\coordinate (gangliupppjx) at (\gangliuxxxj, \gangliuyyyx);
\coordinate (gangliupppjy) at (\gangliuxxxj, \gangliuyyyy);
\coordinate (gangliupppjz) at (\gangliuxxxj, \gangliuyyyz);
\coordinate (gangliupppka) at (\gangliuxxxk, \gangliuyyya);
\coordinate (gangliupppkb) at (\gangliuxxxk, \gangliuyyyb);
\coordinate (gangliupppkc) at (\gangliuxxxk, \gangliuyyyc);
\coordinate (gangliupppkd) at (\gangliuxxxk, \gangliuyyyd);
\coordinate (gangliupppke) at (\gangliuxxxk, \gangliuyyye);
\coordinate (gangliupppkf) at (\gangliuxxxk, \gangliuyyyf);
\coordinate (gangliupppkg) at (\gangliuxxxk, \gangliuyyyg);
\coordinate (gangliupppkh) at (\gangliuxxxk, \gangliuyyyh);
\coordinate (gangliupppki) at (\gangliuxxxk, \gangliuyyyi);
\coordinate (gangliupppkj) at (\gangliuxxxk, \gangliuyyyj);
\coordinate (gangliupppkk) at (\gangliuxxxk, \gangliuyyyk);
\coordinate (gangliupppkl) at (\gangliuxxxk, \gangliuyyyl);
\coordinate (gangliupppkm) at (\gangliuxxxk, \gangliuyyym);
\coordinate (gangliupppkn) at (\gangliuxxxk, \gangliuyyyn);
\coordinate (gangliupppko) at (\gangliuxxxk, \gangliuyyyo);
\coordinate (gangliupppkp) at (\gangliuxxxk, \gangliuyyyp);
\coordinate (gangliupppkq) at (\gangliuxxxk, \gangliuyyyq);
\coordinate (gangliupppkr) at (\gangliuxxxk, \gangliuyyyr);
\coordinate (gangliupppks) at (\gangliuxxxk, \gangliuyyys);
\coordinate (gangliupppkt) at (\gangliuxxxk, \gangliuyyyt);
\coordinate (gangliupppku) at (\gangliuxxxk, \gangliuyyyu);
\coordinate (gangliupppkv) at (\gangliuxxxk, \gangliuyyyv);
\coordinate (gangliupppkw) at (\gangliuxxxk, \gangliuyyyw);
\coordinate (gangliupppkx) at (\gangliuxxxk, \gangliuyyyx);
\coordinate (gangliupppky) at (\gangliuxxxk, \gangliuyyyy);
\coordinate (gangliupppkz) at (\gangliuxxxk, \gangliuyyyz);
\coordinate (gangliupppla) at (\gangliuxxxl, \gangliuyyya);
\coordinate (gangliuppplb) at (\gangliuxxxl, \gangliuyyyb);
\coordinate (gangliuppplc) at (\gangliuxxxl, \gangliuyyyc);
\coordinate (gangliupppld) at (\gangliuxxxl, \gangliuyyyd);
\coordinate (gangliuppple) at (\gangliuxxxl, \gangliuyyye);
\coordinate (gangliuppplf) at (\gangliuxxxl, \gangliuyyyf);
\coordinate (gangliuppplg) at (\gangliuxxxl, \gangliuyyyg);
\coordinate (gangliuppplh) at (\gangliuxxxl, \gangliuyyyh);
\coordinate (gangliupppli) at (\gangliuxxxl, \gangliuyyyi);
\coordinate (gangliuppplj) at (\gangliuxxxl, \gangliuyyyj);
\coordinate (gangliuppplk) at (\gangliuxxxl, \gangliuyyyk);
\coordinate (gangliupppll) at (\gangliuxxxl, \gangliuyyyl);
\coordinate (gangliuppplm) at (\gangliuxxxl, \gangliuyyym);
\coordinate (gangliupppln) at (\gangliuxxxl, \gangliuyyyn);
\coordinate (gangliuppplo) at (\gangliuxxxl, \gangliuyyyo);
\coordinate (gangliuppplp) at (\gangliuxxxl, \gangliuyyyp);
\coordinate (gangliuppplq) at (\gangliuxxxl, \gangliuyyyq);
\coordinate (gangliuppplr) at (\gangliuxxxl, \gangliuyyyr);
\coordinate (gangliupppls) at (\gangliuxxxl, \gangliuyyys);
\coordinate (gangliuppplt) at (\gangliuxxxl, \gangliuyyyt);
\coordinate (gangliuppplu) at (\gangliuxxxl, \gangliuyyyu);
\coordinate (gangliuppplv) at (\gangliuxxxl, \gangliuyyyv);
\coordinate (gangliuppplw) at (\gangliuxxxl, \gangliuyyyw);
\coordinate (gangliuppplx) at (\gangliuxxxl, \gangliuyyyx);
\coordinate (gangliuppply) at (\gangliuxxxl, \gangliuyyyy);
\coordinate (gangliuppplz) at (\gangliuxxxl, \gangliuyyyz);
\coordinate (gangliupppma) at (\gangliuxxxm, \gangliuyyya);
\coordinate (gangliupppmb) at (\gangliuxxxm, \gangliuyyyb);
\coordinate (gangliupppmc) at (\gangliuxxxm, \gangliuyyyc);
\coordinate (gangliupppmd) at (\gangliuxxxm, \gangliuyyyd);
\coordinate (gangliupppme) at (\gangliuxxxm, \gangliuyyye);
\coordinate (gangliupppmf) at (\gangliuxxxm, \gangliuyyyf);
\coordinate (gangliupppmg) at (\gangliuxxxm, \gangliuyyyg);
\coordinate (gangliupppmh) at (\gangliuxxxm, \gangliuyyyh);
\coordinate (gangliupppmi) at (\gangliuxxxm, \gangliuyyyi);
\coordinate (gangliupppmj) at (\gangliuxxxm, \gangliuyyyj);
\coordinate (gangliupppmk) at (\gangliuxxxm, \gangliuyyyk);
\coordinate (gangliupppml) at (\gangliuxxxm, \gangliuyyyl);
\coordinate (gangliupppmm) at (\gangliuxxxm, \gangliuyyym);
\coordinate (gangliupppmn) at (\gangliuxxxm, \gangliuyyyn);
\coordinate (gangliupppmo) at (\gangliuxxxm, \gangliuyyyo);
\coordinate (gangliupppmp) at (\gangliuxxxm, \gangliuyyyp);
\coordinate (gangliupppmq) at (\gangliuxxxm, \gangliuyyyq);
\coordinate (gangliupppmr) at (\gangliuxxxm, \gangliuyyyr);
\coordinate (gangliupppms) at (\gangliuxxxm, \gangliuyyys);
\coordinate (gangliupppmt) at (\gangliuxxxm, \gangliuyyyt);
\coordinate (gangliupppmu) at (\gangliuxxxm, \gangliuyyyu);
\coordinate (gangliupppmv) at (\gangliuxxxm, \gangliuyyyv);
\coordinate (gangliupppmw) at (\gangliuxxxm, \gangliuyyyw);
\coordinate (gangliupppmx) at (\gangliuxxxm, \gangliuyyyx);
\coordinate (gangliupppmy) at (\gangliuxxxm, \gangliuyyyy);
\coordinate (gangliupppmz) at (\gangliuxxxm, \gangliuyyyz);
\coordinate (gangliupppna) at (\gangliuxxxn, \gangliuyyya);
\coordinate (gangliupppnb) at (\gangliuxxxn, \gangliuyyyb);
\coordinate (gangliupppnc) at (\gangliuxxxn, \gangliuyyyc);
\coordinate (gangliupppnd) at (\gangliuxxxn, \gangliuyyyd);
\coordinate (gangliupppne) at (\gangliuxxxn, \gangliuyyye);
\coordinate (gangliupppnf) at (\gangliuxxxn, \gangliuyyyf);
\coordinate (gangliupppng) at (\gangliuxxxn, \gangliuyyyg);
\coordinate (gangliupppnh) at (\gangliuxxxn, \gangliuyyyh);
\coordinate (gangliupppni) at (\gangliuxxxn, \gangliuyyyi);
\coordinate (gangliupppnj) at (\gangliuxxxn, \gangliuyyyj);
\coordinate (gangliupppnk) at (\gangliuxxxn, \gangliuyyyk);
\coordinate (gangliupppnl) at (\gangliuxxxn, \gangliuyyyl);
\coordinate (gangliupppnm) at (\gangliuxxxn, \gangliuyyym);
\coordinate (gangliupppnn) at (\gangliuxxxn, \gangliuyyyn);
\coordinate (gangliupppno) at (\gangliuxxxn, \gangliuyyyo);
\coordinate (gangliupppnp) at (\gangliuxxxn, \gangliuyyyp);
\coordinate (gangliupppnq) at (\gangliuxxxn, \gangliuyyyq);
\coordinate (gangliupppnr) at (\gangliuxxxn, \gangliuyyyr);
\coordinate (gangliupppns) at (\gangliuxxxn, \gangliuyyys);
\coordinate (gangliupppnt) at (\gangliuxxxn, \gangliuyyyt);
\coordinate (gangliupppnu) at (\gangliuxxxn, \gangliuyyyu);
\coordinate (gangliupppnv) at (\gangliuxxxn, \gangliuyyyv);
\coordinate (gangliupppnw) at (\gangliuxxxn, \gangliuyyyw);
\coordinate (gangliupppnx) at (\gangliuxxxn, \gangliuyyyx);
\coordinate (gangliupppny) at (\gangliuxxxn, \gangliuyyyy);
\coordinate (gangliupppnz) at (\gangliuxxxn, \gangliuyyyz);
\coordinate (gangliupppoa) at (\gangliuxxxo, \gangliuyyya);
\coordinate (gangliupppob) at (\gangliuxxxo, \gangliuyyyb);
\coordinate (gangliupppoc) at (\gangliuxxxo, \gangliuyyyc);
\coordinate (gangliupppod) at (\gangliuxxxo, \gangliuyyyd);
\coordinate (gangliupppoe) at (\gangliuxxxo, \gangliuyyye);
\coordinate (gangliupppof) at (\gangliuxxxo, \gangliuyyyf);
\coordinate (gangliupppog) at (\gangliuxxxo, \gangliuyyyg);
\coordinate (gangliupppoh) at (\gangliuxxxo, \gangliuyyyh);
\coordinate (gangliupppoi) at (\gangliuxxxo, \gangliuyyyi);
\coordinate (gangliupppoj) at (\gangliuxxxo, \gangliuyyyj);
\coordinate (gangliupppok) at (\gangliuxxxo, \gangliuyyyk);
\coordinate (gangliupppol) at (\gangliuxxxo, \gangliuyyyl);
\coordinate (gangliupppom) at (\gangliuxxxo, \gangliuyyym);
\coordinate (gangliupppon) at (\gangliuxxxo, \gangliuyyyn);
\coordinate (gangliupppoo) at (\gangliuxxxo, \gangliuyyyo);
\coordinate (gangliupppop) at (\gangliuxxxo, \gangliuyyyp);
\coordinate (gangliupppoq) at (\gangliuxxxo, \gangliuyyyq);
\coordinate (gangliupppor) at (\gangliuxxxo, \gangliuyyyr);
\coordinate (gangliupppos) at (\gangliuxxxo, \gangliuyyys);
\coordinate (gangliupppot) at (\gangliuxxxo, \gangliuyyyt);
\coordinate (gangliupppou) at (\gangliuxxxo, \gangliuyyyu);
\coordinate (gangliupppov) at (\gangliuxxxo, \gangliuyyyv);
\coordinate (gangliupppow) at (\gangliuxxxo, \gangliuyyyw);
\coordinate (gangliupppox) at (\gangliuxxxo, \gangliuyyyx);
\coordinate (gangliupppoy) at (\gangliuxxxo, \gangliuyyyy);
\coordinate (gangliupppoz) at (\gangliuxxxo, \gangliuyyyz);
\coordinate (gangliuppppa) at (\gangliuxxxp, \gangliuyyya);
\coordinate (gangliuppppb) at (\gangliuxxxp, \gangliuyyyb);
\coordinate (gangliuppppc) at (\gangliuxxxp, \gangliuyyyc);
\coordinate (gangliuppppd) at (\gangliuxxxp, \gangliuyyyd);
\coordinate (gangliuppppe) at (\gangliuxxxp, \gangliuyyye);
\coordinate (gangliuppppf) at (\gangliuxxxp, \gangliuyyyf);
\coordinate (gangliuppppg) at (\gangliuxxxp, \gangliuyyyg);
\coordinate (gangliupppph) at (\gangliuxxxp, \gangliuyyyh);
\coordinate (gangliuppppi) at (\gangliuxxxp, \gangliuyyyi);
\coordinate (gangliuppppj) at (\gangliuxxxp, \gangliuyyyj);
\coordinate (gangliuppppk) at (\gangliuxxxp, \gangliuyyyk);
\coordinate (gangliuppppl) at (\gangliuxxxp, \gangliuyyyl);
\coordinate (gangliuppppm) at (\gangliuxxxp, \gangliuyyym);
\coordinate (gangliuppppn) at (\gangliuxxxp, \gangliuyyyn);
\coordinate (gangliuppppo) at (\gangliuxxxp, \gangliuyyyo);
\coordinate (gangliuppppp) at (\gangliuxxxp, \gangliuyyyp);
\coordinate (gangliuppppq) at (\gangliuxxxp, \gangliuyyyq);
\coordinate (gangliuppppr) at (\gangliuxxxp, \gangliuyyyr);
\coordinate (gangliupppps) at (\gangliuxxxp, \gangliuyyys);
\coordinate (gangliuppppt) at (\gangliuxxxp, \gangliuyyyt);
\coordinate (gangliuppppu) at (\gangliuxxxp, \gangliuyyyu);
\coordinate (gangliuppppv) at (\gangliuxxxp, \gangliuyyyv);
\coordinate (gangliuppppw) at (\gangliuxxxp, \gangliuyyyw);
\coordinate (gangliuppppx) at (\gangliuxxxp, \gangliuyyyx);
\coordinate (gangliuppppy) at (\gangliuxxxp, \gangliuyyyy);
\coordinate (gangliuppppz) at (\gangliuxxxp, \gangliuyyyz);
\coordinate (gangliupppqa) at (\gangliuxxxq, \gangliuyyya);
\coordinate (gangliupppqb) at (\gangliuxxxq, \gangliuyyyb);
\coordinate (gangliupppqc) at (\gangliuxxxq, \gangliuyyyc);
\coordinate (gangliupppqd) at (\gangliuxxxq, \gangliuyyyd);
\coordinate (gangliupppqe) at (\gangliuxxxq, \gangliuyyye);
\coordinate (gangliupppqf) at (\gangliuxxxq, \gangliuyyyf);
\coordinate (gangliupppqg) at (\gangliuxxxq, \gangliuyyyg);
\coordinate (gangliupppqh) at (\gangliuxxxq, \gangliuyyyh);
\coordinate (gangliupppqi) at (\gangliuxxxq, \gangliuyyyi);
\coordinate (gangliupppqj) at (\gangliuxxxq, \gangliuyyyj);
\coordinate (gangliupppqk) at (\gangliuxxxq, \gangliuyyyk);
\coordinate (gangliupppql) at (\gangliuxxxq, \gangliuyyyl);
\coordinate (gangliupppqm) at (\gangliuxxxq, \gangliuyyym);
\coordinate (gangliupppqn) at (\gangliuxxxq, \gangliuyyyn);
\coordinate (gangliupppqo) at (\gangliuxxxq, \gangliuyyyo);
\coordinate (gangliupppqp) at (\gangliuxxxq, \gangliuyyyp);
\coordinate (gangliupppqq) at (\gangliuxxxq, \gangliuyyyq);
\coordinate (gangliupppqr) at (\gangliuxxxq, \gangliuyyyr);
\coordinate (gangliupppqs) at (\gangliuxxxq, \gangliuyyys);
\coordinate (gangliupppqt) at (\gangliuxxxq, \gangliuyyyt);
\coordinate (gangliupppqu) at (\gangliuxxxq, \gangliuyyyu);
\coordinate (gangliupppqv) at (\gangliuxxxq, \gangliuyyyv);
\coordinate (gangliupppqw) at (\gangliuxxxq, \gangliuyyyw);
\coordinate (gangliupppqx) at (\gangliuxxxq, \gangliuyyyx);
\coordinate (gangliupppqy) at (\gangliuxxxq, \gangliuyyyy);
\coordinate (gangliupppqz) at (\gangliuxxxq, \gangliuyyyz);
\coordinate (gangliupppra) at (\gangliuxxxr, \gangliuyyya);
\coordinate (gangliuppprb) at (\gangliuxxxr, \gangliuyyyb);
\coordinate (gangliuppprc) at (\gangliuxxxr, \gangliuyyyc);
\coordinate (gangliuppprd) at (\gangliuxxxr, \gangliuyyyd);
\coordinate (gangliupppre) at (\gangliuxxxr, \gangliuyyye);
\coordinate (gangliuppprf) at (\gangliuxxxr, \gangliuyyyf);
\coordinate (gangliuppprg) at (\gangliuxxxr, \gangliuyyyg);
\coordinate (gangliuppprh) at (\gangliuxxxr, \gangliuyyyh);
\coordinate (gangliupppri) at (\gangliuxxxr, \gangliuyyyi);
\coordinate (gangliuppprj) at (\gangliuxxxr, \gangliuyyyj);
\coordinate (gangliuppprk) at (\gangliuxxxr, \gangliuyyyk);
\coordinate (gangliuppprl) at (\gangliuxxxr, \gangliuyyyl);
\coordinate (gangliuppprm) at (\gangliuxxxr, \gangliuyyym);
\coordinate (gangliuppprn) at (\gangliuxxxr, \gangliuyyyn);
\coordinate (gangliupppro) at (\gangliuxxxr, \gangliuyyyo);
\coordinate (gangliuppprp) at (\gangliuxxxr, \gangliuyyyp);
\coordinate (gangliuppprq) at (\gangliuxxxr, \gangliuyyyq);
\coordinate (gangliuppprr) at (\gangliuxxxr, \gangliuyyyr);
\coordinate (gangliuppprs) at (\gangliuxxxr, \gangliuyyys);
\coordinate (gangliuppprt) at (\gangliuxxxr, \gangliuyyyt);
\coordinate (gangliupppru) at (\gangliuxxxr, \gangliuyyyu);
\coordinate (gangliuppprv) at (\gangliuxxxr, \gangliuyyyv);
\coordinate (gangliuppprw) at (\gangliuxxxr, \gangliuyyyw);
\coordinate (gangliuppprx) at (\gangliuxxxr, \gangliuyyyx);
\coordinate (gangliupppry) at (\gangliuxxxr, \gangliuyyyy);
\coordinate (gangliuppprz) at (\gangliuxxxr, \gangliuyyyz);
\coordinate (gangliupppsa) at (\gangliuxxxs, \gangliuyyya);
\coordinate (gangliupppsb) at (\gangliuxxxs, \gangliuyyyb);
\coordinate (gangliupppsc) at (\gangliuxxxs, \gangliuyyyc);
\coordinate (gangliupppsd) at (\gangliuxxxs, \gangliuyyyd);
\coordinate (gangliupppse) at (\gangliuxxxs, \gangliuyyye);
\coordinate (gangliupppsf) at (\gangliuxxxs, \gangliuyyyf);
\coordinate (gangliupppsg) at (\gangliuxxxs, \gangliuyyyg);
\coordinate (gangliupppsh) at (\gangliuxxxs, \gangliuyyyh);
\coordinate (gangliupppsi) at (\gangliuxxxs, \gangliuyyyi);
\coordinate (gangliupppsj) at (\gangliuxxxs, \gangliuyyyj);
\coordinate (gangliupppsk) at (\gangliuxxxs, \gangliuyyyk);
\coordinate (gangliupppsl) at (\gangliuxxxs, \gangliuyyyl);
\coordinate (gangliupppsm) at (\gangliuxxxs, \gangliuyyym);
\coordinate (gangliupppsn) at (\gangliuxxxs, \gangliuyyyn);
\coordinate (gangliupppso) at (\gangliuxxxs, \gangliuyyyo);
\coordinate (gangliupppsp) at (\gangliuxxxs, \gangliuyyyp);
\coordinate (gangliupppsq) at (\gangliuxxxs, \gangliuyyyq);
\coordinate (gangliupppsr) at (\gangliuxxxs, \gangliuyyyr);
\coordinate (gangliupppss) at (\gangliuxxxs, \gangliuyyys);
\coordinate (gangliupppst) at (\gangliuxxxs, \gangliuyyyt);
\coordinate (gangliupppsu) at (\gangliuxxxs, \gangliuyyyu);
\coordinate (gangliupppsv) at (\gangliuxxxs, \gangliuyyyv);
\coordinate (gangliupppsw) at (\gangliuxxxs, \gangliuyyyw);
\coordinate (gangliupppsx) at (\gangliuxxxs, \gangliuyyyx);
\coordinate (gangliupppsy) at (\gangliuxxxs, \gangliuyyyy);
\coordinate (gangliupppsz) at (\gangliuxxxs, \gangliuyyyz);
\coordinate (gangliupppta) at (\gangliuxxxt, \gangliuyyya);
\coordinate (gangliuppptb) at (\gangliuxxxt, \gangliuyyyb);
\coordinate (gangliuppptc) at (\gangliuxxxt, \gangliuyyyc);
\coordinate (gangliuppptd) at (\gangliuxxxt, \gangliuyyyd);
\coordinate (gangliupppte) at (\gangliuxxxt, \gangliuyyye);
\coordinate (gangliuppptf) at (\gangliuxxxt, \gangliuyyyf);
\coordinate (gangliuppptg) at (\gangliuxxxt, \gangliuyyyg);
\coordinate (gangliupppth) at (\gangliuxxxt, \gangliuyyyh);
\coordinate (gangliupppti) at (\gangliuxxxt, \gangliuyyyi);
\coordinate (gangliuppptj) at (\gangliuxxxt, \gangliuyyyj);
\coordinate (gangliuppptk) at (\gangliuxxxt, \gangliuyyyk);
\coordinate (gangliuppptl) at (\gangliuxxxt, \gangliuyyyl);
\coordinate (gangliuppptm) at (\gangliuxxxt, \gangliuyyym);
\coordinate (gangliuppptn) at (\gangliuxxxt, \gangliuyyyn);
\coordinate (gangliupppto) at (\gangliuxxxt, \gangliuyyyo);
\coordinate (gangliuppptp) at (\gangliuxxxt, \gangliuyyyp);
\coordinate (gangliuppptq) at (\gangliuxxxt, \gangliuyyyq);
\coordinate (gangliuppptr) at (\gangliuxxxt, \gangliuyyyr);
\coordinate (gangliupppts) at (\gangliuxxxt, \gangliuyyys);
\coordinate (gangliuppptt) at (\gangliuxxxt, \gangliuyyyt);
\coordinate (gangliuppptu) at (\gangliuxxxt, \gangliuyyyu);
\coordinate (gangliuppptv) at (\gangliuxxxt, \gangliuyyyv);
\coordinate (gangliuppptw) at (\gangliuxxxt, \gangliuyyyw);
\coordinate (gangliuppptx) at (\gangliuxxxt, \gangliuyyyx);
\coordinate (gangliupppty) at (\gangliuxxxt, \gangliuyyyy);
\coordinate (gangliuppptz) at (\gangliuxxxt, \gangliuyyyz);
\coordinate (gangliupppua) at (\gangliuxxxu, \gangliuyyya);
\coordinate (gangliupppub) at (\gangliuxxxu, \gangliuyyyb);
\coordinate (gangliupppuc) at (\gangliuxxxu, \gangliuyyyc);
\coordinate (gangliupppud) at (\gangliuxxxu, \gangliuyyyd);
\coordinate (gangliupppue) at (\gangliuxxxu, \gangliuyyye);
\coordinate (gangliupppuf) at (\gangliuxxxu, \gangliuyyyf);
\coordinate (gangliupppug) at (\gangliuxxxu, \gangliuyyyg);
\coordinate (gangliupppuh) at (\gangliuxxxu, \gangliuyyyh);
\coordinate (gangliupppui) at (\gangliuxxxu, \gangliuyyyi);
\coordinate (gangliupppuj) at (\gangliuxxxu, \gangliuyyyj);
\coordinate (gangliupppuk) at (\gangliuxxxu, \gangliuyyyk);
\coordinate (gangliupppul) at (\gangliuxxxu, \gangliuyyyl);
\coordinate (gangliupppum) at (\gangliuxxxu, \gangliuyyym);
\coordinate (gangliupppun) at (\gangliuxxxu, \gangliuyyyn);
\coordinate (gangliupppuo) at (\gangliuxxxu, \gangliuyyyo);
\coordinate (gangliupppup) at (\gangliuxxxu, \gangliuyyyp);
\coordinate (gangliupppuq) at (\gangliuxxxu, \gangliuyyyq);
\coordinate (gangliupppur) at (\gangliuxxxu, \gangliuyyyr);
\coordinate (gangliupppus) at (\gangliuxxxu, \gangliuyyys);
\coordinate (gangliuppput) at (\gangliuxxxu, \gangliuyyyt);
\coordinate (gangliupppuu) at (\gangliuxxxu, \gangliuyyyu);
\coordinate (gangliupppuv) at (\gangliuxxxu, \gangliuyyyv);
\coordinate (gangliupppuw) at (\gangliuxxxu, \gangliuyyyw);
\coordinate (gangliupppux) at (\gangliuxxxu, \gangliuyyyx);
\coordinate (gangliupppuy) at (\gangliuxxxu, \gangliuyyyy);
\coordinate (gangliupppuz) at (\gangliuxxxu, \gangliuyyyz);
\coordinate (gangliupppva) at (\gangliuxxxv, \gangliuyyya);
\coordinate (gangliupppvb) at (\gangliuxxxv, \gangliuyyyb);
\coordinate (gangliupppvc) at (\gangliuxxxv, \gangliuyyyc);
\coordinate (gangliupppvd) at (\gangliuxxxv, \gangliuyyyd);
\coordinate (gangliupppve) at (\gangliuxxxv, \gangliuyyye);
\coordinate (gangliupppvf) at (\gangliuxxxv, \gangliuyyyf);
\coordinate (gangliupppvg) at (\gangliuxxxv, \gangliuyyyg);
\coordinate (gangliupppvh) at (\gangliuxxxv, \gangliuyyyh);
\coordinate (gangliupppvi) at (\gangliuxxxv, \gangliuyyyi);
\coordinate (gangliupppvj) at (\gangliuxxxv, \gangliuyyyj);
\coordinate (gangliupppvk) at (\gangliuxxxv, \gangliuyyyk);
\coordinate (gangliupppvl) at (\gangliuxxxv, \gangliuyyyl);
\coordinate (gangliupppvm) at (\gangliuxxxv, \gangliuyyym);
\coordinate (gangliupppvn) at (\gangliuxxxv, \gangliuyyyn);
\coordinate (gangliupppvo) at (\gangliuxxxv, \gangliuyyyo);
\coordinate (gangliupppvp) at (\gangliuxxxv, \gangliuyyyp);
\coordinate (gangliupppvq) at (\gangliuxxxv, \gangliuyyyq);
\coordinate (gangliupppvr) at (\gangliuxxxv, \gangliuyyyr);
\coordinate (gangliupppvs) at (\gangliuxxxv, \gangliuyyys);
\coordinate (gangliupppvt) at (\gangliuxxxv, \gangliuyyyt);
\coordinate (gangliupppvu) at (\gangliuxxxv, \gangliuyyyu);
\coordinate (gangliupppvv) at (\gangliuxxxv, \gangliuyyyv);
\coordinate (gangliupppvw) at (\gangliuxxxv, \gangliuyyyw);
\coordinate (gangliupppvx) at (\gangliuxxxv, \gangliuyyyx);
\coordinate (gangliupppvy) at (\gangliuxxxv, \gangliuyyyy);
\coordinate (gangliupppvz) at (\gangliuxxxv, \gangliuyyyz);
\coordinate (gangliupppwa) at (\gangliuxxxw, \gangliuyyya);
\coordinate (gangliupppwb) at (\gangliuxxxw, \gangliuyyyb);
\coordinate (gangliupppwc) at (\gangliuxxxw, \gangliuyyyc);
\coordinate (gangliupppwd) at (\gangliuxxxw, \gangliuyyyd);
\coordinate (gangliupppwe) at (\gangliuxxxw, \gangliuyyye);
\coordinate (gangliupppwf) at (\gangliuxxxw, \gangliuyyyf);
\coordinate (gangliupppwg) at (\gangliuxxxw, \gangliuyyyg);
\coordinate (gangliupppwh) at (\gangliuxxxw, \gangliuyyyh);
\coordinate (gangliupppwi) at (\gangliuxxxw, \gangliuyyyi);
\coordinate (gangliupppwj) at (\gangliuxxxw, \gangliuyyyj);
\coordinate (gangliupppwk) at (\gangliuxxxw, \gangliuyyyk);
\coordinate (gangliupppwl) at (\gangliuxxxw, \gangliuyyyl);
\coordinate (gangliupppwm) at (\gangliuxxxw, \gangliuyyym);
\coordinate (gangliupppwn) at (\gangliuxxxw, \gangliuyyyn);
\coordinate (gangliupppwo) at (\gangliuxxxw, \gangliuyyyo);
\coordinate (gangliupppwp) at (\gangliuxxxw, \gangliuyyyp);
\coordinate (gangliupppwq) at (\gangliuxxxw, \gangliuyyyq);
\coordinate (gangliupppwr) at (\gangliuxxxw, \gangliuyyyr);
\coordinate (gangliupppws) at (\gangliuxxxw, \gangliuyyys);
\coordinate (gangliupppwt) at (\gangliuxxxw, \gangliuyyyt);
\coordinate (gangliupppwu) at (\gangliuxxxw, \gangliuyyyu);
\coordinate (gangliupppwv) at (\gangliuxxxw, \gangliuyyyv);
\coordinate (gangliupppww) at (\gangliuxxxw, \gangliuyyyw);
\coordinate (gangliupppwx) at (\gangliuxxxw, \gangliuyyyx);
\coordinate (gangliupppwy) at (\gangliuxxxw, \gangliuyyyy);
\coordinate (gangliupppwz) at (\gangliuxxxw, \gangliuyyyz);
\coordinate (gangliupppxa) at (\gangliuxxxx, \gangliuyyya);
\coordinate (gangliupppxb) at (\gangliuxxxx, \gangliuyyyb);
\coordinate (gangliupppxc) at (\gangliuxxxx, \gangliuyyyc);
\coordinate (gangliupppxd) at (\gangliuxxxx, \gangliuyyyd);
\coordinate (gangliupppxe) at (\gangliuxxxx, \gangliuyyye);
\coordinate (gangliupppxf) at (\gangliuxxxx, \gangliuyyyf);
\coordinate (gangliupppxg) at (\gangliuxxxx, \gangliuyyyg);
\coordinate (gangliupppxh) at (\gangliuxxxx, \gangliuyyyh);
\coordinate (gangliupppxi) at (\gangliuxxxx, \gangliuyyyi);
\coordinate (gangliupppxj) at (\gangliuxxxx, \gangliuyyyj);
\coordinate (gangliupppxk) at (\gangliuxxxx, \gangliuyyyk);
\coordinate (gangliupppxl) at (\gangliuxxxx, \gangliuyyyl);
\coordinate (gangliupppxm) at (\gangliuxxxx, \gangliuyyym);
\coordinate (gangliupppxn) at (\gangliuxxxx, \gangliuyyyn);
\coordinate (gangliupppxo) at (\gangliuxxxx, \gangliuyyyo);
\coordinate (gangliupppxp) at (\gangliuxxxx, \gangliuyyyp);
\coordinate (gangliupppxq) at (\gangliuxxxx, \gangliuyyyq);
\coordinate (gangliupppxr) at (\gangliuxxxx, \gangliuyyyr);
\coordinate (gangliupppxs) at (\gangliuxxxx, \gangliuyyys);
\coordinate (gangliupppxt) at (\gangliuxxxx, \gangliuyyyt);
\coordinate (gangliupppxu) at (\gangliuxxxx, \gangliuyyyu);
\coordinate (gangliupppxv) at (\gangliuxxxx, \gangliuyyyv);
\coordinate (gangliupppxw) at (\gangliuxxxx, \gangliuyyyw);
\coordinate (gangliupppxx) at (\gangliuxxxx, \gangliuyyyx);
\coordinate (gangliupppxy) at (\gangliuxxxx, \gangliuyyyy);
\coordinate (gangliupppxz) at (\gangliuxxxx, \gangliuyyyz);
\coordinate (gangliupppya) at (\gangliuxxxy, \gangliuyyya);
\coordinate (gangliupppyb) at (\gangliuxxxy, \gangliuyyyb);
\coordinate (gangliupppyc) at (\gangliuxxxy, \gangliuyyyc);
\coordinate (gangliupppyd) at (\gangliuxxxy, \gangliuyyyd);
\coordinate (gangliupppye) at (\gangliuxxxy, \gangliuyyye);
\coordinate (gangliupppyf) at (\gangliuxxxy, \gangliuyyyf);
\coordinate (gangliupppyg) at (\gangliuxxxy, \gangliuyyyg);
\coordinate (gangliupppyh) at (\gangliuxxxy, \gangliuyyyh);
\coordinate (gangliupppyi) at (\gangliuxxxy, \gangliuyyyi);
\coordinate (gangliupppyj) at (\gangliuxxxy, \gangliuyyyj);
\coordinate (gangliupppyk) at (\gangliuxxxy, \gangliuyyyk);
\coordinate (gangliupppyl) at (\gangliuxxxy, \gangliuyyyl);
\coordinate (gangliupppym) at (\gangliuxxxy, \gangliuyyym);
\coordinate (gangliupppyn) at (\gangliuxxxy, \gangliuyyyn);
\coordinate (gangliupppyo) at (\gangliuxxxy, \gangliuyyyo);
\coordinate (gangliupppyp) at (\gangliuxxxy, \gangliuyyyp);
\coordinate (gangliupppyq) at (\gangliuxxxy, \gangliuyyyq);
\coordinate (gangliupppyr) at (\gangliuxxxy, \gangliuyyyr);
\coordinate (gangliupppys) at (\gangliuxxxy, \gangliuyyys);
\coordinate (gangliupppyt) at (\gangliuxxxy, \gangliuyyyt);
\coordinate (gangliupppyu) at (\gangliuxxxy, \gangliuyyyu);
\coordinate (gangliupppyv) at (\gangliuxxxy, \gangliuyyyv);
\coordinate (gangliupppyw) at (\gangliuxxxy, \gangliuyyyw);
\coordinate (gangliupppyx) at (\gangliuxxxy, \gangliuyyyx);
\coordinate (gangliupppyy) at (\gangliuxxxy, \gangliuyyyy);
\coordinate (gangliupppyz) at (\gangliuxxxy, \gangliuyyyz);
\coordinate (gangliupppza) at (\gangliuxxxz, \gangliuyyya);
\coordinate (gangliupppzb) at (\gangliuxxxz, \gangliuyyyb);
\coordinate (gangliupppzc) at (\gangliuxxxz, \gangliuyyyc);
\coordinate (gangliupppzd) at (\gangliuxxxz, \gangliuyyyd);
\coordinate (gangliupppze) at (\gangliuxxxz, \gangliuyyye);
\coordinate (gangliupppzf) at (\gangliuxxxz, \gangliuyyyf);
\coordinate (gangliupppzg) at (\gangliuxxxz, \gangliuyyyg);
\coordinate (gangliupppzh) at (\gangliuxxxz, \gangliuyyyh);
\coordinate (gangliupppzi) at (\gangliuxxxz, \gangliuyyyi);
\coordinate (gangliupppzj) at (\gangliuxxxz, \gangliuyyyj);
\coordinate (gangliupppzk) at (\gangliuxxxz, \gangliuyyyk);
\coordinate (gangliupppzl) at (\gangliuxxxz, \gangliuyyyl);
\coordinate (gangliupppzm) at (\gangliuxxxz, \gangliuyyym);
\coordinate (gangliupppzn) at (\gangliuxxxz, \gangliuyyyn);
\coordinate (gangliupppzo) at (\gangliuxxxz, \gangliuyyyo);
\coordinate (gangliupppzp) at (\gangliuxxxz, \gangliuyyyp);
\coordinate (gangliupppzq) at (\gangliuxxxz, \gangliuyyyq);
\coordinate (gangliupppzr) at (\gangliuxxxz, \gangliuyyyr);
\coordinate (gangliupppzs) at (\gangliuxxxz, \gangliuyyys);
\coordinate (gangliupppzt) at (\gangliuxxxz, \gangliuyyyt);
\coordinate (gangliupppzu) at (\gangliuxxxz, \gangliuyyyu);
\coordinate (gangliupppzv) at (\gangliuxxxz, \gangliuyyyv);
\coordinate (gangliupppzw) at (\gangliuxxxz, \gangliuyyyw);
\coordinate (gangliupppzx) at (\gangliuxxxz, \gangliuyyyx);
\coordinate (gangliupppzy) at (\gangliuxxxz, \gangliuyyyy);
\coordinate (gangliupppzz) at (\gangliuxxxz, \gangliuyyyz);

%\gangprintcoordinateat{(0,0)}{The last coordinate values: }{($(gangliupppzz)$)}; 



\pgfmathsetmacro{\totalganglaxxx}{26}
\pgfmathsetmacro{\totalganglayyy}{26}
\pgfmathsetmacro{\ganglaxxxspacing}{1}
\pgfmathsetmacro{\ganglayyyspacing}{1}
\pgfmathsetmacro{\ganglaxxxa}{\gangliuxxxk}
\pgfmathsetmacro{\ganglayyya}{\gangliuyyyj}

\pgfmathsetmacro{\ganglaxxxb}{\ganglaxxxa + \ganglaxxxspacing + 0.0 }
\pgfmathsetmacro{\ganglaxxxc}{\ganglaxxxb + \ganglaxxxspacing + 0.0 }
\pgfmathsetmacro{\ganglaxxxd}{\ganglaxxxc + \ganglaxxxspacing + 0.0 }
\pgfmathsetmacro{\ganglaxxxe}{\ganglaxxxd + \ganglaxxxspacing + 0.0 }
\pgfmathsetmacro{\ganglaxxxf}{\ganglaxxxe + \ganglaxxxspacing + 0.0 }
\pgfmathsetmacro{\ganglaxxxg}{\ganglaxxxf + \ganglaxxxspacing + 0.0 }
\pgfmathsetmacro{\ganglaxxxh}{\ganglaxxxg + \ganglaxxxspacing + 0.0 }
\pgfmathsetmacro{\ganglaxxxi}{\ganglaxxxh + \ganglaxxxspacing + 0.0 }
\pgfmathsetmacro{\ganglaxxxj}{\ganglaxxxi + \ganglaxxxspacing + 0.0 }
\pgfmathsetmacro{\ganglaxxxk}{\ganglaxxxj + \ganglaxxxspacing + 0.0 }
\pgfmathsetmacro{\ganglaxxxl}{\ganglaxxxk + \ganglaxxxspacing + 0.0 }
\pgfmathsetmacro{\ganglaxxxm}{\ganglaxxxl + \ganglaxxxspacing + 0.0 }
\pgfmathsetmacro{\ganglaxxxn}{\ganglaxxxm + \ganglaxxxspacing + 0.0 }
\pgfmathsetmacro{\ganglaxxxo}{\ganglaxxxn + \ganglaxxxspacing + 0.0 }
\pgfmathsetmacro{\ganglaxxxp}{\ganglaxxxo + \ganglaxxxspacing + 0.0 }
\pgfmathsetmacro{\ganglaxxxq}{\ganglaxxxp + \ganglaxxxspacing + 0.0 }
\pgfmathsetmacro{\ganglaxxxr}{\ganglaxxxq + \ganglaxxxspacing + 0.0 }
\pgfmathsetmacro{\ganglaxxxs}{\ganglaxxxr + \ganglaxxxspacing + 0.0 }
\pgfmathsetmacro{\ganglaxxxt}{\ganglaxxxs + \ganglaxxxspacing + 0.0 }
\pgfmathsetmacro{\ganglaxxxu}{\ganglaxxxt + \ganglaxxxspacing + 0.0 }
\pgfmathsetmacro{\ganglaxxxv}{\ganglaxxxu + \ganglaxxxspacing + 0.0 }
\pgfmathsetmacro{\ganglaxxxw}{\ganglaxxxv + \ganglaxxxspacing + 0.0 }
\pgfmathsetmacro{\ganglaxxxx}{\ganglaxxxw + \ganglaxxxspacing + 0.0 }
\pgfmathsetmacro{\ganglaxxxy}{\ganglaxxxx + \ganglaxxxspacing + 0.0 }
\pgfmathsetmacro{\ganglaxxxz}{\ganglaxxxy + \ganglaxxxspacing + 0.0 }

\pgfmathsetmacro{\ganglayyyb}{\ganglayyya + \ganglayyyspacing + 0.0 }
\pgfmathsetmacro{\ganglayyyc}{\ganglayyyb + \ganglayyyspacing + 0.0 }
\pgfmathsetmacro{\ganglayyyd}{\ganglayyyc + \ganglayyyspacing + 0.0 }
\pgfmathsetmacro{\ganglayyye}{\ganglayyyd + \ganglayyyspacing + 0.0 }
\pgfmathsetmacro{\ganglayyyf}{\ganglayyye + \ganglayyyspacing + 0.0 }
\pgfmathsetmacro{\ganglayyyg}{\ganglayyyf + \ganglayyyspacing + 0.0 }
\pgfmathsetmacro{\ganglayyyh}{\ganglayyyg + \ganglayyyspacing + 0.0 }
\pgfmathsetmacro{\ganglayyyi}{\ganglayyyh + \ganglayyyspacing + 0.0 }
\pgfmathsetmacro{\ganglayyyj}{\ganglayyyi + \ganglayyyspacing + 0.0 }
\pgfmathsetmacro{\ganglayyyk}{\ganglayyyj + \ganglayyyspacing + 0.0 }
\pgfmathsetmacro{\ganglayyyl}{\ganglayyyk + \ganglayyyspacing + 0.0 }
\pgfmathsetmacro{\ganglayyym}{\ganglayyyl + \ganglayyyspacing + 0.0 }
\pgfmathsetmacro{\ganglayyyn}{\ganglayyym + \ganglayyyspacing + 0.0 }
\pgfmathsetmacro{\ganglayyyo}{\ganglayyyn + \ganglayyyspacing + 0.0 }
\pgfmathsetmacro{\ganglayyyp}{\ganglayyyo + \ganglayyyspacing + 0.0 }
\pgfmathsetmacro{\ganglayyyq}{\ganglayyyp + \ganglayyyspacing + 0.0 }
\pgfmathsetmacro{\ganglayyyr}{\ganglayyyq + \ganglayyyspacing + 0.0 }
\pgfmathsetmacro{\ganglayyys}{\ganglayyyr + \ganglayyyspacing + 0.0 }
\pgfmathsetmacro{\ganglayyyt}{\ganglayyys + \ganglayyyspacing + 0.0 }
\pgfmathsetmacro{\ganglayyyu}{\ganglayyyt + \ganglayyyspacing + 0.0 }
\pgfmathsetmacro{\ganglayyyv}{\ganglayyyu + \ganglayyyspacing + 0.0 }
\pgfmathsetmacro{\ganglayyyw}{\ganglayyyv + \ganglayyyspacing + 0.0 }
\pgfmathsetmacro{\ganglayyyx}{\ganglayyyw + \ganglayyyspacing + 0.0 }
\pgfmathsetmacro{\ganglayyyy}{\ganglayyyx + \ganglayyyspacing + 0.0 }
\pgfmathsetmacro{\ganglayyyz}{\ganglayyyy + \ganglayyyspacing + 0.0 }

\coordinate (ganglapppaa) at (\ganglaxxxa, \ganglayyya);
\coordinate (ganglapppab) at (\ganglaxxxa, \ganglayyyb);
\coordinate (ganglapppac) at (\ganglaxxxa, \ganglayyyc);
\coordinate (ganglapppad) at (\ganglaxxxa, \ganglayyyd);
\coordinate (ganglapppae) at (\ganglaxxxa, \ganglayyye);
\coordinate (ganglapppaf) at (\ganglaxxxa, \ganglayyyf);
\coordinate (ganglapppag) at (\ganglaxxxa, \ganglayyyg);
\coordinate (ganglapppah) at (\ganglaxxxa, \ganglayyyh);
\coordinate (ganglapppai) at (\ganglaxxxa, \ganglayyyi);
\coordinate (ganglapppaj) at (\ganglaxxxa, \ganglayyyj);
\coordinate (ganglapppak) at (\ganglaxxxa, \ganglayyyk);
\coordinate (ganglapppal) at (\ganglaxxxa, \ganglayyyl);
\coordinate (ganglapppam) at (\ganglaxxxa, \ganglayyym);
\coordinate (ganglapppan) at (\ganglaxxxa, \ganglayyyn);
\coordinate (ganglapppao) at (\ganglaxxxa, \ganglayyyo);
\coordinate (ganglapppap) at (\ganglaxxxa, \ganglayyyp);
\coordinate (ganglapppaq) at (\ganglaxxxa, \ganglayyyq);
\coordinate (ganglapppar) at (\ganglaxxxa, \ganglayyyr);
\coordinate (ganglapppas) at (\ganglaxxxa, \ganglayyys);
\coordinate (ganglapppat) at (\ganglaxxxa, \ganglayyyt);
\coordinate (ganglapppau) at (\ganglaxxxa, \ganglayyyu);
\coordinate (ganglapppav) at (\ganglaxxxa, \ganglayyyv);
\coordinate (ganglapppaw) at (\ganglaxxxa, \ganglayyyw);
\coordinate (ganglapppax) at (\ganglaxxxa, \ganglayyyx);
\coordinate (ganglapppay) at (\ganglaxxxa, \ganglayyyy);
\coordinate (ganglapppaz) at (\ganglaxxxa, \ganglayyyz);
\coordinate (ganglapppba) at (\ganglaxxxb, \ganglayyya);
\coordinate (ganglapppbb) at (\ganglaxxxb, \ganglayyyb);
\coordinate (ganglapppbc) at (\ganglaxxxb, \ganglayyyc);
\coordinate (ganglapppbd) at (\ganglaxxxb, \ganglayyyd);
\coordinate (ganglapppbe) at (\ganglaxxxb, \ganglayyye);
\coordinate (ganglapppbf) at (\ganglaxxxb, \ganglayyyf);
\coordinate (ganglapppbg) at (\ganglaxxxb, \ganglayyyg);
\coordinate (ganglapppbh) at (\ganglaxxxb, \ganglayyyh);
\coordinate (ganglapppbi) at (\ganglaxxxb, \ganglayyyi);
\coordinate (ganglapppbj) at (\ganglaxxxb, \ganglayyyj);
\coordinate (ganglapppbk) at (\ganglaxxxb, \ganglayyyk);
\coordinate (ganglapppbl) at (\ganglaxxxb, \ganglayyyl);
\coordinate (ganglapppbm) at (\ganglaxxxb, \ganglayyym);
\coordinate (ganglapppbn) at (\ganglaxxxb, \ganglayyyn);
\coordinate (ganglapppbo) at (\ganglaxxxb, \ganglayyyo);
\coordinate (ganglapppbp) at (\ganglaxxxb, \ganglayyyp);
\coordinate (ganglapppbq) at (\ganglaxxxb, \ganglayyyq);
\coordinate (ganglapppbr) at (\ganglaxxxb, \ganglayyyr);
\coordinate (ganglapppbs) at (\ganglaxxxb, \ganglayyys);
\coordinate (ganglapppbt) at (\ganglaxxxb, \ganglayyyt);
\coordinate (ganglapppbu) at (\ganglaxxxb, \ganglayyyu);
\coordinate (ganglapppbv) at (\ganglaxxxb, \ganglayyyv);
\coordinate (ganglapppbw) at (\ganglaxxxb, \ganglayyyw);
\coordinate (ganglapppbx) at (\ganglaxxxb, \ganglayyyx);
\coordinate (ganglapppby) at (\ganglaxxxb, \ganglayyyy);
\coordinate (ganglapppbz) at (\ganglaxxxb, \ganglayyyz);
\coordinate (ganglapppca) at (\ganglaxxxc, \ganglayyya);
\coordinate (ganglapppcb) at (\ganglaxxxc, \ganglayyyb);
\coordinate (ganglapppcc) at (\ganglaxxxc, \ganglayyyc);
\coordinate (ganglapppcd) at (\ganglaxxxc, \ganglayyyd);
\coordinate (ganglapppce) at (\ganglaxxxc, \ganglayyye);
\coordinate (ganglapppcf) at (\ganglaxxxc, \ganglayyyf);
\coordinate (ganglapppcg) at (\ganglaxxxc, \ganglayyyg);
\coordinate (ganglapppch) at (\ganglaxxxc, \ganglayyyh);
\coordinate (ganglapppci) at (\ganglaxxxc, \ganglayyyi);
\coordinate (ganglapppcj) at (\ganglaxxxc, \ganglayyyj);
\coordinate (ganglapppck) at (\ganglaxxxc, \ganglayyyk);
\coordinate (ganglapppcl) at (\ganglaxxxc, \ganglayyyl);
\coordinate (ganglapppcm) at (\ganglaxxxc, \ganglayyym);
\coordinate (ganglapppcn) at (\ganglaxxxc, \ganglayyyn);
\coordinate (ganglapppco) at (\ganglaxxxc, \ganglayyyo);
\coordinate (ganglapppcp) at (\ganglaxxxc, \ganglayyyp);
\coordinate (ganglapppcq) at (\ganglaxxxc, \ganglayyyq);
\coordinate (ganglapppcr) at (\ganglaxxxc, \ganglayyyr);
\coordinate (ganglapppcs) at (\ganglaxxxc, \ganglayyys);
\coordinate (ganglapppct) at (\ganglaxxxc, \ganglayyyt);
\coordinate (ganglapppcu) at (\ganglaxxxc, \ganglayyyu);
\coordinate (ganglapppcv) at (\ganglaxxxc, \ganglayyyv);
\coordinate (ganglapppcw) at (\ganglaxxxc, \ganglayyyw);
\coordinate (ganglapppcx) at (\ganglaxxxc, \ganglayyyx);
\coordinate (ganglapppcy) at (\ganglaxxxc, \ganglayyyy);
\coordinate (ganglapppcz) at (\ganglaxxxc, \ganglayyyz);
\coordinate (ganglapppda) at (\ganglaxxxd, \ganglayyya);
\coordinate (ganglapppdb) at (\ganglaxxxd, \ganglayyyb);
\coordinate (ganglapppdc) at (\ganglaxxxd, \ganglayyyc);
\coordinate (ganglapppdd) at (\ganglaxxxd, \ganglayyyd);
\coordinate (ganglapppde) at (\ganglaxxxd, \ganglayyye);
\coordinate (ganglapppdf) at (\ganglaxxxd, \ganglayyyf);
\coordinate (ganglapppdg) at (\ganglaxxxd, \ganglayyyg);
\coordinate (ganglapppdh) at (\ganglaxxxd, \ganglayyyh);
\coordinate (ganglapppdi) at (\ganglaxxxd, \ganglayyyi);
\coordinate (ganglapppdj) at (\ganglaxxxd, \ganglayyyj);
\coordinate (ganglapppdk) at (\ganglaxxxd, \ganglayyyk);
\coordinate (ganglapppdl) at (\ganglaxxxd, \ganglayyyl);
\coordinate (ganglapppdm) at (\ganglaxxxd, \ganglayyym);
\coordinate (ganglapppdn) at (\ganglaxxxd, \ganglayyyn);
\coordinate (ganglapppdo) at (\ganglaxxxd, \ganglayyyo);
\coordinate (ganglapppdp) at (\ganglaxxxd, \ganglayyyp);
\coordinate (ganglapppdq) at (\ganglaxxxd, \ganglayyyq);
\coordinate (ganglapppdr) at (\ganglaxxxd, \ganglayyyr);
\coordinate (ganglapppds) at (\ganglaxxxd, \ganglayyys);
\coordinate (ganglapppdt) at (\ganglaxxxd, \ganglayyyt);
\coordinate (ganglapppdu) at (\ganglaxxxd, \ganglayyyu);
\coordinate (ganglapppdv) at (\ganglaxxxd, \ganglayyyv);
\coordinate (ganglapppdw) at (\ganglaxxxd, \ganglayyyw);
\coordinate (ganglapppdx) at (\ganglaxxxd, \ganglayyyx);
\coordinate (ganglapppdy) at (\ganglaxxxd, \ganglayyyy);
\coordinate (ganglapppdz) at (\ganglaxxxd, \ganglayyyz);
\coordinate (ganglapppea) at (\ganglaxxxe, \ganglayyya);
\coordinate (ganglapppeb) at (\ganglaxxxe, \ganglayyyb);
\coordinate (ganglapppec) at (\ganglaxxxe, \ganglayyyc);
\coordinate (ganglappped) at (\ganglaxxxe, \ganglayyyd);
\coordinate (ganglapppee) at (\ganglaxxxe, \ganglayyye);
\coordinate (ganglapppef) at (\ganglaxxxe, \ganglayyyf);
\coordinate (ganglapppeg) at (\ganglaxxxe, \ganglayyyg);
\coordinate (ganglapppeh) at (\ganglaxxxe, \ganglayyyh);
\coordinate (ganglapppei) at (\ganglaxxxe, \ganglayyyi);
\coordinate (ganglapppej) at (\ganglaxxxe, \ganglayyyj);
\coordinate (ganglapppek) at (\ganglaxxxe, \ganglayyyk);
\coordinate (ganglapppel) at (\ganglaxxxe, \ganglayyyl);
\coordinate (ganglapppem) at (\ganglaxxxe, \ganglayyym);
\coordinate (ganglapppen) at (\ganglaxxxe, \ganglayyyn);
\coordinate (ganglapppeo) at (\ganglaxxxe, \ganglayyyo);
\coordinate (ganglapppep) at (\ganglaxxxe, \ganglayyyp);
\coordinate (ganglapppeq) at (\ganglaxxxe, \ganglayyyq);
\coordinate (ganglappper) at (\ganglaxxxe, \ganglayyyr);
\coordinate (ganglapppes) at (\ganglaxxxe, \ganglayyys);
\coordinate (ganglapppet) at (\ganglaxxxe, \ganglayyyt);
\coordinate (ganglapppeu) at (\ganglaxxxe, \ganglayyyu);
\coordinate (ganglapppev) at (\ganglaxxxe, \ganglayyyv);
\coordinate (ganglapppew) at (\ganglaxxxe, \ganglayyyw);
\coordinate (ganglapppex) at (\ganglaxxxe, \ganglayyyx);
\coordinate (ganglapppey) at (\ganglaxxxe, \ganglayyyy);
\coordinate (ganglapppez) at (\ganglaxxxe, \ganglayyyz);
\coordinate (ganglapppfa) at (\ganglaxxxf, \ganglayyya);
\coordinate (ganglapppfb) at (\ganglaxxxf, \ganglayyyb);
\coordinate (ganglapppfc) at (\ganglaxxxf, \ganglayyyc);
\coordinate (ganglapppfd) at (\ganglaxxxf, \ganglayyyd);
\coordinate (ganglapppfe) at (\ganglaxxxf, \ganglayyye);
\coordinate (ganglapppff) at (\ganglaxxxf, \ganglayyyf);
\coordinate (ganglapppfg) at (\ganglaxxxf, \ganglayyyg);
\coordinate (ganglapppfh) at (\ganglaxxxf, \ganglayyyh);
\coordinate (ganglapppfi) at (\ganglaxxxf, \ganglayyyi);
\coordinate (ganglapppfj) at (\ganglaxxxf, \ganglayyyj);
\coordinate (ganglapppfk) at (\ganglaxxxf, \ganglayyyk);
\coordinate (ganglapppfl) at (\ganglaxxxf, \ganglayyyl);
\coordinate (ganglapppfm) at (\ganglaxxxf, \ganglayyym);
\coordinate (ganglapppfn) at (\ganglaxxxf, \ganglayyyn);
\coordinate (ganglapppfo) at (\ganglaxxxf, \ganglayyyo);
\coordinate (ganglapppfp) at (\ganglaxxxf, \ganglayyyp);
\coordinate (ganglapppfq) at (\ganglaxxxf, \ganglayyyq);
\coordinate (ganglapppfr) at (\ganglaxxxf, \ganglayyyr);
\coordinate (ganglapppfs) at (\ganglaxxxf, \ganglayyys);
\coordinate (ganglapppft) at (\ganglaxxxf, \ganglayyyt);
\coordinate (ganglapppfu) at (\ganglaxxxf, \ganglayyyu);
\coordinate (ganglapppfv) at (\ganglaxxxf, \ganglayyyv);
\coordinate (ganglapppfw) at (\ganglaxxxf, \ganglayyyw);
\coordinate (ganglapppfx) at (\ganglaxxxf, \ganglayyyx);
\coordinate (ganglapppfy) at (\ganglaxxxf, \ganglayyyy);
\coordinate (ganglapppfz) at (\ganglaxxxf, \ganglayyyz);
\coordinate (ganglapppga) at (\ganglaxxxg, \ganglayyya);
\coordinate (ganglapppgb) at (\ganglaxxxg, \ganglayyyb);
\coordinate (ganglapppgc) at (\ganglaxxxg, \ganglayyyc);
\coordinate (ganglapppgd) at (\ganglaxxxg, \ganglayyyd);
\coordinate (ganglapppge) at (\ganglaxxxg, \ganglayyye);
\coordinate (ganglapppgf) at (\ganglaxxxg, \ganglayyyf);
\coordinate (ganglapppgg) at (\ganglaxxxg, \ganglayyyg);
\coordinate (ganglapppgh) at (\ganglaxxxg, \ganglayyyh);
\coordinate (ganglapppgi) at (\ganglaxxxg, \ganglayyyi);
\coordinate (ganglapppgj) at (\ganglaxxxg, \ganglayyyj);
\coordinate (ganglapppgk) at (\ganglaxxxg, \ganglayyyk);
\coordinate (ganglapppgl) at (\ganglaxxxg, \ganglayyyl);
\coordinate (ganglapppgm) at (\ganglaxxxg, \ganglayyym);
\coordinate (ganglapppgn) at (\ganglaxxxg, \ganglayyyn);
\coordinate (ganglapppgo) at (\ganglaxxxg, \ganglayyyo);
\coordinate (ganglapppgp) at (\ganglaxxxg, \ganglayyyp);
\coordinate (ganglapppgq) at (\ganglaxxxg, \ganglayyyq);
\coordinate (ganglapppgr) at (\ganglaxxxg, \ganglayyyr);
\coordinate (ganglapppgs) at (\ganglaxxxg, \ganglayyys);
\coordinate (ganglapppgt) at (\ganglaxxxg, \ganglayyyt);
\coordinate (ganglapppgu) at (\ganglaxxxg, \ganglayyyu);
\coordinate (ganglapppgv) at (\ganglaxxxg, \ganglayyyv);
\coordinate (ganglapppgw) at (\ganglaxxxg, \ganglayyyw);
\coordinate (ganglapppgx) at (\ganglaxxxg, \ganglayyyx);
\coordinate (ganglapppgy) at (\ganglaxxxg, \ganglayyyy);
\coordinate (ganglapppgz) at (\ganglaxxxg, \ganglayyyz);
\coordinate (ganglapppha) at (\ganglaxxxh, \ganglayyya);
\coordinate (ganglappphb) at (\ganglaxxxh, \ganglayyyb);
\coordinate (ganglappphc) at (\ganglaxxxh, \ganglayyyc);
\coordinate (ganglappphd) at (\ganglaxxxh, \ganglayyyd);
\coordinate (ganglappphe) at (\ganglaxxxh, \ganglayyye);
\coordinate (ganglappphf) at (\ganglaxxxh, \ganglayyyf);
\coordinate (ganglappphg) at (\ganglaxxxh, \ganglayyyg);
\coordinate (ganglappphh) at (\ganglaxxxh, \ganglayyyh);
\coordinate (ganglappphi) at (\ganglaxxxh, \ganglayyyi);
\coordinate (ganglappphj) at (\ganglaxxxh, \ganglayyyj);
\coordinate (ganglappphk) at (\ganglaxxxh, \ganglayyyk);
\coordinate (ganglappphl) at (\ganglaxxxh, \ganglayyyl);
\coordinate (ganglappphm) at (\ganglaxxxh, \ganglayyym);
\coordinate (ganglappphn) at (\ganglaxxxh, \ganglayyyn);
\coordinate (ganglapppho) at (\ganglaxxxh, \ganglayyyo);
\coordinate (ganglappphp) at (\ganglaxxxh, \ganglayyyp);
\coordinate (ganglappphq) at (\ganglaxxxh, \ganglayyyq);
\coordinate (ganglappphr) at (\ganglaxxxh, \ganglayyyr);
\coordinate (ganglappphs) at (\ganglaxxxh, \ganglayyys);
\coordinate (ganglapppht) at (\ganglaxxxh, \ganglayyyt);
\coordinate (ganglappphu) at (\ganglaxxxh, \ganglayyyu);
\coordinate (ganglappphv) at (\ganglaxxxh, \ganglayyyv);
\coordinate (ganglappphw) at (\ganglaxxxh, \ganglayyyw);
\coordinate (ganglappphx) at (\ganglaxxxh, \ganglayyyx);
\coordinate (ganglappphy) at (\ganglaxxxh, \ganglayyyy);
\coordinate (ganglappphz) at (\ganglaxxxh, \ganglayyyz);
\coordinate (ganglapppia) at (\ganglaxxxi, \ganglayyya);
\coordinate (ganglapppib) at (\ganglaxxxi, \ganglayyyb);
\coordinate (ganglapppic) at (\ganglaxxxi, \ganglayyyc);
\coordinate (ganglapppid) at (\ganglaxxxi, \ganglayyyd);
\coordinate (ganglapppie) at (\ganglaxxxi, \ganglayyye);
\coordinate (ganglapppif) at (\ganglaxxxi, \ganglayyyf);
\coordinate (ganglapppig) at (\ganglaxxxi, \ganglayyyg);
\coordinate (ganglapppih) at (\ganglaxxxi, \ganglayyyh);
\coordinate (ganglapppii) at (\ganglaxxxi, \ganglayyyi);
\coordinate (ganglapppij) at (\ganglaxxxi, \ganglayyyj);
\coordinate (ganglapppik) at (\ganglaxxxi, \ganglayyyk);
\coordinate (ganglapppil) at (\ganglaxxxi, \ganglayyyl);
\coordinate (ganglapppim) at (\ganglaxxxi, \ganglayyym);
\coordinate (ganglapppin) at (\ganglaxxxi, \ganglayyyn);
\coordinate (ganglapppio) at (\ganglaxxxi, \ganglayyyo);
\coordinate (ganglapppip) at (\ganglaxxxi, \ganglayyyp);
\coordinate (ganglapppiq) at (\ganglaxxxi, \ganglayyyq);
\coordinate (ganglapppir) at (\ganglaxxxi, \ganglayyyr);
\coordinate (ganglapppis) at (\ganglaxxxi, \ganglayyys);
\coordinate (ganglapppit) at (\ganglaxxxi, \ganglayyyt);
\coordinate (ganglapppiu) at (\ganglaxxxi, \ganglayyyu);
\coordinate (ganglapppiv) at (\ganglaxxxi, \ganglayyyv);
\coordinate (ganglapppiw) at (\ganglaxxxi, \ganglayyyw);
\coordinate (ganglapppix) at (\ganglaxxxi, \ganglayyyx);
\coordinate (ganglapppiy) at (\ganglaxxxi, \ganglayyyy);
\coordinate (ganglapppiz) at (\ganglaxxxi, \ganglayyyz);
\coordinate (ganglapppja) at (\ganglaxxxj, \ganglayyya);
\coordinate (ganglapppjb) at (\ganglaxxxj, \ganglayyyb);
\coordinate (ganglapppjc) at (\ganglaxxxj, \ganglayyyc);
\coordinate (ganglapppjd) at (\ganglaxxxj, \ganglayyyd);
\coordinate (ganglapppje) at (\ganglaxxxj, \ganglayyye);
\coordinate (ganglapppjf) at (\ganglaxxxj, \ganglayyyf);
\coordinate (ganglapppjg) at (\ganglaxxxj, \ganglayyyg);
\coordinate (ganglapppjh) at (\ganglaxxxj, \ganglayyyh);
\coordinate (ganglapppji) at (\ganglaxxxj, \ganglayyyi);
\coordinate (ganglapppjj) at (\ganglaxxxj, \ganglayyyj);
\coordinate (ganglapppjk) at (\ganglaxxxj, \ganglayyyk);
\coordinate (ganglapppjl) at (\ganglaxxxj, \ganglayyyl);
\coordinate (ganglapppjm) at (\ganglaxxxj, \ganglayyym);
\coordinate (ganglapppjn) at (\ganglaxxxj, \ganglayyyn);
\coordinate (ganglapppjo) at (\ganglaxxxj, \ganglayyyo);
\coordinate (ganglapppjp) at (\ganglaxxxj, \ganglayyyp);
\coordinate (ganglapppjq) at (\ganglaxxxj, \ganglayyyq);
\coordinate (ganglapppjr) at (\ganglaxxxj, \ganglayyyr);
\coordinate (ganglapppjs) at (\ganglaxxxj, \ganglayyys);
\coordinate (ganglapppjt) at (\ganglaxxxj, \ganglayyyt);
\coordinate (ganglapppju) at (\ganglaxxxj, \ganglayyyu);
\coordinate (ganglapppjv) at (\ganglaxxxj, \ganglayyyv);
\coordinate (ganglapppjw) at (\ganglaxxxj, \ganglayyyw);
\coordinate (ganglapppjx) at (\ganglaxxxj, \ganglayyyx);
\coordinate (ganglapppjy) at (\ganglaxxxj, \ganglayyyy);
\coordinate (ganglapppjz) at (\ganglaxxxj, \ganglayyyz);
\coordinate (ganglapppka) at (\ganglaxxxk, \ganglayyya);
\coordinate (ganglapppkb) at (\ganglaxxxk, \ganglayyyb);
\coordinate (ganglapppkc) at (\ganglaxxxk, \ganglayyyc);
\coordinate (ganglapppkd) at (\ganglaxxxk, \ganglayyyd);
\coordinate (ganglapppke) at (\ganglaxxxk, \ganglayyye);
\coordinate (ganglapppkf) at (\ganglaxxxk, \ganglayyyf);
\coordinate (ganglapppkg) at (\ganglaxxxk, \ganglayyyg);
\coordinate (ganglapppkh) at (\ganglaxxxk, \ganglayyyh);
\coordinate (ganglapppki) at (\ganglaxxxk, \ganglayyyi);
\coordinate (ganglapppkj) at (\ganglaxxxk, \ganglayyyj);
\coordinate (ganglapppkk) at (\ganglaxxxk, \ganglayyyk);
\coordinate (ganglapppkl) at (\ganglaxxxk, \ganglayyyl);
\coordinate (ganglapppkm) at (\ganglaxxxk, \ganglayyym);
\coordinate (ganglapppkn) at (\ganglaxxxk, \ganglayyyn);
\coordinate (ganglapppko) at (\ganglaxxxk, \ganglayyyo);
\coordinate (ganglapppkp) at (\ganglaxxxk, \ganglayyyp);
\coordinate (ganglapppkq) at (\ganglaxxxk, \ganglayyyq);
\coordinate (ganglapppkr) at (\ganglaxxxk, \ganglayyyr);
\coordinate (ganglapppks) at (\ganglaxxxk, \ganglayyys);
\coordinate (ganglapppkt) at (\ganglaxxxk, \ganglayyyt);
\coordinate (ganglapppku) at (\ganglaxxxk, \ganglayyyu);
\coordinate (ganglapppkv) at (\ganglaxxxk, \ganglayyyv);
\coordinate (ganglapppkw) at (\ganglaxxxk, \ganglayyyw);
\coordinate (ganglapppkx) at (\ganglaxxxk, \ganglayyyx);
\coordinate (ganglapppky) at (\ganglaxxxk, \ganglayyyy);
\coordinate (ganglapppkz) at (\ganglaxxxk, \ganglayyyz);
\coordinate (ganglapppla) at (\ganglaxxxl, \ganglayyya);
\coordinate (ganglappplb) at (\ganglaxxxl, \ganglayyyb);
\coordinate (ganglappplc) at (\ganglaxxxl, \ganglayyyc);
\coordinate (ganglapppld) at (\ganglaxxxl, \ganglayyyd);
\coordinate (ganglappple) at (\ganglaxxxl, \ganglayyye);
\coordinate (ganglappplf) at (\ganglaxxxl, \ganglayyyf);
\coordinate (ganglappplg) at (\ganglaxxxl, \ganglayyyg);
\coordinate (ganglappplh) at (\ganglaxxxl, \ganglayyyh);
\coordinate (ganglapppli) at (\ganglaxxxl, \ganglayyyi);
\coordinate (ganglappplj) at (\ganglaxxxl, \ganglayyyj);
\coordinate (ganglappplk) at (\ganglaxxxl, \ganglayyyk);
\coordinate (ganglapppll) at (\ganglaxxxl, \ganglayyyl);
\coordinate (ganglappplm) at (\ganglaxxxl, \ganglayyym);
\coordinate (ganglapppln) at (\ganglaxxxl, \ganglayyyn);
\coordinate (ganglappplo) at (\ganglaxxxl, \ganglayyyo);
\coordinate (ganglappplp) at (\ganglaxxxl, \ganglayyyp);
\coordinate (ganglappplq) at (\ganglaxxxl, \ganglayyyq);
\coordinate (ganglappplr) at (\ganglaxxxl, \ganglayyyr);
\coordinate (ganglapppls) at (\ganglaxxxl, \ganglayyys);
\coordinate (ganglappplt) at (\ganglaxxxl, \ganglayyyt);
\coordinate (ganglappplu) at (\ganglaxxxl, \ganglayyyu);
\coordinate (ganglappplv) at (\ganglaxxxl, \ganglayyyv);
\coordinate (ganglappplw) at (\ganglaxxxl, \ganglayyyw);
\coordinate (ganglappplx) at (\ganglaxxxl, \ganglayyyx);
\coordinate (ganglappply) at (\ganglaxxxl, \ganglayyyy);
\coordinate (ganglappplz) at (\ganglaxxxl, \ganglayyyz);
\coordinate (ganglapppma) at (\ganglaxxxm, \ganglayyya);
\coordinate (ganglapppmb) at (\ganglaxxxm, \ganglayyyb);
\coordinate (ganglapppmc) at (\ganglaxxxm, \ganglayyyc);
\coordinate (ganglapppmd) at (\ganglaxxxm, \ganglayyyd);
\coordinate (ganglapppme) at (\ganglaxxxm, \ganglayyye);
\coordinate (ganglapppmf) at (\ganglaxxxm, \ganglayyyf);
\coordinate (ganglapppmg) at (\ganglaxxxm, \ganglayyyg);
\coordinate (ganglapppmh) at (\ganglaxxxm, \ganglayyyh);
\coordinate (ganglapppmi) at (\ganglaxxxm, \ganglayyyi);
\coordinate (ganglapppmj) at (\ganglaxxxm, \ganglayyyj);
\coordinate (ganglapppmk) at (\ganglaxxxm, \ganglayyyk);
\coordinate (ganglapppml) at (\ganglaxxxm, \ganglayyyl);
\coordinate (ganglapppmm) at (\ganglaxxxm, \ganglayyym);
\coordinate (ganglapppmn) at (\ganglaxxxm, \ganglayyyn);
\coordinate (ganglapppmo) at (\ganglaxxxm, \ganglayyyo);
\coordinate (ganglapppmp) at (\ganglaxxxm, \ganglayyyp);
\coordinate (ganglapppmq) at (\ganglaxxxm, \ganglayyyq);
\coordinate (ganglapppmr) at (\ganglaxxxm, \ganglayyyr);
\coordinate (ganglapppms) at (\ganglaxxxm, \ganglayyys);
\coordinate (ganglapppmt) at (\ganglaxxxm, \ganglayyyt);
\coordinate (ganglapppmu) at (\ganglaxxxm, \ganglayyyu);
\coordinate (ganglapppmv) at (\ganglaxxxm, \ganglayyyv);
\coordinate (ganglapppmw) at (\ganglaxxxm, \ganglayyyw);
\coordinate (ganglapppmx) at (\ganglaxxxm, \ganglayyyx);
\coordinate (ganglapppmy) at (\ganglaxxxm, \ganglayyyy);
\coordinate (ganglapppmz) at (\ganglaxxxm, \ganglayyyz);
\coordinate (ganglapppna) at (\ganglaxxxn, \ganglayyya);
\coordinate (ganglapppnb) at (\ganglaxxxn, \ganglayyyb);
\coordinate (ganglapppnc) at (\ganglaxxxn, \ganglayyyc);
\coordinate (ganglapppnd) at (\ganglaxxxn, \ganglayyyd);
\coordinate (ganglapppne) at (\ganglaxxxn, \ganglayyye);
\coordinate (ganglapppnf) at (\ganglaxxxn, \ganglayyyf);
\coordinate (ganglapppng) at (\ganglaxxxn, \ganglayyyg);
\coordinate (ganglapppnh) at (\ganglaxxxn, \ganglayyyh);
\coordinate (ganglapppni) at (\ganglaxxxn, \ganglayyyi);
\coordinate (ganglapppnj) at (\ganglaxxxn, \ganglayyyj);
\coordinate (ganglapppnk) at (\ganglaxxxn, \ganglayyyk);
\coordinate (ganglapppnl) at (\ganglaxxxn, \ganglayyyl);
\coordinate (ganglapppnm) at (\ganglaxxxn, \ganglayyym);
\coordinate (ganglapppnn) at (\ganglaxxxn, \ganglayyyn);
\coordinate (ganglapppno) at (\ganglaxxxn, \ganglayyyo);
\coordinate (ganglapppnp) at (\ganglaxxxn, \ganglayyyp);
\coordinate (ganglapppnq) at (\ganglaxxxn, \ganglayyyq);
\coordinate (ganglapppnr) at (\ganglaxxxn, \ganglayyyr);
\coordinate (ganglapppns) at (\ganglaxxxn, \ganglayyys);
\coordinate (ganglapppnt) at (\ganglaxxxn, \ganglayyyt);
\coordinate (ganglapppnu) at (\ganglaxxxn, \ganglayyyu);
\coordinate (ganglapppnv) at (\ganglaxxxn, \ganglayyyv);
\coordinate (ganglapppnw) at (\ganglaxxxn, \ganglayyyw);
\coordinate (ganglapppnx) at (\ganglaxxxn, \ganglayyyx);
\coordinate (ganglapppny) at (\ganglaxxxn, \ganglayyyy);
\coordinate (ganglapppnz) at (\ganglaxxxn, \ganglayyyz);
\coordinate (ganglapppoa) at (\ganglaxxxo, \ganglayyya);
\coordinate (ganglapppob) at (\ganglaxxxo, \ganglayyyb);
\coordinate (ganglapppoc) at (\ganglaxxxo, \ganglayyyc);
\coordinate (ganglapppod) at (\ganglaxxxo, \ganglayyyd);
\coordinate (ganglapppoe) at (\ganglaxxxo, \ganglayyye);
\coordinate (ganglapppof) at (\ganglaxxxo, \ganglayyyf);
\coordinate (ganglapppog) at (\ganglaxxxo, \ganglayyyg);
\coordinate (ganglapppoh) at (\ganglaxxxo, \ganglayyyh);
\coordinate (ganglapppoi) at (\ganglaxxxo, \ganglayyyi);
\coordinate (ganglapppoj) at (\ganglaxxxo, \ganglayyyj);
\coordinate (ganglapppok) at (\ganglaxxxo, \ganglayyyk);
\coordinate (ganglapppol) at (\ganglaxxxo, \ganglayyyl);
\coordinate (ganglapppom) at (\ganglaxxxo, \ganglayyym);
\coordinate (ganglapppon) at (\ganglaxxxo, \ganglayyyn);
\coordinate (ganglapppoo) at (\ganglaxxxo, \ganglayyyo);
\coordinate (ganglapppop) at (\ganglaxxxo, \ganglayyyp);
\coordinate (ganglapppoq) at (\ganglaxxxo, \ganglayyyq);
\coordinate (ganglapppor) at (\ganglaxxxo, \ganglayyyr);
\coordinate (ganglapppos) at (\ganglaxxxo, \ganglayyys);
\coordinate (ganglapppot) at (\ganglaxxxo, \ganglayyyt);
\coordinate (ganglapppou) at (\ganglaxxxo, \ganglayyyu);
\coordinate (ganglapppov) at (\ganglaxxxo, \ganglayyyv);
\coordinate (ganglapppow) at (\ganglaxxxo, \ganglayyyw);
\coordinate (ganglapppox) at (\ganglaxxxo, \ganglayyyx);
\coordinate (ganglapppoy) at (\ganglaxxxo, \ganglayyyy);
\coordinate (ganglapppoz) at (\ganglaxxxo, \ganglayyyz);
\coordinate (ganglappppa) at (\ganglaxxxp, \ganglayyya);
\coordinate (ganglappppb) at (\ganglaxxxp, \ganglayyyb);
\coordinate (ganglappppc) at (\ganglaxxxp, \ganglayyyc);
\coordinate (ganglappppd) at (\ganglaxxxp, \ganglayyyd);
\coordinate (ganglappppe) at (\ganglaxxxp, \ganglayyye);
\coordinate (ganglappppf) at (\ganglaxxxp, \ganglayyyf);
\coordinate (ganglappppg) at (\ganglaxxxp, \ganglayyyg);
\coordinate (ganglapppph) at (\ganglaxxxp, \ganglayyyh);
\coordinate (ganglappppi) at (\ganglaxxxp, \ganglayyyi);
\coordinate (ganglappppj) at (\ganglaxxxp, \ganglayyyj);
\coordinate (ganglappppk) at (\ganglaxxxp, \ganglayyyk);
\coordinate (ganglappppl) at (\ganglaxxxp, \ganglayyyl);
\coordinate (ganglappppm) at (\ganglaxxxp, \ganglayyym);
\coordinate (ganglappppn) at (\ganglaxxxp, \ganglayyyn);
\coordinate (ganglappppo) at (\ganglaxxxp, \ganglayyyo);
\coordinate (ganglappppp) at (\ganglaxxxp, \ganglayyyp);
\coordinate (ganglappppq) at (\ganglaxxxp, \ganglayyyq);
\coordinate (ganglappppr) at (\ganglaxxxp, \ganglayyyr);
\coordinate (ganglapppps) at (\ganglaxxxp, \ganglayyys);
\coordinate (ganglappppt) at (\ganglaxxxp, \ganglayyyt);
\coordinate (ganglappppu) at (\ganglaxxxp, \ganglayyyu);
\coordinate (ganglappppv) at (\ganglaxxxp, \ganglayyyv);
\coordinate (ganglappppw) at (\ganglaxxxp, \ganglayyyw);
\coordinate (ganglappppx) at (\ganglaxxxp, \ganglayyyx);
\coordinate (ganglappppy) at (\ganglaxxxp, \ganglayyyy);
\coordinate (ganglappppz) at (\ganglaxxxp, \ganglayyyz);
\coordinate (ganglapppqa) at (\ganglaxxxq, \ganglayyya);
\coordinate (ganglapppqb) at (\ganglaxxxq, \ganglayyyb);
\coordinate (ganglapppqc) at (\ganglaxxxq, \ganglayyyc);
\coordinate (ganglapppqd) at (\ganglaxxxq, \ganglayyyd);
\coordinate (ganglapppqe) at (\ganglaxxxq, \ganglayyye);
\coordinate (ganglapppqf) at (\ganglaxxxq, \ganglayyyf);
\coordinate (ganglapppqg) at (\ganglaxxxq, \ganglayyyg);
\coordinate (ganglapppqh) at (\ganglaxxxq, \ganglayyyh);
\coordinate (ganglapppqi) at (\ganglaxxxq, \ganglayyyi);
\coordinate (ganglapppqj) at (\ganglaxxxq, \ganglayyyj);
\coordinate (ganglapppqk) at (\ganglaxxxq, \ganglayyyk);
\coordinate (ganglapppql) at (\ganglaxxxq, \ganglayyyl);
\coordinate (ganglapppqm) at (\ganglaxxxq, \ganglayyym);
\coordinate (ganglapppqn) at (\ganglaxxxq, \ganglayyyn);
\coordinate (ganglapppqo) at (\ganglaxxxq, \ganglayyyo);
\coordinate (ganglapppqp) at (\ganglaxxxq, \ganglayyyp);
\coordinate (ganglapppqq) at (\ganglaxxxq, \ganglayyyq);
\coordinate (ganglapppqr) at (\ganglaxxxq, \ganglayyyr);
\coordinate (ganglapppqs) at (\ganglaxxxq, \ganglayyys);
\coordinate (ganglapppqt) at (\ganglaxxxq, \ganglayyyt);
\coordinate (ganglapppqu) at (\ganglaxxxq, \ganglayyyu);
\coordinate (ganglapppqv) at (\ganglaxxxq, \ganglayyyv);
\coordinate (ganglapppqw) at (\ganglaxxxq, \ganglayyyw);
\coordinate (ganglapppqx) at (\ganglaxxxq, \ganglayyyx);
\coordinate (ganglapppqy) at (\ganglaxxxq, \ganglayyyy);
\coordinate (ganglapppqz) at (\ganglaxxxq, \ganglayyyz);
\coordinate (ganglapppra) at (\ganglaxxxr, \ganglayyya);
\coordinate (ganglappprb) at (\ganglaxxxr, \ganglayyyb);
\coordinate (ganglappprc) at (\ganglaxxxr, \ganglayyyc);
\coordinate (ganglappprd) at (\ganglaxxxr, \ganglayyyd);
\coordinate (ganglapppre) at (\ganglaxxxr, \ganglayyye);
\coordinate (ganglappprf) at (\ganglaxxxr, \ganglayyyf);
\coordinate (ganglappprg) at (\ganglaxxxr, \ganglayyyg);
\coordinate (ganglappprh) at (\ganglaxxxr, \ganglayyyh);
\coordinate (ganglapppri) at (\ganglaxxxr, \ganglayyyi);
\coordinate (ganglappprj) at (\ganglaxxxr, \ganglayyyj);
\coordinate (ganglappprk) at (\ganglaxxxr, \ganglayyyk);
\coordinate (ganglappprl) at (\ganglaxxxr, \ganglayyyl);
\coordinate (ganglappprm) at (\ganglaxxxr, \ganglayyym);
\coordinate (ganglappprn) at (\ganglaxxxr, \ganglayyyn);
\coordinate (ganglapppro) at (\ganglaxxxr, \ganglayyyo);
\coordinate (ganglappprp) at (\ganglaxxxr, \ganglayyyp);
\coordinate (ganglappprq) at (\ganglaxxxr, \ganglayyyq);
\coordinate (ganglappprr) at (\ganglaxxxr, \ganglayyyr);
\coordinate (ganglappprs) at (\ganglaxxxr, \ganglayyys);
\coordinate (ganglappprt) at (\ganglaxxxr, \ganglayyyt);
\coordinate (ganglapppru) at (\ganglaxxxr, \ganglayyyu);
\coordinate (ganglappprv) at (\ganglaxxxr, \ganglayyyv);
\coordinate (ganglappprw) at (\ganglaxxxr, \ganglayyyw);
\coordinate (ganglappprx) at (\ganglaxxxr, \ganglayyyx);
\coordinate (ganglapppry) at (\ganglaxxxr, \ganglayyyy);
\coordinate (ganglappprz) at (\ganglaxxxr, \ganglayyyz);
\coordinate (ganglapppsa) at (\ganglaxxxs, \ganglayyya);
\coordinate (ganglapppsb) at (\ganglaxxxs, \ganglayyyb);
\coordinate (ganglapppsc) at (\ganglaxxxs, \ganglayyyc);
\coordinate (ganglapppsd) at (\ganglaxxxs, \ganglayyyd);
\coordinate (ganglapppse) at (\ganglaxxxs, \ganglayyye);
\coordinate (ganglapppsf) at (\ganglaxxxs, \ganglayyyf);
\coordinate (ganglapppsg) at (\ganglaxxxs, \ganglayyyg);
\coordinate (ganglapppsh) at (\ganglaxxxs, \ganglayyyh);
\coordinate (ganglapppsi) at (\ganglaxxxs, \ganglayyyi);
\coordinate (ganglapppsj) at (\ganglaxxxs, \ganglayyyj);
\coordinate (ganglapppsk) at (\ganglaxxxs, \ganglayyyk);
\coordinate (ganglapppsl) at (\ganglaxxxs, \ganglayyyl);
\coordinate (ganglapppsm) at (\ganglaxxxs, \ganglayyym);
\coordinate (ganglapppsn) at (\ganglaxxxs, \ganglayyyn);
\coordinate (ganglapppso) at (\ganglaxxxs, \ganglayyyo);
\coordinate (ganglapppsp) at (\ganglaxxxs, \ganglayyyp);
\coordinate (ganglapppsq) at (\ganglaxxxs, \ganglayyyq);
\coordinate (ganglapppsr) at (\ganglaxxxs, \ganglayyyr);
\coordinate (ganglapppss) at (\ganglaxxxs, \ganglayyys);
\coordinate (ganglapppst) at (\ganglaxxxs, \ganglayyyt);
\coordinate (ganglapppsu) at (\ganglaxxxs, \ganglayyyu);
\coordinate (ganglapppsv) at (\ganglaxxxs, \ganglayyyv);
\coordinate (ganglapppsw) at (\ganglaxxxs, \ganglayyyw);
\coordinate (ganglapppsx) at (\ganglaxxxs, \ganglayyyx);
\coordinate (ganglapppsy) at (\ganglaxxxs, \ganglayyyy);
\coordinate (ganglapppsz) at (\ganglaxxxs, \ganglayyyz);
\coordinate (ganglapppta) at (\ganglaxxxt, \ganglayyya);
\coordinate (ganglappptb) at (\ganglaxxxt, \ganglayyyb);
\coordinate (ganglappptc) at (\ganglaxxxt, \ganglayyyc);
\coordinate (ganglappptd) at (\ganglaxxxt, \ganglayyyd);
\coordinate (ganglapppte) at (\ganglaxxxt, \ganglayyye);
\coordinate (ganglappptf) at (\ganglaxxxt, \ganglayyyf);
\coordinate (ganglappptg) at (\ganglaxxxt, \ganglayyyg);
\coordinate (ganglapppth) at (\ganglaxxxt, \ganglayyyh);
\coordinate (ganglapppti) at (\ganglaxxxt, \ganglayyyi);
\coordinate (ganglappptj) at (\ganglaxxxt, \ganglayyyj);
\coordinate (ganglappptk) at (\ganglaxxxt, \ganglayyyk);
\coordinate (ganglappptl) at (\ganglaxxxt, \ganglayyyl);
\coordinate (ganglappptm) at (\ganglaxxxt, \ganglayyym);
\coordinate (ganglappptn) at (\ganglaxxxt, \ganglayyyn);
\coordinate (ganglapppto) at (\ganglaxxxt, \ganglayyyo);
\coordinate (ganglappptp) at (\ganglaxxxt, \ganglayyyp);
\coordinate (ganglappptq) at (\ganglaxxxt, \ganglayyyq);
\coordinate (ganglappptr) at (\ganglaxxxt, \ganglayyyr);
\coordinate (ganglapppts) at (\ganglaxxxt, \ganglayyys);
\coordinate (ganglappptt) at (\ganglaxxxt, \ganglayyyt);
\coordinate (ganglappptu) at (\ganglaxxxt, \ganglayyyu);
\coordinate (ganglappptv) at (\ganglaxxxt, \ganglayyyv);
\coordinate (ganglappptw) at (\ganglaxxxt, \ganglayyyw);
\coordinate (ganglappptx) at (\ganglaxxxt, \ganglayyyx);
\coordinate (ganglapppty) at (\ganglaxxxt, \ganglayyyy);
\coordinate (ganglappptz) at (\ganglaxxxt, \ganglayyyz);
\coordinate (ganglapppua) at (\ganglaxxxu, \ganglayyya);
\coordinate (ganglapppub) at (\ganglaxxxu, \ganglayyyb);
\coordinate (ganglapppuc) at (\ganglaxxxu, \ganglayyyc);
\coordinate (ganglapppud) at (\ganglaxxxu, \ganglayyyd);
\coordinate (ganglapppue) at (\ganglaxxxu, \ganglayyye);
\coordinate (ganglapppuf) at (\ganglaxxxu, \ganglayyyf);
\coordinate (ganglapppug) at (\ganglaxxxu, \ganglayyyg);
\coordinate (ganglapppuh) at (\ganglaxxxu, \ganglayyyh);
\coordinate (ganglapppui) at (\ganglaxxxu, \ganglayyyi);
\coordinate (ganglapppuj) at (\ganglaxxxu, \ganglayyyj);
\coordinate (ganglapppuk) at (\ganglaxxxu, \ganglayyyk);
\coordinate (ganglapppul) at (\ganglaxxxu, \ganglayyyl);
\coordinate (ganglapppum) at (\ganglaxxxu, \ganglayyym);
\coordinate (ganglapppun) at (\ganglaxxxu, \ganglayyyn);
\coordinate (ganglapppuo) at (\ganglaxxxu, \ganglayyyo);
\coordinate (ganglapppup) at (\ganglaxxxu, \ganglayyyp);
\coordinate (ganglapppuq) at (\ganglaxxxu, \ganglayyyq);
\coordinate (ganglapppur) at (\ganglaxxxu, \ganglayyyr);
\coordinate (ganglapppus) at (\ganglaxxxu, \ganglayyys);
\coordinate (ganglappput) at (\ganglaxxxu, \ganglayyyt);
\coordinate (ganglapppuu) at (\ganglaxxxu, \ganglayyyu);
\coordinate (ganglapppuv) at (\ganglaxxxu, \ganglayyyv);
\coordinate (ganglapppuw) at (\ganglaxxxu, \ganglayyyw);
\coordinate (ganglapppux) at (\ganglaxxxu, \ganglayyyx);
\coordinate (ganglapppuy) at (\ganglaxxxu, \ganglayyyy);
\coordinate (ganglapppuz) at (\ganglaxxxu, \ganglayyyz);
\coordinate (ganglapppva) at (\ganglaxxxv, \ganglayyya);
\coordinate (ganglapppvb) at (\ganglaxxxv, \ganglayyyb);
\coordinate (ganglapppvc) at (\ganglaxxxv, \ganglayyyc);
\coordinate (ganglapppvd) at (\ganglaxxxv, \ganglayyyd);
\coordinate (ganglapppve) at (\ganglaxxxv, \ganglayyye);
\coordinate (ganglapppvf) at (\ganglaxxxv, \ganglayyyf);
\coordinate (ganglapppvg) at (\ganglaxxxv, \ganglayyyg);
\coordinate (ganglapppvh) at (\ganglaxxxv, \ganglayyyh);
\coordinate (ganglapppvi) at (\ganglaxxxv, \ganglayyyi);
\coordinate (ganglapppvj) at (\ganglaxxxv, \ganglayyyj);
\coordinate (ganglapppvk) at (\ganglaxxxv, \ganglayyyk);
\coordinate (ganglapppvl) at (\ganglaxxxv, \ganglayyyl);
\coordinate (ganglapppvm) at (\ganglaxxxv, \ganglayyym);
\coordinate (ganglapppvn) at (\ganglaxxxv, \ganglayyyn);
\coordinate (ganglapppvo) at (\ganglaxxxv, \ganglayyyo);
\coordinate (ganglapppvp) at (\ganglaxxxv, \ganglayyyp);
\coordinate (ganglapppvq) at (\ganglaxxxv, \ganglayyyq);
\coordinate (ganglapppvr) at (\ganglaxxxv, \ganglayyyr);
\coordinate (ganglapppvs) at (\ganglaxxxv, \ganglayyys);
\coordinate (ganglapppvt) at (\ganglaxxxv, \ganglayyyt);
\coordinate (ganglapppvu) at (\ganglaxxxv, \ganglayyyu);
\coordinate (ganglapppvv) at (\ganglaxxxv, \ganglayyyv);
\coordinate (ganglapppvw) at (\ganglaxxxv, \ganglayyyw);
\coordinate (ganglapppvx) at (\ganglaxxxv, \ganglayyyx);
\coordinate (ganglapppvy) at (\ganglaxxxv, \ganglayyyy);
\coordinate (ganglapppvz) at (\ganglaxxxv, \ganglayyyz);
\coordinate (ganglapppwa) at (\ganglaxxxw, \ganglayyya);
\coordinate (ganglapppwb) at (\ganglaxxxw, \ganglayyyb);
\coordinate (ganglapppwc) at (\ganglaxxxw, \ganglayyyc);
\coordinate (ganglapppwd) at (\ganglaxxxw, \ganglayyyd);
\coordinate (ganglapppwe) at (\ganglaxxxw, \ganglayyye);
\coordinate (ganglapppwf) at (\ganglaxxxw, \ganglayyyf);
\coordinate (ganglapppwg) at (\ganglaxxxw, \ganglayyyg);
\coordinate (ganglapppwh) at (\ganglaxxxw, \ganglayyyh);
\coordinate (ganglapppwi) at (\ganglaxxxw, \ganglayyyi);
\coordinate (ganglapppwj) at (\ganglaxxxw, \ganglayyyj);
\coordinate (ganglapppwk) at (\ganglaxxxw, \ganglayyyk);
\coordinate (ganglapppwl) at (\ganglaxxxw, \ganglayyyl);
\coordinate (ganglapppwm) at (\ganglaxxxw, \ganglayyym);
\coordinate (ganglapppwn) at (\ganglaxxxw, \ganglayyyn);
\coordinate (ganglapppwo) at (\ganglaxxxw, \ganglayyyo);
\coordinate (ganglapppwp) at (\ganglaxxxw, \ganglayyyp);
\coordinate (ganglapppwq) at (\ganglaxxxw, \ganglayyyq);
\coordinate (ganglapppwr) at (\ganglaxxxw, \ganglayyyr);
\coordinate (ganglapppws) at (\ganglaxxxw, \ganglayyys);
\coordinate (ganglapppwt) at (\ganglaxxxw, \ganglayyyt);
\coordinate (ganglapppwu) at (\ganglaxxxw, \ganglayyyu);
\coordinate (ganglapppwv) at (\ganglaxxxw, \ganglayyyv);
\coordinate (ganglapppww) at (\ganglaxxxw, \ganglayyyw);
\coordinate (ganglapppwx) at (\ganglaxxxw, \ganglayyyx);
\coordinate (ganglapppwy) at (\ganglaxxxw, \ganglayyyy);
\coordinate (ganglapppwz) at (\ganglaxxxw, \ganglayyyz);
\coordinate (ganglapppxa) at (\ganglaxxxx, \ganglayyya);
\coordinate (ganglapppxb) at (\ganglaxxxx, \ganglayyyb);
\coordinate (ganglapppxc) at (\ganglaxxxx, \ganglayyyc);
\coordinate (ganglapppxd) at (\ganglaxxxx, \ganglayyyd);
\coordinate (ganglapppxe) at (\ganglaxxxx, \ganglayyye);
\coordinate (ganglapppxf) at (\ganglaxxxx, \ganglayyyf);
\coordinate (ganglapppxg) at (\ganglaxxxx, \ganglayyyg);
\coordinate (ganglapppxh) at (\ganglaxxxx, \ganglayyyh);
\coordinate (ganglapppxi) at (\ganglaxxxx, \ganglayyyi);
\coordinate (ganglapppxj) at (\ganglaxxxx, \ganglayyyj);
\coordinate (ganglapppxk) at (\ganglaxxxx, \ganglayyyk);
\coordinate (ganglapppxl) at (\ganglaxxxx, \ganglayyyl);
\coordinate (ganglapppxm) at (\ganglaxxxx, \ganglayyym);
\coordinate (ganglapppxn) at (\ganglaxxxx, \ganglayyyn);
\coordinate (ganglapppxo) at (\ganglaxxxx, \ganglayyyo);
\coordinate (ganglapppxp) at (\ganglaxxxx, \ganglayyyp);
\coordinate (ganglapppxq) at (\ganglaxxxx, \ganglayyyq);
\coordinate (ganglapppxr) at (\ganglaxxxx, \ganglayyyr);
\coordinate (ganglapppxs) at (\ganglaxxxx, \ganglayyys);
\coordinate (ganglapppxt) at (\ganglaxxxx, \ganglayyyt);
\coordinate (ganglapppxu) at (\ganglaxxxx, \ganglayyyu);
\coordinate (ganglapppxv) at (\ganglaxxxx, \ganglayyyv);
\coordinate (ganglapppxw) at (\ganglaxxxx, \ganglayyyw);
\coordinate (ganglapppxx) at (\ganglaxxxx, \ganglayyyx);
\coordinate (ganglapppxy) at (\ganglaxxxx, \ganglayyyy);
\coordinate (ganglapppxz) at (\ganglaxxxx, \ganglayyyz);
\coordinate (ganglapppya) at (\ganglaxxxy, \ganglayyya);
\coordinate (ganglapppyb) at (\ganglaxxxy, \ganglayyyb);
\coordinate (ganglapppyc) at (\ganglaxxxy, \ganglayyyc);
\coordinate (ganglapppyd) at (\ganglaxxxy, \ganglayyyd);
\coordinate (ganglapppye) at (\ganglaxxxy, \ganglayyye);
\coordinate (ganglapppyf) at (\ganglaxxxy, \ganglayyyf);
\coordinate (ganglapppyg) at (\ganglaxxxy, \ganglayyyg);
\coordinate (ganglapppyh) at (\ganglaxxxy, \ganglayyyh);
\coordinate (ganglapppyi) at (\ganglaxxxy, \ganglayyyi);
\coordinate (ganglapppyj) at (\ganglaxxxy, \ganglayyyj);
\coordinate (ganglapppyk) at (\ganglaxxxy, \ganglayyyk);
\coordinate (ganglapppyl) at (\ganglaxxxy, \ganglayyyl);
\coordinate (ganglapppym) at (\ganglaxxxy, \ganglayyym);
\coordinate (ganglapppyn) at (\ganglaxxxy, \ganglayyyn);
\coordinate (ganglapppyo) at (\ganglaxxxy, \ganglayyyo);
\coordinate (ganglapppyp) at (\ganglaxxxy, \ganglayyyp);
\coordinate (ganglapppyq) at (\ganglaxxxy, \ganglayyyq);
\coordinate (ganglapppyr) at (\ganglaxxxy, \ganglayyyr);
\coordinate (ganglapppys) at (\ganglaxxxy, \ganglayyys);
\coordinate (ganglapppyt) at (\ganglaxxxy, \ganglayyyt);
\coordinate (ganglapppyu) at (\ganglaxxxy, \ganglayyyu);
\coordinate (ganglapppyv) at (\ganglaxxxy, \ganglayyyv);
\coordinate (ganglapppyw) at (\ganglaxxxy, \ganglayyyw);
\coordinate (ganglapppyx) at (\ganglaxxxy, \ganglayyyx);
\coordinate (ganglapppyy) at (\ganglaxxxy, \ganglayyyy);
\coordinate (ganglapppyz) at (\ganglaxxxy, \ganglayyyz);
\coordinate (ganglapppza) at (\ganglaxxxz, \ganglayyya);
\coordinate (ganglapppzb) at (\ganglaxxxz, \ganglayyyb);
\coordinate (ganglapppzc) at (\ganglaxxxz, \ganglayyyc);
\coordinate (ganglapppzd) at (\ganglaxxxz, \ganglayyyd);
\coordinate (ganglapppze) at (\ganglaxxxz, \ganglayyye);
\coordinate (ganglapppzf) at (\ganglaxxxz, \ganglayyyf);
\coordinate (ganglapppzg) at (\ganglaxxxz, \ganglayyyg);
\coordinate (ganglapppzh) at (\ganglaxxxz, \ganglayyyh);
\coordinate (ganglapppzi) at (\ganglaxxxz, \ganglayyyi);
\coordinate (ganglapppzj) at (\ganglaxxxz, \ganglayyyj);
\coordinate (ganglapppzk) at (\ganglaxxxz, \ganglayyyk);
\coordinate (ganglapppzl) at (\ganglaxxxz, \ganglayyyl);
\coordinate (ganglapppzm) at (\ganglaxxxz, \ganglayyym);
\coordinate (ganglapppzn) at (\ganglaxxxz, \ganglayyyn);
\coordinate (ganglapppzo) at (\ganglaxxxz, \ganglayyyo);
\coordinate (ganglapppzp) at (\ganglaxxxz, \ganglayyyp);
\coordinate (ganglapppzq) at (\ganglaxxxz, \ganglayyyq);
\coordinate (ganglapppzr) at (\ganglaxxxz, \ganglayyyr);
\coordinate (ganglapppzs) at (\ganglaxxxz, \ganglayyys);
\coordinate (ganglapppzt) at (\ganglaxxxz, \ganglayyyt);
\coordinate (ganglapppzu) at (\ganglaxxxz, \ganglayyyu);
\coordinate (ganglapppzv) at (\ganglaxxxz, \ganglayyyv);
\coordinate (ganglapppzw) at (\ganglaxxxz, \ganglayyyw);
\coordinate (ganglapppzx) at (\ganglaxxxz, \ganglayyyx);
\coordinate (ganglapppzy) at (\ganglaxxxz, \ganglayyyy);
\coordinate (ganglapppzz) at (\ganglaxxxz, \ganglayyyz);

%\gangprintcoordinateat{(0,0)}{The last coordinate values: }{($(ganglapppzz)$)}; 



% Draw related part of the coordinate system with dashed helplines with letters as background, which would help to determine all coordinates. 
\coordinatebackgroundxy{gangliu} {f}{g}{v} {f}{g}{q};

%%%%%% The next line is for fig. 3.
\coordinatebackgroundxy{gangla}{a}{b}{h} {a}{b}{h};

\draw [white] (gangliupppdp) -- (gangliupppgp);


% Draw the Opamp at the coordinate (gangliupppli) and name it as "myopamp".
\draw (gangliupppli) 
      node [op amp] (myopamp) {} ; 

% Retrieve the x- and y-components of the coordinates of the "+", "-", and "out" pins of myopamp, supposing we have no idea about them beforehand. 
\getxyingivenunit{cm}{(myopamp.+)}
                 {\myopamppx}{\myopamppy};
\getxyingivenunit{cm}{(myopamp.-)}
                 {\myopampnx}{\myopampny};
\getxyingivenunit{cm}{(myopamp.out)}
                 {\myopampox}{\myopampoy};

\draw (myopamp.-) -| (gangliupppjj) 
      to [R, l_=$R_F \text{=} 300K \Omega$,
                label/align=rotate] 
      (gangliupppjm) 
%%%%%% removed for fig. 2: -| (gangliupppni)
      ;

\draw [-o] (myopamp.out) 
      to[short, xshift=1mm] 
      (\gangliuxxxq, \myopampoy) node [anchor=north, yshift=-1mm] {$V_0$};

\draw (\gangliuxxxj, \myopampny) -- 
      (\gangliuxxxi, \myopampny) 
      to [R, l_=$R \text{=} 100 K\Omega$]  (\gangliuxxxg, \myopampny) -- (gangliupppgi) node [ground]{};;

\draw [-o] (myopamp.+) 
      to[short, xshift=-1mm] 
      (\gangliuxxxj, \myopamppy) node [anchor=north, yshift=-1mm] {$V_i$};
      
      
      

%%%%%% The rest are added for fig. 2.      
%%%%%% The rest are added for fig. 2.      

\draw (gangliuppppi) |- (gangliupppno) --
      (gangliupppnn);

\draw (gangliupppnk) node [ground]{} --
      (gangliupppnl) 
      to[american potentiometer,n=mypot, 
         l_=$R_P \text{=} 56 K \Omega$, label/align=rotate] 
      (gangliupppnn);

%%%%%% Remove the following line for Fig. 3. 
%\draw (mypot.wiper) 
%      node [red, anchor=south east] {$V_A$} -| 
%      (gangliupppjm);


%%%%%% Add the rest lines for Fig. 3. 
\getxyingivenunit{cm}{(mypot.wiper)}
                 {\mypotwiperx}{\mypotwipery};

\draw (mypot.wiper) 
      node [red, anchor=south east] {$V_A$} -- (\ganglaxxxe, \mypotwipery);

\draw (\ganglaxxxe, \mypotwipery)  
      to[variable american resistor, n=resistora, 
         l=$R_D \text{=} 560K \Omega$] 
      (\ganglaxxxc, \mypotwipery) -- 
      (\ganglaxxxb, \mypotwipery) 
      node [red, anchor=south] {$V_B$};

\draw (\ganglaxxxb, \mypotwipery) -- 
      (ganglapppbd)
      to [C, l=$C_D \text{=} 100 \mu F$] 
      (ganglapppbb) node [ground] {};

\draw (gangliupppjm) -- (\ganglaxxxb, \mypotwipery);


\end{circuitikz}



\end{document}
