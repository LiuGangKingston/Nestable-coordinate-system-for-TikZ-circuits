%   This is an accessory  file for 
%   https://github.com/LiuGangKingston/Nestable-coordinate-system-for-Tikz-circuits.git
%            Version 1.0
%   free for non-commercial use.
%   Please send us emails for any problems/suggestions/comments.
%   Please be advised that none of us accept any responsibility
%   for any consequences arising out of the usage of this
%   software, especially for damage.
%   For usage, please refer to the README file and the following lines.
%   This code was written by
%        Gang Liu (gl.cell@outlook)
%                 (http://orcid.org/0000-0003-1575-9290)
%          and
%        Shiwei Huang (huang937@gmail.com)
%   Copyright (c) 2021
%
%
%  The following command is to get the x-component and y-component 
%  of a coordinate. The command is
%  \getxyofcoordinate{the coordinate}{x-component}{y-component};
\newcommand{\getxyofcoordinate}[3]{%
\coordinate (tempcoord) at ($#1$);
\path (tempcoord) node {};
\pgfgetlastxy{\tempx}{\tempy};
\pgfmathsetmacro{#2}{\tempx}
\pgfmathsetmacro{#3}{\tempy}
}


%  The following command is the same as above but for given unit.
%  The command is
%  \getxyingivenunit{the unit like cm}{the coordinate}{x-component}{y-component};
\newcommand{\getxyingivenunit}[4]{%
\coordinate (tempcoord) at (1#1,1#1);
\path (tempcoord) node {};
\pgfgetlastxy{\tempxunit}{\tempyunit};
\coordinate (tempcoord) at ($#2$);
\path (tempcoord) node {};
\pgfgetlastxy{\tempx}{\tempy};
\pgfmathsetmacro{#3}{\tempx/\tempxunit}
\pgfmathsetmacro{#4}{\tempy/\tempyunit}
}


%  The following command is to print the value of a coordinate with some words at the first coordinate postion 
%  The command is
%  \printcoordinateat{the first coordinate}{the words}{the coordinate};
\newcommand{\printcoordinateat}[3]{%
\getxyingivenunit{cm}{#3}{\tempxx}{\tempyy}
\node at #1 {#2 ($\tempxx$, $\tempyy$).};
}


%  The following command is to print a keyworded coordinate system as a background.
%  The command is
%  \coordinatebackground{the KEYWORD}
%                                            {the first letter in both x and y directions}
%                                       {the second letter in both x and y directions}
%                                             {the last letter in both x and y directions};
\newcommand{\coordinatebackground}[4]{
\pgfmathsetmacro{\colourpercent}{30}
\foreach \i in {#2,#3,...,#4} 
{\node [black!\colourpercent] at (#1ppp\i\i) {\i};}
\foreach \i in {#2,#4} 
{\node [white] at (#1ppp\i\i) {\i};}
\coordinatebackgroundxy{#1}{#2}{#3}{#4}{#2}{#3}{#4};
}


%  The following command is to print a keyworded coordinate system as a background.
%  The command is
%  \coordinatebackgroundxy{the KEYWORD}
%                                                {the first letter in the x direction}
%                                           {the second letter in the x direction}
%                                                 {the last letter in the x direction}
%                                                {the first letter in the y direction}
%                                           {the second letter in the y direction}
%                                                 {the last letter in the y direction};
\newcommand{\coordinatebackgroundxy}[7]{
\pgfmathsetmacro{\bordercolourpercent}{60}
\pgfmathsetmacro{\colourpercent}{30}

\foreach \i in {#2,#3,...,#4} 
\foreach \j in {#5} 
\foreach \k in {#7} 
{\draw [dashed,black!\colourpercent] (#1ppp\i\j) -- (#1ppp\i\k);}

\foreach \i in {#5,#6,...,#7} 
\foreach \j in {#2} 
\foreach \k in {#4} 
{\draw [dashed,black!\colourpercent] (#1ppp\j\i) -- (#1ppp\k\i);}

\foreach \i in {#2,#4} 
\foreach \j in {#5} 
\foreach \k in {#7} 
{\draw [dashed,black!\bordercolourpercent] (#1ppp\i\j) -- (#1ppp\i\k);}

\foreach \i in {#5,#7} 
\foreach \j in {#2} 
\foreach \k in {#4} 
{\draw [dashed,black!\bordercolourpercent] (#1ppp\j\i) -- (#1ppp\k\i);}

\foreach \i in {#2,#3,...,#4} 
\foreach \j in {#5} 
\foreach \k in {#7} 
{
\node [black!\bordercolourpercent] at ($(#1ppp\i\j) + (0,-.2)$) {\i};
\node [black!\bordercolourpercent] at ($(#1ppp\i\k) + (0,.2)$) {\i};
}

\foreach \i in {#5,#6,...,#7} 
\foreach \j in {#2} 
\foreach \k in {#4} 
{
\node [black!\bordercolourpercent] at ($(#1ppp\k\i) + (.2,0)$) {\i};
\node [black!\bordercolourpercent] at ($(#1ppp\j\i) + (-.2,0)$) {\i};
}

}



