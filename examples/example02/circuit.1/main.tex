% This is circuit 1 of example 02 of
% https://github.com/LiuGangKingston/Nestable-coordinate-system-for-TikZ-circuits.git


\documentclass[tikz,border=5mm]{standalone}
\usepackage[siunitx]{circuitikz}
\usetikzlibrary{shapes,arrows,positioning}
\input{afewmorecommandsintikzpiture}


\begin{document}


\ctikzset{
 /tikz/circuitikz/diodes/scale=0.8,
 /tikz/circuitikz/bipoles/length=1cm
}


 
\begin{circuitikz} [scale=0.8]
% https://github.com/LiuGangKingston/Nestable-coordinate-system-for-Tikz-circuits.git
% https://github.com/LiuGangKingston/Nestable-coordinate-system-for-Tikz-circuits.git


\pgfmathsetmacro{\totalgangliuxxx}{26}
\pgfmathsetmacro{\totalgangliuyyy}{26}
\pgfmathsetmacro{\gangliuxxxspacing}{1}
\pgfmathsetmacro{\gangliuyyyspacing}{1}
\pgfmathsetmacro{\gangliuxxxa}{-8}
\pgfmathsetmacro{\gangliuyyya}{-8}

\pgfmathsetmacro{\gangliuxxxb}{\gangliuxxxa + \gangliuxxxspacing + 0.0 }
\pgfmathsetmacro{\gangliuxxxc}{\gangliuxxxb + \gangliuxxxspacing + 0.0 }
\pgfmathsetmacro{\gangliuxxxd}{\gangliuxxxc + \gangliuxxxspacing + 0.0 }
\pgfmathsetmacro{\gangliuxxxe}{\gangliuxxxd + \gangliuxxxspacing + 0.0 }
\pgfmathsetmacro{\gangliuxxxf}{\gangliuxxxe + \gangliuxxxspacing + 0.0 }
\pgfmathsetmacro{\gangliuxxxg}{\gangliuxxxf + \gangliuxxxspacing + 0.0 }
\pgfmathsetmacro{\gangliuxxxh}{\gangliuxxxg + \gangliuxxxspacing + 0.0 }
\pgfmathsetmacro{\gangliuxxxi}{\gangliuxxxh + \gangliuxxxspacing + 0.0 }
\pgfmathsetmacro{\gangliuxxxj}{\gangliuxxxi + \gangliuxxxspacing + 0.0 }
\pgfmathsetmacro{\gangliuxxxk}{\gangliuxxxj + \gangliuxxxspacing + 0.0 }
\pgfmathsetmacro{\gangliuxxxl}{\gangliuxxxk + \gangliuxxxspacing + 0.0 }
\pgfmathsetmacro{\gangliuxxxm}{\gangliuxxxl + \gangliuxxxspacing + 0.0 }
\pgfmathsetmacro{\gangliuxxxn}{\gangliuxxxm + \gangliuxxxspacing + 2.0 }
\pgfmathsetmacro{\gangliuxxxo}{\gangliuxxxn + \gangliuxxxspacing + 0.0 }
\pgfmathsetmacro{\gangliuxxxp}{\gangliuxxxo + \gangliuxxxspacing + 0.0 }
\pgfmathsetmacro{\gangliuxxxq}{\gangliuxxxp + \gangliuxxxspacing + 0.0 }
\pgfmathsetmacro{\gangliuxxxr}{\gangliuxxxq + \gangliuxxxspacing + 0.0 }
\pgfmathsetmacro{\gangliuxxxs}{\gangliuxxxr + \gangliuxxxspacing + 0.0 }
\pgfmathsetmacro{\gangliuxxxt}{\gangliuxxxs + \gangliuxxxspacing + 0.0 }
\pgfmathsetmacro{\gangliuxxxu}{\gangliuxxxt + \gangliuxxxspacing + 0.0 }
\pgfmathsetmacro{\gangliuxxxv}{\gangliuxxxu + \gangliuxxxspacing + 0.0 }
\pgfmathsetmacro{\gangliuxxxw}{\gangliuxxxv + \gangliuxxxspacing + 0.0 }
\pgfmathsetmacro{\gangliuxxxx}{\gangliuxxxw + \gangliuxxxspacing + 0.0 }
\pgfmathsetmacro{\gangliuxxxy}{\gangliuxxxx + \gangliuxxxspacing + 0.0 }
\pgfmathsetmacro{\gangliuxxxz}{\gangliuxxxy + \gangliuxxxspacing + 0.0 }

\pgfmathsetmacro{\gangliuyyyb}{\gangliuyyya + \gangliuyyyspacing + 0.0 }
\pgfmathsetmacro{\gangliuyyyc}{\gangliuyyyb + \gangliuyyyspacing + 0.0 }
\pgfmathsetmacro{\gangliuyyyd}{\gangliuyyyc + \gangliuyyyspacing + 0.0 }
\pgfmathsetmacro{\gangliuyyye}{\gangliuyyyd + \gangliuyyyspacing + 0.0 }
\pgfmathsetmacro{\gangliuyyyf}{\gangliuyyye + \gangliuyyyspacing + 0.0 }
\pgfmathsetmacro{\gangliuyyyg}{\gangliuyyyf + \gangliuyyyspacing + 0.0 }
\pgfmathsetmacro{\gangliuyyyh}{\gangliuyyyg + \gangliuyyyspacing + 0.0 }
\pgfmathsetmacro{\gangliuyyyi}{\gangliuyyyh + \gangliuyyyspacing + 0.0 }
\pgfmathsetmacro{\gangliuyyyj}{\gangliuyyyi + \gangliuyyyspacing + 0.0 }
\pgfmathsetmacro{\gangliuyyyk}{\gangliuyyyj + \gangliuyyyspacing + 0.0 }
\pgfmathsetmacro{\gangliuyyyl}{\gangliuyyyk + \gangliuyyyspacing + 1.0 }
\pgfmathsetmacro{\gangliuyyym}{\gangliuyyyl + \gangliuyyyspacing + 0.0 }
\pgfmathsetmacro{\gangliuyyyn}{\gangliuyyym + \gangliuyyyspacing + 0.0 }
\pgfmathsetmacro{\gangliuyyyo}{\gangliuyyyn + \gangliuyyyspacing + 0.0 }
\pgfmathsetmacro{\gangliuyyyp}{\gangliuyyyo + \gangliuyyyspacing + 0.0 }
\pgfmathsetmacro{\gangliuyyyq}{\gangliuyyyp + \gangliuyyyspacing + 0.0 }
\pgfmathsetmacro{\gangliuyyyr}{\gangliuyyyq + \gangliuyyyspacing + 0.0 }
\pgfmathsetmacro{\gangliuyyys}{\gangliuyyyr + \gangliuyyyspacing + 0.0 }
\pgfmathsetmacro{\gangliuyyyt}{\gangliuyyys + \gangliuyyyspacing + 0.0 }
\pgfmathsetmacro{\gangliuyyyu}{\gangliuyyyt + \gangliuyyyspacing + 0.0 }
\pgfmathsetmacro{\gangliuyyyv}{\gangliuyyyu + \gangliuyyyspacing + 0.0 }
\pgfmathsetmacro{\gangliuyyyw}{\gangliuyyyv + \gangliuyyyspacing + 0.0 }
\pgfmathsetmacro{\gangliuyyyx}{\gangliuyyyw + \gangliuyyyspacing + 0.0 }
\pgfmathsetmacro{\gangliuyyyy}{\gangliuyyyx + \gangliuyyyspacing + 0.0 }
\pgfmathsetmacro{\gangliuyyyz}{\gangliuyyyy + \gangliuyyyspacing + 0.0 }

\coordinate (gangliupppaa) at (\gangliuxxxa, \gangliuyyya);
\coordinate (gangliupppab) at (\gangliuxxxa, \gangliuyyyb);
\coordinate (gangliupppac) at (\gangliuxxxa, \gangliuyyyc);
\coordinate (gangliupppad) at (\gangliuxxxa, \gangliuyyyd);
\coordinate (gangliupppae) at (\gangliuxxxa, \gangliuyyye);
\coordinate (gangliupppaf) at (\gangliuxxxa, \gangliuyyyf);
\coordinate (gangliupppag) at (\gangliuxxxa, \gangliuyyyg);
\coordinate (gangliupppah) at (\gangliuxxxa, \gangliuyyyh);
\coordinate (gangliupppai) at (\gangliuxxxa, \gangliuyyyi);
\coordinate (gangliupppaj) at (\gangliuxxxa, \gangliuyyyj);
\coordinate (gangliupppak) at (\gangliuxxxa, \gangliuyyyk);
\coordinate (gangliupppal) at (\gangliuxxxa, \gangliuyyyl);
\coordinate (gangliupppam) at (\gangliuxxxa, \gangliuyyym);
\coordinate (gangliupppan) at (\gangliuxxxa, \gangliuyyyn);
\coordinate (gangliupppao) at (\gangliuxxxa, \gangliuyyyo);
\coordinate (gangliupppap) at (\gangliuxxxa, \gangliuyyyp);
\coordinate (gangliupppaq) at (\gangliuxxxa, \gangliuyyyq);
\coordinate (gangliupppar) at (\gangliuxxxa, \gangliuyyyr);
\coordinate (gangliupppas) at (\gangliuxxxa, \gangliuyyys);
\coordinate (gangliupppat) at (\gangliuxxxa, \gangliuyyyt);
\coordinate (gangliupppau) at (\gangliuxxxa, \gangliuyyyu);
\coordinate (gangliupppav) at (\gangliuxxxa, \gangliuyyyv);
\coordinate (gangliupppaw) at (\gangliuxxxa, \gangliuyyyw);
\coordinate (gangliupppax) at (\gangliuxxxa, \gangliuyyyx);
\coordinate (gangliupppay) at (\gangliuxxxa, \gangliuyyyy);
\coordinate (gangliupppaz) at (\gangliuxxxa, \gangliuyyyz);
\coordinate (gangliupppba) at (\gangliuxxxb, \gangliuyyya);
\coordinate (gangliupppbb) at (\gangliuxxxb, \gangliuyyyb);
\coordinate (gangliupppbc) at (\gangliuxxxb, \gangliuyyyc);
\coordinate (gangliupppbd) at (\gangliuxxxb, \gangliuyyyd);
\coordinate (gangliupppbe) at (\gangliuxxxb, \gangliuyyye);
\coordinate (gangliupppbf) at (\gangliuxxxb, \gangliuyyyf);
\coordinate (gangliupppbg) at (\gangliuxxxb, \gangliuyyyg);
\coordinate (gangliupppbh) at (\gangliuxxxb, \gangliuyyyh);
\coordinate (gangliupppbi) at (\gangliuxxxb, \gangliuyyyi);
\coordinate (gangliupppbj) at (\gangliuxxxb, \gangliuyyyj);
\coordinate (gangliupppbk) at (\gangliuxxxb, \gangliuyyyk);
\coordinate (gangliupppbl) at (\gangliuxxxb, \gangliuyyyl);
\coordinate (gangliupppbm) at (\gangliuxxxb, \gangliuyyym);
\coordinate (gangliupppbn) at (\gangliuxxxb, \gangliuyyyn);
\coordinate (gangliupppbo) at (\gangliuxxxb, \gangliuyyyo);
\coordinate (gangliupppbp) at (\gangliuxxxb, \gangliuyyyp);
\coordinate (gangliupppbq) at (\gangliuxxxb, \gangliuyyyq);
\coordinate (gangliupppbr) at (\gangliuxxxb, \gangliuyyyr);
\coordinate (gangliupppbs) at (\gangliuxxxb, \gangliuyyys);
\coordinate (gangliupppbt) at (\gangliuxxxb, \gangliuyyyt);
\coordinate (gangliupppbu) at (\gangliuxxxb, \gangliuyyyu);
\coordinate (gangliupppbv) at (\gangliuxxxb, \gangliuyyyv);
\coordinate (gangliupppbw) at (\gangliuxxxb, \gangliuyyyw);
\coordinate (gangliupppbx) at (\gangliuxxxb, \gangliuyyyx);
\coordinate (gangliupppby) at (\gangliuxxxb, \gangliuyyyy);
\coordinate (gangliupppbz) at (\gangliuxxxb, \gangliuyyyz);
\coordinate (gangliupppca) at (\gangliuxxxc, \gangliuyyya);
\coordinate (gangliupppcb) at (\gangliuxxxc, \gangliuyyyb);
\coordinate (gangliupppcc) at (\gangliuxxxc, \gangliuyyyc);
\coordinate (gangliupppcd) at (\gangliuxxxc, \gangliuyyyd);
\coordinate (gangliupppce) at (\gangliuxxxc, \gangliuyyye);
\coordinate (gangliupppcf) at (\gangliuxxxc, \gangliuyyyf);
\coordinate (gangliupppcg) at (\gangliuxxxc, \gangliuyyyg);
\coordinate (gangliupppch) at (\gangliuxxxc, \gangliuyyyh);
\coordinate (gangliupppci) at (\gangliuxxxc, \gangliuyyyi);
\coordinate (gangliupppcj) at (\gangliuxxxc, \gangliuyyyj);
\coordinate (gangliupppck) at (\gangliuxxxc, \gangliuyyyk);
\coordinate (gangliupppcl) at (\gangliuxxxc, \gangliuyyyl);
\coordinate (gangliupppcm) at (\gangliuxxxc, \gangliuyyym);
\coordinate (gangliupppcn) at (\gangliuxxxc, \gangliuyyyn);
\coordinate (gangliupppco) at (\gangliuxxxc, \gangliuyyyo);
\coordinate (gangliupppcp) at (\gangliuxxxc, \gangliuyyyp);
\coordinate (gangliupppcq) at (\gangliuxxxc, \gangliuyyyq);
\coordinate (gangliupppcr) at (\gangliuxxxc, \gangliuyyyr);
\coordinate (gangliupppcs) at (\gangliuxxxc, \gangliuyyys);
\coordinate (gangliupppct) at (\gangliuxxxc, \gangliuyyyt);
\coordinate (gangliupppcu) at (\gangliuxxxc, \gangliuyyyu);
\coordinate (gangliupppcv) at (\gangliuxxxc, \gangliuyyyv);
\coordinate (gangliupppcw) at (\gangliuxxxc, \gangliuyyyw);
\coordinate (gangliupppcx) at (\gangliuxxxc, \gangliuyyyx);
\coordinate (gangliupppcy) at (\gangliuxxxc, \gangliuyyyy);
\coordinate (gangliupppcz) at (\gangliuxxxc, \gangliuyyyz);
\coordinate (gangliupppda) at (\gangliuxxxd, \gangliuyyya);
\coordinate (gangliupppdb) at (\gangliuxxxd, \gangliuyyyb);
\coordinate (gangliupppdc) at (\gangliuxxxd, \gangliuyyyc);
\coordinate (gangliupppdd) at (\gangliuxxxd, \gangliuyyyd);
\coordinate (gangliupppde) at (\gangliuxxxd, \gangliuyyye);
\coordinate (gangliupppdf) at (\gangliuxxxd, \gangliuyyyf);
\coordinate (gangliupppdg) at (\gangliuxxxd, \gangliuyyyg);
\coordinate (gangliupppdh) at (\gangliuxxxd, \gangliuyyyh);
\coordinate (gangliupppdi) at (\gangliuxxxd, \gangliuyyyi);
\coordinate (gangliupppdj) at (\gangliuxxxd, \gangliuyyyj);
\coordinate (gangliupppdk) at (\gangliuxxxd, \gangliuyyyk);
\coordinate (gangliupppdl) at (\gangliuxxxd, \gangliuyyyl);
\coordinate (gangliupppdm) at (\gangliuxxxd, \gangliuyyym);
\coordinate (gangliupppdn) at (\gangliuxxxd, \gangliuyyyn);
\coordinate (gangliupppdo) at (\gangliuxxxd, \gangliuyyyo);
\coordinate (gangliupppdp) at (\gangliuxxxd, \gangliuyyyp);
\coordinate (gangliupppdq) at (\gangliuxxxd, \gangliuyyyq);
\coordinate (gangliupppdr) at (\gangliuxxxd, \gangliuyyyr);
\coordinate (gangliupppds) at (\gangliuxxxd, \gangliuyyys);
\coordinate (gangliupppdt) at (\gangliuxxxd, \gangliuyyyt);
\coordinate (gangliupppdu) at (\gangliuxxxd, \gangliuyyyu);
\coordinate (gangliupppdv) at (\gangliuxxxd, \gangliuyyyv);
\coordinate (gangliupppdw) at (\gangliuxxxd, \gangliuyyyw);
\coordinate (gangliupppdx) at (\gangliuxxxd, \gangliuyyyx);
\coordinate (gangliupppdy) at (\gangliuxxxd, \gangliuyyyy);
\coordinate (gangliupppdz) at (\gangliuxxxd, \gangliuyyyz);
\coordinate (gangliupppea) at (\gangliuxxxe, \gangliuyyya);
\coordinate (gangliupppeb) at (\gangliuxxxe, \gangliuyyyb);
\coordinate (gangliupppec) at (\gangliuxxxe, \gangliuyyyc);
\coordinate (gangliuppped) at (\gangliuxxxe, \gangliuyyyd);
\coordinate (gangliupppee) at (\gangliuxxxe, \gangliuyyye);
\coordinate (gangliupppef) at (\gangliuxxxe, \gangliuyyyf);
\coordinate (gangliupppeg) at (\gangliuxxxe, \gangliuyyyg);
\coordinate (gangliupppeh) at (\gangliuxxxe, \gangliuyyyh);
\coordinate (gangliupppei) at (\gangliuxxxe, \gangliuyyyi);
\coordinate (gangliupppej) at (\gangliuxxxe, \gangliuyyyj);
\coordinate (gangliupppek) at (\gangliuxxxe, \gangliuyyyk);
\coordinate (gangliupppel) at (\gangliuxxxe, \gangliuyyyl);
\coordinate (gangliupppem) at (\gangliuxxxe, \gangliuyyym);
\coordinate (gangliupppen) at (\gangliuxxxe, \gangliuyyyn);
\coordinate (gangliupppeo) at (\gangliuxxxe, \gangliuyyyo);
\coordinate (gangliupppep) at (\gangliuxxxe, \gangliuyyyp);
\coordinate (gangliupppeq) at (\gangliuxxxe, \gangliuyyyq);
\coordinate (gangliuppper) at (\gangliuxxxe, \gangliuyyyr);
\coordinate (gangliupppes) at (\gangliuxxxe, \gangliuyyys);
\coordinate (gangliupppet) at (\gangliuxxxe, \gangliuyyyt);
\coordinate (gangliupppeu) at (\gangliuxxxe, \gangliuyyyu);
\coordinate (gangliupppev) at (\gangliuxxxe, \gangliuyyyv);
\coordinate (gangliupppew) at (\gangliuxxxe, \gangliuyyyw);
\coordinate (gangliupppex) at (\gangliuxxxe, \gangliuyyyx);
\coordinate (gangliupppey) at (\gangliuxxxe, \gangliuyyyy);
\coordinate (gangliupppez) at (\gangliuxxxe, \gangliuyyyz);
\coordinate (gangliupppfa) at (\gangliuxxxf, \gangliuyyya);
\coordinate (gangliupppfb) at (\gangliuxxxf, \gangliuyyyb);
\coordinate (gangliupppfc) at (\gangliuxxxf, \gangliuyyyc);
\coordinate (gangliupppfd) at (\gangliuxxxf, \gangliuyyyd);
\coordinate (gangliupppfe) at (\gangliuxxxf, \gangliuyyye);
\coordinate (gangliupppff) at (\gangliuxxxf, \gangliuyyyf);
\coordinate (gangliupppfg) at (\gangliuxxxf, \gangliuyyyg);
\coordinate (gangliupppfh) at (\gangliuxxxf, \gangliuyyyh);
\coordinate (gangliupppfi) at (\gangliuxxxf, \gangliuyyyi);
\coordinate (gangliupppfj) at (\gangliuxxxf, \gangliuyyyj);
\coordinate (gangliupppfk) at (\gangliuxxxf, \gangliuyyyk);
\coordinate (gangliupppfl) at (\gangliuxxxf, \gangliuyyyl);
\coordinate (gangliupppfm) at (\gangliuxxxf, \gangliuyyym);
\coordinate (gangliupppfn) at (\gangliuxxxf, \gangliuyyyn);
\coordinate (gangliupppfo) at (\gangliuxxxf, \gangliuyyyo);
\coordinate (gangliupppfp) at (\gangliuxxxf, \gangliuyyyp);
\coordinate (gangliupppfq) at (\gangliuxxxf, \gangliuyyyq);
\coordinate (gangliupppfr) at (\gangliuxxxf, \gangliuyyyr);
\coordinate (gangliupppfs) at (\gangliuxxxf, \gangliuyyys);
\coordinate (gangliupppft) at (\gangliuxxxf, \gangliuyyyt);
\coordinate (gangliupppfu) at (\gangliuxxxf, \gangliuyyyu);
\coordinate (gangliupppfv) at (\gangliuxxxf, \gangliuyyyv);
\coordinate (gangliupppfw) at (\gangliuxxxf, \gangliuyyyw);
\coordinate (gangliupppfx) at (\gangliuxxxf, \gangliuyyyx);
\coordinate (gangliupppfy) at (\gangliuxxxf, \gangliuyyyy);
\coordinate (gangliupppfz) at (\gangliuxxxf, \gangliuyyyz);
\coordinate (gangliupppga) at (\gangliuxxxg, \gangliuyyya);
\coordinate (gangliupppgb) at (\gangliuxxxg, \gangliuyyyb);
\coordinate (gangliupppgc) at (\gangliuxxxg, \gangliuyyyc);
\coordinate (gangliupppgd) at (\gangliuxxxg, \gangliuyyyd);
\coordinate (gangliupppge) at (\gangliuxxxg, \gangliuyyye);
\coordinate (gangliupppgf) at (\gangliuxxxg, \gangliuyyyf);
\coordinate (gangliupppgg) at (\gangliuxxxg, \gangliuyyyg);
\coordinate (gangliupppgh) at (\gangliuxxxg, \gangliuyyyh);
\coordinate (gangliupppgi) at (\gangliuxxxg, \gangliuyyyi);
\coordinate (gangliupppgj) at (\gangliuxxxg, \gangliuyyyj);
\coordinate (gangliupppgk) at (\gangliuxxxg, \gangliuyyyk);
\coordinate (gangliupppgl) at (\gangliuxxxg, \gangliuyyyl);
\coordinate (gangliupppgm) at (\gangliuxxxg, \gangliuyyym);
\coordinate (gangliupppgn) at (\gangliuxxxg, \gangliuyyyn);
\coordinate (gangliupppgo) at (\gangliuxxxg, \gangliuyyyo);
\coordinate (gangliupppgp) at (\gangliuxxxg, \gangliuyyyp);
\coordinate (gangliupppgq) at (\gangliuxxxg, \gangliuyyyq);
\coordinate (gangliupppgr) at (\gangliuxxxg, \gangliuyyyr);
\coordinate (gangliupppgs) at (\gangliuxxxg, \gangliuyyys);
\coordinate (gangliupppgt) at (\gangliuxxxg, \gangliuyyyt);
\coordinate (gangliupppgu) at (\gangliuxxxg, \gangliuyyyu);
\coordinate (gangliupppgv) at (\gangliuxxxg, \gangliuyyyv);
\coordinate (gangliupppgw) at (\gangliuxxxg, \gangliuyyyw);
\coordinate (gangliupppgx) at (\gangliuxxxg, \gangliuyyyx);
\coordinate (gangliupppgy) at (\gangliuxxxg, \gangliuyyyy);
\coordinate (gangliupppgz) at (\gangliuxxxg, \gangliuyyyz);
\coordinate (gangliupppha) at (\gangliuxxxh, \gangliuyyya);
\coordinate (gangliuppphb) at (\gangliuxxxh, \gangliuyyyb);
\coordinate (gangliuppphc) at (\gangliuxxxh, \gangliuyyyc);
\coordinate (gangliuppphd) at (\gangliuxxxh, \gangliuyyyd);
\coordinate (gangliuppphe) at (\gangliuxxxh, \gangliuyyye);
\coordinate (gangliuppphf) at (\gangliuxxxh, \gangliuyyyf);
\coordinate (gangliuppphg) at (\gangliuxxxh, \gangliuyyyg);
\coordinate (gangliuppphh) at (\gangliuxxxh, \gangliuyyyh);
\coordinate (gangliuppphi) at (\gangliuxxxh, \gangliuyyyi);
\coordinate (gangliuppphj) at (\gangliuxxxh, \gangliuyyyj);
\coordinate (gangliuppphk) at (\gangliuxxxh, \gangliuyyyk);
\coordinate (gangliuppphl) at (\gangliuxxxh, \gangliuyyyl);
\coordinate (gangliuppphm) at (\gangliuxxxh, \gangliuyyym);
\coordinate (gangliuppphn) at (\gangliuxxxh, \gangliuyyyn);
\coordinate (gangliupppho) at (\gangliuxxxh, \gangliuyyyo);
\coordinate (gangliuppphp) at (\gangliuxxxh, \gangliuyyyp);
\coordinate (gangliuppphq) at (\gangliuxxxh, \gangliuyyyq);
\coordinate (gangliuppphr) at (\gangliuxxxh, \gangliuyyyr);
\coordinate (gangliuppphs) at (\gangliuxxxh, \gangliuyyys);
\coordinate (gangliupppht) at (\gangliuxxxh, \gangliuyyyt);
\coordinate (gangliuppphu) at (\gangliuxxxh, \gangliuyyyu);
\coordinate (gangliuppphv) at (\gangliuxxxh, \gangliuyyyv);
\coordinate (gangliuppphw) at (\gangliuxxxh, \gangliuyyyw);
\coordinate (gangliuppphx) at (\gangliuxxxh, \gangliuyyyx);
\coordinate (gangliuppphy) at (\gangliuxxxh, \gangliuyyyy);
\coordinate (gangliuppphz) at (\gangliuxxxh, \gangliuyyyz);
\coordinate (gangliupppia) at (\gangliuxxxi, \gangliuyyya);
\coordinate (gangliupppib) at (\gangliuxxxi, \gangliuyyyb);
\coordinate (gangliupppic) at (\gangliuxxxi, \gangliuyyyc);
\coordinate (gangliupppid) at (\gangliuxxxi, \gangliuyyyd);
\coordinate (gangliupppie) at (\gangliuxxxi, \gangliuyyye);
\coordinate (gangliupppif) at (\gangliuxxxi, \gangliuyyyf);
\coordinate (gangliupppig) at (\gangliuxxxi, \gangliuyyyg);
\coordinate (gangliupppih) at (\gangliuxxxi, \gangliuyyyh);
\coordinate (gangliupppii) at (\gangliuxxxi, \gangliuyyyi);
\coordinate (gangliupppij) at (\gangliuxxxi, \gangliuyyyj);
\coordinate (gangliupppik) at (\gangliuxxxi, \gangliuyyyk);
\coordinate (gangliupppil) at (\gangliuxxxi, \gangliuyyyl);
\coordinate (gangliupppim) at (\gangliuxxxi, \gangliuyyym);
\coordinate (gangliupppin) at (\gangliuxxxi, \gangliuyyyn);
\coordinate (gangliupppio) at (\gangliuxxxi, \gangliuyyyo);
\coordinate (gangliupppip) at (\gangliuxxxi, \gangliuyyyp);
\coordinate (gangliupppiq) at (\gangliuxxxi, \gangliuyyyq);
\coordinate (gangliupppir) at (\gangliuxxxi, \gangliuyyyr);
\coordinate (gangliupppis) at (\gangliuxxxi, \gangliuyyys);
\coordinate (gangliupppit) at (\gangliuxxxi, \gangliuyyyt);
\coordinate (gangliupppiu) at (\gangliuxxxi, \gangliuyyyu);
\coordinate (gangliupppiv) at (\gangliuxxxi, \gangliuyyyv);
\coordinate (gangliupppiw) at (\gangliuxxxi, \gangliuyyyw);
\coordinate (gangliupppix) at (\gangliuxxxi, \gangliuyyyx);
\coordinate (gangliupppiy) at (\gangliuxxxi, \gangliuyyyy);
\coordinate (gangliupppiz) at (\gangliuxxxi, \gangliuyyyz);
\coordinate (gangliupppja) at (\gangliuxxxj, \gangliuyyya);
\coordinate (gangliupppjb) at (\gangliuxxxj, \gangliuyyyb);
\coordinate (gangliupppjc) at (\gangliuxxxj, \gangliuyyyc);
\coordinate (gangliupppjd) at (\gangliuxxxj, \gangliuyyyd);
\coordinate (gangliupppje) at (\gangliuxxxj, \gangliuyyye);
\coordinate (gangliupppjf) at (\gangliuxxxj, \gangliuyyyf);
\coordinate (gangliupppjg) at (\gangliuxxxj, \gangliuyyyg);
\coordinate (gangliupppjh) at (\gangliuxxxj, \gangliuyyyh);
\coordinate (gangliupppji) at (\gangliuxxxj, \gangliuyyyi);
\coordinate (gangliupppjj) at (\gangliuxxxj, \gangliuyyyj);
\coordinate (gangliupppjk) at (\gangliuxxxj, \gangliuyyyk);
\coordinate (gangliupppjl) at (\gangliuxxxj, \gangliuyyyl);
\coordinate (gangliupppjm) at (\gangliuxxxj, \gangliuyyym);
\coordinate (gangliupppjn) at (\gangliuxxxj, \gangliuyyyn);
\coordinate (gangliupppjo) at (\gangliuxxxj, \gangliuyyyo);
\coordinate (gangliupppjp) at (\gangliuxxxj, \gangliuyyyp);
\coordinate (gangliupppjq) at (\gangliuxxxj, \gangliuyyyq);
\coordinate (gangliupppjr) at (\gangliuxxxj, \gangliuyyyr);
\coordinate (gangliupppjs) at (\gangliuxxxj, \gangliuyyys);
\coordinate (gangliupppjt) at (\gangliuxxxj, \gangliuyyyt);
\coordinate (gangliupppju) at (\gangliuxxxj, \gangliuyyyu);
\coordinate (gangliupppjv) at (\gangliuxxxj, \gangliuyyyv);
\coordinate (gangliupppjw) at (\gangliuxxxj, \gangliuyyyw);
\coordinate (gangliupppjx) at (\gangliuxxxj, \gangliuyyyx);
\coordinate (gangliupppjy) at (\gangliuxxxj, \gangliuyyyy);
\coordinate (gangliupppjz) at (\gangliuxxxj, \gangliuyyyz);
\coordinate (gangliupppka) at (\gangliuxxxk, \gangliuyyya);
\coordinate (gangliupppkb) at (\gangliuxxxk, \gangliuyyyb);
\coordinate (gangliupppkc) at (\gangliuxxxk, \gangliuyyyc);
\coordinate (gangliupppkd) at (\gangliuxxxk, \gangliuyyyd);
\coordinate (gangliupppke) at (\gangliuxxxk, \gangliuyyye);
\coordinate (gangliupppkf) at (\gangliuxxxk, \gangliuyyyf);
\coordinate (gangliupppkg) at (\gangliuxxxk, \gangliuyyyg);
\coordinate (gangliupppkh) at (\gangliuxxxk, \gangliuyyyh);
\coordinate (gangliupppki) at (\gangliuxxxk, \gangliuyyyi);
\coordinate (gangliupppkj) at (\gangliuxxxk, \gangliuyyyj);
\coordinate (gangliupppkk) at (\gangliuxxxk, \gangliuyyyk);
\coordinate (gangliupppkl) at (\gangliuxxxk, \gangliuyyyl);
\coordinate (gangliupppkm) at (\gangliuxxxk, \gangliuyyym);
\coordinate (gangliupppkn) at (\gangliuxxxk, \gangliuyyyn);
\coordinate (gangliupppko) at (\gangliuxxxk, \gangliuyyyo);
\coordinate (gangliupppkp) at (\gangliuxxxk, \gangliuyyyp);
\coordinate (gangliupppkq) at (\gangliuxxxk, \gangliuyyyq);
\coordinate (gangliupppkr) at (\gangliuxxxk, \gangliuyyyr);
\coordinate (gangliupppks) at (\gangliuxxxk, \gangliuyyys);
\coordinate (gangliupppkt) at (\gangliuxxxk, \gangliuyyyt);
\coordinate (gangliupppku) at (\gangliuxxxk, \gangliuyyyu);
\coordinate (gangliupppkv) at (\gangliuxxxk, \gangliuyyyv);
\coordinate (gangliupppkw) at (\gangliuxxxk, \gangliuyyyw);
\coordinate (gangliupppkx) at (\gangliuxxxk, \gangliuyyyx);
\coordinate (gangliupppky) at (\gangliuxxxk, \gangliuyyyy);
\coordinate (gangliupppkz) at (\gangliuxxxk, \gangliuyyyz);
\coordinate (gangliupppla) at (\gangliuxxxl, \gangliuyyya);
\coordinate (gangliuppplb) at (\gangliuxxxl, \gangliuyyyb);
\coordinate (gangliuppplc) at (\gangliuxxxl, \gangliuyyyc);
\coordinate (gangliupppld) at (\gangliuxxxl, \gangliuyyyd);
\coordinate (gangliuppple) at (\gangliuxxxl, \gangliuyyye);
\coordinate (gangliuppplf) at (\gangliuxxxl, \gangliuyyyf);
\coordinate (gangliuppplg) at (\gangliuxxxl, \gangliuyyyg);
\coordinate (gangliuppplh) at (\gangliuxxxl, \gangliuyyyh);
\coordinate (gangliupppli) at (\gangliuxxxl, \gangliuyyyi);
\coordinate (gangliuppplj) at (\gangliuxxxl, \gangliuyyyj);
\coordinate (gangliuppplk) at (\gangliuxxxl, \gangliuyyyk);
\coordinate (gangliupppll) at (\gangliuxxxl, \gangliuyyyl);
\coordinate (gangliuppplm) at (\gangliuxxxl, \gangliuyyym);
\coordinate (gangliupppln) at (\gangliuxxxl, \gangliuyyyn);
\coordinate (gangliuppplo) at (\gangliuxxxl, \gangliuyyyo);
\coordinate (gangliuppplp) at (\gangliuxxxl, \gangliuyyyp);
\coordinate (gangliuppplq) at (\gangliuxxxl, \gangliuyyyq);
\coordinate (gangliuppplr) at (\gangliuxxxl, \gangliuyyyr);
\coordinate (gangliupppls) at (\gangliuxxxl, \gangliuyyys);
\coordinate (gangliuppplt) at (\gangliuxxxl, \gangliuyyyt);
\coordinate (gangliuppplu) at (\gangliuxxxl, \gangliuyyyu);
\coordinate (gangliuppplv) at (\gangliuxxxl, \gangliuyyyv);
\coordinate (gangliuppplw) at (\gangliuxxxl, \gangliuyyyw);
\coordinate (gangliuppplx) at (\gangliuxxxl, \gangliuyyyx);
\coordinate (gangliuppply) at (\gangliuxxxl, \gangliuyyyy);
\coordinate (gangliuppplz) at (\gangliuxxxl, \gangliuyyyz);
\coordinate (gangliupppma) at (\gangliuxxxm, \gangliuyyya);
\coordinate (gangliupppmb) at (\gangliuxxxm, \gangliuyyyb);
\coordinate (gangliupppmc) at (\gangliuxxxm, \gangliuyyyc);
\coordinate (gangliupppmd) at (\gangliuxxxm, \gangliuyyyd);
\coordinate (gangliupppme) at (\gangliuxxxm, \gangliuyyye);
\coordinate (gangliupppmf) at (\gangliuxxxm, \gangliuyyyf);
\coordinate (gangliupppmg) at (\gangliuxxxm, \gangliuyyyg);
\coordinate (gangliupppmh) at (\gangliuxxxm, \gangliuyyyh);
\coordinate (gangliupppmi) at (\gangliuxxxm, \gangliuyyyi);
\coordinate (gangliupppmj) at (\gangliuxxxm, \gangliuyyyj);
\coordinate (gangliupppmk) at (\gangliuxxxm, \gangliuyyyk);
\coordinate (gangliupppml) at (\gangliuxxxm, \gangliuyyyl);
\coordinate (gangliupppmm) at (\gangliuxxxm, \gangliuyyym);
\coordinate (gangliupppmn) at (\gangliuxxxm, \gangliuyyyn);
\coordinate (gangliupppmo) at (\gangliuxxxm, \gangliuyyyo);
\coordinate (gangliupppmp) at (\gangliuxxxm, \gangliuyyyp);
\coordinate (gangliupppmq) at (\gangliuxxxm, \gangliuyyyq);
\coordinate (gangliupppmr) at (\gangliuxxxm, \gangliuyyyr);
\coordinate (gangliupppms) at (\gangliuxxxm, \gangliuyyys);
\coordinate (gangliupppmt) at (\gangliuxxxm, \gangliuyyyt);
\coordinate (gangliupppmu) at (\gangliuxxxm, \gangliuyyyu);
\coordinate (gangliupppmv) at (\gangliuxxxm, \gangliuyyyv);
\coordinate (gangliupppmw) at (\gangliuxxxm, \gangliuyyyw);
\coordinate (gangliupppmx) at (\gangliuxxxm, \gangliuyyyx);
\coordinate (gangliupppmy) at (\gangliuxxxm, \gangliuyyyy);
\coordinate (gangliupppmz) at (\gangliuxxxm, \gangliuyyyz);
\coordinate (gangliupppna) at (\gangliuxxxn, \gangliuyyya);
\coordinate (gangliupppnb) at (\gangliuxxxn, \gangliuyyyb);
\coordinate (gangliupppnc) at (\gangliuxxxn, \gangliuyyyc);
\coordinate (gangliupppnd) at (\gangliuxxxn, \gangliuyyyd);
\coordinate (gangliupppne) at (\gangliuxxxn, \gangliuyyye);
\coordinate (gangliupppnf) at (\gangliuxxxn, \gangliuyyyf);
\coordinate (gangliupppng) at (\gangliuxxxn, \gangliuyyyg);
\coordinate (gangliupppnh) at (\gangliuxxxn, \gangliuyyyh);
\coordinate (gangliupppni) at (\gangliuxxxn, \gangliuyyyi);
\coordinate (gangliupppnj) at (\gangliuxxxn, \gangliuyyyj);
\coordinate (gangliupppnk) at (\gangliuxxxn, \gangliuyyyk);
\coordinate (gangliupppnl) at (\gangliuxxxn, \gangliuyyyl);
\coordinate (gangliupppnm) at (\gangliuxxxn, \gangliuyyym);
\coordinate (gangliupppnn) at (\gangliuxxxn, \gangliuyyyn);
\coordinate (gangliupppno) at (\gangliuxxxn, \gangliuyyyo);
\coordinate (gangliupppnp) at (\gangliuxxxn, \gangliuyyyp);
\coordinate (gangliupppnq) at (\gangliuxxxn, \gangliuyyyq);
\coordinate (gangliupppnr) at (\gangliuxxxn, \gangliuyyyr);
\coordinate (gangliupppns) at (\gangliuxxxn, \gangliuyyys);
\coordinate (gangliupppnt) at (\gangliuxxxn, \gangliuyyyt);
\coordinate (gangliupppnu) at (\gangliuxxxn, \gangliuyyyu);
\coordinate (gangliupppnv) at (\gangliuxxxn, \gangliuyyyv);
\coordinate (gangliupppnw) at (\gangliuxxxn, \gangliuyyyw);
\coordinate (gangliupppnx) at (\gangliuxxxn, \gangliuyyyx);
\coordinate (gangliupppny) at (\gangliuxxxn, \gangliuyyyy);
\coordinate (gangliupppnz) at (\gangliuxxxn, \gangliuyyyz);
\coordinate (gangliupppoa) at (\gangliuxxxo, \gangliuyyya);
\coordinate (gangliupppob) at (\gangliuxxxo, \gangliuyyyb);
\coordinate (gangliupppoc) at (\gangliuxxxo, \gangliuyyyc);
\coordinate (gangliupppod) at (\gangliuxxxo, \gangliuyyyd);
\coordinate (gangliupppoe) at (\gangliuxxxo, \gangliuyyye);
\coordinate (gangliupppof) at (\gangliuxxxo, \gangliuyyyf);
\coordinate (gangliupppog) at (\gangliuxxxo, \gangliuyyyg);
\coordinate (gangliupppoh) at (\gangliuxxxo, \gangliuyyyh);
\coordinate (gangliupppoi) at (\gangliuxxxo, \gangliuyyyi);
\coordinate (gangliupppoj) at (\gangliuxxxo, \gangliuyyyj);
\coordinate (gangliupppok) at (\gangliuxxxo, \gangliuyyyk);
\coordinate (gangliupppol) at (\gangliuxxxo, \gangliuyyyl);
\coordinate (gangliupppom) at (\gangliuxxxo, \gangliuyyym);
\coordinate (gangliupppon) at (\gangliuxxxo, \gangliuyyyn);
\coordinate (gangliupppoo) at (\gangliuxxxo, \gangliuyyyo);
\coordinate (gangliupppop) at (\gangliuxxxo, \gangliuyyyp);
\coordinate (gangliupppoq) at (\gangliuxxxo, \gangliuyyyq);
\coordinate (gangliupppor) at (\gangliuxxxo, \gangliuyyyr);
\coordinate (gangliupppos) at (\gangliuxxxo, \gangliuyyys);
\coordinate (gangliupppot) at (\gangliuxxxo, \gangliuyyyt);
\coordinate (gangliupppou) at (\gangliuxxxo, \gangliuyyyu);
\coordinate (gangliupppov) at (\gangliuxxxo, \gangliuyyyv);
\coordinate (gangliupppow) at (\gangliuxxxo, \gangliuyyyw);
\coordinate (gangliupppox) at (\gangliuxxxo, \gangliuyyyx);
\coordinate (gangliupppoy) at (\gangliuxxxo, \gangliuyyyy);
\coordinate (gangliupppoz) at (\gangliuxxxo, \gangliuyyyz);
\coordinate (gangliuppppa) at (\gangliuxxxp, \gangliuyyya);
\coordinate (gangliuppppb) at (\gangliuxxxp, \gangliuyyyb);
\coordinate (gangliuppppc) at (\gangliuxxxp, \gangliuyyyc);
\coordinate (gangliuppppd) at (\gangliuxxxp, \gangliuyyyd);
\coordinate (gangliuppppe) at (\gangliuxxxp, \gangliuyyye);
\coordinate (gangliuppppf) at (\gangliuxxxp, \gangliuyyyf);
\coordinate (gangliuppppg) at (\gangliuxxxp, \gangliuyyyg);
\coordinate (gangliupppph) at (\gangliuxxxp, \gangliuyyyh);
\coordinate (gangliuppppi) at (\gangliuxxxp, \gangliuyyyi);
\coordinate (gangliuppppj) at (\gangliuxxxp, \gangliuyyyj);
\coordinate (gangliuppppk) at (\gangliuxxxp, \gangliuyyyk);
\coordinate (gangliuppppl) at (\gangliuxxxp, \gangliuyyyl);
\coordinate (gangliuppppm) at (\gangliuxxxp, \gangliuyyym);
\coordinate (gangliuppppn) at (\gangliuxxxp, \gangliuyyyn);
\coordinate (gangliuppppo) at (\gangliuxxxp, \gangliuyyyo);
\coordinate (gangliuppppp) at (\gangliuxxxp, \gangliuyyyp);
\coordinate (gangliuppppq) at (\gangliuxxxp, \gangliuyyyq);
\coordinate (gangliuppppr) at (\gangliuxxxp, \gangliuyyyr);
\coordinate (gangliupppps) at (\gangliuxxxp, \gangliuyyys);
\coordinate (gangliuppppt) at (\gangliuxxxp, \gangliuyyyt);
\coordinate (gangliuppppu) at (\gangliuxxxp, \gangliuyyyu);
\coordinate (gangliuppppv) at (\gangliuxxxp, \gangliuyyyv);
\coordinate (gangliuppppw) at (\gangliuxxxp, \gangliuyyyw);
\coordinate (gangliuppppx) at (\gangliuxxxp, \gangliuyyyx);
\coordinate (gangliuppppy) at (\gangliuxxxp, \gangliuyyyy);
\coordinate (gangliuppppz) at (\gangliuxxxp, \gangliuyyyz);
\coordinate (gangliupppqa) at (\gangliuxxxq, \gangliuyyya);
\coordinate (gangliupppqb) at (\gangliuxxxq, \gangliuyyyb);
\coordinate (gangliupppqc) at (\gangliuxxxq, \gangliuyyyc);
\coordinate (gangliupppqd) at (\gangliuxxxq, \gangliuyyyd);
\coordinate (gangliupppqe) at (\gangliuxxxq, \gangliuyyye);
\coordinate (gangliupppqf) at (\gangliuxxxq, \gangliuyyyf);
\coordinate (gangliupppqg) at (\gangliuxxxq, \gangliuyyyg);
\coordinate (gangliupppqh) at (\gangliuxxxq, \gangliuyyyh);
\coordinate (gangliupppqi) at (\gangliuxxxq, \gangliuyyyi);
\coordinate (gangliupppqj) at (\gangliuxxxq, \gangliuyyyj);
\coordinate (gangliupppqk) at (\gangliuxxxq, \gangliuyyyk);
\coordinate (gangliupppql) at (\gangliuxxxq, \gangliuyyyl);
\coordinate (gangliupppqm) at (\gangliuxxxq, \gangliuyyym);
\coordinate (gangliupppqn) at (\gangliuxxxq, \gangliuyyyn);
\coordinate (gangliupppqo) at (\gangliuxxxq, \gangliuyyyo);
\coordinate (gangliupppqp) at (\gangliuxxxq, \gangliuyyyp);
\coordinate (gangliupppqq) at (\gangliuxxxq, \gangliuyyyq);
\coordinate (gangliupppqr) at (\gangliuxxxq, \gangliuyyyr);
\coordinate (gangliupppqs) at (\gangliuxxxq, \gangliuyyys);
\coordinate (gangliupppqt) at (\gangliuxxxq, \gangliuyyyt);
\coordinate (gangliupppqu) at (\gangliuxxxq, \gangliuyyyu);
\coordinate (gangliupppqv) at (\gangliuxxxq, \gangliuyyyv);
\coordinate (gangliupppqw) at (\gangliuxxxq, \gangliuyyyw);
\coordinate (gangliupppqx) at (\gangliuxxxq, \gangliuyyyx);
\coordinate (gangliupppqy) at (\gangliuxxxq, \gangliuyyyy);
\coordinate (gangliupppqz) at (\gangliuxxxq, \gangliuyyyz);
\coordinate (gangliupppra) at (\gangliuxxxr, \gangliuyyya);
\coordinate (gangliuppprb) at (\gangliuxxxr, \gangliuyyyb);
\coordinate (gangliuppprc) at (\gangliuxxxr, \gangliuyyyc);
\coordinate (gangliuppprd) at (\gangliuxxxr, \gangliuyyyd);
\coordinate (gangliupppre) at (\gangliuxxxr, \gangliuyyye);
\coordinate (gangliuppprf) at (\gangliuxxxr, \gangliuyyyf);
\coordinate (gangliuppprg) at (\gangliuxxxr, \gangliuyyyg);
\coordinate (gangliuppprh) at (\gangliuxxxr, \gangliuyyyh);
\coordinate (gangliupppri) at (\gangliuxxxr, \gangliuyyyi);
\coordinate (gangliuppprj) at (\gangliuxxxr, \gangliuyyyj);
\coordinate (gangliuppprk) at (\gangliuxxxr, \gangliuyyyk);
\coordinate (gangliuppprl) at (\gangliuxxxr, \gangliuyyyl);
\coordinate (gangliuppprm) at (\gangliuxxxr, \gangliuyyym);
\coordinate (gangliuppprn) at (\gangliuxxxr, \gangliuyyyn);
\coordinate (gangliupppro) at (\gangliuxxxr, \gangliuyyyo);
\coordinate (gangliuppprp) at (\gangliuxxxr, \gangliuyyyp);
\coordinate (gangliuppprq) at (\gangliuxxxr, \gangliuyyyq);
\coordinate (gangliuppprr) at (\gangliuxxxr, \gangliuyyyr);
\coordinate (gangliuppprs) at (\gangliuxxxr, \gangliuyyys);
\coordinate (gangliuppprt) at (\gangliuxxxr, \gangliuyyyt);
\coordinate (gangliupppru) at (\gangliuxxxr, \gangliuyyyu);
\coordinate (gangliuppprv) at (\gangliuxxxr, \gangliuyyyv);
\coordinate (gangliuppprw) at (\gangliuxxxr, \gangliuyyyw);
\coordinate (gangliuppprx) at (\gangliuxxxr, \gangliuyyyx);
\coordinate (gangliupppry) at (\gangliuxxxr, \gangliuyyyy);
\coordinate (gangliuppprz) at (\gangliuxxxr, \gangliuyyyz);
\coordinate (gangliupppsa) at (\gangliuxxxs, \gangliuyyya);
\coordinate (gangliupppsb) at (\gangliuxxxs, \gangliuyyyb);
\coordinate (gangliupppsc) at (\gangliuxxxs, \gangliuyyyc);
\coordinate (gangliupppsd) at (\gangliuxxxs, \gangliuyyyd);
\coordinate (gangliupppse) at (\gangliuxxxs, \gangliuyyye);
\coordinate (gangliupppsf) at (\gangliuxxxs, \gangliuyyyf);
\coordinate (gangliupppsg) at (\gangliuxxxs, \gangliuyyyg);
\coordinate (gangliupppsh) at (\gangliuxxxs, \gangliuyyyh);
\coordinate (gangliupppsi) at (\gangliuxxxs, \gangliuyyyi);
\coordinate (gangliupppsj) at (\gangliuxxxs, \gangliuyyyj);
\coordinate (gangliupppsk) at (\gangliuxxxs, \gangliuyyyk);
\coordinate (gangliupppsl) at (\gangliuxxxs, \gangliuyyyl);
\coordinate (gangliupppsm) at (\gangliuxxxs, \gangliuyyym);
\coordinate (gangliupppsn) at (\gangliuxxxs, \gangliuyyyn);
\coordinate (gangliupppso) at (\gangliuxxxs, \gangliuyyyo);
\coordinate (gangliupppsp) at (\gangliuxxxs, \gangliuyyyp);
\coordinate (gangliupppsq) at (\gangliuxxxs, \gangliuyyyq);
\coordinate (gangliupppsr) at (\gangliuxxxs, \gangliuyyyr);
\coordinate (gangliupppss) at (\gangliuxxxs, \gangliuyyys);
\coordinate (gangliupppst) at (\gangliuxxxs, \gangliuyyyt);
\coordinate (gangliupppsu) at (\gangliuxxxs, \gangliuyyyu);
\coordinate (gangliupppsv) at (\gangliuxxxs, \gangliuyyyv);
\coordinate (gangliupppsw) at (\gangliuxxxs, \gangliuyyyw);
\coordinate (gangliupppsx) at (\gangliuxxxs, \gangliuyyyx);
\coordinate (gangliupppsy) at (\gangliuxxxs, \gangliuyyyy);
\coordinate (gangliupppsz) at (\gangliuxxxs, \gangliuyyyz);
\coordinate (gangliupppta) at (\gangliuxxxt, \gangliuyyya);
\coordinate (gangliuppptb) at (\gangliuxxxt, \gangliuyyyb);
\coordinate (gangliuppptc) at (\gangliuxxxt, \gangliuyyyc);
\coordinate (gangliuppptd) at (\gangliuxxxt, \gangliuyyyd);
\coordinate (gangliupppte) at (\gangliuxxxt, \gangliuyyye);
\coordinate (gangliuppptf) at (\gangliuxxxt, \gangliuyyyf);
\coordinate (gangliuppptg) at (\gangliuxxxt, \gangliuyyyg);
\coordinate (gangliupppth) at (\gangliuxxxt, \gangliuyyyh);
\coordinate (gangliupppti) at (\gangliuxxxt, \gangliuyyyi);
\coordinate (gangliuppptj) at (\gangliuxxxt, \gangliuyyyj);
\coordinate (gangliuppptk) at (\gangliuxxxt, \gangliuyyyk);
\coordinate (gangliuppptl) at (\gangliuxxxt, \gangliuyyyl);
\coordinate (gangliuppptm) at (\gangliuxxxt, \gangliuyyym);
\coordinate (gangliuppptn) at (\gangliuxxxt, \gangliuyyyn);
\coordinate (gangliupppto) at (\gangliuxxxt, \gangliuyyyo);
\coordinate (gangliuppptp) at (\gangliuxxxt, \gangliuyyyp);
\coordinate (gangliuppptq) at (\gangliuxxxt, \gangliuyyyq);
\coordinate (gangliuppptr) at (\gangliuxxxt, \gangliuyyyr);
\coordinate (gangliupppts) at (\gangliuxxxt, \gangliuyyys);
\coordinate (gangliuppptt) at (\gangliuxxxt, \gangliuyyyt);
\coordinate (gangliuppptu) at (\gangliuxxxt, \gangliuyyyu);
\coordinate (gangliuppptv) at (\gangliuxxxt, \gangliuyyyv);
\coordinate (gangliuppptw) at (\gangliuxxxt, \gangliuyyyw);
\coordinate (gangliuppptx) at (\gangliuxxxt, \gangliuyyyx);
\coordinate (gangliupppty) at (\gangliuxxxt, \gangliuyyyy);
\coordinate (gangliuppptz) at (\gangliuxxxt, \gangliuyyyz);
\coordinate (gangliupppua) at (\gangliuxxxu, \gangliuyyya);
\coordinate (gangliupppub) at (\gangliuxxxu, \gangliuyyyb);
\coordinate (gangliupppuc) at (\gangliuxxxu, \gangliuyyyc);
\coordinate (gangliupppud) at (\gangliuxxxu, \gangliuyyyd);
\coordinate (gangliupppue) at (\gangliuxxxu, \gangliuyyye);
\coordinate (gangliupppuf) at (\gangliuxxxu, \gangliuyyyf);
\coordinate (gangliupppug) at (\gangliuxxxu, \gangliuyyyg);
\coordinate (gangliupppuh) at (\gangliuxxxu, \gangliuyyyh);
\coordinate (gangliupppui) at (\gangliuxxxu, \gangliuyyyi);
\coordinate (gangliupppuj) at (\gangliuxxxu, \gangliuyyyj);
\coordinate (gangliupppuk) at (\gangliuxxxu, \gangliuyyyk);
\coordinate (gangliupppul) at (\gangliuxxxu, \gangliuyyyl);
\coordinate (gangliupppum) at (\gangliuxxxu, \gangliuyyym);
\coordinate (gangliupppun) at (\gangliuxxxu, \gangliuyyyn);
\coordinate (gangliupppuo) at (\gangliuxxxu, \gangliuyyyo);
\coordinate (gangliupppup) at (\gangliuxxxu, \gangliuyyyp);
\coordinate (gangliupppuq) at (\gangliuxxxu, \gangliuyyyq);
\coordinate (gangliupppur) at (\gangliuxxxu, \gangliuyyyr);
\coordinate (gangliupppus) at (\gangliuxxxu, \gangliuyyys);
\coordinate (gangliuppput) at (\gangliuxxxu, \gangliuyyyt);
\coordinate (gangliupppuu) at (\gangliuxxxu, \gangliuyyyu);
\coordinate (gangliupppuv) at (\gangliuxxxu, \gangliuyyyv);
\coordinate (gangliupppuw) at (\gangliuxxxu, \gangliuyyyw);
\coordinate (gangliupppux) at (\gangliuxxxu, \gangliuyyyx);
\coordinate (gangliupppuy) at (\gangliuxxxu, \gangliuyyyy);
\coordinate (gangliupppuz) at (\gangliuxxxu, \gangliuyyyz);
\coordinate (gangliupppva) at (\gangliuxxxv, \gangliuyyya);
\coordinate (gangliupppvb) at (\gangliuxxxv, \gangliuyyyb);
\coordinate (gangliupppvc) at (\gangliuxxxv, \gangliuyyyc);
\coordinate (gangliupppvd) at (\gangliuxxxv, \gangliuyyyd);
\coordinate (gangliupppve) at (\gangliuxxxv, \gangliuyyye);
\coordinate (gangliupppvf) at (\gangliuxxxv, \gangliuyyyf);
\coordinate (gangliupppvg) at (\gangliuxxxv, \gangliuyyyg);
\coordinate (gangliupppvh) at (\gangliuxxxv, \gangliuyyyh);
\coordinate (gangliupppvi) at (\gangliuxxxv, \gangliuyyyi);
\coordinate (gangliupppvj) at (\gangliuxxxv, \gangliuyyyj);
\coordinate (gangliupppvk) at (\gangliuxxxv, \gangliuyyyk);
\coordinate (gangliupppvl) at (\gangliuxxxv, \gangliuyyyl);
\coordinate (gangliupppvm) at (\gangliuxxxv, \gangliuyyym);
\coordinate (gangliupppvn) at (\gangliuxxxv, \gangliuyyyn);
\coordinate (gangliupppvo) at (\gangliuxxxv, \gangliuyyyo);
\coordinate (gangliupppvp) at (\gangliuxxxv, \gangliuyyyp);
\coordinate (gangliupppvq) at (\gangliuxxxv, \gangliuyyyq);
\coordinate (gangliupppvr) at (\gangliuxxxv, \gangliuyyyr);
\coordinate (gangliupppvs) at (\gangliuxxxv, \gangliuyyys);
\coordinate (gangliupppvt) at (\gangliuxxxv, \gangliuyyyt);
\coordinate (gangliupppvu) at (\gangliuxxxv, \gangliuyyyu);
\coordinate (gangliupppvv) at (\gangliuxxxv, \gangliuyyyv);
\coordinate (gangliupppvw) at (\gangliuxxxv, \gangliuyyyw);
\coordinate (gangliupppvx) at (\gangliuxxxv, \gangliuyyyx);
\coordinate (gangliupppvy) at (\gangliuxxxv, \gangliuyyyy);
\coordinate (gangliupppvz) at (\gangliuxxxv, \gangliuyyyz);
\coordinate (gangliupppwa) at (\gangliuxxxw, \gangliuyyya);
\coordinate (gangliupppwb) at (\gangliuxxxw, \gangliuyyyb);
\coordinate (gangliupppwc) at (\gangliuxxxw, \gangliuyyyc);
\coordinate (gangliupppwd) at (\gangliuxxxw, \gangliuyyyd);
\coordinate (gangliupppwe) at (\gangliuxxxw, \gangliuyyye);
\coordinate (gangliupppwf) at (\gangliuxxxw, \gangliuyyyf);
\coordinate (gangliupppwg) at (\gangliuxxxw, \gangliuyyyg);
\coordinate (gangliupppwh) at (\gangliuxxxw, \gangliuyyyh);
\coordinate (gangliupppwi) at (\gangliuxxxw, \gangliuyyyi);
\coordinate (gangliupppwj) at (\gangliuxxxw, \gangliuyyyj);
\coordinate (gangliupppwk) at (\gangliuxxxw, \gangliuyyyk);
\coordinate (gangliupppwl) at (\gangliuxxxw, \gangliuyyyl);
\coordinate (gangliupppwm) at (\gangliuxxxw, \gangliuyyym);
\coordinate (gangliupppwn) at (\gangliuxxxw, \gangliuyyyn);
\coordinate (gangliupppwo) at (\gangliuxxxw, \gangliuyyyo);
\coordinate (gangliupppwp) at (\gangliuxxxw, \gangliuyyyp);
\coordinate (gangliupppwq) at (\gangliuxxxw, \gangliuyyyq);
\coordinate (gangliupppwr) at (\gangliuxxxw, \gangliuyyyr);
\coordinate (gangliupppws) at (\gangliuxxxw, \gangliuyyys);
\coordinate (gangliupppwt) at (\gangliuxxxw, \gangliuyyyt);
\coordinate (gangliupppwu) at (\gangliuxxxw, \gangliuyyyu);
\coordinate (gangliupppwv) at (\gangliuxxxw, \gangliuyyyv);
\coordinate (gangliupppww) at (\gangliuxxxw, \gangliuyyyw);
\coordinate (gangliupppwx) at (\gangliuxxxw, \gangliuyyyx);
\coordinate (gangliupppwy) at (\gangliuxxxw, \gangliuyyyy);
\coordinate (gangliupppwz) at (\gangliuxxxw, \gangliuyyyz);
\coordinate (gangliupppxa) at (\gangliuxxxx, \gangliuyyya);
\coordinate (gangliupppxb) at (\gangliuxxxx, \gangliuyyyb);
\coordinate (gangliupppxc) at (\gangliuxxxx, \gangliuyyyc);
\coordinate (gangliupppxd) at (\gangliuxxxx, \gangliuyyyd);
\coordinate (gangliupppxe) at (\gangliuxxxx, \gangliuyyye);
\coordinate (gangliupppxf) at (\gangliuxxxx, \gangliuyyyf);
\coordinate (gangliupppxg) at (\gangliuxxxx, \gangliuyyyg);
\coordinate (gangliupppxh) at (\gangliuxxxx, \gangliuyyyh);
\coordinate (gangliupppxi) at (\gangliuxxxx, \gangliuyyyi);
\coordinate (gangliupppxj) at (\gangliuxxxx, \gangliuyyyj);
\coordinate (gangliupppxk) at (\gangliuxxxx, \gangliuyyyk);
\coordinate (gangliupppxl) at (\gangliuxxxx, \gangliuyyyl);
\coordinate (gangliupppxm) at (\gangliuxxxx, \gangliuyyym);
\coordinate (gangliupppxn) at (\gangliuxxxx, \gangliuyyyn);
\coordinate (gangliupppxo) at (\gangliuxxxx, \gangliuyyyo);
\coordinate (gangliupppxp) at (\gangliuxxxx, \gangliuyyyp);
\coordinate (gangliupppxq) at (\gangliuxxxx, \gangliuyyyq);
\coordinate (gangliupppxr) at (\gangliuxxxx, \gangliuyyyr);
\coordinate (gangliupppxs) at (\gangliuxxxx, \gangliuyyys);
\coordinate (gangliupppxt) at (\gangliuxxxx, \gangliuyyyt);
\coordinate (gangliupppxu) at (\gangliuxxxx, \gangliuyyyu);
\coordinate (gangliupppxv) at (\gangliuxxxx, \gangliuyyyv);
\coordinate (gangliupppxw) at (\gangliuxxxx, \gangliuyyyw);
\coordinate (gangliupppxx) at (\gangliuxxxx, \gangliuyyyx);
\coordinate (gangliupppxy) at (\gangliuxxxx, \gangliuyyyy);
\coordinate (gangliupppxz) at (\gangliuxxxx, \gangliuyyyz);
\coordinate (gangliupppya) at (\gangliuxxxy, \gangliuyyya);
\coordinate (gangliupppyb) at (\gangliuxxxy, \gangliuyyyb);
\coordinate (gangliupppyc) at (\gangliuxxxy, \gangliuyyyc);
\coordinate (gangliupppyd) at (\gangliuxxxy, \gangliuyyyd);
\coordinate (gangliupppye) at (\gangliuxxxy, \gangliuyyye);
\coordinate (gangliupppyf) at (\gangliuxxxy, \gangliuyyyf);
\coordinate (gangliupppyg) at (\gangliuxxxy, \gangliuyyyg);
\coordinate (gangliupppyh) at (\gangliuxxxy, \gangliuyyyh);
\coordinate (gangliupppyi) at (\gangliuxxxy, \gangliuyyyi);
\coordinate (gangliupppyj) at (\gangliuxxxy, \gangliuyyyj);
\coordinate (gangliupppyk) at (\gangliuxxxy, \gangliuyyyk);
\coordinate (gangliupppyl) at (\gangliuxxxy, \gangliuyyyl);
\coordinate (gangliupppym) at (\gangliuxxxy, \gangliuyyym);
\coordinate (gangliupppyn) at (\gangliuxxxy, \gangliuyyyn);
\coordinate (gangliupppyo) at (\gangliuxxxy, \gangliuyyyo);
\coordinate (gangliupppyp) at (\gangliuxxxy, \gangliuyyyp);
\coordinate (gangliupppyq) at (\gangliuxxxy, \gangliuyyyq);
\coordinate (gangliupppyr) at (\gangliuxxxy, \gangliuyyyr);
\coordinate (gangliupppys) at (\gangliuxxxy, \gangliuyyys);
\coordinate (gangliupppyt) at (\gangliuxxxy, \gangliuyyyt);
\coordinate (gangliupppyu) at (\gangliuxxxy, \gangliuyyyu);
\coordinate (gangliupppyv) at (\gangliuxxxy, \gangliuyyyv);
\coordinate (gangliupppyw) at (\gangliuxxxy, \gangliuyyyw);
\coordinate (gangliupppyx) at (\gangliuxxxy, \gangliuyyyx);
\coordinate (gangliupppyy) at (\gangliuxxxy, \gangliuyyyy);
\coordinate (gangliupppyz) at (\gangliuxxxy, \gangliuyyyz);
\coordinate (gangliupppza) at (\gangliuxxxz, \gangliuyyya);
\coordinate (gangliupppzb) at (\gangliuxxxz, \gangliuyyyb);
\coordinate (gangliupppzc) at (\gangliuxxxz, \gangliuyyyc);
\coordinate (gangliupppzd) at (\gangliuxxxz, \gangliuyyyd);
\coordinate (gangliupppze) at (\gangliuxxxz, \gangliuyyye);
\coordinate (gangliupppzf) at (\gangliuxxxz, \gangliuyyyf);
\coordinate (gangliupppzg) at (\gangliuxxxz, \gangliuyyyg);
\coordinate (gangliupppzh) at (\gangliuxxxz, \gangliuyyyh);
\coordinate (gangliupppzi) at (\gangliuxxxz, \gangliuyyyi);
\coordinate (gangliupppzj) at (\gangliuxxxz, \gangliuyyyj);
\coordinate (gangliupppzk) at (\gangliuxxxz, \gangliuyyyk);
\coordinate (gangliupppzl) at (\gangliuxxxz, \gangliuyyyl);
\coordinate (gangliupppzm) at (\gangliuxxxz, \gangliuyyym);
\coordinate (gangliupppzn) at (\gangliuxxxz, \gangliuyyyn);
\coordinate (gangliupppzo) at (\gangliuxxxz, \gangliuyyyo);
\coordinate (gangliupppzp) at (\gangliuxxxz, \gangliuyyyp);
\coordinate (gangliupppzq) at (\gangliuxxxz, \gangliuyyyq);
\coordinate (gangliupppzr) at (\gangliuxxxz, \gangliuyyyr);
\coordinate (gangliupppzs) at (\gangliuxxxz, \gangliuyyys);
\coordinate (gangliupppzt) at (\gangliuxxxz, \gangliuyyyt);
\coordinate (gangliupppzu) at (\gangliuxxxz, \gangliuyyyu);
\coordinate (gangliupppzv) at (\gangliuxxxz, \gangliuyyyv);
\coordinate (gangliupppzw) at (\gangliuxxxz, \gangliuyyyw);
\coordinate (gangliupppzx) at (\gangliuxxxz, \gangliuyyyx);
\coordinate (gangliupppzy) at (\gangliuxxxz, \gangliuyyyy);
\coordinate (gangliupppzz) at (\gangliuxxxz, \gangliuyyyz);

%\gangprintcoordinateat{(0,0)}{The last coordinate values: }{($(gangliupppzz)$)}; 




% Draw related part of the coordinate system with dashed helplines with letters as background. 
\coordinatebackgroundxy{gangliu}{b}{c}{v}{a}{b}{s};


\node [nigfetd](nigfetd) at (gangliupppon) {F3055L};

\node [anchor=south] at (nigfetd.G) {G};
\node [anchor= west] at (nigfetd.D) {D};
\node [anchor= west] at (nigfetd.S) {S};

% To retrieve x- and y-component of the coordinates  (nigfetd.G), (nigfetd.D), and (nigfetd.S) separately. 
\getxyingivenunit{cm}{(nigfetd.G)}
                     {\nigfetdgx} {\nigfetdgy};
\getxyingivenunit{cm}{(nigfetd.D)}
                     {\nigfetddx} {\nigfetddy};
\getxyingivenunit{cm}{(nigfetd.S)}
                     {\nigfetdsx} {\nigfetdsy};

\draw  [o-] (gangliupppbr) node [anchor=east] {$5V$} --
       (\nigfetddx, \gangliuyyyr);

\draw  (\nigfetddx, \gangliuyyyr)
       to [Telmech=M1, n=motor]
       (\nigfetddx, \gangliuyyyp) --
       (nigfetd.D);

\node [xshift= 2mm] at (motor.block north west) {$-$};
\node [xshift= 2mm] at (motor.block south west) {$+$};

\draw  (\nigfetddx, \gangliuyyyp) --
       (gangliupppnp)
       to [full diode = 1N4001, label/align=rotate] 
       (gangliupppnr); 

\draw  (nigfetd.S) -- 
       (\nigfetdsx, \gangliuyyyk)
          node [ground] {};

 
% To draw LM555
\draw [blue, line width=0.5mm] 
      (gangliuppphk) rectangle (gangliupppkq);
 
\node [blue, xshift=4mm] at (gangliupppio)
      {\underline{LM555}};

\draw (gangliupppir) -- 
      (gangliupppiq) node [anchor=north] {8};

\draw (gangliupppjr) -- 
      (gangliupppjq) node [anchor=north] {4};


\draw (nigfetd.G) -- 
      (\gangliuxxxn, \nigfetdgy) 
      to [R, l_=$R_2 \text{=} 330 \Omega $] 
      (\gangliuxxxk, \nigfetdgy) 
      node [anchor=east] {3};
 
\draw (gangliupppkm) node [anchor=east] {1}  --
      (\nigfetdsx, \gangliuyyym);

\draw (gangliupppkl) node [anchor=east] {5} 
      to [C, l_=$C_2  \text{=} 0.01 \mu F$] 
      (gangliupppnl) -- 
      (\nigfetdsx, \gangliuyyyl);


\draw (gangliupppdr) 
      to [R = $R_1 \text{=} 1k \Omega$] 
      (gangliupppdp) -- 
      (gangliuppphp) node [anchor=west] {7};
 
\draw (gangliupppdn) 
      to [full diode = 1N4001, label/align=rotate]
      (gangliupppdp);
 
\draw (gangliupppgp) 
      to [full diode = 1N4001, label/align=rotate]
      (gangliupppgn);
 

\draw (gangliupppgn) 
      to [potentiometer, l_=$R_3\text{=} 100k \Omega$,                                                       n=mypot]
      (gangliupppdn);

\getxyingivenunit{cm}{(mypot.wiper)}
                     {\mypotwiperx}{\mypotwipery};


\draw (mypot.wiper) -- 
      (\mypotwiperx, \gangliuyyym) -- 
      (gangliuppphm) node [anchor=west] {6};

\draw  (\mypotwiperx, \gangliuyyym) -- 
       (\mypotwiperx, \gangliuyyyl) -- 
       (gangliuppphl) node [anchor=west] {2};
 

\draw  (\mypotwiperx, \gangliuyyyl) 
       to [C, n = capacitorl] 
       (\mypotwiperx, \gangliuyyyk) node[ground]{};

\node [anchor=north west, xshift=2mm, yshift=.7mm] 
      at (capacitorl) {$C_1 \text{=} 0.1 \mu F$};




% Move the whole circuit toward the paper center a little bit.
\draw [white] (\gangliuxxxa - 3.0, \gangliuyyyv)   --
           (gangliupppkv) ;


\end{circuitikz}



\end{document}