% This is circuit 2 of example 02 of
% https://github.com/LiuGangKingston/Nestable-coordinate-system-for-TikZ-circuits.git


\documentclass[tikz,border=5mm]{standalone}
\usepackage[siunitx]{circuitikz}
\usetikzlibrary{shapes,arrows,positioning}
%   This is an accessory  file for 
%   https://github.com/LiuGangKingston/Nestable-coordinate-system-for-Tikz-circuits.git
%            Version 1.0
%   free for non-commercial use.
%   Please send us emails for any problems/suggestions/comments.
%   Please be advised that none of us accept any responsibility
%   for any consequences arising out of the usage of this
%   software, especially for damage.
%   For usage, please refer to the README file and the following lines.
%   This code was written by
%        Gang Liu (gl.cell@outlook)
%                 (http://orcid.org/0000-0003-1575-9290)
%          and
%        Shiwei Huang (huang937@gmail.com)
%   Copyright (c) 2021
%
%
%  The following command is to get the x-component and y-component 
%  of a coordinate. The command is
%  \getxyofcoordinate{the coordinate}{x-component}{y-component};
\newcommand{\getxyofcoordinate}[3]{%
\coordinate (tempcoord) at ($#1$);
\path (tempcoord) node {};
\pgfgetlastxy{\tempx}{\tempy};
\pgfmathsetmacro{#2}{\tempx}
\pgfmathsetmacro{#3}{\tempy}
}


%  The following command is the same as above but for given unit.
%  The command is
%  \getxyingivenunit{the unit like cm}{the coordinate}{x-component}{y-component};
\newcommand{\getxyingivenunit}[4]{%
\coordinate (tempcoord) at (1#1,1#1);
\path (tempcoord) node {};
\pgfgetlastxy{\tempxunit}{\tempyunit};
\coordinate (tempcoord) at ($#2$);
\path (tempcoord) node {};
\pgfgetlastxy{\tempx}{\tempy};
\pgfmathsetmacro{#3}{\tempx/\tempxunit}
\pgfmathsetmacro{#4}{\tempy/\tempyunit}
}


%  The following command is to print the value of a coordinate with some words at the first coordinate postion 
%  The command is
%  \printcoordinateat{the first coordinate}{the words}{the coordinate};
\newcommand{\printcoordinateat}[3]{%
\getxyingivenunit{cm}{#3}{\tempxx}{\tempyy}
\node at #1 {#2 ($\tempxx$, $\tempyy$).};
}


%  The following command is to print a keyworded coordinate system as a background.
%  The command is
%  \coordinatebackground{the KEYWORD}
%                                            {the first letter in both x and y directions}
%                                       {the second letter in both x and y directions}
%                                             {the last letter in both x and y directions};
\newcommand{\coordinatebackground}[4]{
\pgfmathsetmacro{\colourpercent}{30}
\foreach \i in {#2,#3,...,#4} 
{\node [black!\colourpercent] at (#1ppp\i\i) {\i};}
\foreach \i in {#2,#4} 
{\node [white] at (#1ppp\i\i) {\i};}
\coordinatebackgroundxy{#1}{#2}{#3}{#4}{#2}{#3}{#4};
}


%  The following command is to print a keyworded coordinate system as a background.
%  The command is
%  \coordinatebackgroundxy{the KEYWORD}
%                                                {the first letter in the x direction}
%                                           {the second letter in the x direction}
%                                                 {the last letter in the x direction}
%                                                {the first letter in the y direction}
%                                           {the second letter in the y direction}
%                                                 {the last letter in the y direction};
\newcommand{\coordinatebackgroundxy}[7]{
\pgfmathsetmacro{\bordercolourpercent}{60}
\pgfmathsetmacro{\colourpercent}{30}

\foreach \i in {#2,#3,...,#4} 
\foreach \j in {#5} 
\foreach \k in {#7} 
{\draw [dashed,black!\colourpercent] (#1ppp\i\j) -- (#1ppp\i\k);}

\foreach \i in {#5,#6,...,#7} 
\foreach \j in {#2} 
\foreach \k in {#4} 
{\draw [dashed,black!\colourpercent] (#1ppp\j\i) -- (#1ppp\k\i);}

\foreach \i in {#2,#4} 
\foreach \j in {#5} 
\foreach \k in {#7} 
{\draw [dashed,black!\bordercolourpercent] (#1ppp\i\j) -- (#1ppp\i\k);}

\foreach \i in {#5,#7} 
\foreach \j in {#2} 
\foreach \k in {#4} 
{\draw [dashed,black!\bordercolourpercent] (#1ppp\j\i) -- (#1ppp\k\i);}

\foreach \i in {#2,#3,...,#4} 
\foreach \j in {#5} 
\foreach \k in {#7} 
{
\node [black!\bordercolourpercent] at ($(#1ppp\i\j) + (0,-.2)$) {\i};
\node [black!\bordercolourpercent] at ($(#1ppp\i\k) + (0,.2)$) {\i};
}

\foreach \i in {#5,#6,...,#7} 
\foreach \j in {#2} 
\foreach \k in {#4} 
{
\node [black!\bordercolourpercent] at ($(#1ppp\k\i) + (.2,0)$) {\i};
\node [black!\bordercolourpercent] at ($(#1ppp\j\i) + (-.2,0)$) {\i};
}

}






\begin{document}


\ctikzset{
 /tikz/circuitikz/diodes/scale=0.8,
 /tikz/circuitikz/bipoles/length=1cm
}


 
\begin{circuitikz} [scale=0.8]
% https://github.com/LiuGangKingston/Nestable-coordinate-system-for-Tikz-circuits.git
% https://github.com/LiuGangKingston/Nestable-coordinate-system-for-Tikz-circuits.git



% https://github.com/LiuGangKingston/Nestable-coordinate-system-for-Tikz-circuits.git
% https://github.com/LiuGangKingston/Nestable-coordinate-system-for-Tikz-circuits.git


% https://github.com/LiuGangKingston/Nestable-coordinate-system-for-Tikz-circuits.git
% https://github.com/LiuGangKingston/Nestable-coordinate-system-for-Tikz-circuits.git


\pgfmathsetmacro{\totalgangliuxxx}{26}
\pgfmathsetmacro{\totalgangliuyyy}{26}
\pgfmathsetmacro{\gangliuxxxspacing}{1}
\pgfmathsetmacro{\gangliuyyyspacing}{1}
\pgfmathsetmacro{\gangliuxxxa}{-8}
\pgfmathsetmacro{\gangliuyyya}{-8}

\pgfmathsetmacro{\gangliuxxxb}{\gangliuxxxa + \gangliuxxxspacing + 0.0 }
\pgfmathsetmacro{\gangliuxxxc}{\gangliuxxxb + \gangliuxxxspacing + 0.0 }
\pgfmathsetmacro{\gangliuxxxd}{\gangliuxxxc + \gangliuxxxspacing + 0.0 }
\pgfmathsetmacro{\gangliuxxxe}{\gangliuxxxd + \gangliuxxxspacing + 0.0 }
\pgfmathsetmacro{\gangliuxxxf}{\gangliuxxxe + \gangliuxxxspacing + 0.0 }
\pgfmathsetmacro{\gangliuxxxg}{\gangliuxxxf + \gangliuxxxspacing + 0.0 }
\pgfmathsetmacro{\gangliuxxxh}{\gangliuxxxg + \gangliuxxxspacing + 0.0 }
\pgfmathsetmacro{\gangliuxxxi}{\gangliuxxxh + \gangliuxxxspacing + 0.0 }
\pgfmathsetmacro{\gangliuxxxj}{\gangliuxxxi + \gangliuxxxspacing + 0.0 }
\pgfmathsetmacro{\gangliuxxxk}{\gangliuxxxj + \gangliuxxxspacing + 0.0 }
\pgfmathsetmacro{\gangliuxxxl}{\gangliuxxxk + \gangliuxxxspacing + 0.0 }
\pgfmathsetmacro{\gangliuxxxm}{\gangliuxxxl + \gangliuxxxspacing + 0.0 }
\pgfmathsetmacro{\gangliuxxxn}{\gangliuxxxm + \gangliuxxxspacing + 0.0 }
\pgfmathsetmacro{\gangliuxxxo}{\gangliuxxxn + \gangliuxxxspacing + 0.0 }
\pgfmathsetmacro{\gangliuxxxp}{\gangliuxxxo + \gangliuxxxspacing + 0.0 }
\pgfmathsetmacro{\gangliuxxxq}{\gangliuxxxp + \gangliuxxxspacing + 0.0 }
\pgfmathsetmacro{\gangliuxxxr}{\gangliuxxxq + \gangliuxxxspacing + 0.0 }
\pgfmathsetmacro{\gangliuxxxs}{\gangliuxxxr + \gangliuxxxspacing + 0.0 }
\pgfmathsetmacro{\gangliuxxxt}{\gangliuxxxs + \gangliuxxxspacing + 0.0 }
\pgfmathsetmacro{\gangliuxxxu}{\gangliuxxxt + \gangliuxxxspacing + 0.0 }
\pgfmathsetmacro{\gangliuxxxv}{\gangliuxxxu + \gangliuxxxspacing + 0.0 }
\pgfmathsetmacro{\gangliuxxxw}{\gangliuxxxv + \gangliuxxxspacing + 0.0 }
\pgfmathsetmacro{\gangliuxxxx}{\gangliuxxxw + \gangliuxxxspacing + 0.0 }
\pgfmathsetmacro{\gangliuxxxy}{\gangliuxxxx + \gangliuxxxspacing + 0.0 }
\pgfmathsetmacro{\gangliuxxxz}{\gangliuxxxy + \gangliuxxxspacing + 0.0 }

\pgfmathsetmacro{\gangliuyyyb}{\gangliuyyya + \gangliuyyyspacing + 0.0 }
\pgfmathsetmacro{\gangliuyyyc}{\gangliuyyyb + \gangliuyyyspacing + 0.0 }
\pgfmathsetmacro{\gangliuyyyd}{\gangliuyyyc + \gangliuyyyspacing + 0.0 }
\pgfmathsetmacro{\gangliuyyye}{\gangliuyyyd + \gangliuyyyspacing + 0.0 }
\pgfmathsetmacro{\gangliuyyyf}{\gangliuyyye + \gangliuyyyspacing + 0.0 }
\pgfmathsetmacro{\gangliuyyyg}{\gangliuyyyf + \gangliuyyyspacing + 0.0 }
\pgfmathsetmacro{\gangliuyyyh}{\gangliuyyyg + \gangliuyyyspacing + 0.0 }
\pgfmathsetmacro{\gangliuyyyi}{\gangliuyyyh + \gangliuyyyspacing + 0.0 }
\pgfmathsetmacro{\gangliuyyyj}{\gangliuyyyi + \gangliuyyyspacing + 0.0 }
\pgfmathsetmacro{\gangliuyyyk}{\gangliuyyyj + \gangliuyyyspacing + 0.0 }
\pgfmathsetmacro{\gangliuyyyl}{\gangliuyyyk + \gangliuyyyspacing + 0.0 }
\pgfmathsetmacro{\gangliuyyym}{\gangliuyyyl + \gangliuyyyspacing + 0.0 }
\pgfmathsetmacro{\gangliuyyyn}{\gangliuyyym + \gangliuyyyspacing + 0.0 }
\pgfmathsetmacro{\gangliuyyyo}{\gangliuyyyn + \gangliuyyyspacing + 0.0 }
\pgfmathsetmacro{\gangliuyyyp}{\gangliuyyyo + \gangliuyyyspacing + 0.0 }
\pgfmathsetmacro{\gangliuyyyq}{\gangliuyyyp + \gangliuyyyspacing + 0.0 }
\pgfmathsetmacro{\gangliuyyyr}{\gangliuyyyq + \gangliuyyyspacing + 0.0 }
\pgfmathsetmacro{\gangliuyyys}{\gangliuyyyr + \gangliuyyyspacing + 0.0 }
\pgfmathsetmacro{\gangliuyyyt}{\gangliuyyys + \gangliuyyyspacing + 0.0 }
\pgfmathsetmacro{\gangliuyyyu}{\gangliuyyyt + \gangliuyyyspacing + 0.0 }
\pgfmathsetmacro{\gangliuyyyv}{\gangliuyyyu + \gangliuyyyspacing + 0.0 }
\pgfmathsetmacro{\gangliuyyyw}{\gangliuyyyv + \gangliuyyyspacing + 0.0 }
\pgfmathsetmacro{\gangliuyyyx}{\gangliuyyyw + \gangliuyyyspacing + 0.0 }
\pgfmathsetmacro{\gangliuyyyy}{\gangliuyyyx + \gangliuyyyspacing + 0.0 }
\pgfmathsetmacro{\gangliuyyyz}{\gangliuyyyy + \gangliuyyyspacing + 0.0 }

\coordinate (gangliupppaa) at (\gangliuxxxa, \gangliuyyya);
\coordinate (gangliupppab) at (\gangliuxxxa, \gangliuyyyb);
\coordinate (gangliupppac) at (\gangliuxxxa, \gangliuyyyc);
\coordinate (gangliupppad) at (\gangliuxxxa, \gangliuyyyd);
\coordinate (gangliupppae) at (\gangliuxxxa, \gangliuyyye);
\coordinate (gangliupppaf) at (\gangliuxxxa, \gangliuyyyf);
\coordinate (gangliupppag) at (\gangliuxxxa, \gangliuyyyg);
\coordinate (gangliupppah) at (\gangliuxxxa, \gangliuyyyh);
\coordinate (gangliupppai) at (\gangliuxxxa, \gangliuyyyi);
\coordinate (gangliupppaj) at (\gangliuxxxa, \gangliuyyyj);
\coordinate (gangliupppak) at (\gangliuxxxa, \gangliuyyyk);
\coordinate (gangliupppal) at (\gangliuxxxa, \gangliuyyyl);
\coordinate (gangliupppam) at (\gangliuxxxa, \gangliuyyym);
\coordinate (gangliupppan) at (\gangliuxxxa, \gangliuyyyn);
\coordinate (gangliupppao) at (\gangliuxxxa, \gangliuyyyo);
\coordinate (gangliupppap) at (\gangliuxxxa, \gangliuyyyp);
\coordinate (gangliupppaq) at (\gangliuxxxa, \gangliuyyyq);
\coordinate (gangliupppar) at (\gangliuxxxa, \gangliuyyyr);
\coordinate (gangliupppas) at (\gangliuxxxa, \gangliuyyys);
\coordinate (gangliupppat) at (\gangliuxxxa, \gangliuyyyt);
\coordinate (gangliupppau) at (\gangliuxxxa, \gangliuyyyu);
\coordinate (gangliupppav) at (\gangliuxxxa, \gangliuyyyv);
\coordinate (gangliupppaw) at (\gangliuxxxa, \gangliuyyyw);
\coordinate (gangliupppax) at (\gangliuxxxa, \gangliuyyyx);
\coordinate (gangliupppay) at (\gangliuxxxa, \gangliuyyyy);
\coordinate (gangliupppaz) at (\gangliuxxxa, \gangliuyyyz);
\coordinate (gangliupppba) at (\gangliuxxxb, \gangliuyyya);
\coordinate (gangliupppbb) at (\gangliuxxxb, \gangliuyyyb);
\coordinate (gangliupppbc) at (\gangliuxxxb, \gangliuyyyc);
\coordinate (gangliupppbd) at (\gangliuxxxb, \gangliuyyyd);
\coordinate (gangliupppbe) at (\gangliuxxxb, \gangliuyyye);
\coordinate (gangliupppbf) at (\gangliuxxxb, \gangliuyyyf);
\coordinate (gangliupppbg) at (\gangliuxxxb, \gangliuyyyg);
\coordinate (gangliupppbh) at (\gangliuxxxb, \gangliuyyyh);
\coordinate (gangliupppbi) at (\gangliuxxxb, \gangliuyyyi);
\coordinate (gangliupppbj) at (\gangliuxxxb, \gangliuyyyj);
\coordinate (gangliupppbk) at (\gangliuxxxb, \gangliuyyyk);
\coordinate (gangliupppbl) at (\gangliuxxxb, \gangliuyyyl);
\coordinate (gangliupppbm) at (\gangliuxxxb, \gangliuyyym);
\coordinate (gangliupppbn) at (\gangliuxxxb, \gangliuyyyn);
\coordinate (gangliupppbo) at (\gangliuxxxb, \gangliuyyyo);
\coordinate (gangliupppbp) at (\gangliuxxxb, \gangliuyyyp);
\coordinate (gangliupppbq) at (\gangliuxxxb, \gangliuyyyq);
\coordinate (gangliupppbr) at (\gangliuxxxb, \gangliuyyyr);
\coordinate (gangliupppbs) at (\gangliuxxxb, \gangliuyyys);
\coordinate (gangliupppbt) at (\gangliuxxxb, \gangliuyyyt);
\coordinate (gangliupppbu) at (\gangliuxxxb, \gangliuyyyu);
\coordinate (gangliupppbv) at (\gangliuxxxb, \gangliuyyyv);
\coordinate (gangliupppbw) at (\gangliuxxxb, \gangliuyyyw);
\coordinate (gangliupppbx) at (\gangliuxxxb, \gangliuyyyx);
\coordinate (gangliupppby) at (\gangliuxxxb, \gangliuyyyy);
\coordinate (gangliupppbz) at (\gangliuxxxb, \gangliuyyyz);
\coordinate (gangliupppca) at (\gangliuxxxc, \gangliuyyya);
\coordinate (gangliupppcb) at (\gangliuxxxc, \gangliuyyyb);
\coordinate (gangliupppcc) at (\gangliuxxxc, \gangliuyyyc);
\coordinate (gangliupppcd) at (\gangliuxxxc, \gangliuyyyd);
\coordinate (gangliupppce) at (\gangliuxxxc, \gangliuyyye);
\coordinate (gangliupppcf) at (\gangliuxxxc, \gangliuyyyf);
\coordinate (gangliupppcg) at (\gangliuxxxc, \gangliuyyyg);
\coordinate (gangliupppch) at (\gangliuxxxc, \gangliuyyyh);
\coordinate (gangliupppci) at (\gangliuxxxc, \gangliuyyyi);
\coordinate (gangliupppcj) at (\gangliuxxxc, \gangliuyyyj);
\coordinate (gangliupppck) at (\gangliuxxxc, \gangliuyyyk);
\coordinate (gangliupppcl) at (\gangliuxxxc, \gangliuyyyl);
\coordinate (gangliupppcm) at (\gangliuxxxc, \gangliuyyym);
\coordinate (gangliupppcn) at (\gangliuxxxc, \gangliuyyyn);
\coordinate (gangliupppco) at (\gangliuxxxc, \gangliuyyyo);
\coordinate (gangliupppcp) at (\gangliuxxxc, \gangliuyyyp);
\coordinate (gangliupppcq) at (\gangliuxxxc, \gangliuyyyq);
\coordinate (gangliupppcr) at (\gangliuxxxc, \gangliuyyyr);
\coordinate (gangliupppcs) at (\gangliuxxxc, \gangliuyyys);
\coordinate (gangliupppct) at (\gangliuxxxc, \gangliuyyyt);
\coordinate (gangliupppcu) at (\gangliuxxxc, \gangliuyyyu);
\coordinate (gangliupppcv) at (\gangliuxxxc, \gangliuyyyv);
\coordinate (gangliupppcw) at (\gangliuxxxc, \gangliuyyyw);
\coordinate (gangliupppcx) at (\gangliuxxxc, \gangliuyyyx);
\coordinate (gangliupppcy) at (\gangliuxxxc, \gangliuyyyy);
\coordinate (gangliupppcz) at (\gangliuxxxc, \gangliuyyyz);
\coordinate (gangliupppda) at (\gangliuxxxd, \gangliuyyya);
\coordinate (gangliupppdb) at (\gangliuxxxd, \gangliuyyyb);
\coordinate (gangliupppdc) at (\gangliuxxxd, \gangliuyyyc);
\coordinate (gangliupppdd) at (\gangliuxxxd, \gangliuyyyd);
\coordinate (gangliupppde) at (\gangliuxxxd, \gangliuyyye);
\coordinate (gangliupppdf) at (\gangliuxxxd, \gangliuyyyf);
\coordinate (gangliupppdg) at (\gangliuxxxd, \gangliuyyyg);
\coordinate (gangliupppdh) at (\gangliuxxxd, \gangliuyyyh);
\coordinate (gangliupppdi) at (\gangliuxxxd, \gangliuyyyi);
\coordinate (gangliupppdj) at (\gangliuxxxd, \gangliuyyyj);
\coordinate (gangliupppdk) at (\gangliuxxxd, \gangliuyyyk);
\coordinate (gangliupppdl) at (\gangliuxxxd, \gangliuyyyl);
\coordinate (gangliupppdm) at (\gangliuxxxd, \gangliuyyym);
\coordinate (gangliupppdn) at (\gangliuxxxd, \gangliuyyyn);
\coordinate (gangliupppdo) at (\gangliuxxxd, \gangliuyyyo);
\coordinate (gangliupppdp) at (\gangliuxxxd, \gangliuyyyp);
\coordinate (gangliupppdq) at (\gangliuxxxd, \gangliuyyyq);
\coordinate (gangliupppdr) at (\gangliuxxxd, \gangliuyyyr);
\coordinate (gangliupppds) at (\gangliuxxxd, \gangliuyyys);
\coordinate (gangliupppdt) at (\gangliuxxxd, \gangliuyyyt);
\coordinate (gangliupppdu) at (\gangliuxxxd, \gangliuyyyu);
\coordinate (gangliupppdv) at (\gangliuxxxd, \gangliuyyyv);
\coordinate (gangliupppdw) at (\gangliuxxxd, \gangliuyyyw);
\coordinate (gangliupppdx) at (\gangliuxxxd, \gangliuyyyx);
\coordinate (gangliupppdy) at (\gangliuxxxd, \gangliuyyyy);
\coordinate (gangliupppdz) at (\gangliuxxxd, \gangliuyyyz);
\coordinate (gangliupppea) at (\gangliuxxxe, \gangliuyyya);
\coordinate (gangliupppeb) at (\gangliuxxxe, \gangliuyyyb);
\coordinate (gangliupppec) at (\gangliuxxxe, \gangliuyyyc);
\coordinate (gangliuppped) at (\gangliuxxxe, \gangliuyyyd);
\coordinate (gangliupppee) at (\gangliuxxxe, \gangliuyyye);
\coordinate (gangliupppef) at (\gangliuxxxe, \gangliuyyyf);
\coordinate (gangliupppeg) at (\gangliuxxxe, \gangliuyyyg);
\coordinate (gangliupppeh) at (\gangliuxxxe, \gangliuyyyh);
\coordinate (gangliupppei) at (\gangliuxxxe, \gangliuyyyi);
\coordinate (gangliupppej) at (\gangliuxxxe, \gangliuyyyj);
\coordinate (gangliupppek) at (\gangliuxxxe, \gangliuyyyk);
\coordinate (gangliupppel) at (\gangliuxxxe, \gangliuyyyl);
\coordinate (gangliupppem) at (\gangliuxxxe, \gangliuyyym);
\coordinate (gangliupppen) at (\gangliuxxxe, \gangliuyyyn);
\coordinate (gangliupppeo) at (\gangliuxxxe, \gangliuyyyo);
\coordinate (gangliupppep) at (\gangliuxxxe, \gangliuyyyp);
\coordinate (gangliupppeq) at (\gangliuxxxe, \gangliuyyyq);
\coordinate (gangliuppper) at (\gangliuxxxe, \gangliuyyyr);
\coordinate (gangliupppes) at (\gangliuxxxe, \gangliuyyys);
\coordinate (gangliupppet) at (\gangliuxxxe, \gangliuyyyt);
\coordinate (gangliupppeu) at (\gangliuxxxe, \gangliuyyyu);
\coordinate (gangliupppev) at (\gangliuxxxe, \gangliuyyyv);
\coordinate (gangliupppew) at (\gangliuxxxe, \gangliuyyyw);
\coordinate (gangliupppex) at (\gangliuxxxe, \gangliuyyyx);
\coordinate (gangliupppey) at (\gangliuxxxe, \gangliuyyyy);
\coordinate (gangliupppez) at (\gangliuxxxe, \gangliuyyyz);
\coordinate (gangliupppfa) at (\gangliuxxxf, \gangliuyyya);
\coordinate (gangliupppfb) at (\gangliuxxxf, \gangliuyyyb);
\coordinate (gangliupppfc) at (\gangliuxxxf, \gangliuyyyc);
\coordinate (gangliupppfd) at (\gangliuxxxf, \gangliuyyyd);
\coordinate (gangliupppfe) at (\gangliuxxxf, \gangliuyyye);
\coordinate (gangliupppff) at (\gangliuxxxf, \gangliuyyyf);
\coordinate (gangliupppfg) at (\gangliuxxxf, \gangliuyyyg);
\coordinate (gangliupppfh) at (\gangliuxxxf, \gangliuyyyh);
\coordinate (gangliupppfi) at (\gangliuxxxf, \gangliuyyyi);
\coordinate (gangliupppfj) at (\gangliuxxxf, \gangliuyyyj);
\coordinate (gangliupppfk) at (\gangliuxxxf, \gangliuyyyk);
\coordinate (gangliupppfl) at (\gangliuxxxf, \gangliuyyyl);
\coordinate (gangliupppfm) at (\gangliuxxxf, \gangliuyyym);
\coordinate (gangliupppfn) at (\gangliuxxxf, \gangliuyyyn);
\coordinate (gangliupppfo) at (\gangliuxxxf, \gangliuyyyo);
\coordinate (gangliupppfp) at (\gangliuxxxf, \gangliuyyyp);
\coordinate (gangliupppfq) at (\gangliuxxxf, \gangliuyyyq);
\coordinate (gangliupppfr) at (\gangliuxxxf, \gangliuyyyr);
\coordinate (gangliupppfs) at (\gangliuxxxf, \gangliuyyys);
\coordinate (gangliupppft) at (\gangliuxxxf, \gangliuyyyt);
\coordinate (gangliupppfu) at (\gangliuxxxf, \gangliuyyyu);
\coordinate (gangliupppfv) at (\gangliuxxxf, \gangliuyyyv);
\coordinate (gangliupppfw) at (\gangliuxxxf, \gangliuyyyw);
\coordinate (gangliupppfx) at (\gangliuxxxf, \gangliuyyyx);
\coordinate (gangliupppfy) at (\gangliuxxxf, \gangliuyyyy);
\coordinate (gangliupppfz) at (\gangliuxxxf, \gangliuyyyz);
\coordinate (gangliupppga) at (\gangliuxxxg, \gangliuyyya);
\coordinate (gangliupppgb) at (\gangliuxxxg, \gangliuyyyb);
\coordinate (gangliupppgc) at (\gangliuxxxg, \gangliuyyyc);
\coordinate (gangliupppgd) at (\gangliuxxxg, \gangliuyyyd);
\coordinate (gangliupppge) at (\gangliuxxxg, \gangliuyyye);
\coordinate (gangliupppgf) at (\gangliuxxxg, \gangliuyyyf);
\coordinate (gangliupppgg) at (\gangliuxxxg, \gangliuyyyg);
\coordinate (gangliupppgh) at (\gangliuxxxg, \gangliuyyyh);
\coordinate (gangliupppgi) at (\gangliuxxxg, \gangliuyyyi);
\coordinate (gangliupppgj) at (\gangliuxxxg, \gangliuyyyj);
\coordinate (gangliupppgk) at (\gangliuxxxg, \gangliuyyyk);
\coordinate (gangliupppgl) at (\gangliuxxxg, \gangliuyyyl);
\coordinate (gangliupppgm) at (\gangliuxxxg, \gangliuyyym);
\coordinate (gangliupppgn) at (\gangliuxxxg, \gangliuyyyn);
\coordinate (gangliupppgo) at (\gangliuxxxg, \gangliuyyyo);
\coordinate (gangliupppgp) at (\gangliuxxxg, \gangliuyyyp);
\coordinate (gangliupppgq) at (\gangliuxxxg, \gangliuyyyq);
\coordinate (gangliupppgr) at (\gangliuxxxg, \gangliuyyyr);
\coordinate (gangliupppgs) at (\gangliuxxxg, \gangliuyyys);
\coordinate (gangliupppgt) at (\gangliuxxxg, \gangliuyyyt);
\coordinate (gangliupppgu) at (\gangliuxxxg, \gangliuyyyu);
\coordinate (gangliupppgv) at (\gangliuxxxg, \gangliuyyyv);
\coordinate (gangliupppgw) at (\gangliuxxxg, \gangliuyyyw);
\coordinate (gangliupppgx) at (\gangliuxxxg, \gangliuyyyx);
\coordinate (gangliupppgy) at (\gangliuxxxg, \gangliuyyyy);
\coordinate (gangliupppgz) at (\gangliuxxxg, \gangliuyyyz);
\coordinate (gangliupppha) at (\gangliuxxxh, \gangliuyyya);
\coordinate (gangliuppphb) at (\gangliuxxxh, \gangliuyyyb);
\coordinate (gangliuppphc) at (\gangliuxxxh, \gangliuyyyc);
\coordinate (gangliuppphd) at (\gangliuxxxh, \gangliuyyyd);
\coordinate (gangliuppphe) at (\gangliuxxxh, \gangliuyyye);
\coordinate (gangliuppphf) at (\gangliuxxxh, \gangliuyyyf);
\coordinate (gangliuppphg) at (\gangliuxxxh, \gangliuyyyg);
\coordinate (gangliuppphh) at (\gangliuxxxh, \gangliuyyyh);
\coordinate (gangliuppphi) at (\gangliuxxxh, \gangliuyyyi);
\coordinate (gangliuppphj) at (\gangliuxxxh, \gangliuyyyj);
\coordinate (gangliuppphk) at (\gangliuxxxh, \gangliuyyyk);
\coordinate (gangliuppphl) at (\gangliuxxxh, \gangliuyyyl);
\coordinate (gangliuppphm) at (\gangliuxxxh, \gangliuyyym);
\coordinate (gangliuppphn) at (\gangliuxxxh, \gangliuyyyn);
\coordinate (gangliupppho) at (\gangliuxxxh, \gangliuyyyo);
\coordinate (gangliuppphp) at (\gangliuxxxh, \gangliuyyyp);
\coordinate (gangliuppphq) at (\gangliuxxxh, \gangliuyyyq);
\coordinate (gangliuppphr) at (\gangliuxxxh, \gangliuyyyr);
\coordinate (gangliuppphs) at (\gangliuxxxh, \gangliuyyys);
\coordinate (gangliupppht) at (\gangliuxxxh, \gangliuyyyt);
\coordinate (gangliuppphu) at (\gangliuxxxh, \gangliuyyyu);
\coordinate (gangliuppphv) at (\gangliuxxxh, \gangliuyyyv);
\coordinate (gangliuppphw) at (\gangliuxxxh, \gangliuyyyw);
\coordinate (gangliuppphx) at (\gangliuxxxh, \gangliuyyyx);
\coordinate (gangliuppphy) at (\gangliuxxxh, \gangliuyyyy);
\coordinate (gangliuppphz) at (\gangliuxxxh, \gangliuyyyz);
\coordinate (gangliupppia) at (\gangliuxxxi, \gangliuyyya);
\coordinate (gangliupppib) at (\gangliuxxxi, \gangliuyyyb);
\coordinate (gangliupppic) at (\gangliuxxxi, \gangliuyyyc);
\coordinate (gangliupppid) at (\gangliuxxxi, \gangliuyyyd);
\coordinate (gangliupppie) at (\gangliuxxxi, \gangliuyyye);
\coordinate (gangliupppif) at (\gangliuxxxi, \gangliuyyyf);
\coordinate (gangliupppig) at (\gangliuxxxi, \gangliuyyyg);
\coordinate (gangliupppih) at (\gangliuxxxi, \gangliuyyyh);
\coordinate (gangliupppii) at (\gangliuxxxi, \gangliuyyyi);
\coordinate (gangliupppij) at (\gangliuxxxi, \gangliuyyyj);
\coordinate (gangliupppik) at (\gangliuxxxi, \gangliuyyyk);
\coordinate (gangliupppil) at (\gangliuxxxi, \gangliuyyyl);
\coordinate (gangliupppim) at (\gangliuxxxi, \gangliuyyym);
\coordinate (gangliupppin) at (\gangliuxxxi, \gangliuyyyn);
\coordinate (gangliupppio) at (\gangliuxxxi, \gangliuyyyo);
\coordinate (gangliupppip) at (\gangliuxxxi, \gangliuyyyp);
\coordinate (gangliupppiq) at (\gangliuxxxi, \gangliuyyyq);
\coordinate (gangliupppir) at (\gangliuxxxi, \gangliuyyyr);
\coordinate (gangliupppis) at (\gangliuxxxi, \gangliuyyys);
\coordinate (gangliupppit) at (\gangliuxxxi, \gangliuyyyt);
\coordinate (gangliupppiu) at (\gangliuxxxi, \gangliuyyyu);
\coordinate (gangliupppiv) at (\gangliuxxxi, \gangliuyyyv);
\coordinate (gangliupppiw) at (\gangliuxxxi, \gangliuyyyw);
\coordinate (gangliupppix) at (\gangliuxxxi, \gangliuyyyx);
\coordinate (gangliupppiy) at (\gangliuxxxi, \gangliuyyyy);
\coordinate (gangliupppiz) at (\gangliuxxxi, \gangliuyyyz);
\coordinate (gangliupppja) at (\gangliuxxxj, \gangliuyyya);
\coordinate (gangliupppjb) at (\gangliuxxxj, \gangliuyyyb);
\coordinate (gangliupppjc) at (\gangliuxxxj, \gangliuyyyc);
\coordinate (gangliupppjd) at (\gangliuxxxj, \gangliuyyyd);
\coordinate (gangliupppje) at (\gangliuxxxj, \gangliuyyye);
\coordinate (gangliupppjf) at (\gangliuxxxj, \gangliuyyyf);
\coordinate (gangliupppjg) at (\gangliuxxxj, \gangliuyyyg);
\coordinate (gangliupppjh) at (\gangliuxxxj, \gangliuyyyh);
\coordinate (gangliupppji) at (\gangliuxxxj, \gangliuyyyi);
\coordinate (gangliupppjj) at (\gangliuxxxj, \gangliuyyyj);
\coordinate (gangliupppjk) at (\gangliuxxxj, \gangliuyyyk);
\coordinate (gangliupppjl) at (\gangliuxxxj, \gangliuyyyl);
\coordinate (gangliupppjm) at (\gangliuxxxj, \gangliuyyym);
\coordinate (gangliupppjn) at (\gangliuxxxj, \gangliuyyyn);
\coordinate (gangliupppjo) at (\gangliuxxxj, \gangliuyyyo);
\coordinate (gangliupppjp) at (\gangliuxxxj, \gangliuyyyp);
\coordinate (gangliupppjq) at (\gangliuxxxj, \gangliuyyyq);
\coordinate (gangliupppjr) at (\gangliuxxxj, \gangliuyyyr);
\coordinate (gangliupppjs) at (\gangliuxxxj, \gangliuyyys);
\coordinate (gangliupppjt) at (\gangliuxxxj, \gangliuyyyt);
\coordinate (gangliupppju) at (\gangliuxxxj, \gangliuyyyu);
\coordinate (gangliupppjv) at (\gangliuxxxj, \gangliuyyyv);
\coordinate (gangliupppjw) at (\gangliuxxxj, \gangliuyyyw);
\coordinate (gangliupppjx) at (\gangliuxxxj, \gangliuyyyx);
\coordinate (gangliupppjy) at (\gangliuxxxj, \gangliuyyyy);
\coordinate (gangliupppjz) at (\gangliuxxxj, \gangliuyyyz);
\coordinate (gangliupppka) at (\gangliuxxxk, \gangliuyyya);
\coordinate (gangliupppkb) at (\gangliuxxxk, \gangliuyyyb);
\coordinate (gangliupppkc) at (\gangliuxxxk, \gangliuyyyc);
\coordinate (gangliupppkd) at (\gangliuxxxk, \gangliuyyyd);
\coordinate (gangliupppke) at (\gangliuxxxk, \gangliuyyye);
\coordinate (gangliupppkf) at (\gangliuxxxk, \gangliuyyyf);
\coordinate (gangliupppkg) at (\gangliuxxxk, \gangliuyyyg);
\coordinate (gangliupppkh) at (\gangliuxxxk, \gangliuyyyh);
\coordinate (gangliupppki) at (\gangliuxxxk, \gangliuyyyi);
\coordinate (gangliupppkj) at (\gangliuxxxk, \gangliuyyyj);
\coordinate (gangliupppkk) at (\gangliuxxxk, \gangliuyyyk);
\coordinate (gangliupppkl) at (\gangliuxxxk, \gangliuyyyl);
\coordinate (gangliupppkm) at (\gangliuxxxk, \gangliuyyym);
\coordinate (gangliupppkn) at (\gangliuxxxk, \gangliuyyyn);
\coordinate (gangliupppko) at (\gangliuxxxk, \gangliuyyyo);
\coordinate (gangliupppkp) at (\gangliuxxxk, \gangliuyyyp);
\coordinate (gangliupppkq) at (\gangliuxxxk, \gangliuyyyq);
\coordinate (gangliupppkr) at (\gangliuxxxk, \gangliuyyyr);
\coordinate (gangliupppks) at (\gangliuxxxk, \gangliuyyys);
\coordinate (gangliupppkt) at (\gangliuxxxk, \gangliuyyyt);
\coordinate (gangliupppku) at (\gangliuxxxk, \gangliuyyyu);
\coordinate (gangliupppkv) at (\gangliuxxxk, \gangliuyyyv);
\coordinate (gangliupppkw) at (\gangliuxxxk, \gangliuyyyw);
\coordinate (gangliupppkx) at (\gangliuxxxk, \gangliuyyyx);
\coordinate (gangliupppky) at (\gangliuxxxk, \gangliuyyyy);
\coordinate (gangliupppkz) at (\gangliuxxxk, \gangliuyyyz);
\coordinate (gangliupppla) at (\gangliuxxxl, \gangliuyyya);
\coordinate (gangliuppplb) at (\gangliuxxxl, \gangliuyyyb);
\coordinate (gangliuppplc) at (\gangliuxxxl, \gangliuyyyc);
\coordinate (gangliupppld) at (\gangliuxxxl, \gangliuyyyd);
\coordinate (gangliuppple) at (\gangliuxxxl, \gangliuyyye);
\coordinate (gangliuppplf) at (\gangliuxxxl, \gangliuyyyf);
\coordinate (gangliuppplg) at (\gangliuxxxl, \gangliuyyyg);
\coordinate (gangliuppplh) at (\gangliuxxxl, \gangliuyyyh);
\coordinate (gangliupppli) at (\gangliuxxxl, \gangliuyyyi);
\coordinate (gangliuppplj) at (\gangliuxxxl, \gangliuyyyj);
\coordinate (gangliuppplk) at (\gangliuxxxl, \gangliuyyyk);
\coordinate (gangliupppll) at (\gangliuxxxl, \gangliuyyyl);
\coordinate (gangliuppplm) at (\gangliuxxxl, \gangliuyyym);
\coordinate (gangliupppln) at (\gangliuxxxl, \gangliuyyyn);
\coordinate (gangliuppplo) at (\gangliuxxxl, \gangliuyyyo);
\coordinate (gangliuppplp) at (\gangliuxxxl, \gangliuyyyp);
\coordinate (gangliuppplq) at (\gangliuxxxl, \gangliuyyyq);
\coordinate (gangliuppplr) at (\gangliuxxxl, \gangliuyyyr);
\coordinate (gangliupppls) at (\gangliuxxxl, \gangliuyyys);
\coordinate (gangliuppplt) at (\gangliuxxxl, \gangliuyyyt);
\coordinate (gangliuppplu) at (\gangliuxxxl, \gangliuyyyu);
\coordinate (gangliuppplv) at (\gangliuxxxl, \gangliuyyyv);
\coordinate (gangliuppplw) at (\gangliuxxxl, \gangliuyyyw);
\coordinate (gangliuppplx) at (\gangliuxxxl, \gangliuyyyx);
\coordinate (gangliuppply) at (\gangliuxxxl, \gangliuyyyy);
\coordinate (gangliuppplz) at (\gangliuxxxl, \gangliuyyyz);
\coordinate (gangliupppma) at (\gangliuxxxm, \gangliuyyya);
\coordinate (gangliupppmb) at (\gangliuxxxm, \gangliuyyyb);
\coordinate (gangliupppmc) at (\gangliuxxxm, \gangliuyyyc);
\coordinate (gangliupppmd) at (\gangliuxxxm, \gangliuyyyd);
\coordinate (gangliupppme) at (\gangliuxxxm, \gangliuyyye);
\coordinate (gangliupppmf) at (\gangliuxxxm, \gangliuyyyf);
\coordinate (gangliupppmg) at (\gangliuxxxm, \gangliuyyyg);
\coordinate (gangliupppmh) at (\gangliuxxxm, \gangliuyyyh);
\coordinate (gangliupppmi) at (\gangliuxxxm, \gangliuyyyi);
\coordinate (gangliupppmj) at (\gangliuxxxm, \gangliuyyyj);
\coordinate (gangliupppmk) at (\gangliuxxxm, \gangliuyyyk);
\coordinate (gangliupppml) at (\gangliuxxxm, \gangliuyyyl);
\coordinate (gangliupppmm) at (\gangliuxxxm, \gangliuyyym);
\coordinate (gangliupppmn) at (\gangliuxxxm, \gangliuyyyn);
\coordinate (gangliupppmo) at (\gangliuxxxm, \gangliuyyyo);
\coordinate (gangliupppmp) at (\gangliuxxxm, \gangliuyyyp);
\coordinate (gangliupppmq) at (\gangliuxxxm, \gangliuyyyq);
\coordinate (gangliupppmr) at (\gangliuxxxm, \gangliuyyyr);
\coordinate (gangliupppms) at (\gangliuxxxm, \gangliuyyys);
\coordinate (gangliupppmt) at (\gangliuxxxm, \gangliuyyyt);
\coordinate (gangliupppmu) at (\gangliuxxxm, \gangliuyyyu);
\coordinate (gangliupppmv) at (\gangliuxxxm, \gangliuyyyv);
\coordinate (gangliupppmw) at (\gangliuxxxm, \gangliuyyyw);
\coordinate (gangliupppmx) at (\gangliuxxxm, \gangliuyyyx);
\coordinate (gangliupppmy) at (\gangliuxxxm, \gangliuyyyy);
\coordinate (gangliupppmz) at (\gangliuxxxm, \gangliuyyyz);
\coordinate (gangliupppna) at (\gangliuxxxn, \gangliuyyya);
\coordinate (gangliupppnb) at (\gangliuxxxn, \gangliuyyyb);
\coordinate (gangliupppnc) at (\gangliuxxxn, \gangliuyyyc);
\coordinate (gangliupppnd) at (\gangliuxxxn, \gangliuyyyd);
\coordinate (gangliupppne) at (\gangliuxxxn, \gangliuyyye);
\coordinate (gangliupppnf) at (\gangliuxxxn, \gangliuyyyf);
\coordinate (gangliupppng) at (\gangliuxxxn, \gangliuyyyg);
\coordinate (gangliupppnh) at (\gangliuxxxn, \gangliuyyyh);
\coordinate (gangliupppni) at (\gangliuxxxn, \gangliuyyyi);
\coordinate (gangliupppnj) at (\gangliuxxxn, \gangliuyyyj);
\coordinate (gangliupppnk) at (\gangliuxxxn, \gangliuyyyk);
\coordinate (gangliupppnl) at (\gangliuxxxn, \gangliuyyyl);
\coordinate (gangliupppnm) at (\gangliuxxxn, \gangliuyyym);
\coordinate (gangliupppnn) at (\gangliuxxxn, \gangliuyyyn);
\coordinate (gangliupppno) at (\gangliuxxxn, \gangliuyyyo);
\coordinate (gangliupppnp) at (\gangliuxxxn, \gangliuyyyp);
\coordinate (gangliupppnq) at (\gangliuxxxn, \gangliuyyyq);
\coordinate (gangliupppnr) at (\gangliuxxxn, \gangliuyyyr);
\coordinate (gangliupppns) at (\gangliuxxxn, \gangliuyyys);
\coordinate (gangliupppnt) at (\gangliuxxxn, \gangliuyyyt);
\coordinate (gangliupppnu) at (\gangliuxxxn, \gangliuyyyu);
\coordinate (gangliupppnv) at (\gangliuxxxn, \gangliuyyyv);
\coordinate (gangliupppnw) at (\gangliuxxxn, \gangliuyyyw);
\coordinate (gangliupppnx) at (\gangliuxxxn, \gangliuyyyx);
\coordinate (gangliupppny) at (\gangliuxxxn, \gangliuyyyy);
\coordinate (gangliupppnz) at (\gangliuxxxn, \gangliuyyyz);
\coordinate (gangliupppoa) at (\gangliuxxxo, \gangliuyyya);
\coordinate (gangliupppob) at (\gangliuxxxo, \gangliuyyyb);
\coordinate (gangliupppoc) at (\gangliuxxxo, \gangliuyyyc);
\coordinate (gangliupppod) at (\gangliuxxxo, \gangliuyyyd);
\coordinate (gangliupppoe) at (\gangliuxxxo, \gangliuyyye);
\coordinate (gangliupppof) at (\gangliuxxxo, \gangliuyyyf);
\coordinate (gangliupppog) at (\gangliuxxxo, \gangliuyyyg);
\coordinate (gangliupppoh) at (\gangliuxxxo, \gangliuyyyh);
\coordinate (gangliupppoi) at (\gangliuxxxo, \gangliuyyyi);
\coordinate (gangliupppoj) at (\gangliuxxxo, \gangliuyyyj);
\coordinate (gangliupppok) at (\gangliuxxxo, \gangliuyyyk);
\coordinate (gangliupppol) at (\gangliuxxxo, \gangliuyyyl);
\coordinate (gangliupppom) at (\gangliuxxxo, \gangliuyyym);
\coordinate (gangliupppon) at (\gangliuxxxo, \gangliuyyyn);
\coordinate (gangliupppoo) at (\gangliuxxxo, \gangliuyyyo);
\coordinate (gangliupppop) at (\gangliuxxxo, \gangliuyyyp);
\coordinate (gangliupppoq) at (\gangliuxxxo, \gangliuyyyq);
\coordinate (gangliupppor) at (\gangliuxxxo, \gangliuyyyr);
\coordinate (gangliupppos) at (\gangliuxxxo, \gangliuyyys);
\coordinate (gangliupppot) at (\gangliuxxxo, \gangliuyyyt);
\coordinate (gangliupppou) at (\gangliuxxxo, \gangliuyyyu);
\coordinate (gangliupppov) at (\gangliuxxxo, \gangliuyyyv);
\coordinate (gangliupppow) at (\gangliuxxxo, \gangliuyyyw);
\coordinate (gangliupppox) at (\gangliuxxxo, \gangliuyyyx);
\coordinate (gangliupppoy) at (\gangliuxxxo, \gangliuyyyy);
\coordinate (gangliupppoz) at (\gangliuxxxo, \gangliuyyyz);
\coordinate (gangliuppppa) at (\gangliuxxxp, \gangliuyyya);
\coordinate (gangliuppppb) at (\gangliuxxxp, \gangliuyyyb);
\coordinate (gangliuppppc) at (\gangliuxxxp, \gangliuyyyc);
\coordinate (gangliuppppd) at (\gangliuxxxp, \gangliuyyyd);
\coordinate (gangliuppppe) at (\gangliuxxxp, \gangliuyyye);
\coordinate (gangliuppppf) at (\gangliuxxxp, \gangliuyyyf);
\coordinate (gangliuppppg) at (\gangliuxxxp, \gangliuyyyg);
\coordinate (gangliupppph) at (\gangliuxxxp, \gangliuyyyh);
\coordinate (gangliuppppi) at (\gangliuxxxp, \gangliuyyyi);
\coordinate (gangliuppppj) at (\gangliuxxxp, \gangliuyyyj);
\coordinate (gangliuppppk) at (\gangliuxxxp, \gangliuyyyk);
\coordinate (gangliuppppl) at (\gangliuxxxp, \gangliuyyyl);
\coordinate (gangliuppppm) at (\gangliuxxxp, \gangliuyyym);
\coordinate (gangliuppppn) at (\gangliuxxxp, \gangliuyyyn);
\coordinate (gangliuppppo) at (\gangliuxxxp, \gangliuyyyo);
\coordinate (gangliuppppp) at (\gangliuxxxp, \gangliuyyyp);
\coordinate (gangliuppppq) at (\gangliuxxxp, \gangliuyyyq);
\coordinate (gangliuppppr) at (\gangliuxxxp, \gangliuyyyr);
\coordinate (gangliupppps) at (\gangliuxxxp, \gangliuyyys);
\coordinate (gangliuppppt) at (\gangliuxxxp, \gangliuyyyt);
\coordinate (gangliuppppu) at (\gangliuxxxp, \gangliuyyyu);
\coordinate (gangliuppppv) at (\gangliuxxxp, \gangliuyyyv);
\coordinate (gangliuppppw) at (\gangliuxxxp, \gangliuyyyw);
\coordinate (gangliuppppx) at (\gangliuxxxp, \gangliuyyyx);
\coordinate (gangliuppppy) at (\gangliuxxxp, \gangliuyyyy);
\coordinate (gangliuppppz) at (\gangliuxxxp, \gangliuyyyz);
\coordinate (gangliupppqa) at (\gangliuxxxq, \gangliuyyya);
\coordinate (gangliupppqb) at (\gangliuxxxq, \gangliuyyyb);
\coordinate (gangliupppqc) at (\gangliuxxxq, \gangliuyyyc);
\coordinate (gangliupppqd) at (\gangliuxxxq, \gangliuyyyd);
\coordinate (gangliupppqe) at (\gangliuxxxq, \gangliuyyye);
\coordinate (gangliupppqf) at (\gangliuxxxq, \gangliuyyyf);
\coordinate (gangliupppqg) at (\gangliuxxxq, \gangliuyyyg);
\coordinate (gangliupppqh) at (\gangliuxxxq, \gangliuyyyh);
\coordinate (gangliupppqi) at (\gangliuxxxq, \gangliuyyyi);
\coordinate (gangliupppqj) at (\gangliuxxxq, \gangliuyyyj);
\coordinate (gangliupppqk) at (\gangliuxxxq, \gangliuyyyk);
\coordinate (gangliupppql) at (\gangliuxxxq, \gangliuyyyl);
\coordinate (gangliupppqm) at (\gangliuxxxq, \gangliuyyym);
\coordinate (gangliupppqn) at (\gangliuxxxq, \gangliuyyyn);
\coordinate (gangliupppqo) at (\gangliuxxxq, \gangliuyyyo);
\coordinate (gangliupppqp) at (\gangliuxxxq, \gangliuyyyp);
\coordinate (gangliupppqq) at (\gangliuxxxq, \gangliuyyyq);
\coordinate (gangliupppqr) at (\gangliuxxxq, \gangliuyyyr);
\coordinate (gangliupppqs) at (\gangliuxxxq, \gangliuyyys);
\coordinate (gangliupppqt) at (\gangliuxxxq, \gangliuyyyt);
\coordinate (gangliupppqu) at (\gangliuxxxq, \gangliuyyyu);
\coordinate (gangliupppqv) at (\gangliuxxxq, \gangliuyyyv);
\coordinate (gangliupppqw) at (\gangliuxxxq, \gangliuyyyw);
\coordinate (gangliupppqx) at (\gangliuxxxq, \gangliuyyyx);
\coordinate (gangliupppqy) at (\gangliuxxxq, \gangliuyyyy);
\coordinate (gangliupppqz) at (\gangliuxxxq, \gangliuyyyz);
\coordinate (gangliupppra) at (\gangliuxxxr, \gangliuyyya);
\coordinate (gangliuppprb) at (\gangliuxxxr, \gangliuyyyb);
\coordinate (gangliuppprc) at (\gangliuxxxr, \gangliuyyyc);
\coordinate (gangliuppprd) at (\gangliuxxxr, \gangliuyyyd);
\coordinate (gangliupppre) at (\gangliuxxxr, \gangliuyyye);
\coordinate (gangliuppprf) at (\gangliuxxxr, \gangliuyyyf);
\coordinate (gangliuppprg) at (\gangliuxxxr, \gangliuyyyg);
\coordinate (gangliuppprh) at (\gangliuxxxr, \gangliuyyyh);
\coordinate (gangliupppri) at (\gangliuxxxr, \gangliuyyyi);
\coordinate (gangliuppprj) at (\gangliuxxxr, \gangliuyyyj);
\coordinate (gangliuppprk) at (\gangliuxxxr, \gangliuyyyk);
\coordinate (gangliuppprl) at (\gangliuxxxr, \gangliuyyyl);
\coordinate (gangliuppprm) at (\gangliuxxxr, \gangliuyyym);
\coordinate (gangliuppprn) at (\gangliuxxxr, \gangliuyyyn);
\coordinate (gangliupppro) at (\gangliuxxxr, \gangliuyyyo);
\coordinate (gangliuppprp) at (\gangliuxxxr, \gangliuyyyp);
\coordinate (gangliuppprq) at (\gangliuxxxr, \gangliuyyyq);
\coordinate (gangliuppprr) at (\gangliuxxxr, \gangliuyyyr);
\coordinate (gangliuppprs) at (\gangliuxxxr, \gangliuyyys);
\coordinate (gangliuppprt) at (\gangliuxxxr, \gangliuyyyt);
\coordinate (gangliupppru) at (\gangliuxxxr, \gangliuyyyu);
\coordinate (gangliuppprv) at (\gangliuxxxr, \gangliuyyyv);
\coordinate (gangliuppprw) at (\gangliuxxxr, \gangliuyyyw);
\coordinate (gangliuppprx) at (\gangliuxxxr, \gangliuyyyx);
\coordinate (gangliupppry) at (\gangliuxxxr, \gangliuyyyy);
\coordinate (gangliuppprz) at (\gangliuxxxr, \gangliuyyyz);
\coordinate (gangliupppsa) at (\gangliuxxxs, \gangliuyyya);
\coordinate (gangliupppsb) at (\gangliuxxxs, \gangliuyyyb);
\coordinate (gangliupppsc) at (\gangliuxxxs, \gangliuyyyc);
\coordinate (gangliupppsd) at (\gangliuxxxs, \gangliuyyyd);
\coordinate (gangliupppse) at (\gangliuxxxs, \gangliuyyye);
\coordinate (gangliupppsf) at (\gangliuxxxs, \gangliuyyyf);
\coordinate (gangliupppsg) at (\gangliuxxxs, \gangliuyyyg);
\coordinate (gangliupppsh) at (\gangliuxxxs, \gangliuyyyh);
\coordinate (gangliupppsi) at (\gangliuxxxs, \gangliuyyyi);
\coordinate (gangliupppsj) at (\gangliuxxxs, \gangliuyyyj);
\coordinate (gangliupppsk) at (\gangliuxxxs, \gangliuyyyk);
\coordinate (gangliupppsl) at (\gangliuxxxs, \gangliuyyyl);
\coordinate (gangliupppsm) at (\gangliuxxxs, \gangliuyyym);
\coordinate (gangliupppsn) at (\gangliuxxxs, \gangliuyyyn);
\coordinate (gangliupppso) at (\gangliuxxxs, \gangliuyyyo);
\coordinate (gangliupppsp) at (\gangliuxxxs, \gangliuyyyp);
\coordinate (gangliupppsq) at (\gangliuxxxs, \gangliuyyyq);
\coordinate (gangliupppsr) at (\gangliuxxxs, \gangliuyyyr);
\coordinate (gangliupppss) at (\gangliuxxxs, \gangliuyyys);
\coordinate (gangliupppst) at (\gangliuxxxs, \gangliuyyyt);
\coordinate (gangliupppsu) at (\gangliuxxxs, \gangliuyyyu);
\coordinate (gangliupppsv) at (\gangliuxxxs, \gangliuyyyv);
\coordinate (gangliupppsw) at (\gangliuxxxs, \gangliuyyyw);
\coordinate (gangliupppsx) at (\gangliuxxxs, \gangliuyyyx);
\coordinate (gangliupppsy) at (\gangliuxxxs, \gangliuyyyy);
\coordinate (gangliupppsz) at (\gangliuxxxs, \gangliuyyyz);
\coordinate (gangliupppta) at (\gangliuxxxt, \gangliuyyya);
\coordinate (gangliuppptb) at (\gangliuxxxt, \gangliuyyyb);
\coordinate (gangliuppptc) at (\gangliuxxxt, \gangliuyyyc);
\coordinate (gangliuppptd) at (\gangliuxxxt, \gangliuyyyd);
\coordinate (gangliupppte) at (\gangliuxxxt, \gangliuyyye);
\coordinate (gangliuppptf) at (\gangliuxxxt, \gangliuyyyf);
\coordinate (gangliuppptg) at (\gangliuxxxt, \gangliuyyyg);
\coordinate (gangliupppth) at (\gangliuxxxt, \gangliuyyyh);
\coordinate (gangliupppti) at (\gangliuxxxt, \gangliuyyyi);
\coordinate (gangliuppptj) at (\gangliuxxxt, \gangliuyyyj);
\coordinate (gangliuppptk) at (\gangliuxxxt, \gangliuyyyk);
\coordinate (gangliuppptl) at (\gangliuxxxt, \gangliuyyyl);
\coordinate (gangliuppptm) at (\gangliuxxxt, \gangliuyyym);
\coordinate (gangliuppptn) at (\gangliuxxxt, \gangliuyyyn);
\coordinate (gangliupppto) at (\gangliuxxxt, \gangliuyyyo);
\coordinate (gangliuppptp) at (\gangliuxxxt, \gangliuyyyp);
\coordinate (gangliuppptq) at (\gangliuxxxt, \gangliuyyyq);
\coordinate (gangliuppptr) at (\gangliuxxxt, \gangliuyyyr);
\coordinate (gangliupppts) at (\gangliuxxxt, \gangliuyyys);
\coordinate (gangliuppptt) at (\gangliuxxxt, \gangliuyyyt);
\coordinate (gangliuppptu) at (\gangliuxxxt, \gangliuyyyu);
\coordinate (gangliuppptv) at (\gangliuxxxt, \gangliuyyyv);
\coordinate (gangliuppptw) at (\gangliuxxxt, \gangliuyyyw);
\coordinate (gangliuppptx) at (\gangliuxxxt, \gangliuyyyx);
\coordinate (gangliupppty) at (\gangliuxxxt, \gangliuyyyy);
\coordinate (gangliuppptz) at (\gangliuxxxt, \gangliuyyyz);
\coordinate (gangliupppua) at (\gangliuxxxu, \gangliuyyya);
\coordinate (gangliupppub) at (\gangliuxxxu, \gangliuyyyb);
\coordinate (gangliupppuc) at (\gangliuxxxu, \gangliuyyyc);
\coordinate (gangliupppud) at (\gangliuxxxu, \gangliuyyyd);
\coordinate (gangliupppue) at (\gangliuxxxu, \gangliuyyye);
\coordinate (gangliupppuf) at (\gangliuxxxu, \gangliuyyyf);
\coordinate (gangliupppug) at (\gangliuxxxu, \gangliuyyyg);
\coordinate (gangliupppuh) at (\gangliuxxxu, \gangliuyyyh);
\coordinate (gangliupppui) at (\gangliuxxxu, \gangliuyyyi);
\coordinate (gangliupppuj) at (\gangliuxxxu, \gangliuyyyj);
\coordinate (gangliupppuk) at (\gangliuxxxu, \gangliuyyyk);
\coordinate (gangliupppul) at (\gangliuxxxu, \gangliuyyyl);
\coordinate (gangliupppum) at (\gangliuxxxu, \gangliuyyym);
\coordinate (gangliupppun) at (\gangliuxxxu, \gangliuyyyn);
\coordinate (gangliupppuo) at (\gangliuxxxu, \gangliuyyyo);
\coordinate (gangliupppup) at (\gangliuxxxu, \gangliuyyyp);
\coordinate (gangliupppuq) at (\gangliuxxxu, \gangliuyyyq);
\coordinate (gangliupppur) at (\gangliuxxxu, \gangliuyyyr);
\coordinate (gangliupppus) at (\gangliuxxxu, \gangliuyyys);
\coordinate (gangliuppput) at (\gangliuxxxu, \gangliuyyyt);
\coordinate (gangliupppuu) at (\gangliuxxxu, \gangliuyyyu);
\coordinate (gangliupppuv) at (\gangliuxxxu, \gangliuyyyv);
\coordinate (gangliupppuw) at (\gangliuxxxu, \gangliuyyyw);
\coordinate (gangliupppux) at (\gangliuxxxu, \gangliuyyyx);
\coordinate (gangliupppuy) at (\gangliuxxxu, \gangliuyyyy);
\coordinate (gangliupppuz) at (\gangliuxxxu, \gangliuyyyz);
\coordinate (gangliupppva) at (\gangliuxxxv, \gangliuyyya);
\coordinate (gangliupppvb) at (\gangliuxxxv, \gangliuyyyb);
\coordinate (gangliupppvc) at (\gangliuxxxv, \gangliuyyyc);
\coordinate (gangliupppvd) at (\gangliuxxxv, \gangliuyyyd);
\coordinate (gangliupppve) at (\gangliuxxxv, \gangliuyyye);
\coordinate (gangliupppvf) at (\gangliuxxxv, \gangliuyyyf);
\coordinate (gangliupppvg) at (\gangliuxxxv, \gangliuyyyg);
\coordinate (gangliupppvh) at (\gangliuxxxv, \gangliuyyyh);
\coordinate (gangliupppvi) at (\gangliuxxxv, \gangliuyyyi);
\coordinate (gangliupppvj) at (\gangliuxxxv, \gangliuyyyj);
\coordinate (gangliupppvk) at (\gangliuxxxv, \gangliuyyyk);
\coordinate (gangliupppvl) at (\gangliuxxxv, \gangliuyyyl);
\coordinate (gangliupppvm) at (\gangliuxxxv, \gangliuyyym);
\coordinate (gangliupppvn) at (\gangliuxxxv, \gangliuyyyn);
\coordinate (gangliupppvo) at (\gangliuxxxv, \gangliuyyyo);
\coordinate (gangliupppvp) at (\gangliuxxxv, \gangliuyyyp);
\coordinate (gangliupppvq) at (\gangliuxxxv, \gangliuyyyq);
\coordinate (gangliupppvr) at (\gangliuxxxv, \gangliuyyyr);
\coordinate (gangliupppvs) at (\gangliuxxxv, \gangliuyyys);
\coordinate (gangliupppvt) at (\gangliuxxxv, \gangliuyyyt);
\coordinate (gangliupppvu) at (\gangliuxxxv, \gangliuyyyu);
\coordinate (gangliupppvv) at (\gangliuxxxv, \gangliuyyyv);
\coordinate (gangliupppvw) at (\gangliuxxxv, \gangliuyyyw);
\coordinate (gangliupppvx) at (\gangliuxxxv, \gangliuyyyx);
\coordinate (gangliupppvy) at (\gangliuxxxv, \gangliuyyyy);
\coordinate (gangliupppvz) at (\gangliuxxxv, \gangliuyyyz);
\coordinate (gangliupppwa) at (\gangliuxxxw, \gangliuyyya);
\coordinate (gangliupppwb) at (\gangliuxxxw, \gangliuyyyb);
\coordinate (gangliupppwc) at (\gangliuxxxw, \gangliuyyyc);
\coordinate (gangliupppwd) at (\gangliuxxxw, \gangliuyyyd);
\coordinate (gangliupppwe) at (\gangliuxxxw, \gangliuyyye);
\coordinate (gangliupppwf) at (\gangliuxxxw, \gangliuyyyf);
\coordinate (gangliupppwg) at (\gangliuxxxw, \gangliuyyyg);
\coordinate (gangliupppwh) at (\gangliuxxxw, \gangliuyyyh);
\coordinate (gangliupppwi) at (\gangliuxxxw, \gangliuyyyi);
\coordinate (gangliupppwj) at (\gangliuxxxw, \gangliuyyyj);
\coordinate (gangliupppwk) at (\gangliuxxxw, \gangliuyyyk);
\coordinate (gangliupppwl) at (\gangliuxxxw, \gangliuyyyl);
\coordinate (gangliupppwm) at (\gangliuxxxw, \gangliuyyym);
\coordinate (gangliupppwn) at (\gangliuxxxw, \gangliuyyyn);
\coordinate (gangliupppwo) at (\gangliuxxxw, \gangliuyyyo);
\coordinate (gangliupppwp) at (\gangliuxxxw, \gangliuyyyp);
\coordinate (gangliupppwq) at (\gangliuxxxw, \gangliuyyyq);
\coordinate (gangliupppwr) at (\gangliuxxxw, \gangliuyyyr);
\coordinate (gangliupppws) at (\gangliuxxxw, \gangliuyyys);
\coordinate (gangliupppwt) at (\gangliuxxxw, \gangliuyyyt);
\coordinate (gangliupppwu) at (\gangliuxxxw, \gangliuyyyu);
\coordinate (gangliupppwv) at (\gangliuxxxw, \gangliuyyyv);
\coordinate (gangliupppww) at (\gangliuxxxw, \gangliuyyyw);
\coordinate (gangliupppwx) at (\gangliuxxxw, \gangliuyyyx);
\coordinate (gangliupppwy) at (\gangliuxxxw, \gangliuyyyy);
\coordinate (gangliupppwz) at (\gangliuxxxw, \gangliuyyyz);
\coordinate (gangliupppxa) at (\gangliuxxxx, \gangliuyyya);
\coordinate (gangliupppxb) at (\gangliuxxxx, \gangliuyyyb);
\coordinate (gangliupppxc) at (\gangliuxxxx, \gangliuyyyc);
\coordinate (gangliupppxd) at (\gangliuxxxx, \gangliuyyyd);
\coordinate (gangliupppxe) at (\gangliuxxxx, \gangliuyyye);
\coordinate (gangliupppxf) at (\gangliuxxxx, \gangliuyyyf);
\coordinate (gangliupppxg) at (\gangliuxxxx, \gangliuyyyg);
\coordinate (gangliupppxh) at (\gangliuxxxx, \gangliuyyyh);
\coordinate (gangliupppxi) at (\gangliuxxxx, \gangliuyyyi);
\coordinate (gangliupppxj) at (\gangliuxxxx, \gangliuyyyj);
\coordinate (gangliupppxk) at (\gangliuxxxx, \gangliuyyyk);
\coordinate (gangliupppxl) at (\gangliuxxxx, \gangliuyyyl);
\coordinate (gangliupppxm) at (\gangliuxxxx, \gangliuyyym);
\coordinate (gangliupppxn) at (\gangliuxxxx, \gangliuyyyn);
\coordinate (gangliupppxo) at (\gangliuxxxx, \gangliuyyyo);
\coordinate (gangliupppxp) at (\gangliuxxxx, \gangliuyyyp);
\coordinate (gangliupppxq) at (\gangliuxxxx, \gangliuyyyq);
\coordinate (gangliupppxr) at (\gangliuxxxx, \gangliuyyyr);
\coordinate (gangliupppxs) at (\gangliuxxxx, \gangliuyyys);
\coordinate (gangliupppxt) at (\gangliuxxxx, \gangliuyyyt);
\coordinate (gangliupppxu) at (\gangliuxxxx, \gangliuyyyu);
\coordinate (gangliupppxv) at (\gangliuxxxx, \gangliuyyyv);
\coordinate (gangliupppxw) at (\gangliuxxxx, \gangliuyyyw);
\coordinate (gangliupppxx) at (\gangliuxxxx, \gangliuyyyx);
\coordinate (gangliupppxy) at (\gangliuxxxx, \gangliuyyyy);
\coordinate (gangliupppxz) at (\gangliuxxxx, \gangliuyyyz);
\coordinate (gangliupppya) at (\gangliuxxxy, \gangliuyyya);
\coordinate (gangliupppyb) at (\gangliuxxxy, \gangliuyyyb);
\coordinate (gangliupppyc) at (\gangliuxxxy, \gangliuyyyc);
\coordinate (gangliupppyd) at (\gangliuxxxy, \gangliuyyyd);
\coordinate (gangliupppye) at (\gangliuxxxy, \gangliuyyye);
\coordinate (gangliupppyf) at (\gangliuxxxy, \gangliuyyyf);
\coordinate (gangliupppyg) at (\gangliuxxxy, \gangliuyyyg);
\coordinate (gangliupppyh) at (\gangliuxxxy, \gangliuyyyh);
\coordinate (gangliupppyi) at (\gangliuxxxy, \gangliuyyyi);
\coordinate (gangliupppyj) at (\gangliuxxxy, \gangliuyyyj);
\coordinate (gangliupppyk) at (\gangliuxxxy, \gangliuyyyk);
\coordinate (gangliupppyl) at (\gangliuxxxy, \gangliuyyyl);
\coordinate (gangliupppym) at (\gangliuxxxy, \gangliuyyym);
\coordinate (gangliupppyn) at (\gangliuxxxy, \gangliuyyyn);
\coordinate (gangliupppyo) at (\gangliuxxxy, \gangliuyyyo);
\coordinate (gangliupppyp) at (\gangliuxxxy, \gangliuyyyp);
\coordinate (gangliupppyq) at (\gangliuxxxy, \gangliuyyyq);
\coordinate (gangliupppyr) at (\gangliuxxxy, \gangliuyyyr);
\coordinate (gangliupppys) at (\gangliuxxxy, \gangliuyyys);
\coordinate (gangliupppyt) at (\gangliuxxxy, \gangliuyyyt);
\coordinate (gangliupppyu) at (\gangliuxxxy, \gangliuyyyu);
\coordinate (gangliupppyv) at (\gangliuxxxy, \gangliuyyyv);
\coordinate (gangliupppyw) at (\gangliuxxxy, \gangliuyyyw);
\coordinate (gangliupppyx) at (\gangliuxxxy, \gangliuyyyx);
\coordinate (gangliupppyy) at (\gangliuxxxy, \gangliuyyyy);
\coordinate (gangliupppyz) at (\gangliuxxxy, \gangliuyyyz);
\coordinate (gangliupppza) at (\gangliuxxxz, \gangliuyyya);
\coordinate (gangliupppzb) at (\gangliuxxxz, \gangliuyyyb);
\coordinate (gangliupppzc) at (\gangliuxxxz, \gangliuyyyc);
\coordinate (gangliupppzd) at (\gangliuxxxz, \gangliuyyyd);
\coordinate (gangliupppze) at (\gangliuxxxz, \gangliuyyye);
\coordinate (gangliupppzf) at (\gangliuxxxz, \gangliuyyyf);
\coordinate (gangliupppzg) at (\gangliuxxxz, \gangliuyyyg);
\coordinate (gangliupppzh) at (\gangliuxxxz, \gangliuyyyh);
\coordinate (gangliupppzi) at (\gangliuxxxz, \gangliuyyyi);
\coordinate (gangliupppzj) at (\gangliuxxxz, \gangliuyyyj);
\coordinate (gangliupppzk) at (\gangliuxxxz, \gangliuyyyk);
\coordinate (gangliupppzl) at (\gangliuxxxz, \gangliuyyyl);
\coordinate (gangliupppzm) at (\gangliuxxxz, \gangliuyyym);
\coordinate (gangliupppzn) at (\gangliuxxxz, \gangliuyyyn);
\coordinate (gangliupppzo) at (\gangliuxxxz, \gangliuyyyo);
\coordinate (gangliupppzp) at (\gangliuxxxz, \gangliuyyyp);
\coordinate (gangliupppzq) at (\gangliuxxxz, \gangliuyyyq);
\coordinate (gangliupppzr) at (\gangliuxxxz, \gangliuyyyr);
\coordinate (gangliupppzs) at (\gangliuxxxz, \gangliuyyys);
\coordinate (gangliupppzt) at (\gangliuxxxz, \gangliuyyyt);
\coordinate (gangliupppzu) at (\gangliuxxxz, \gangliuyyyu);
\coordinate (gangliupppzv) at (\gangliuxxxz, \gangliuyyyv);
\coordinate (gangliupppzw) at (\gangliuxxxz, \gangliuyyyw);
\coordinate (gangliupppzx) at (\gangliuxxxz, \gangliuyyyx);
\coordinate (gangliupppzy) at (\gangliuxxxz, \gangliuyyyy);
\coordinate (gangliupppzz) at (\gangliuxxxz, \gangliuyyyz);

%\gangprintcoordinateat{(0,0)}{The last coordinate values: }{($(gangliupppzz)$)}; 



\pgfmathsetmacro{\totalglaxxx}{26}
\pgfmathsetmacro{\totalglayyy}{26}
\pgfmathsetmacro{\glaxxxspacing}{1}
\pgfmathsetmacro{\glayyyspacing}{1}
\pgfmathsetmacro{\glaxxxa}{-8}
\pgfmathsetmacro{\glayyya}{\gangliuxxxa -8.0}

\pgfmathsetmacro{\glaxxxb}{\glaxxxa + \glaxxxspacing + 0.0 }
\pgfmathsetmacro{\glaxxxc}{\glaxxxb + \glaxxxspacing + 0.0 }
\pgfmathsetmacro{\glaxxxd}{\glaxxxc + \glaxxxspacing + 0.0 }
\pgfmathsetmacro{\glaxxxe}{\glaxxxd + \glaxxxspacing + 0.0 }
\pgfmathsetmacro{\glaxxxf}{\glaxxxe + \glaxxxspacing + 0.0 }
\pgfmathsetmacro{\glaxxxg}{\glaxxxf + \glaxxxspacing + 0.0 }
\pgfmathsetmacro{\glaxxxh}{\glaxxxg + \glaxxxspacing + 0.0 }
\pgfmathsetmacro{\glaxxxi}{\glaxxxh + \glaxxxspacing + 0.0 }
\pgfmathsetmacro{\glaxxxj}{\glaxxxi + \glaxxxspacing + 0.0 }
\pgfmathsetmacro{\glaxxxk}{\glaxxxj + \glaxxxspacing + 0.0 }
\pgfmathsetmacro{\glaxxxl}{\glaxxxk + \glaxxxspacing + 0.0 }
\pgfmathsetmacro{\glaxxxm}{\glaxxxl + \glaxxxspacing + 0.0 }
\pgfmathsetmacro{\glaxxxn}{\glaxxxm + \glaxxxspacing + 0.0 }
\pgfmathsetmacro{\glaxxxo}{\glaxxxn + \glaxxxspacing + 0.0 }
\pgfmathsetmacro{\glaxxxp}{\glaxxxo + \glaxxxspacing + 0.0 }
\pgfmathsetmacro{\glaxxxq}{\glaxxxp + \glaxxxspacing + 0.0 }
\pgfmathsetmacro{\glaxxxr}{\glaxxxq + \glaxxxspacing + 0.0 }
\pgfmathsetmacro{\glaxxxs}{\glaxxxr + \glaxxxspacing + 0.0 }
\pgfmathsetmacro{\glaxxxt}{\glaxxxs + \glaxxxspacing + 0.0 }
\pgfmathsetmacro{\glaxxxu}{\glaxxxt + \glaxxxspacing + 0.0 }
\pgfmathsetmacro{\glaxxxv}{\glaxxxu + \glaxxxspacing + 0.0 }
\pgfmathsetmacro{\glaxxxw}{\glaxxxv + \glaxxxspacing + 0.0 }
\pgfmathsetmacro{\glaxxxx}{\glaxxxw + \glaxxxspacing + 0.0 }
\pgfmathsetmacro{\glaxxxy}{\glaxxxx + \glaxxxspacing + 0.0 }
\pgfmathsetmacro{\glaxxxz}{\glaxxxy + \glaxxxspacing + 0.0 }

\pgfmathsetmacro{\glayyyb}{\glayyya + \glayyyspacing + 0.0 }
\pgfmathsetmacro{\glayyyc}{\glayyyb + \glayyyspacing + 0.0 }
\pgfmathsetmacro{\glayyyd}{\glayyyc + \glayyyspacing + 0.0 }
\pgfmathsetmacro{\glayyye}{\glayyyd + \glayyyspacing + 0.0 }
\pgfmathsetmacro{\glayyyf}{\glayyye + \glayyyspacing + 0.0 }
\pgfmathsetmacro{\glayyyg}{\glayyyf + \glayyyspacing + 0.0 }
\pgfmathsetmacro{\glayyyh}{\glayyyg + \glayyyspacing + 0.0 }
\pgfmathsetmacro{\glayyyi}{\glayyyh + \glayyyspacing + 0.0 }
\pgfmathsetmacro{\glayyyj}{\glayyyi + \glayyyspacing + 0.0 }
\pgfmathsetmacro{\glayyyk}{\glayyyj + \glayyyspacing + 0.0 }
\pgfmathsetmacro{\glayyyl}{\glayyyk + \glayyyspacing + 0.0 }
\pgfmathsetmacro{\glayyym}{\glayyyl + \glayyyspacing + 0.0 }
\pgfmathsetmacro{\glayyyn}{\glayyym + \glayyyspacing + 0.0 }
\pgfmathsetmacro{\glayyyo}{\glayyyn + \glayyyspacing + 0.0 }
\pgfmathsetmacro{\glayyyp}{\glayyyo + \glayyyspacing + 0.0 }
\pgfmathsetmacro{\glayyyq}{\glayyyp + \glayyyspacing + 0.0 }
\pgfmathsetmacro{\glayyyr}{\glayyyq + \glayyyspacing + 0.0 }
\pgfmathsetmacro{\glayyys}{\glayyyr + \glayyyspacing + 0.0 }
\pgfmathsetmacro{\glayyyt}{\glayyys + \glayyyspacing + 0.0 }
\pgfmathsetmacro{\glayyyu}{\glayyyt + \glayyyspacing + 0.0 }
\pgfmathsetmacro{\glayyyv}{\glayyyu + \glayyyspacing + 0.0 }
\pgfmathsetmacro{\glayyyw}{\glayyyv + \glayyyspacing + 0.0 }
\pgfmathsetmacro{\glayyyx}{\glayyyw + \glayyyspacing + 0.0 }
\pgfmathsetmacro{\glayyyy}{\glayyyx + \glayyyspacing + 0.0 }
\pgfmathsetmacro{\glayyyz}{\glayyyy + \glayyyspacing + 0.0 }

\coordinate (glapppaa) at (\glaxxxa, \glayyya);
\coordinate (glapppab) at (\glaxxxa, \glayyyb);
\coordinate (glapppac) at (\glaxxxa, \glayyyc);
\coordinate (glapppad) at (\glaxxxa, \glayyyd);
\coordinate (glapppae) at (\glaxxxa, \glayyye);
\coordinate (glapppaf) at (\glaxxxa, \glayyyf);
\coordinate (glapppag) at (\glaxxxa, \glayyyg);
\coordinate (glapppah) at (\glaxxxa, \glayyyh);
\coordinate (glapppai) at (\glaxxxa, \glayyyi);
\coordinate (glapppaj) at (\glaxxxa, \glayyyj);
\coordinate (glapppak) at (\glaxxxa, \glayyyk);
\coordinate (glapppal) at (\glaxxxa, \glayyyl);
\coordinate (glapppam) at (\glaxxxa, \glayyym);
\coordinate (glapppan) at (\glaxxxa, \glayyyn);
\coordinate (glapppao) at (\glaxxxa, \glayyyo);
\coordinate (glapppap) at (\glaxxxa, \glayyyp);
\coordinate (glapppaq) at (\glaxxxa, \glayyyq);
\coordinate (glapppar) at (\glaxxxa, \glayyyr);
\coordinate (glapppas) at (\glaxxxa, \glayyys);
\coordinate (glapppat) at (\glaxxxa, \glayyyt);
\coordinate (glapppau) at (\glaxxxa, \glayyyu);
\coordinate (glapppav) at (\glaxxxa, \glayyyv);
\coordinate (glapppaw) at (\glaxxxa, \glayyyw);
\coordinate (glapppax) at (\glaxxxa, \glayyyx);
\coordinate (glapppay) at (\glaxxxa, \glayyyy);
\coordinate (glapppaz) at (\glaxxxa, \glayyyz);
\coordinate (glapppba) at (\glaxxxb, \glayyya);
\coordinate (glapppbb) at (\glaxxxb, \glayyyb);
\coordinate (glapppbc) at (\glaxxxb, \glayyyc);
\coordinate (glapppbd) at (\glaxxxb, \glayyyd);
\coordinate (glapppbe) at (\glaxxxb, \glayyye);
\coordinate (glapppbf) at (\glaxxxb, \glayyyf);
\coordinate (glapppbg) at (\glaxxxb, \glayyyg);
\coordinate (glapppbh) at (\glaxxxb, \glayyyh);
\coordinate (glapppbi) at (\glaxxxb, \glayyyi);
\coordinate (glapppbj) at (\glaxxxb, \glayyyj);
\coordinate (glapppbk) at (\glaxxxb, \glayyyk);
\coordinate (glapppbl) at (\glaxxxb, \glayyyl);
\coordinate (glapppbm) at (\glaxxxb, \glayyym);
\coordinate (glapppbn) at (\glaxxxb, \glayyyn);
\coordinate (glapppbo) at (\glaxxxb, \glayyyo);
\coordinate (glapppbp) at (\glaxxxb, \glayyyp);
\coordinate (glapppbq) at (\glaxxxb, \glayyyq);
\coordinate (glapppbr) at (\glaxxxb, \glayyyr);
\coordinate (glapppbs) at (\glaxxxb, \glayyys);
\coordinate (glapppbt) at (\glaxxxb, \glayyyt);
\coordinate (glapppbu) at (\glaxxxb, \glayyyu);
\coordinate (glapppbv) at (\glaxxxb, \glayyyv);
\coordinate (glapppbw) at (\glaxxxb, \glayyyw);
\coordinate (glapppbx) at (\glaxxxb, \glayyyx);
\coordinate (glapppby) at (\glaxxxb, \glayyyy);
\coordinate (glapppbz) at (\glaxxxb, \glayyyz);
\coordinate (glapppca) at (\glaxxxc, \glayyya);
\coordinate (glapppcb) at (\glaxxxc, \glayyyb);
\coordinate (glapppcc) at (\glaxxxc, \glayyyc);
\coordinate (glapppcd) at (\glaxxxc, \glayyyd);
\coordinate (glapppce) at (\glaxxxc, \glayyye);
\coordinate (glapppcf) at (\glaxxxc, \glayyyf);
\coordinate (glapppcg) at (\glaxxxc, \glayyyg);
\coordinate (glapppch) at (\glaxxxc, \glayyyh);
\coordinate (glapppci) at (\glaxxxc, \glayyyi);
\coordinate (glapppcj) at (\glaxxxc, \glayyyj);
\coordinate (glapppck) at (\glaxxxc, \glayyyk);
\coordinate (glapppcl) at (\glaxxxc, \glayyyl);
\coordinate (glapppcm) at (\glaxxxc, \glayyym);
\coordinate (glapppcn) at (\glaxxxc, \glayyyn);
\coordinate (glapppco) at (\glaxxxc, \glayyyo);
\coordinate (glapppcp) at (\glaxxxc, \glayyyp);
\coordinate (glapppcq) at (\glaxxxc, \glayyyq);
\coordinate (glapppcr) at (\glaxxxc, \glayyyr);
\coordinate (glapppcs) at (\glaxxxc, \glayyys);
\coordinate (glapppct) at (\glaxxxc, \glayyyt);
\coordinate (glapppcu) at (\glaxxxc, \glayyyu);
\coordinate (glapppcv) at (\glaxxxc, \glayyyv);
\coordinate (glapppcw) at (\glaxxxc, \glayyyw);
\coordinate (glapppcx) at (\glaxxxc, \glayyyx);
\coordinate (glapppcy) at (\glaxxxc, \glayyyy);
\coordinate (glapppcz) at (\glaxxxc, \glayyyz);
\coordinate (glapppda) at (\glaxxxd, \glayyya);
\coordinate (glapppdb) at (\glaxxxd, \glayyyb);
\coordinate (glapppdc) at (\glaxxxd, \glayyyc);
\coordinate (glapppdd) at (\glaxxxd, \glayyyd);
\coordinate (glapppde) at (\glaxxxd, \glayyye);
\coordinate (glapppdf) at (\glaxxxd, \glayyyf);
\coordinate (glapppdg) at (\glaxxxd, \glayyyg);
\coordinate (glapppdh) at (\glaxxxd, \glayyyh);
\coordinate (glapppdi) at (\glaxxxd, \glayyyi);
\coordinate (glapppdj) at (\glaxxxd, \glayyyj);
\coordinate (glapppdk) at (\glaxxxd, \glayyyk);
\coordinate (glapppdl) at (\glaxxxd, \glayyyl);
\coordinate (glapppdm) at (\glaxxxd, \glayyym);
\coordinate (glapppdn) at (\glaxxxd, \glayyyn);
\coordinate (glapppdo) at (\glaxxxd, \glayyyo);
\coordinate (glapppdp) at (\glaxxxd, \glayyyp);
\coordinate (glapppdq) at (\glaxxxd, \glayyyq);
\coordinate (glapppdr) at (\glaxxxd, \glayyyr);
\coordinate (glapppds) at (\glaxxxd, \glayyys);
\coordinate (glapppdt) at (\glaxxxd, \glayyyt);
\coordinate (glapppdu) at (\glaxxxd, \glayyyu);
\coordinate (glapppdv) at (\glaxxxd, \glayyyv);
\coordinate (glapppdw) at (\glaxxxd, \glayyyw);
\coordinate (glapppdx) at (\glaxxxd, \glayyyx);
\coordinate (glapppdy) at (\glaxxxd, \glayyyy);
\coordinate (glapppdz) at (\glaxxxd, \glayyyz);
\coordinate (glapppea) at (\glaxxxe, \glayyya);
\coordinate (glapppeb) at (\glaxxxe, \glayyyb);
\coordinate (glapppec) at (\glaxxxe, \glayyyc);
\coordinate (glappped) at (\glaxxxe, \glayyyd);
\coordinate (glapppee) at (\glaxxxe, \glayyye);
\coordinate (glapppef) at (\glaxxxe, \glayyyf);
\coordinate (glapppeg) at (\glaxxxe, \glayyyg);
\coordinate (glapppeh) at (\glaxxxe, \glayyyh);
\coordinate (glapppei) at (\glaxxxe, \glayyyi);
\coordinate (glapppej) at (\glaxxxe, \glayyyj);
\coordinate (glapppek) at (\glaxxxe, \glayyyk);
\coordinate (glapppel) at (\glaxxxe, \glayyyl);
\coordinate (glapppem) at (\glaxxxe, \glayyym);
\coordinate (glapppen) at (\glaxxxe, \glayyyn);
\coordinate (glapppeo) at (\glaxxxe, \glayyyo);
\coordinate (glapppep) at (\glaxxxe, \glayyyp);
\coordinate (glapppeq) at (\glaxxxe, \glayyyq);
\coordinate (glappper) at (\glaxxxe, \glayyyr);
\coordinate (glapppes) at (\glaxxxe, \glayyys);
\coordinate (glapppet) at (\glaxxxe, \glayyyt);
\coordinate (glapppeu) at (\glaxxxe, \glayyyu);
\coordinate (glapppev) at (\glaxxxe, \glayyyv);
\coordinate (glapppew) at (\glaxxxe, \glayyyw);
\coordinate (glapppex) at (\glaxxxe, \glayyyx);
\coordinate (glapppey) at (\glaxxxe, \glayyyy);
\coordinate (glapppez) at (\glaxxxe, \glayyyz);
\coordinate (glapppfa) at (\glaxxxf, \glayyya);
\coordinate (glapppfb) at (\glaxxxf, \glayyyb);
\coordinate (glapppfc) at (\glaxxxf, \glayyyc);
\coordinate (glapppfd) at (\glaxxxf, \glayyyd);
\coordinate (glapppfe) at (\glaxxxf, \glayyye);
\coordinate (glapppff) at (\glaxxxf, \glayyyf);
\coordinate (glapppfg) at (\glaxxxf, \glayyyg);
\coordinate (glapppfh) at (\glaxxxf, \glayyyh);
\coordinate (glapppfi) at (\glaxxxf, \glayyyi);
\coordinate (glapppfj) at (\glaxxxf, \glayyyj);
\coordinate (glapppfk) at (\glaxxxf, \glayyyk);
\coordinate (glapppfl) at (\glaxxxf, \glayyyl);
\coordinate (glapppfm) at (\glaxxxf, \glayyym);
\coordinate (glapppfn) at (\glaxxxf, \glayyyn);
\coordinate (glapppfo) at (\glaxxxf, \glayyyo);
\coordinate (glapppfp) at (\glaxxxf, \glayyyp);
\coordinate (glapppfq) at (\glaxxxf, \glayyyq);
\coordinate (glapppfr) at (\glaxxxf, \glayyyr);
\coordinate (glapppfs) at (\glaxxxf, \glayyys);
\coordinate (glapppft) at (\glaxxxf, \glayyyt);
\coordinate (glapppfu) at (\glaxxxf, \glayyyu);
\coordinate (glapppfv) at (\glaxxxf, \glayyyv);
\coordinate (glapppfw) at (\glaxxxf, \glayyyw);
\coordinate (glapppfx) at (\glaxxxf, \glayyyx);
\coordinate (glapppfy) at (\glaxxxf, \glayyyy);
\coordinate (glapppfz) at (\glaxxxf, \glayyyz);
\coordinate (glapppga) at (\glaxxxg, \glayyya);
\coordinate (glapppgb) at (\glaxxxg, \glayyyb);
\coordinate (glapppgc) at (\glaxxxg, \glayyyc);
\coordinate (glapppgd) at (\glaxxxg, \glayyyd);
\coordinate (glapppge) at (\glaxxxg, \glayyye);
\coordinate (glapppgf) at (\glaxxxg, \glayyyf);
\coordinate (glapppgg) at (\glaxxxg, \glayyyg);
\coordinate (glapppgh) at (\glaxxxg, \glayyyh);
\coordinate (glapppgi) at (\glaxxxg, \glayyyi);
\coordinate (glapppgj) at (\glaxxxg, \glayyyj);
\coordinate (glapppgk) at (\glaxxxg, \glayyyk);
\coordinate (glapppgl) at (\glaxxxg, \glayyyl);
\coordinate (glapppgm) at (\glaxxxg, \glayyym);
\coordinate (glapppgn) at (\glaxxxg, \glayyyn);
\coordinate (glapppgo) at (\glaxxxg, \glayyyo);
\coordinate (glapppgp) at (\glaxxxg, \glayyyp);
\coordinate (glapppgq) at (\glaxxxg, \glayyyq);
\coordinate (glapppgr) at (\glaxxxg, \glayyyr);
\coordinate (glapppgs) at (\glaxxxg, \glayyys);
\coordinate (glapppgt) at (\glaxxxg, \glayyyt);
\coordinate (glapppgu) at (\glaxxxg, \glayyyu);
\coordinate (glapppgv) at (\glaxxxg, \glayyyv);
\coordinate (glapppgw) at (\glaxxxg, \glayyyw);
\coordinate (glapppgx) at (\glaxxxg, \glayyyx);
\coordinate (glapppgy) at (\glaxxxg, \glayyyy);
\coordinate (glapppgz) at (\glaxxxg, \glayyyz);
\coordinate (glapppha) at (\glaxxxh, \glayyya);
\coordinate (glappphb) at (\glaxxxh, \glayyyb);
\coordinate (glappphc) at (\glaxxxh, \glayyyc);
\coordinate (glappphd) at (\glaxxxh, \glayyyd);
\coordinate (glappphe) at (\glaxxxh, \glayyye);
\coordinate (glappphf) at (\glaxxxh, \glayyyf);
\coordinate (glappphg) at (\glaxxxh, \glayyyg);
\coordinate (glappphh) at (\glaxxxh, \glayyyh);
\coordinate (glappphi) at (\glaxxxh, \glayyyi);
\coordinate (glappphj) at (\glaxxxh, \glayyyj);
\coordinate (glappphk) at (\glaxxxh, \glayyyk);
\coordinate (glappphl) at (\glaxxxh, \glayyyl);
\coordinate (glappphm) at (\glaxxxh, \glayyym);
\coordinate (glappphn) at (\glaxxxh, \glayyyn);
\coordinate (glapppho) at (\glaxxxh, \glayyyo);
\coordinate (glappphp) at (\glaxxxh, \glayyyp);
\coordinate (glappphq) at (\glaxxxh, \glayyyq);
\coordinate (glappphr) at (\glaxxxh, \glayyyr);
\coordinate (glappphs) at (\glaxxxh, \glayyys);
\coordinate (glapppht) at (\glaxxxh, \glayyyt);
\coordinate (glappphu) at (\glaxxxh, \glayyyu);
\coordinate (glappphv) at (\glaxxxh, \glayyyv);
\coordinate (glappphw) at (\glaxxxh, \glayyyw);
\coordinate (glappphx) at (\glaxxxh, \glayyyx);
\coordinate (glappphy) at (\glaxxxh, \glayyyy);
\coordinate (glappphz) at (\glaxxxh, \glayyyz);
\coordinate (glapppia) at (\glaxxxi, \glayyya);
\coordinate (glapppib) at (\glaxxxi, \glayyyb);
\coordinate (glapppic) at (\glaxxxi, \glayyyc);
\coordinate (glapppid) at (\glaxxxi, \glayyyd);
\coordinate (glapppie) at (\glaxxxi, \glayyye);
\coordinate (glapppif) at (\glaxxxi, \glayyyf);
\coordinate (glapppig) at (\glaxxxi, \glayyyg);
\coordinate (glapppih) at (\glaxxxi, \glayyyh);
\coordinate (glapppii) at (\glaxxxi, \glayyyi);
\coordinate (glapppij) at (\glaxxxi, \glayyyj);
\coordinate (glapppik) at (\glaxxxi, \glayyyk);
\coordinate (glapppil) at (\glaxxxi, \glayyyl);
\coordinate (glapppim) at (\glaxxxi, \glayyym);
\coordinate (glapppin) at (\glaxxxi, \glayyyn);
\coordinate (glapppio) at (\glaxxxi, \glayyyo);
\coordinate (glapppip) at (\glaxxxi, \glayyyp);
\coordinate (glapppiq) at (\glaxxxi, \glayyyq);
\coordinate (glapppir) at (\glaxxxi, \glayyyr);
\coordinate (glapppis) at (\glaxxxi, \glayyys);
\coordinate (glapppit) at (\glaxxxi, \glayyyt);
\coordinate (glapppiu) at (\glaxxxi, \glayyyu);
\coordinate (glapppiv) at (\glaxxxi, \glayyyv);
\coordinate (glapppiw) at (\glaxxxi, \glayyyw);
\coordinate (glapppix) at (\glaxxxi, \glayyyx);
\coordinate (glapppiy) at (\glaxxxi, \glayyyy);
\coordinate (glapppiz) at (\glaxxxi, \glayyyz);
\coordinate (glapppja) at (\glaxxxj, \glayyya);
\coordinate (glapppjb) at (\glaxxxj, \glayyyb);
\coordinate (glapppjc) at (\glaxxxj, \glayyyc);
\coordinate (glapppjd) at (\glaxxxj, \glayyyd);
\coordinate (glapppje) at (\glaxxxj, \glayyye);
\coordinate (glapppjf) at (\glaxxxj, \glayyyf);
\coordinate (glapppjg) at (\glaxxxj, \glayyyg);
\coordinate (glapppjh) at (\glaxxxj, \glayyyh);
\coordinate (glapppji) at (\glaxxxj, \glayyyi);
\coordinate (glapppjj) at (\glaxxxj, \glayyyj);
\coordinate (glapppjk) at (\glaxxxj, \glayyyk);
\coordinate (glapppjl) at (\glaxxxj, \glayyyl);
\coordinate (glapppjm) at (\glaxxxj, \glayyym);
\coordinate (glapppjn) at (\glaxxxj, \glayyyn);
\coordinate (glapppjo) at (\glaxxxj, \glayyyo);
\coordinate (glapppjp) at (\glaxxxj, \glayyyp);
\coordinate (glapppjq) at (\glaxxxj, \glayyyq);
\coordinate (glapppjr) at (\glaxxxj, \glayyyr);
\coordinate (glapppjs) at (\glaxxxj, \glayyys);
\coordinate (glapppjt) at (\glaxxxj, \glayyyt);
\coordinate (glapppju) at (\glaxxxj, \glayyyu);
\coordinate (glapppjv) at (\glaxxxj, \glayyyv);
\coordinate (glapppjw) at (\glaxxxj, \glayyyw);
\coordinate (glapppjx) at (\glaxxxj, \glayyyx);
\coordinate (glapppjy) at (\glaxxxj, \glayyyy);
\coordinate (glapppjz) at (\glaxxxj, \glayyyz);
\coordinate (glapppka) at (\glaxxxk, \glayyya);
\coordinate (glapppkb) at (\glaxxxk, \glayyyb);
\coordinate (glapppkc) at (\glaxxxk, \glayyyc);
\coordinate (glapppkd) at (\glaxxxk, \glayyyd);
\coordinate (glapppke) at (\glaxxxk, \glayyye);
\coordinate (glapppkf) at (\glaxxxk, \glayyyf);
\coordinate (glapppkg) at (\glaxxxk, \glayyyg);
\coordinate (glapppkh) at (\glaxxxk, \glayyyh);
\coordinate (glapppki) at (\glaxxxk, \glayyyi);
\coordinate (glapppkj) at (\glaxxxk, \glayyyj);
\coordinate (glapppkk) at (\glaxxxk, \glayyyk);
\coordinate (glapppkl) at (\glaxxxk, \glayyyl);
\coordinate (glapppkm) at (\glaxxxk, \glayyym);
\coordinate (glapppkn) at (\glaxxxk, \glayyyn);
\coordinate (glapppko) at (\glaxxxk, \glayyyo);
\coordinate (glapppkp) at (\glaxxxk, \glayyyp);
\coordinate (glapppkq) at (\glaxxxk, \glayyyq);
\coordinate (glapppkr) at (\glaxxxk, \glayyyr);
\coordinate (glapppks) at (\glaxxxk, \glayyys);
\coordinate (glapppkt) at (\glaxxxk, \glayyyt);
\coordinate (glapppku) at (\glaxxxk, \glayyyu);
\coordinate (glapppkv) at (\glaxxxk, \glayyyv);
\coordinate (glapppkw) at (\glaxxxk, \glayyyw);
\coordinate (glapppkx) at (\glaxxxk, \glayyyx);
\coordinate (glapppky) at (\glaxxxk, \glayyyy);
\coordinate (glapppkz) at (\glaxxxk, \glayyyz);
\coordinate (glapppla) at (\glaxxxl, \glayyya);
\coordinate (glappplb) at (\glaxxxl, \glayyyb);
\coordinate (glappplc) at (\glaxxxl, \glayyyc);
\coordinate (glapppld) at (\glaxxxl, \glayyyd);
\coordinate (glappple) at (\glaxxxl, \glayyye);
\coordinate (glappplf) at (\glaxxxl, \glayyyf);
\coordinate (glappplg) at (\glaxxxl, \glayyyg);
\coordinate (glappplh) at (\glaxxxl, \glayyyh);
\coordinate (glapppli) at (\glaxxxl, \glayyyi);
\coordinate (glappplj) at (\glaxxxl, \glayyyj);
\coordinate (glappplk) at (\glaxxxl, \glayyyk);
\coordinate (glapppll) at (\glaxxxl, \glayyyl);
\coordinate (glappplm) at (\glaxxxl, \glayyym);
\coordinate (glapppln) at (\glaxxxl, \glayyyn);
\coordinate (glappplo) at (\glaxxxl, \glayyyo);
\coordinate (glappplp) at (\glaxxxl, \glayyyp);
\coordinate (glappplq) at (\glaxxxl, \glayyyq);
\coordinate (glappplr) at (\glaxxxl, \glayyyr);
\coordinate (glapppls) at (\glaxxxl, \glayyys);
\coordinate (glappplt) at (\glaxxxl, \glayyyt);
\coordinate (glappplu) at (\glaxxxl, \glayyyu);
\coordinate (glappplv) at (\glaxxxl, \glayyyv);
\coordinate (glappplw) at (\glaxxxl, \glayyyw);
\coordinate (glappplx) at (\glaxxxl, \glayyyx);
\coordinate (glappply) at (\glaxxxl, \glayyyy);
\coordinate (glappplz) at (\glaxxxl, \glayyyz);
\coordinate (glapppma) at (\glaxxxm, \glayyya);
\coordinate (glapppmb) at (\glaxxxm, \glayyyb);
\coordinate (glapppmc) at (\glaxxxm, \glayyyc);
\coordinate (glapppmd) at (\glaxxxm, \glayyyd);
\coordinate (glapppme) at (\glaxxxm, \glayyye);
\coordinate (glapppmf) at (\glaxxxm, \glayyyf);
\coordinate (glapppmg) at (\glaxxxm, \glayyyg);
\coordinate (glapppmh) at (\glaxxxm, \glayyyh);
\coordinate (glapppmi) at (\glaxxxm, \glayyyi);
\coordinate (glapppmj) at (\glaxxxm, \glayyyj);
\coordinate (glapppmk) at (\glaxxxm, \glayyyk);
\coordinate (glapppml) at (\glaxxxm, \glayyyl);
\coordinate (glapppmm) at (\glaxxxm, \glayyym);
\coordinate (glapppmn) at (\glaxxxm, \glayyyn);
\coordinate (glapppmo) at (\glaxxxm, \glayyyo);
\coordinate (glapppmp) at (\glaxxxm, \glayyyp);
\coordinate (glapppmq) at (\glaxxxm, \glayyyq);
\coordinate (glapppmr) at (\glaxxxm, \glayyyr);
\coordinate (glapppms) at (\glaxxxm, \glayyys);
\coordinate (glapppmt) at (\glaxxxm, \glayyyt);
\coordinate (glapppmu) at (\glaxxxm, \glayyyu);
\coordinate (glapppmv) at (\glaxxxm, \glayyyv);
\coordinate (glapppmw) at (\glaxxxm, \glayyyw);
\coordinate (glapppmx) at (\glaxxxm, \glayyyx);
\coordinate (glapppmy) at (\glaxxxm, \glayyyy);
\coordinate (glapppmz) at (\glaxxxm, \glayyyz);
\coordinate (glapppna) at (\glaxxxn, \glayyya);
\coordinate (glapppnb) at (\glaxxxn, \glayyyb);
\coordinate (glapppnc) at (\glaxxxn, \glayyyc);
\coordinate (glapppnd) at (\glaxxxn, \glayyyd);
\coordinate (glapppne) at (\glaxxxn, \glayyye);
\coordinate (glapppnf) at (\glaxxxn, \glayyyf);
\coordinate (glapppng) at (\glaxxxn, \glayyyg);
\coordinate (glapppnh) at (\glaxxxn, \glayyyh);
\coordinate (glapppni) at (\glaxxxn, \glayyyi);
\coordinate (glapppnj) at (\glaxxxn, \glayyyj);
\coordinate (glapppnk) at (\glaxxxn, \glayyyk);
\coordinate (glapppnl) at (\glaxxxn, \glayyyl);
\coordinate (glapppnm) at (\glaxxxn, \glayyym);
\coordinate (glapppnn) at (\glaxxxn, \glayyyn);
\coordinate (glapppno) at (\glaxxxn, \glayyyo);
\coordinate (glapppnp) at (\glaxxxn, \glayyyp);
\coordinate (glapppnq) at (\glaxxxn, \glayyyq);
\coordinate (glapppnr) at (\glaxxxn, \glayyyr);
\coordinate (glapppns) at (\glaxxxn, \glayyys);
\coordinate (glapppnt) at (\glaxxxn, \glayyyt);
\coordinate (glapppnu) at (\glaxxxn, \glayyyu);
\coordinate (glapppnv) at (\glaxxxn, \glayyyv);
\coordinate (glapppnw) at (\glaxxxn, \glayyyw);
\coordinate (glapppnx) at (\glaxxxn, \glayyyx);
\coordinate (glapppny) at (\glaxxxn, \glayyyy);
\coordinate (glapppnz) at (\glaxxxn, \glayyyz);
\coordinate (glapppoa) at (\glaxxxo, \glayyya);
\coordinate (glapppob) at (\glaxxxo, \glayyyb);
\coordinate (glapppoc) at (\glaxxxo, \glayyyc);
\coordinate (glapppod) at (\glaxxxo, \glayyyd);
\coordinate (glapppoe) at (\glaxxxo, \glayyye);
\coordinate (glapppof) at (\glaxxxo, \glayyyf);
\coordinate (glapppog) at (\glaxxxo, \glayyyg);
\coordinate (glapppoh) at (\glaxxxo, \glayyyh);
\coordinate (glapppoi) at (\glaxxxo, \glayyyi);
\coordinate (glapppoj) at (\glaxxxo, \glayyyj);
\coordinate (glapppok) at (\glaxxxo, \glayyyk);
\coordinate (glapppol) at (\glaxxxo, \glayyyl);
\coordinate (glapppom) at (\glaxxxo, \glayyym);
\coordinate (glapppon) at (\glaxxxo, \glayyyn);
\coordinate (glapppoo) at (\glaxxxo, \glayyyo);
\coordinate (glapppop) at (\glaxxxo, \glayyyp);
\coordinate (glapppoq) at (\glaxxxo, \glayyyq);
\coordinate (glapppor) at (\glaxxxo, \glayyyr);
\coordinate (glapppos) at (\glaxxxo, \glayyys);
\coordinate (glapppot) at (\glaxxxo, \glayyyt);
\coordinate (glapppou) at (\glaxxxo, \glayyyu);
\coordinate (glapppov) at (\glaxxxo, \glayyyv);
\coordinate (glapppow) at (\glaxxxo, \glayyyw);
\coordinate (glapppox) at (\glaxxxo, \glayyyx);
\coordinate (glapppoy) at (\glaxxxo, \glayyyy);
\coordinate (glapppoz) at (\glaxxxo, \glayyyz);
\coordinate (glappppa) at (\glaxxxp, \glayyya);
\coordinate (glappppb) at (\glaxxxp, \glayyyb);
\coordinate (glappppc) at (\glaxxxp, \glayyyc);
\coordinate (glappppd) at (\glaxxxp, \glayyyd);
\coordinate (glappppe) at (\glaxxxp, \glayyye);
\coordinate (glappppf) at (\glaxxxp, \glayyyf);
\coordinate (glappppg) at (\glaxxxp, \glayyyg);
\coordinate (glapppph) at (\glaxxxp, \glayyyh);
\coordinate (glappppi) at (\glaxxxp, \glayyyi);
\coordinate (glappppj) at (\glaxxxp, \glayyyj);
\coordinate (glappppk) at (\glaxxxp, \glayyyk);
\coordinate (glappppl) at (\glaxxxp, \glayyyl);
\coordinate (glappppm) at (\glaxxxp, \glayyym);
\coordinate (glappppn) at (\glaxxxp, \glayyyn);
\coordinate (glappppo) at (\glaxxxp, \glayyyo);
\coordinate (glappppp) at (\glaxxxp, \glayyyp);
\coordinate (glappppq) at (\glaxxxp, \glayyyq);
\coordinate (glappppr) at (\glaxxxp, \glayyyr);
\coordinate (glapppps) at (\glaxxxp, \glayyys);
\coordinate (glappppt) at (\glaxxxp, \glayyyt);
\coordinate (glappppu) at (\glaxxxp, \glayyyu);
\coordinate (glappppv) at (\glaxxxp, \glayyyv);
\coordinate (glappppw) at (\glaxxxp, \glayyyw);
\coordinate (glappppx) at (\glaxxxp, \glayyyx);
\coordinate (glappppy) at (\glaxxxp, \glayyyy);
\coordinate (glappppz) at (\glaxxxp, \glayyyz);
\coordinate (glapppqa) at (\glaxxxq, \glayyya);
\coordinate (glapppqb) at (\glaxxxq, \glayyyb);
\coordinate (glapppqc) at (\glaxxxq, \glayyyc);
\coordinate (glapppqd) at (\glaxxxq, \glayyyd);
\coordinate (glapppqe) at (\glaxxxq, \glayyye);
\coordinate (glapppqf) at (\glaxxxq, \glayyyf);
\coordinate (glapppqg) at (\glaxxxq, \glayyyg);
\coordinate (glapppqh) at (\glaxxxq, \glayyyh);
\coordinate (glapppqi) at (\glaxxxq, \glayyyi);
\coordinate (glapppqj) at (\glaxxxq, \glayyyj);
\coordinate (glapppqk) at (\glaxxxq, \glayyyk);
\coordinate (glapppql) at (\glaxxxq, \glayyyl);
\coordinate (glapppqm) at (\glaxxxq, \glayyym);
\coordinate (glapppqn) at (\glaxxxq, \glayyyn);
\coordinate (glapppqo) at (\glaxxxq, \glayyyo);
\coordinate (glapppqp) at (\glaxxxq, \glayyyp);
\coordinate (glapppqq) at (\glaxxxq, \glayyyq);
\coordinate (glapppqr) at (\glaxxxq, \glayyyr);
\coordinate (glapppqs) at (\glaxxxq, \glayyys);
\coordinate (glapppqt) at (\glaxxxq, \glayyyt);
\coordinate (glapppqu) at (\glaxxxq, \glayyyu);
\coordinate (glapppqv) at (\glaxxxq, \glayyyv);
\coordinate (glapppqw) at (\glaxxxq, \glayyyw);
\coordinate (glapppqx) at (\glaxxxq, \glayyyx);
\coordinate (glapppqy) at (\glaxxxq, \glayyyy);
\coordinate (glapppqz) at (\glaxxxq, \glayyyz);
\coordinate (glapppra) at (\glaxxxr, \glayyya);
\coordinate (glappprb) at (\glaxxxr, \glayyyb);
\coordinate (glappprc) at (\glaxxxr, \glayyyc);
\coordinate (glappprd) at (\glaxxxr, \glayyyd);
\coordinate (glapppre) at (\glaxxxr, \glayyye);
\coordinate (glappprf) at (\glaxxxr, \glayyyf);
\coordinate (glappprg) at (\glaxxxr, \glayyyg);
\coordinate (glappprh) at (\glaxxxr, \glayyyh);
\coordinate (glapppri) at (\glaxxxr, \glayyyi);
\coordinate (glappprj) at (\glaxxxr, \glayyyj);
\coordinate (glappprk) at (\glaxxxr, \glayyyk);
\coordinate (glappprl) at (\glaxxxr, \glayyyl);
\coordinate (glappprm) at (\glaxxxr, \glayyym);
\coordinate (glappprn) at (\glaxxxr, \glayyyn);
\coordinate (glapppro) at (\glaxxxr, \glayyyo);
\coordinate (glappprp) at (\glaxxxr, \glayyyp);
\coordinate (glappprq) at (\glaxxxr, \glayyyq);
\coordinate (glappprr) at (\glaxxxr, \glayyyr);
\coordinate (glappprs) at (\glaxxxr, \glayyys);
\coordinate (glappprt) at (\glaxxxr, \glayyyt);
\coordinate (glapppru) at (\glaxxxr, \glayyyu);
\coordinate (glappprv) at (\glaxxxr, \glayyyv);
\coordinate (glappprw) at (\glaxxxr, \glayyyw);
\coordinate (glappprx) at (\glaxxxr, \glayyyx);
\coordinate (glapppry) at (\glaxxxr, \glayyyy);
\coordinate (glappprz) at (\glaxxxr, \glayyyz);
\coordinate (glapppsa) at (\glaxxxs, \glayyya);
\coordinate (glapppsb) at (\glaxxxs, \glayyyb);
\coordinate (glapppsc) at (\glaxxxs, \glayyyc);
\coordinate (glapppsd) at (\glaxxxs, \glayyyd);
\coordinate (glapppse) at (\glaxxxs, \glayyye);
\coordinate (glapppsf) at (\glaxxxs, \glayyyf);
\coordinate (glapppsg) at (\glaxxxs, \glayyyg);
\coordinate (glapppsh) at (\glaxxxs, \glayyyh);
\coordinate (glapppsi) at (\glaxxxs, \glayyyi);
\coordinate (glapppsj) at (\glaxxxs, \glayyyj);
\coordinate (glapppsk) at (\glaxxxs, \glayyyk);
\coordinate (glapppsl) at (\glaxxxs, \glayyyl);
\coordinate (glapppsm) at (\glaxxxs, \glayyym);
\coordinate (glapppsn) at (\glaxxxs, \glayyyn);
\coordinate (glapppso) at (\glaxxxs, \glayyyo);
\coordinate (glapppsp) at (\glaxxxs, \glayyyp);
\coordinate (glapppsq) at (\glaxxxs, \glayyyq);
\coordinate (glapppsr) at (\glaxxxs, \glayyyr);
\coordinate (glapppss) at (\glaxxxs, \glayyys);
\coordinate (glapppst) at (\glaxxxs, \glayyyt);
\coordinate (glapppsu) at (\glaxxxs, \glayyyu);
\coordinate (glapppsv) at (\glaxxxs, \glayyyv);
\coordinate (glapppsw) at (\glaxxxs, \glayyyw);
\coordinate (glapppsx) at (\glaxxxs, \glayyyx);
\coordinate (glapppsy) at (\glaxxxs, \glayyyy);
\coordinate (glapppsz) at (\glaxxxs, \glayyyz);
\coordinate (glapppta) at (\glaxxxt, \glayyya);
\coordinate (glappptb) at (\glaxxxt, \glayyyb);
\coordinate (glappptc) at (\glaxxxt, \glayyyc);
\coordinate (glappptd) at (\glaxxxt, \glayyyd);
\coordinate (glapppte) at (\glaxxxt, \glayyye);
\coordinate (glappptf) at (\glaxxxt, \glayyyf);
\coordinate (glappptg) at (\glaxxxt, \glayyyg);
\coordinate (glapppth) at (\glaxxxt, \glayyyh);
\coordinate (glapppti) at (\glaxxxt, \glayyyi);
\coordinate (glappptj) at (\glaxxxt, \glayyyj);
\coordinate (glappptk) at (\glaxxxt, \glayyyk);
\coordinate (glappptl) at (\glaxxxt, \glayyyl);
\coordinate (glappptm) at (\glaxxxt, \glayyym);
\coordinate (glappptn) at (\glaxxxt, \glayyyn);
\coordinate (glapppto) at (\glaxxxt, \glayyyo);
\coordinate (glappptp) at (\glaxxxt, \glayyyp);
\coordinate (glappptq) at (\glaxxxt, \glayyyq);
\coordinate (glappptr) at (\glaxxxt, \glayyyr);
\coordinate (glapppts) at (\glaxxxt, \glayyys);
\coordinate (glappptt) at (\glaxxxt, \glayyyt);
\coordinate (glappptu) at (\glaxxxt, \glayyyu);
\coordinate (glappptv) at (\glaxxxt, \glayyyv);
\coordinate (glappptw) at (\glaxxxt, \glayyyw);
\coordinate (glappptx) at (\glaxxxt, \glayyyx);
\coordinate (glapppty) at (\glaxxxt, \glayyyy);
\coordinate (glappptz) at (\glaxxxt, \glayyyz);
\coordinate (glapppua) at (\glaxxxu, \glayyya);
\coordinate (glapppub) at (\glaxxxu, \glayyyb);
\coordinate (glapppuc) at (\glaxxxu, \glayyyc);
\coordinate (glapppud) at (\glaxxxu, \glayyyd);
\coordinate (glapppue) at (\glaxxxu, \glayyye);
\coordinate (glapppuf) at (\glaxxxu, \glayyyf);
\coordinate (glapppug) at (\glaxxxu, \glayyyg);
\coordinate (glapppuh) at (\glaxxxu, \glayyyh);
\coordinate (glapppui) at (\glaxxxu, \glayyyi);
\coordinate (glapppuj) at (\glaxxxu, \glayyyj);
\coordinate (glapppuk) at (\glaxxxu, \glayyyk);
\coordinate (glapppul) at (\glaxxxu, \glayyyl);
\coordinate (glapppum) at (\glaxxxu, \glayyym);
\coordinate (glapppun) at (\glaxxxu, \glayyyn);
\coordinate (glapppuo) at (\glaxxxu, \glayyyo);
\coordinate (glapppup) at (\glaxxxu, \glayyyp);
\coordinate (glapppuq) at (\glaxxxu, \glayyyq);
\coordinate (glapppur) at (\glaxxxu, \glayyyr);
\coordinate (glapppus) at (\glaxxxu, \glayyys);
\coordinate (glappput) at (\glaxxxu, \glayyyt);
\coordinate (glapppuu) at (\glaxxxu, \glayyyu);
\coordinate (glapppuv) at (\glaxxxu, \glayyyv);
\coordinate (glapppuw) at (\glaxxxu, \glayyyw);
\coordinate (glapppux) at (\glaxxxu, \glayyyx);
\coordinate (glapppuy) at (\glaxxxu, \glayyyy);
\coordinate (glapppuz) at (\glaxxxu, \glayyyz);
\coordinate (glapppva) at (\glaxxxv, \glayyya);
\coordinate (glapppvb) at (\glaxxxv, \glayyyb);
\coordinate (glapppvc) at (\glaxxxv, \glayyyc);
\coordinate (glapppvd) at (\glaxxxv, \glayyyd);
\coordinate (glapppve) at (\glaxxxv, \glayyye);
\coordinate (glapppvf) at (\glaxxxv, \glayyyf);
\coordinate (glapppvg) at (\glaxxxv, \glayyyg);
\coordinate (glapppvh) at (\glaxxxv, \glayyyh);
\coordinate (glapppvi) at (\glaxxxv, \glayyyi);
\coordinate (glapppvj) at (\glaxxxv, \glayyyj);
\coordinate (glapppvk) at (\glaxxxv, \glayyyk);
\coordinate (glapppvl) at (\glaxxxv, \glayyyl);
\coordinate (glapppvm) at (\glaxxxv, \glayyym);
\coordinate (glapppvn) at (\glaxxxv, \glayyyn);
\coordinate (glapppvo) at (\glaxxxv, \glayyyo);
\coordinate (glapppvp) at (\glaxxxv, \glayyyp);
\coordinate (glapppvq) at (\glaxxxv, \glayyyq);
\coordinate (glapppvr) at (\glaxxxv, \glayyyr);
\coordinate (glapppvs) at (\glaxxxv, \glayyys);
\coordinate (glapppvt) at (\glaxxxv, \glayyyt);
\coordinate (glapppvu) at (\glaxxxv, \glayyyu);
\coordinate (glapppvv) at (\glaxxxv, \glayyyv);
\coordinate (glapppvw) at (\glaxxxv, \glayyyw);
\coordinate (glapppvx) at (\glaxxxv, \glayyyx);
\coordinate (glapppvy) at (\glaxxxv, \glayyyy);
\coordinate (glapppvz) at (\glaxxxv, \glayyyz);
\coordinate (glapppwa) at (\glaxxxw, \glayyya);
\coordinate (glapppwb) at (\glaxxxw, \glayyyb);
\coordinate (glapppwc) at (\glaxxxw, \glayyyc);
\coordinate (glapppwd) at (\glaxxxw, \glayyyd);
\coordinate (glapppwe) at (\glaxxxw, \glayyye);
\coordinate (glapppwf) at (\glaxxxw, \glayyyf);
\coordinate (glapppwg) at (\glaxxxw, \glayyyg);
\coordinate (glapppwh) at (\glaxxxw, \glayyyh);
\coordinate (glapppwi) at (\glaxxxw, \glayyyi);
\coordinate (glapppwj) at (\glaxxxw, \glayyyj);
\coordinate (glapppwk) at (\glaxxxw, \glayyyk);
\coordinate (glapppwl) at (\glaxxxw, \glayyyl);
\coordinate (glapppwm) at (\glaxxxw, \glayyym);
\coordinate (glapppwn) at (\glaxxxw, \glayyyn);
\coordinate (glapppwo) at (\glaxxxw, \glayyyo);
\coordinate (glapppwp) at (\glaxxxw, \glayyyp);
\coordinate (glapppwq) at (\glaxxxw, \glayyyq);
\coordinate (glapppwr) at (\glaxxxw, \glayyyr);
\coordinate (glapppws) at (\glaxxxw, \glayyys);
\coordinate (glapppwt) at (\glaxxxw, \glayyyt);
\coordinate (glapppwu) at (\glaxxxw, \glayyyu);
\coordinate (glapppwv) at (\glaxxxw, \glayyyv);
\coordinate (glapppww) at (\glaxxxw, \glayyyw);
\coordinate (glapppwx) at (\glaxxxw, \glayyyx);
\coordinate (glapppwy) at (\glaxxxw, \glayyyy);
\coordinate (glapppwz) at (\glaxxxw, \glayyyz);
\coordinate (glapppxa) at (\glaxxxx, \glayyya);
\coordinate (glapppxb) at (\glaxxxx, \glayyyb);
\coordinate (glapppxc) at (\glaxxxx, \glayyyc);
\coordinate (glapppxd) at (\glaxxxx, \glayyyd);
\coordinate (glapppxe) at (\glaxxxx, \glayyye);
\coordinate (glapppxf) at (\glaxxxx, \glayyyf);
\coordinate (glapppxg) at (\glaxxxx, \glayyyg);
\coordinate (glapppxh) at (\glaxxxx, \glayyyh);
\coordinate (glapppxi) at (\glaxxxx, \glayyyi);
\coordinate (glapppxj) at (\glaxxxx, \glayyyj);
\coordinate (glapppxk) at (\glaxxxx, \glayyyk);
\coordinate (glapppxl) at (\glaxxxx, \glayyyl);
\coordinate (glapppxm) at (\glaxxxx, \glayyym);
\coordinate (glapppxn) at (\glaxxxx, \glayyyn);
\coordinate (glapppxo) at (\glaxxxx, \glayyyo);
\coordinate (glapppxp) at (\glaxxxx, \glayyyp);
\coordinate (glapppxq) at (\glaxxxx, \glayyyq);
\coordinate (glapppxr) at (\glaxxxx, \glayyyr);
\coordinate (glapppxs) at (\glaxxxx, \glayyys);
\coordinate (glapppxt) at (\glaxxxx, \glayyyt);
\coordinate (glapppxu) at (\glaxxxx, \glayyyu);
\coordinate (glapppxv) at (\glaxxxx, \glayyyv);
\coordinate (glapppxw) at (\glaxxxx, \glayyyw);
\coordinate (glapppxx) at (\glaxxxx, \glayyyx);
\coordinate (glapppxy) at (\glaxxxx, \glayyyy);
\coordinate (glapppxz) at (\glaxxxx, \glayyyz);
\coordinate (glapppya) at (\glaxxxy, \glayyya);
\coordinate (glapppyb) at (\glaxxxy, \glayyyb);
\coordinate (glapppyc) at (\glaxxxy, \glayyyc);
\coordinate (glapppyd) at (\glaxxxy, \glayyyd);
\coordinate (glapppye) at (\glaxxxy, \glayyye);
\coordinate (glapppyf) at (\glaxxxy, \glayyyf);
\coordinate (glapppyg) at (\glaxxxy, \glayyyg);
\coordinate (glapppyh) at (\glaxxxy, \glayyyh);
\coordinate (glapppyi) at (\glaxxxy, \glayyyi);
\coordinate (glapppyj) at (\glaxxxy, \glayyyj);
\coordinate (glapppyk) at (\glaxxxy, \glayyyk);
\coordinate (glapppyl) at (\glaxxxy, \glayyyl);
\coordinate (glapppym) at (\glaxxxy, \glayyym);
\coordinate (glapppyn) at (\glaxxxy, \glayyyn);
\coordinate (glapppyo) at (\glaxxxy, \glayyyo);
\coordinate (glapppyp) at (\glaxxxy, \glayyyp);
\coordinate (glapppyq) at (\glaxxxy, \glayyyq);
\coordinate (glapppyr) at (\glaxxxy, \glayyyr);
\coordinate (glapppys) at (\glaxxxy, \glayyys);
\coordinate (glapppyt) at (\glaxxxy, \glayyyt);
\coordinate (glapppyu) at (\glaxxxy, \glayyyu);
\coordinate (glapppyv) at (\glaxxxy, \glayyyv);
\coordinate (glapppyw) at (\glaxxxy, \glayyyw);
\coordinate (glapppyx) at (\glaxxxy, \glayyyx);
\coordinate (glapppyy) at (\glaxxxy, \glayyyy);
\coordinate (glapppyz) at (\glaxxxy, \glayyyz);
\coordinate (glapppza) at (\glaxxxz, \glayyya);
\coordinate (glapppzb) at (\glaxxxz, \glayyyb);
\coordinate (glapppzc) at (\glaxxxz, \glayyyc);
\coordinate (glapppzd) at (\glaxxxz, \glayyyd);
\coordinate (glapppze) at (\glaxxxz, \glayyye);
\coordinate (glapppzf) at (\glaxxxz, \glayyyf);
\coordinate (glapppzg) at (\glaxxxz, \glayyyg);
\coordinate (glapppzh) at (\glaxxxz, \glayyyh);
\coordinate (glapppzi) at (\glaxxxz, \glayyyi);
\coordinate (glapppzj) at (\glaxxxz, \glayyyj);
\coordinate (glapppzk) at (\glaxxxz, \glayyyk);
\coordinate (glapppzl) at (\glaxxxz, \glayyyl);
\coordinate (glapppzm) at (\glaxxxz, \glayyym);
\coordinate (glapppzn) at (\glaxxxz, \glayyyn);
\coordinate (glapppzo) at (\glaxxxz, \glayyyo);
\coordinate (glapppzp) at (\glaxxxz, \glayyyp);
\coordinate (glapppzq) at (\glaxxxz, \glayyyq);
\coordinate (glapppzr) at (\glaxxxz, \glayyyr);
\coordinate (glapppzs) at (\glaxxxz, \glayyys);
\coordinate (glapppzt) at (\glaxxxz, \glayyyt);
\coordinate (glapppzu) at (\glaxxxz, \glayyyu);
\coordinate (glapppzv) at (\glaxxxz, \glayyyv);
\coordinate (glapppzw) at (\glaxxxz, \glayyyw);
\coordinate (glapppzx) at (\glaxxxz, \glayyyx);
\coordinate (glapppzy) at (\glaxxxz, \glayyyy);
\coordinate (glapppzz) at (\glaxxxz, \glayyyz);

%\gangprintcoordinateat{(0,0)}{The last coordinate values: }{($(glapppzz)$)}; 



\pgfmathsetmacro{\totalglbxxx}{26}
\pgfmathsetmacro{\totalglbyyy}{26}
\pgfmathsetmacro{\glbxxxspacing}{1}
\pgfmathsetmacro{\glbyyyspacing}{1}
\pgfmathsetmacro{\glbxxxa}{-8}
\pgfmathsetmacro{\glbyyya}{\gangliuxxxa - 16.0}

\pgfmathsetmacro{\glbxxxb}{\glbxxxa + \glbxxxspacing + 0.0 }
\pgfmathsetmacro{\glbxxxc}{\glbxxxb + \glbxxxspacing + 0.0 }
\pgfmathsetmacro{\glbxxxd}{\glbxxxc + \glbxxxspacing + 0.0 }
\pgfmathsetmacro{\glbxxxe}{\glbxxxd + \glbxxxspacing + 0.0 }
\pgfmathsetmacro{\glbxxxf}{\glbxxxe + \glbxxxspacing + 0.0 }
\pgfmathsetmacro{\glbxxxg}{\glbxxxf + \glbxxxspacing + 0.0 }
\pgfmathsetmacro{\glbxxxh}{\glbxxxg + \glbxxxspacing + 0.0 }
\pgfmathsetmacro{\glbxxxi}{\glbxxxh + \glbxxxspacing + 0.0 }
\pgfmathsetmacro{\glbxxxj}{\glbxxxi + \glbxxxspacing + 0.0 }
\pgfmathsetmacro{\glbxxxk}{\glbxxxj + \glbxxxspacing + 0.0 }
\pgfmathsetmacro{\glbxxxl}{\glbxxxk + \glbxxxspacing + 0.0 }
\pgfmathsetmacro{\glbxxxm}{\glbxxxl + \glbxxxspacing + 0.0 }
\pgfmathsetmacro{\glbxxxn}{\glbxxxm + \glbxxxspacing + 0.0 }
\pgfmathsetmacro{\glbxxxo}{\glbxxxn + \glbxxxspacing + 0.0 }
\pgfmathsetmacro{\glbxxxp}{\glbxxxo + \glbxxxspacing + 0.0 }
\pgfmathsetmacro{\glbxxxq}{\glbxxxp + \glbxxxspacing + 0.0 }
\pgfmathsetmacro{\glbxxxr}{\glbxxxq + \glbxxxspacing + 0.0 }
\pgfmathsetmacro{\glbxxxs}{\glbxxxr + \glbxxxspacing + 0.0 }
\pgfmathsetmacro{\glbxxxt}{\glbxxxs + \glbxxxspacing + 0.0 }
\pgfmathsetmacro{\glbxxxu}{\glbxxxt + \glbxxxspacing + 0.0 }
\pgfmathsetmacro{\glbxxxv}{\glbxxxu + \glbxxxspacing + 0.0 }
\pgfmathsetmacro{\glbxxxw}{\glbxxxv + \glbxxxspacing + 0.0 }
\pgfmathsetmacro{\glbxxxx}{\glbxxxw + \glbxxxspacing + 0.0 }
\pgfmathsetmacro{\glbxxxy}{\glbxxxx + \glbxxxspacing + 0.0 }
\pgfmathsetmacro{\glbxxxz}{\glbxxxy + \glbxxxspacing + 0.0 }

\pgfmathsetmacro{\glbyyyb}{\glbyyya + \glbyyyspacing + 0.0 }
\pgfmathsetmacro{\glbyyyc}{\glbyyyb + \glbyyyspacing + 0.0 }
\pgfmathsetmacro{\glbyyyd}{\glbyyyc + \glbyyyspacing + 0.0 }
\pgfmathsetmacro{\glbyyye}{\glbyyyd + \glbyyyspacing + 0.0 }
\pgfmathsetmacro{\glbyyyf}{\glbyyye + \glbyyyspacing + 0.0 }
\pgfmathsetmacro{\glbyyyg}{\glbyyyf + \glbyyyspacing + 0.0 }
\pgfmathsetmacro{\glbyyyh}{\glbyyyg + \glbyyyspacing + 0.0 }
\pgfmathsetmacro{\glbyyyi}{\glbyyyh + \glbyyyspacing + 0.0 }
\pgfmathsetmacro{\glbyyyj}{\glbyyyi + \glbyyyspacing + 0.0 }
\pgfmathsetmacro{\glbyyyk}{\glbyyyj + \glbyyyspacing + 0.0 }
\pgfmathsetmacro{\glbyyyl}{\glbyyyk + \glbyyyspacing + 0.0 }
\pgfmathsetmacro{\glbyyym}{\glbyyyl + \glbyyyspacing + 0.0 }
\pgfmathsetmacro{\glbyyyn}{\glbyyym + \glbyyyspacing + 0.0 }
\pgfmathsetmacro{\glbyyyo}{\glbyyyn + \glbyyyspacing + 0.0 }
\pgfmathsetmacro{\glbyyyp}{\glbyyyo + \glbyyyspacing + 0.0 }
\pgfmathsetmacro{\glbyyyq}{\glbyyyp + \glbyyyspacing + 0.0 }
\pgfmathsetmacro{\glbyyyr}{\glbyyyq + \glbyyyspacing + 0.0 }
\pgfmathsetmacro{\glbyyys}{\glbyyyr + \glbyyyspacing + 0.0 }
\pgfmathsetmacro{\glbyyyt}{\glbyyys + \glbyyyspacing + 0.0 }
\pgfmathsetmacro{\glbyyyu}{\glbyyyt + \glbyyyspacing + 0.0 }
\pgfmathsetmacro{\glbyyyv}{\glbyyyu + \glbyyyspacing + 0.0 }
\pgfmathsetmacro{\glbyyyw}{\glbyyyv + \glbyyyspacing + 0.0 }
\pgfmathsetmacro{\glbyyyx}{\glbyyyw + \glbyyyspacing + 0.0 }
\pgfmathsetmacro{\glbyyyy}{\glbyyyx + \glbyyyspacing + 0.0 }
\pgfmathsetmacro{\glbyyyz}{\glbyyyy + \glbyyyspacing + 0.0 }

\coordinate (glbpppaa) at (\glbxxxa, \glbyyya);
\coordinate (glbpppab) at (\glbxxxa, \glbyyyb);
\coordinate (glbpppac) at (\glbxxxa, \glbyyyc);
\coordinate (glbpppad) at (\glbxxxa, \glbyyyd);
\coordinate (glbpppae) at (\glbxxxa, \glbyyye);
\coordinate (glbpppaf) at (\glbxxxa, \glbyyyf);
\coordinate (glbpppag) at (\glbxxxa, \glbyyyg);
\coordinate (glbpppah) at (\glbxxxa, \glbyyyh);
\coordinate (glbpppai) at (\glbxxxa, \glbyyyi);
\coordinate (glbpppaj) at (\glbxxxa, \glbyyyj);
\coordinate (glbpppak) at (\glbxxxa, \glbyyyk);
\coordinate (glbpppal) at (\glbxxxa, \glbyyyl);
\coordinate (glbpppam) at (\glbxxxa, \glbyyym);
\coordinate (glbpppan) at (\glbxxxa, \glbyyyn);
\coordinate (glbpppao) at (\glbxxxa, \glbyyyo);
\coordinate (glbpppap) at (\glbxxxa, \glbyyyp);
\coordinate (glbpppaq) at (\glbxxxa, \glbyyyq);
\coordinate (glbpppar) at (\glbxxxa, \glbyyyr);
\coordinate (glbpppas) at (\glbxxxa, \glbyyys);
\coordinate (glbpppat) at (\glbxxxa, \glbyyyt);
\coordinate (glbpppau) at (\glbxxxa, \glbyyyu);
\coordinate (glbpppav) at (\glbxxxa, \glbyyyv);
\coordinate (glbpppaw) at (\glbxxxa, \glbyyyw);
\coordinate (glbpppax) at (\glbxxxa, \glbyyyx);
\coordinate (glbpppay) at (\glbxxxa, \glbyyyy);
\coordinate (glbpppaz) at (\glbxxxa, \glbyyyz);
\coordinate (glbpppba) at (\glbxxxb, \glbyyya);
\coordinate (glbpppbb) at (\glbxxxb, \glbyyyb);
\coordinate (glbpppbc) at (\glbxxxb, \glbyyyc);
\coordinate (glbpppbd) at (\glbxxxb, \glbyyyd);
\coordinate (glbpppbe) at (\glbxxxb, \glbyyye);
\coordinate (glbpppbf) at (\glbxxxb, \glbyyyf);
\coordinate (glbpppbg) at (\glbxxxb, \glbyyyg);
\coordinate (glbpppbh) at (\glbxxxb, \glbyyyh);
\coordinate (glbpppbi) at (\glbxxxb, \glbyyyi);
\coordinate (glbpppbj) at (\glbxxxb, \glbyyyj);
\coordinate (glbpppbk) at (\glbxxxb, \glbyyyk);
\coordinate (glbpppbl) at (\glbxxxb, \glbyyyl);
\coordinate (glbpppbm) at (\glbxxxb, \glbyyym);
\coordinate (glbpppbn) at (\glbxxxb, \glbyyyn);
\coordinate (glbpppbo) at (\glbxxxb, \glbyyyo);
\coordinate (glbpppbp) at (\glbxxxb, \glbyyyp);
\coordinate (glbpppbq) at (\glbxxxb, \glbyyyq);
\coordinate (glbpppbr) at (\glbxxxb, \glbyyyr);
\coordinate (glbpppbs) at (\glbxxxb, \glbyyys);
\coordinate (glbpppbt) at (\glbxxxb, \glbyyyt);
\coordinate (glbpppbu) at (\glbxxxb, \glbyyyu);
\coordinate (glbpppbv) at (\glbxxxb, \glbyyyv);
\coordinate (glbpppbw) at (\glbxxxb, \glbyyyw);
\coordinate (glbpppbx) at (\glbxxxb, \glbyyyx);
\coordinate (glbpppby) at (\glbxxxb, \glbyyyy);
\coordinate (glbpppbz) at (\glbxxxb, \glbyyyz);
\coordinate (glbpppca) at (\glbxxxc, \glbyyya);
\coordinate (glbpppcb) at (\glbxxxc, \glbyyyb);
\coordinate (glbpppcc) at (\glbxxxc, \glbyyyc);
\coordinate (glbpppcd) at (\glbxxxc, \glbyyyd);
\coordinate (glbpppce) at (\glbxxxc, \glbyyye);
\coordinate (glbpppcf) at (\glbxxxc, \glbyyyf);
\coordinate (glbpppcg) at (\glbxxxc, \glbyyyg);
\coordinate (glbpppch) at (\glbxxxc, \glbyyyh);
\coordinate (glbpppci) at (\glbxxxc, \glbyyyi);
\coordinate (glbpppcj) at (\glbxxxc, \glbyyyj);
\coordinate (glbpppck) at (\glbxxxc, \glbyyyk);
\coordinate (glbpppcl) at (\glbxxxc, \glbyyyl);
\coordinate (glbpppcm) at (\glbxxxc, \glbyyym);
\coordinate (glbpppcn) at (\glbxxxc, \glbyyyn);
\coordinate (glbpppco) at (\glbxxxc, \glbyyyo);
\coordinate (glbpppcp) at (\glbxxxc, \glbyyyp);
\coordinate (glbpppcq) at (\glbxxxc, \glbyyyq);
\coordinate (glbpppcr) at (\glbxxxc, \glbyyyr);
\coordinate (glbpppcs) at (\glbxxxc, \glbyyys);
\coordinate (glbpppct) at (\glbxxxc, \glbyyyt);
\coordinate (glbpppcu) at (\glbxxxc, \glbyyyu);
\coordinate (glbpppcv) at (\glbxxxc, \glbyyyv);
\coordinate (glbpppcw) at (\glbxxxc, \glbyyyw);
\coordinate (glbpppcx) at (\glbxxxc, \glbyyyx);
\coordinate (glbpppcy) at (\glbxxxc, \glbyyyy);
\coordinate (glbpppcz) at (\glbxxxc, \glbyyyz);
\coordinate (glbpppda) at (\glbxxxd, \glbyyya);
\coordinate (glbpppdb) at (\glbxxxd, \glbyyyb);
\coordinate (glbpppdc) at (\glbxxxd, \glbyyyc);
\coordinate (glbpppdd) at (\glbxxxd, \glbyyyd);
\coordinate (glbpppde) at (\glbxxxd, \glbyyye);
\coordinate (glbpppdf) at (\glbxxxd, \glbyyyf);
\coordinate (glbpppdg) at (\glbxxxd, \glbyyyg);
\coordinate (glbpppdh) at (\glbxxxd, \glbyyyh);
\coordinate (glbpppdi) at (\glbxxxd, \glbyyyi);
\coordinate (glbpppdj) at (\glbxxxd, \glbyyyj);
\coordinate (glbpppdk) at (\glbxxxd, \glbyyyk);
\coordinate (glbpppdl) at (\glbxxxd, \glbyyyl);
\coordinate (glbpppdm) at (\glbxxxd, \glbyyym);
\coordinate (glbpppdn) at (\glbxxxd, \glbyyyn);
\coordinate (glbpppdo) at (\glbxxxd, \glbyyyo);
\coordinate (glbpppdp) at (\glbxxxd, \glbyyyp);
\coordinate (glbpppdq) at (\glbxxxd, \glbyyyq);
\coordinate (glbpppdr) at (\glbxxxd, \glbyyyr);
\coordinate (glbpppds) at (\glbxxxd, \glbyyys);
\coordinate (glbpppdt) at (\glbxxxd, \glbyyyt);
\coordinate (glbpppdu) at (\glbxxxd, \glbyyyu);
\coordinate (glbpppdv) at (\glbxxxd, \glbyyyv);
\coordinate (glbpppdw) at (\glbxxxd, \glbyyyw);
\coordinate (glbpppdx) at (\glbxxxd, \glbyyyx);
\coordinate (glbpppdy) at (\glbxxxd, \glbyyyy);
\coordinate (glbpppdz) at (\glbxxxd, \glbyyyz);
\coordinate (glbpppea) at (\glbxxxe, \glbyyya);
\coordinate (glbpppeb) at (\glbxxxe, \glbyyyb);
\coordinate (glbpppec) at (\glbxxxe, \glbyyyc);
\coordinate (glbppped) at (\glbxxxe, \glbyyyd);
\coordinate (glbpppee) at (\glbxxxe, \glbyyye);
\coordinate (glbpppef) at (\glbxxxe, \glbyyyf);
\coordinate (glbpppeg) at (\glbxxxe, \glbyyyg);
\coordinate (glbpppeh) at (\glbxxxe, \glbyyyh);
\coordinate (glbpppei) at (\glbxxxe, \glbyyyi);
\coordinate (glbpppej) at (\glbxxxe, \glbyyyj);
\coordinate (glbpppek) at (\glbxxxe, \glbyyyk);
\coordinate (glbpppel) at (\glbxxxe, \glbyyyl);
\coordinate (glbpppem) at (\glbxxxe, \glbyyym);
\coordinate (glbpppen) at (\glbxxxe, \glbyyyn);
\coordinate (glbpppeo) at (\glbxxxe, \glbyyyo);
\coordinate (glbpppep) at (\glbxxxe, \glbyyyp);
\coordinate (glbpppeq) at (\glbxxxe, \glbyyyq);
\coordinate (glbppper) at (\glbxxxe, \glbyyyr);
\coordinate (glbpppes) at (\glbxxxe, \glbyyys);
\coordinate (glbpppet) at (\glbxxxe, \glbyyyt);
\coordinate (glbpppeu) at (\glbxxxe, \glbyyyu);
\coordinate (glbpppev) at (\glbxxxe, \glbyyyv);
\coordinate (glbpppew) at (\glbxxxe, \glbyyyw);
\coordinate (glbpppex) at (\glbxxxe, \glbyyyx);
\coordinate (glbpppey) at (\glbxxxe, \glbyyyy);
\coordinate (glbpppez) at (\glbxxxe, \glbyyyz);
\coordinate (glbpppfa) at (\glbxxxf, \glbyyya);
\coordinate (glbpppfb) at (\glbxxxf, \glbyyyb);
\coordinate (glbpppfc) at (\glbxxxf, \glbyyyc);
\coordinate (glbpppfd) at (\glbxxxf, \glbyyyd);
\coordinate (glbpppfe) at (\glbxxxf, \glbyyye);
\coordinate (glbpppff) at (\glbxxxf, \glbyyyf);
\coordinate (glbpppfg) at (\glbxxxf, \glbyyyg);
\coordinate (glbpppfh) at (\glbxxxf, \glbyyyh);
\coordinate (glbpppfi) at (\glbxxxf, \glbyyyi);
\coordinate (glbpppfj) at (\glbxxxf, \glbyyyj);
\coordinate (glbpppfk) at (\glbxxxf, \glbyyyk);
\coordinate (glbpppfl) at (\glbxxxf, \glbyyyl);
\coordinate (glbpppfm) at (\glbxxxf, \glbyyym);
\coordinate (glbpppfn) at (\glbxxxf, \glbyyyn);
\coordinate (glbpppfo) at (\glbxxxf, \glbyyyo);
\coordinate (glbpppfp) at (\glbxxxf, \glbyyyp);
\coordinate (glbpppfq) at (\glbxxxf, \glbyyyq);
\coordinate (glbpppfr) at (\glbxxxf, \glbyyyr);
\coordinate (glbpppfs) at (\glbxxxf, \glbyyys);
\coordinate (glbpppft) at (\glbxxxf, \glbyyyt);
\coordinate (glbpppfu) at (\glbxxxf, \glbyyyu);
\coordinate (glbpppfv) at (\glbxxxf, \glbyyyv);
\coordinate (glbpppfw) at (\glbxxxf, \glbyyyw);
\coordinate (glbpppfx) at (\glbxxxf, \glbyyyx);
\coordinate (glbpppfy) at (\glbxxxf, \glbyyyy);
\coordinate (glbpppfz) at (\glbxxxf, \glbyyyz);
\coordinate (glbpppga) at (\glbxxxg, \glbyyya);
\coordinate (glbpppgb) at (\glbxxxg, \glbyyyb);
\coordinate (glbpppgc) at (\glbxxxg, \glbyyyc);
\coordinate (glbpppgd) at (\glbxxxg, \glbyyyd);
\coordinate (glbpppge) at (\glbxxxg, \glbyyye);
\coordinate (glbpppgf) at (\glbxxxg, \glbyyyf);
\coordinate (glbpppgg) at (\glbxxxg, \glbyyyg);
\coordinate (glbpppgh) at (\glbxxxg, \glbyyyh);
\coordinate (glbpppgi) at (\glbxxxg, \glbyyyi);
\coordinate (glbpppgj) at (\glbxxxg, \glbyyyj);
\coordinate (glbpppgk) at (\glbxxxg, \glbyyyk);
\coordinate (glbpppgl) at (\glbxxxg, \glbyyyl);
\coordinate (glbpppgm) at (\glbxxxg, \glbyyym);
\coordinate (glbpppgn) at (\glbxxxg, \glbyyyn);
\coordinate (glbpppgo) at (\glbxxxg, \glbyyyo);
\coordinate (glbpppgp) at (\glbxxxg, \glbyyyp);
\coordinate (glbpppgq) at (\glbxxxg, \glbyyyq);
\coordinate (glbpppgr) at (\glbxxxg, \glbyyyr);
\coordinate (glbpppgs) at (\glbxxxg, \glbyyys);
\coordinate (glbpppgt) at (\glbxxxg, \glbyyyt);
\coordinate (glbpppgu) at (\glbxxxg, \glbyyyu);
\coordinate (glbpppgv) at (\glbxxxg, \glbyyyv);
\coordinate (glbpppgw) at (\glbxxxg, \glbyyyw);
\coordinate (glbpppgx) at (\glbxxxg, \glbyyyx);
\coordinate (glbpppgy) at (\glbxxxg, \glbyyyy);
\coordinate (glbpppgz) at (\glbxxxg, \glbyyyz);
\coordinate (glbpppha) at (\glbxxxh, \glbyyya);
\coordinate (glbppphb) at (\glbxxxh, \glbyyyb);
\coordinate (glbppphc) at (\glbxxxh, \glbyyyc);
\coordinate (glbppphd) at (\glbxxxh, \glbyyyd);
\coordinate (glbppphe) at (\glbxxxh, \glbyyye);
\coordinate (glbppphf) at (\glbxxxh, \glbyyyf);
\coordinate (glbppphg) at (\glbxxxh, \glbyyyg);
\coordinate (glbppphh) at (\glbxxxh, \glbyyyh);
\coordinate (glbppphi) at (\glbxxxh, \glbyyyi);
\coordinate (glbppphj) at (\glbxxxh, \glbyyyj);
\coordinate (glbppphk) at (\glbxxxh, \glbyyyk);
\coordinate (glbppphl) at (\glbxxxh, \glbyyyl);
\coordinate (glbppphm) at (\glbxxxh, \glbyyym);
\coordinate (glbppphn) at (\glbxxxh, \glbyyyn);
\coordinate (glbpppho) at (\glbxxxh, \glbyyyo);
\coordinate (glbppphp) at (\glbxxxh, \glbyyyp);
\coordinate (glbppphq) at (\glbxxxh, \glbyyyq);
\coordinate (glbppphr) at (\glbxxxh, \glbyyyr);
\coordinate (glbppphs) at (\glbxxxh, \glbyyys);
\coordinate (glbpppht) at (\glbxxxh, \glbyyyt);
\coordinate (glbppphu) at (\glbxxxh, \glbyyyu);
\coordinate (glbppphv) at (\glbxxxh, \glbyyyv);
\coordinate (glbppphw) at (\glbxxxh, \glbyyyw);
\coordinate (glbppphx) at (\glbxxxh, \glbyyyx);
\coordinate (glbppphy) at (\glbxxxh, \glbyyyy);
\coordinate (glbppphz) at (\glbxxxh, \glbyyyz);
\coordinate (glbpppia) at (\glbxxxi, \glbyyya);
\coordinate (glbpppib) at (\glbxxxi, \glbyyyb);
\coordinate (glbpppic) at (\glbxxxi, \glbyyyc);
\coordinate (glbpppid) at (\glbxxxi, \glbyyyd);
\coordinate (glbpppie) at (\glbxxxi, \glbyyye);
\coordinate (glbpppif) at (\glbxxxi, \glbyyyf);
\coordinate (glbpppig) at (\glbxxxi, \glbyyyg);
\coordinate (glbpppih) at (\glbxxxi, \glbyyyh);
\coordinate (glbpppii) at (\glbxxxi, \glbyyyi);
\coordinate (glbpppij) at (\glbxxxi, \glbyyyj);
\coordinate (glbpppik) at (\glbxxxi, \glbyyyk);
\coordinate (glbpppil) at (\glbxxxi, \glbyyyl);
\coordinate (glbpppim) at (\glbxxxi, \glbyyym);
\coordinate (glbpppin) at (\glbxxxi, \glbyyyn);
\coordinate (glbpppio) at (\glbxxxi, \glbyyyo);
\coordinate (glbpppip) at (\glbxxxi, \glbyyyp);
\coordinate (glbpppiq) at (\glbxxxi, \glbyyyq);
\coordinate (glbpppir) at (\glbxxxi, \glbyyyr);
\coordinate (glbpppis) at (\glbxxxi, \glbyyys);
\coordinate (glbpppit) at (\glbxxxi, \glbyyyt);
\coordinate (glbpppiu) at (\glbxxxi, \glbyyyu);
\coordinate (glbpppiv) at (\glbxxxi, \glbyyyv);
\coordinate (glbpppiw) at (\glbxxxi, \glbyyyw);
\coordinate (glbpppix) at (\glbxxxi, \glbyyyx);
\coordinate (glbpppiy) at (\glbxxxi, \glbyyyy);
\coordinate (glbpppiz) at (\glbxxxi, \glbyyyz);
\coordinate (glbpppja) at (\glbxxxj, \glbyyya);
\coordinate (glbpppjb) at (\glbxxxj, \glbyyyb);
\coordinate (glbpppjc) at (\glbxxxj, \glbyyyc);
\coordinate (glbpppjd) at (\glbxxxj, \glbyyyd);
\coordinate (glbpppje) at (\glbxxxj, \glbyyye);
\coordinate (glbpppjf) at (\glbxxxj, \glbyyyf);
\coordinate (glbpppjg) at (\glbxxxj, \glbyyyg);
\coordinate (glbpppjh) at (\glbxxxj, \glbyyyh);
\coordinate (glbpppji) at (\glbxxxj, \glbyyyi);
\coordinate (glbpppjj) at (\glbxxxj, \glbyyyj);
\coordinate (glbpppjk) at (\glbxxxj, \glbyyyk);
\coordinate (glbpppjl) at (\glbxxxj, \glbyyyl);
\coordinate (glbpppjm) at (\glbxxxj, \glbyyym);
\coordinate (glbpppjn) at (\glbxxxj, \glbyyyn);
\coordinate (glbpppjo) at (\glbxxxj, \glbyyyo);
\coordinate (glbpppjp) at (\glbxxxj, \glbyyyp);
\coordinate (glbpppjq) at (\glbxxxj, \glbyyyq);
\coordinate (glbpppjr) at (\glbxxxj, \glbyyyr);
\coordinate (glbpppjs) at (\glbxxxj, \glbyyys);
\coordinate (glbpppjt) at (\glbxxxj, \glbyyyt);
\coordinate (glbpppju) at (\glbxxxj, \glbyyyu);
\coordinate (glbpppjv) at (\glbxxxj, \glbyyyv);
\coordinate (glbpppjw) at (\glbxxxj, \glbyyyw);
\coordinate (glbpppjx) at (\glbxxxj, \glbyyyx);
\coordinate (glbpppjy) at (\glbxxxj, \glbyyyy);
\coordinate (glbpppjz) at (\glbxxxj, \glbyyyz);
\coordinate (glbpppka) at (\glbxxxk, \glbyyya);
\coordinate (glbpppkb) at (\glbxxxk, \glbyyyb);
\coordinate (glbpppkc) at (\glbxxxk, \glbyyyc);
\coordinate (glbpppkd) at (\glbxxxk, \glbyyyd);
\coordinate (glbpppke) at (\glbxxxk, \glbyyye);
\coordinate (glbpppkf) at (\glbxxxk, \glbyyyf);
\coordinate (glbpppkg) at (\glbxxxk, \glbyyyg);
\coordinate (glbpppkh) at (\glbxxxk, \glbyyyh);
\coordinate (glbpppki) at (\glbxxxk, \glbyyyi);
\coordinate (glbpppkj) at (\glbxxxk, \glbyyyj);
\coordinate (glbpppkk) at (\glbxxxk, \glbyyyk);
\coordinate (glbpppkl) at (\glbxxxk, \glbyyyl);
\coordinate (glbpppkm) at (\glbxxxk, \glbyyym);
\coordinate (glbpppkn) at (\glbxxxk, \glbyyyn);
\coordinate (glbpppko) at (\glbxxxk, \glbyyyo);
\coordinate (glbpppkp) at (\glbxxxk, \glbyyyp);
\coordinate (glbpppkq) at (\glbxxxk, \glbyyyq);
\coordinate (glbpppkr) at (\glbxxxk, \glbyyyr);
\coordinate (glbpppks) at (\glbxxxk, \glbyyys);
\coordinate (glbpppkt) at (\glbxxxk, \glbyyyt);
\coordinate (glbpppku) at (\glbxxxk, \glbyyyu);
\coordinate (glbpppkv) at (\glbxxxk, \glbyyyv);
\coordinate (glbpppkw) at (\glbxxxk, \glbyyyw);
\coordinate (glbpppkx) at (\glbxxxk, \glbyyyx);
\coordinate (glbpppky) at (\glbxxxk, \glbyyyy);
\coordinate (glbpppkz) at (\glbxxxk, \glbyyyz);
\coordinate (glbpppla) at (\glbxxxl, \glbyyya);
\coordinate (glbppplb) at (\glbxxxl, \glbyyyb);
\coordinate (glbppplc) at (\glbxxxl, \glbyyyc);
\coordinate (glbpppld) at (\glbxxxl, \glbyyyd);
\coordinate (glbppple) at (\glbxxxl, \glbyyye);
\coordinate (glbppplf) at (\glbxxxl, \glbyyyf);
\coordinate (glbppplg) at (\glbxxxl, \glbyyyg);
\coordinate (glbppplh) at (\glbxxxl, \glbyyyh);
\coordinate (glbpppli) at (\glbxxxl, \glbyyyi);
\coordinate (glbppplj) at (\glbxxxl, \glbyyyj);
\coordinate (glbppplk) at (\glbxxxl, \glbyyyk);
\coordinate (glbpppll) at (\glbxxxl, \glbyyyl);
\coordinate (glbppplm) at (\glbxxxl, \glbyyym);
\coordinate (glbpppln) at (\glbxxxl, \glbyyyn);
\coordinate (glbppplo) at (\glbxxxl, \glbyyyo);
\coordinate (glbppplp) at (\glbxxxl, \glbyyyp);
\coordinate (glbppplq) at (\glbxxxl, \glbyyyq);
\coordinate (glbppplr) at (\glbxxxl, \glbyyyr);
\coordinate (glbpppls) at (\glbxxxl, \glbyyys);
\coordinate (glbppplt) at (\glbxxxl, \glbyyyt);
\coordinate (glbppplu) at (\glbxxxl, \glbyyyu);
\coordinate (glbppplv) at (\glbxxxl, \glbyyyv);
\coordinate (glbppplw) at (\glbxxxl, \glbyyyw);
\coordinate (glbppplx) at (\glbxxxl, \glbyyyx);
\coordinate (glbppply) at (\glbxxxl, \glbyyyy);
\coordinate (glbppplz) at (\glbxxxl, \glbyyyz);
\coordinate (glbpppma) at (\glbxxxm, \glbyyya);
\coordinate (glbpppmb) at (\glbxxxm, \glbyyyb);
\coordinate (glbpppmc) at (\glbxxxm, \glbyyyc);
\coordinate (glbpppmd) at (\glbxxxm, \glbyyyd);
\coordinate (glbpppme) at (\glbxxxm, \glbyyye);
\coordinate (glbpppmf) at (\glbxxxm, \glbyyyf);
\coordinate (glbpppmg) at (\glbxxxm, \glbyyyg);
\coordinate (glbpppmh) at (\glbxxxm, \glbyyyh);
\coordinate (glbpppmi) at (\glbxxxm, \glbyyyi);
\coordinate (glbpppmj) at (\glbxxxm, \glbyyyj);
\coordinate (glbpppmk) at (\glbxxxm, \glbyyyk);
\coordinate (glbpppml) at (\glbxxxm, \glbyyyl);
\coordinate (glbpppmm) at (\glbxxxm, \glbyyym);
\coordinate (glbpppmn) at (\glbxxxm, \glbyyyn);
\coordinate (glbpppmo) at (\glbxxxm, \glbyyyo);
\coordinate (glbpppmp) at (\glbxxxm, \glbyyyp);
\coordinate (glbpppmq) at (\glbxxxm, \glbyyyq);
\coordinate (glbpppmr) at (\glbxxxm, \glbyyyr);
\coordinate (glbpppms) at (\glbxxxm, \glbyyys);
\coordinate (glbpppmt) at (\glbxxxm, \glbyyyt);
\coordinate (glbpppmu) at (\glbxxxm, \glbyyyu);
\coordinate (glbpppmv) at (\glbxxxm, \glbyyyv);
\coordinate (glbpppmw) at (\glbxxxm, \glbyyyw);
\coordinate (glbpppmx) at (\glbxxxm, \glbyyyx);
\coordinate (glbpppmy) at (\glbxxxm, \glbyyyy);
\coordinate (glbpppmz) at (\glbxxxm, \glbyyyz);
\coordinate (glbpppna) at (\glbxxxn, \glbyyya);
\coordinate (glbpppnb) at (\glbxxxn, \glbyyyb);
\coordinate (glbpppnc) at (\glbxxxn, \glbyyyc);
\coordinate (glbpppnd) at (\glbxxxn, \glbyyyd);
\coordinate (glbpppne) at (\glbxxxn, \glbyyye);
\coordinate (glbpppnf) at (\glbxxxn, \glbyyyf);
\coordinate (glbpppng) at (\glbxxxn, \glbyyyg);
\coordinate (glbpppnh) at (\glbxxxn, \glbyyyh);
\coordinate (glbpppni) at (\glbxxxn, \glbyyyi);
\coordinate (glbpppnj) at (\glbxxxn, \glbyyyj);
\coordinate (glbpppnk) at (\glbxxxn, \glbyyyk);
\coordinate (glbpppnl) at (\glbxxxn, \glbyyyl);
\coordinate (glbpppnm) at (\glbxxxn, \glbyyym);
\coordinate (glbpppnn) at (\glbxxxn, \glbyyyn);
\coordinate (glbpppno) at (\glbxxxn, \glbyyyo);
\coordinate (glbpppnp) at (\glbxxxn, \glbyyyp);
\coordinate (glbpppnq) at (\glbxxxn, \glbyyyq);
\coordinate (glbpppnr) at (\glbxxxn, \glbyyyr);
\coordinate (glbpppns) at (\glbxxxn, \glbyyys);
\coordinate (glbpppnt) at (\glbxxxn, \glbyyyt);
\coordinate (glbpppnu) at (\glbxxxn, \glbyyyu);
\coordinate (glbpppnv) at (\glbxxxn, \glbyyyv);
\coordinate (glbpppnw) at (\glbxxxn, \glbyyyw);
\coordinate (glbpppnx) at (\glbxxxn, \glbyyyx);
\coordinate (glbpppny) at (\glbxxxn, \glbyyyy);
\coordinate (glbpppnz) at (\glbxxxn, \glbyyyz);
\coordinate (glbpppoa) at (\glbxxxo, \glbyyya);
\coordinate (glbpppob) at (\glbxxxo, \glbyyyb);
\coordinate (glbpppoc) at (\glbxxxo, \glbyyyc);
\coordinate (glbpppod) at (\glbxxxo, \glbyyyd);
\coordinate (glbpppoe) at (\glbxxxo, \glbyyye);
\coordinate (glbpppof) at (\glbxxxo, \glbyyyf);
\coordinate (glbpppog) at (\glbxxxo, \glbyyyg);
\coordinate (glbpppoh) at (\glbxxxo, \glbyyyh);
\coordinate (glbpppoi) at (\glbxxxo, \glbyyyi);
\coordinate (glbpppoj) at (\glbxxxo, \glbyyyj);
\coordinate (glbpppok) at (\glbxxxo, \glbyyyk);
\coordinate (glbpppol) at (\glbxxxo, \glbyyyl);
\coordinate (glbpppom) at (\glbxxxo, \glbyyym);
\coordinate (glbpppon) at (\glbxxxo, \glbyyyn);
\coordinate (glbpppoo) at (\glbxxxo, \glbyyyo);
\coordinate (glbpppop) at (\glbxxxo, \glbyyyp);
\coordinate (glbpppoq) at (\glbxxxo, \glbyyyq);
\coordinate (glbpppor) at (\glbxxxo, \glbyyyr);
\coordinate (glbpppos) at (\glbxxxo, \glbyyys);
\coordinate (glbpppot) at (\glbxxxo, \glbyyyt);
\coordinate (glbpppou) at (\glbxxxo, \glbyyyu);
\coordinate (glbpppov) at (\glbxxxo, \glbyyyv);
\coordinate (glbpppow) at (\glbxxxo, \glbyyyw);
\coordinate (glbpppox) at (\glbxxxo, \glbyyyx);
\coordinate (glbpppoy) at (\glbxxxo, \glbyyyy);
\coordinate (glbpppoz) at (\glbxxxo, \glbyyyz);
\coordinate (glbppppa) at (\glbxxxp, \glbyyya);
\coordinate (glbppppb) at (\glbxxxp, \glbyyyb);
\coordinate (glbppppc) at (\glbxxxp, \glbyyyc);
\coordinate (glbppppd) at (\glbxxxp, \glbyyyd);
\coordinate (glbppppe) at (\glbxxxp, \glbyyye);
\coordinate (glbppppf) at (\glbxxxp, \glbyyyf);
\coordinate (glbppppg) at (\glbxxxp, \glbyyyg);
\coordinate (glbpppph) at (\glbxxxp, \glbyyyh);
\coordinate (glbppppi) at (\glbxxxp, \glbyyyi);
\coordinate (glbppppj) at (\glbxxxp, \glbyyyj);
\coordinate (glbppppk) at (\glbxxxp, \glbyyyk);
\coordinate (glbppppl) at (\glbxxxp, \glbyyyl);
\coordinate (glbppppm) at (\glbxxxp, \glbyyym);
\coordinate (glbppppn) at (\glbxxxp, \glbyyyn);
\coordinate (glbppppo) at (\glbxxxp, \glbyyyo);
\coordinate (glbppppp) at (\glbxxxp, \glbyyyp);
\coordinate (glbppppq) at (\glbxxxp, \glbyyyq);
\coordinate (glbppppr) at (\glbxxxp, \glbyyyr);
\coordinate (glbpppps) at (\glbxxxp, \glbyyys);
\coordinate (glbppppt) at (\glbxxxp, \glbyyyt);
\coordinate (glbppppu) at (\glbxxxp, \glbyyyu);
\coordinate (glbppppv) at (\glbxxxp, \glbyyyv);
\coordinate (glbppppw) at (\glbxxxp, \glbyyyw);
\coordinate (glbppppx) at (\glbxxxp, \glbyyyx);
\coordinate (glbppppy) at (\glbxxxp, \glbyyyy);
\coordinate (glbppppz) at (\glbxxxp, \glbyyyz);
\coordinate (glbpppqa) at (\glbxxxq, \glbyyya);
\coordinate (glbpppqb) at (\glbxxxq, \glbyyyb);
\coordinate (glbpppqc) at (\glbxxxq, \glbyyyc);
\coordinate (glbpppqd) at (\glbxxxq, \glbyyyd);
\coordinate (glbpppqe) at (\glbxxxq, \glbyyye);
\coordinate (glbpppqf) at (\glbxxxq, \glbyyyf);
\coordinate (glbpppqg) at (\glbxxxq, \glbyyyg);
\coordinate (glbpppqh) at (\glbxxxq, \glbyyyh);
\coordinate (glbpppqi) at (\glbxxxq, \glbyyyi);
\coordinate (glbpppqj) at (\glbxxxq, \glbyyyj);
\coordinate (glbpppqk) at (\glbxxxq, \glbyyyk);
\coordinate (glbpppql) at (\glbxxxq, \glbyyyl);
\coordinate (glbpppqm) at (\glbxxxq, \glbyyym);
\coordinate (glbpppqn) at (\glbxxxq, \glbyyyn);
\coordinate (glbpppqo) at (\glbxxxq, \glbyyyo);
\coordinate (glbpppqp) at (\glbxxxq, \glbyyyp);
\coordinate (glbpppqq) at (\glbxxxq, \glbyyyq);
\coordinate (glbpppqr) at (\glbxxxq, \glbyyyr);
\coordinate (glbpppqs) at (\glbxxxq, \glbyyys);
\coordinate (glbpppqt) at (\glbxxxq, \glbyyyt);
\coordinate (glbpppqu) at (\glbxxxq, \glbyyyu);
\coordinate (glbpppqv) at (\glbxxxq, \glbyyyv);
\coordinate (glbpppqw) at (\glbxxxq, \glbyyyw);
\coordinate (glbpppqx) at (\glbxxxq, \glbyyyx);
\coordinate (glbpppqy) at (\glbxxxq, \glbyyyy);
\coordinate (glbpppqz) at (\glbxxxq, \glbyyyz);
\coordinate (glbpppra) at (\glbxxxr, \glbyyya);
\coordinate (glbppprb) at (\glbxxxr, \glbyyyb);
\coordinate (glbppprc) at (\glbxxxr, \glbyyyc);
\coordinate (glbppprd) at (\glbxxxr, \glbyyyd);
\coordinate (glbpppre) at (\glbxxxr, \glbyyye);
\coordinate (glbppprf) at (\glbxxxr, \glbyyyf);
\coordinate (glbppprg) at (\glbxxxr, \glbyyyg);
\coordinate (glbppprh) at (\glbxxxr, \glbyyyh);
\coordinate (glbpppri) at (\glbxxxr, \glbyyyi);
\coordinate (glbppprj) at (\glbxxxr, \glbyyyj);
\coordinate (glbppprk) at (\glbxxxr, \glbyyyk);
\coordinate (glbppprl) at (\glbxxxr, \glbyyyl);
\coordinate (glbppprm) at (\glbxxxr, \glbyyym);
\coordinate (glbppprn) at (\glbxxxr, \glbyyyn);
\coordinate (glbpppro) at (\glbxxxr, \glbyyyo);
\coordinate (glbppprp) at (\glbxxxr, \glbyyyp);
\coordinate (glbppprq) at (\glbxxxr, \glbyyyq);
\coordinate (glbppprr) at (\glbxxxr, \glbyyyr);
\coordinate (glbppprs) at (\glbxxxr, \glbyyys);
\coordinate (glbppprt) at (\glbxxxr, \glbyyyt);
\coordinate (glbpppru) at (\glbxxxr, \glbyyyu);
\coordinate (glbppprv) at (\glbxxxr, \glbyyyv);
\coordinate (glbppprw) at (\glbxxxr, \glbyyyw);
\coordinate (glbppprx) at (\glbxxxr, \glbyyyx);
\coordinate (glbpppry) at (\glbxxxr, \glbyyyy);
\coordinate (glbppprz) at (\glbxxxr, \glbyyyz);
\coordinate (glbpppsa) at (\glbxxxs, \glbyyya);
\coordinate (glbpppsb) at (\glbxxxs, \glbyyyb);
\coordinate (glbpppsc) at (\glbxxxs, \glbyyyc);
\coordinate (glbpppsd) at (\glbxxxs, \glbyyyd);
\coordinate (glbpppse) at (\glbxxxs, \glbyyye);
\coordinate (glbpppsf) at (\glbxxxs, \glbyyyf);
\coordinate (glbpppsg) at (\glbxxxs, \glbyyyg);
\coordinate (glbpppsh) at (\glbxxxs, \glbyyyh);
\coordinate (glbpppsi) at (\glbxxxs, \glbyyyi);
\coordinate (glbpppsj) at (\glbxxxs, \glbyyyj);
\coordinate (glbpppsk) at (\glbxxxs, \glbyyyk);
\coordinate (glbpppsl) at (\glbxxxs, \glbyyyl);
\coordinate (glbpppsm) at (\glbxxxs, \glbyyym);
\coordinate (glbpppsn) at (\glbxxxs, \glbyyyn);
\coordinate (glbpppso) at (\glbxxxs, \glbyyyo);
\coordinate (glbpppsp) at (\glbxxxs, \glbyyyp);
\coordinate (glbpppsq) at (\glbxxxs, \glbyyyq);
\coordinate (glbpppsr) at (\glbxxxs, \glbyyyr);
\coordinate (glbpppss) at (\glbxxxs, \glbyyys);
\coordinate (glbpppst) at (\glbxxxs, \glbyyyt);
\coordinate (glbpppsu) at (\glbxxxs, \glbyyyu);
\coordinate (glbpppsv) at (\glbxxxs, \glbyyyv);
\coordinate (glbpppsw) at (\glbxxxs, \glbyyyw);
\coordinate (glbpppsx) at (\glbxxxs, \glbyyyx);
\coordinate (glbpppsy) at (\glbxxxs, \glbyyyy);
\coordinate (glbpppsz) at (\glbxxxs, \glbyyyz);
\coordinate (glbpppta) at (\glbxxxt, \glbyyya);
\coordinate (glbppptb) at (\glbxxxt, \glbyyyb);
\coordinate (glbppptc) at (\glbxxxt, \glbyyyc);
\coordinate (glbppptd) at (\glbxxxt, \glbyyyd);
\coordinate (glbpppte) at (\glbxxxt, \glbyyye);
\coordinate (glbppptf) at (\glbxxxt, \glbyyyf);
\coordinate (glbppptg) at (\glbxxxt, \glbyyyg);
\coordinate (glbpppth) at (\glbxxxt, \glbyyyh);
\coordinate (glbpppti) at (\glbxxxt, \glbyyyi);
\coordinate (glbppptj) at (\glbxxxt, \glbyyyj);
\coordinate (glbppptk) at (\glbxxxt, \glbyyyk);
\coordinate (glbppptl) at (\glbxxxt, \glbyyyl);
\coordinate (glbppptm) at (\glbxxxt, \glbyyym);
\coordinate (glbppptn) at (\glbxxxt, \glbyyyn);
\coordinate (glbpppto) at (\glbxxxt, \glbyyyo);
\coordinate (glbppptp) at (\glbxxxt, \glbyyyp);
\coordinate (glbppptq) at (\glbxxxt, \glbyyyq);
\coordinate (glbppptr) at (\glbxxxt, \glbyyyr);
\coordinate (glbpppts) at (\glbxxxt, \glbyyys);
\coordinate (glbppptt) at (\glbxxxt, \glbyyyt);
\coordinate (glbppptu) at (\glbxxxt, \glbyyyu);
\coordinate (glbppptv) at (\glbxxxt, \glbyyyv);
\coordinate (glbppptw) at (\glbxxxt, \glbyyyw);
\coordinate (glbppptx) at (\glbxxxt, \glbyyyx);
\coordinate (glbpppty) at (\glbxxxt, \glbyyyy);
\coordinate (glbppptz) at (\glbxxxt, \glbyyyz);
\coordinate (glbpppua) at (\glbxxxu, \glbyyya);
\coordinate (glbpppub) at (\glbxxxu, \glbyyyb);
\coordinate (glbpppuc) at (\glbxxxu, \glbyyyc);
\coordinate (glbpppud) at (\glbxxxu, \glbyyyd);
\coordinate (glbpppue) at (\glbxxxu, \glbyyye);
\coordinate (glbpppuf) at (\glbxxxu, \glbyyyf);
\coordinate (glbpppug) at (\glbxxxu, \glbyyyg);
\coordinate (glbpppuh) at (\glbxxxu, \glbyyyh);
\coordinate (glbpppui) at (\glbxxxu, \glbyyyi);
\coordinate (glbpppuj) at (\glbxxxu, \glbyyyj);
\coordinate (glbpppuk) at (\glbxxxu, \glbyyyk);
\coordinate (glbpppul) at (\glbxxxu, \glbyyyl);
\coordinate (glbpppum) at (\glbxxxu, \glbyyym);
\coordinate (glbpppun) at (\glbxxxu, \glbyyyn);
\coordinate (glbpppuo) at (\glbxxxu, \glbyyyo);
\coordinate (glbpppup) at (\glbxxxu, \glbyyyp);
\coordinate (glbpppuq) at (\glbxxxu, \glbyyyq);
\coordinate (glbpppur) at (\glbxxxu, \glbyyyr);
\coordinate (glbpppus) at (\glbxxxu, \glbyyys);
\coordinate (glbppput) at (\glbxxxu, \glbyyyt);
\coordinate (glbpppuu) at (\glbxxxu, \glbyyyu);
\coordinate (glbpppuv) at (\glbxxxu, \glbyyyv);
\coordinate (glbpppuw) at (\glbxxxu, \glbyyyw);
\coordinate (glbpppux) at (\glbxxxu, \glbyyyx);
\coordinate (glbpppuy) at (\glbxxxu, \glbyyyy);
\coordinate (glbpppuz) at (\glbxxxu, \glbyyyz);
\coordinate (glbpppva) at (\glbxxxv, \glbyyya);
\coordinate (glbpppvb) at (\glbxxxv, \glbyyyb);
\coordinate (glbpppvc) at (\glbxxxv, \glbyyyc);
\coordinate (glbpppvd) at (\glbxxxv, \glbyyyd);
\coordinate (glbpppve) at (\glbxxxv, \glbyyye);
\coordinate (glbpppvf) at (\glbxxxv, \glbyyyf);
\coordinate (glbpppvg) at (\glbxxxv, \glbyyyg);
\coordinate (glbpppvh) at (\glbxxxv, \glbyyyh);
\coordinate (glbpppvi) at (\glbxxxv, \glbyyyi);
\coordinate (glbpppvj) at (\glbxxxv, \glbyyyj);
\coordinate (glbpppvk) at (\glbxxxv, \glbyyyk);
\coordinate (glbpppvl) at (\glbxxxv, \glbyyyl);
\coordinate (glbpppvm) at (\glbxxxv, \glbyyym);
\coordinate (glbpppvn) at (\glbxxxv, \glbyyyn);
\coordinate (glbpppvo) at (\glbxxxv, \glbyyyo);
\coordinate (glbpppvp) at (\glbxxxv, \glbyyyp);
\coordinate (glbpppvq) at (\glbxxxv, \glbyyyq);
\coordinate (glbpppvr) at (\glbxxxv, \glbyyyr);
\coordinate (glbpppvs) at (\glbxxxv, \glbyyys);
\coordinate (glbpppvt) at (\glbxxxv, \glbyyyt);
\coordinate (glbpppvu) at (\glbxxxv, \glbyyyu);
\coordinate (glbpppvv) at (\glbxxxv, \glbyyyv);
\coordinate (glbpppvw) at (\glbxxxv, \glbyyyw);
\coordinate (glbpppvx) at (\glbxxxv, \glbyyyx);
\coordinate (glbpppvy) at (\glbxxxv, \glbyyyy);
\coordinate (glbpppvz) at (\glbxxxv, \glbyyyz);
\coordinate (glbpppwa) at (\glbxxxw, \glbyyya);
\coordinate (glbpppwb) at (\glbxxxw, \glbyyyb);
\coordinate (glbpppwc) at (\glbxxxw, \glbyyyc);
\coordinate (glbpppwd) at (\glbxxxw, \glbyyyd);
\coordinate (glbpppwe) at (\glbxxxw, \glbyyye);
\coordinate (glbpppwf) at (\glbxxxw, \glbyyyf);
\coordinate (glbpppwg) at (\glbxxxw, \glbyyyg);
\coordinate (glbpppwh) at (\glbxxxw, \glbyyyh);
\coordinate (glbpppwi) at (\glbxxxw, \glbyyyi);
\coordinate (glbpppwj) at (\glbxxxw, \glbyyyj);
\coordinate (glbpppwk) at (\glbxxxw, \glbyyyk);
\coordinate (glbpppwl) at (\glbxxxw, \glbyyyl);
\coordinate (glbpppwm) at (\glbxxxw, \glbyyym);
\coordinate (glbpppwn) at (\glbxxxw, \glbyyyn);
\coordinate (glbpppwo) at (\glbxxxw, \glbyyyo);
\coordinate (glbpppwp) at (\glbxxxw, \glbyyyp);
\coordinate (glbpppwq) at (\glbxxxw, \glbyyyq);
\coordinate (glbpppwr) at (\glbxxxw, \glbyyyr);
\coordinate (glbpppws) at (\glbxxxw, \glbyyys);
\coordinate (glbpppwt) at (\glbxxxw, \glbyyyt);
\coordinate (glbpppwu) at (\glbxxxw, \glbyyyu);
\coordinate (glbpppwv) at (\glbxxxw, \glbyyyv);
\coordinate (glbpppww) at (\glbxxxw, \glbyyyw);
\coordinate (glbpppwx) at (\glbxxxw, \glbyyyx);
\coordinate (glbpppwy) at (\glbxxxw, \glbyyyy);
\coordinate (glbpppwz) at (\glbxxxw, \glbyyyz);
\coordinate (glbpppxa) at (\glbxxxx, \glbyyya);
\coordinate (glbpppxb) at (\glbxxxx, \glbyyyb);
\coordinate (glbpppxc) at (\glbxxxx, \glbyyyc);
\coordinate (glbpppxd) at (\glbxxxx, \glbyyyd);
\coordinate (glbpppxe) at (\glbxxxx, \glbyyye);
\coordinate (glbpppxf) at (\glbxxxx, \glbyyyf);
\coordinate (glbpppxg) at (\glbxxxx, \glbyyyg);
\coordinate (glbpppxh) at (\glbxxxx, \glbyyyh);
\coordinate (glbpppxi) at (\glbxxxx, \glbyyyi);
\coordinate (glbpppxj) at (\glbxxxx, \glbyyyj);
\coordinate (glbpppxk) at (\glbxxxx, \glbyyyk);
\coordinate (glbpppxl) at (\glbxxxx, \glbyyyl);
\coordinate (glbpppxm) at (\glbxxxx, \glbyyym);
\coordinate (glbpppxn) at (\glbxxxx, \glbyyyn);
\coordinate (glbpppxo) at (\glbxxxx, \glbyyyo);
\coordinate (glbpppxp) at (\glbxxxx, \glbyyyp);
\coordinate (glbpppxq) at (\glbxxxx, \glbyyyq);
\coordinate (glbpppxr) at (\glbxxxx, \glbyyyr);
\coordinate (glbpppxs) at (\glbxxxx, \glbyyys);
\coordinate (glbpppxt) at (\glbxxxx, \glbyyyt);
\coordinate (glbpppxu) at (\glbxxxx, \glbyyyu);
\coordinate (glbpppxv) at (\glbxxxx, \glbyyyv);
\coordinate (glbpppxw) at (\glbxxxx, \glbyyyw);
\coordinate (glbpppxx) at (\glbxxxx, \glbyyyx);
\coordinate (glbpppxy) at (\glbxxxx, \glbyyyy);
\coordinate (glbpppxz) at (\glbxxxx, \glbyyyz);
\coordinate (glbpppya) at (\glbxxxy, \glbyyya);
\coordinate (glbpppyb) at (\glbxxxy, \glbyyyb);
\coordinate (glbpppyc) at (\glbxxxy, \glbyyyc);
\coordinate (glbpppyd) at (\glbxxxy, \glbyyyd);
\coordinate (glbpppye) at (\glbxxxy, \glbyyye);
\coordinate (glbpppyf) at (\glbxxxy, \glbyyyf);
\coordinate (glbpppyg) at (\glbxxxy, \glbyyyg);
\coordinate (glbpppyh) at (\glbxxxy, \glbyyyh);
\coordinate (glbpppyi) at (\glbxxxy, \glbyyyi);
\coordinate (glbpppyj) at (\glbxxxy, \glbyyyj);
\coordinate (glbpppyk) at (\glbxxxy, \glbyyyk);
\coordinate (glbpppyl) at (\glbxxxy, \glbyyyl);
\coordinate (glbpppym) at (\glbxxxy, \glbyyym);
\coordinate (glbpppyn) at (\glbxxxy, \glbyyyn);
\coordinate (glbpppyo) at (\glbxxxy, \glbyyyo);
\coordinate (glbpppyp) at (\glbxxxy, \glbyyyp);
\coordinate (glbpppyq) at (\glbxxxy, \glbyyyq);
\coordinate (glbpppyr) at (\glbxxxy, \glbyyyr);
\coordinate (glbpppys) at (\glbxxxy, \glbyyys);
\coordinate (glbpppyt) at (\glbxxxy, \glbyyyt);
\coordinate (glbpppyu) at (\glbxxxy, \glbyyyu);
\coordinate (glbpppyv) at (\glbxxxy, \glbyyyv);
\coordinate (glbpppyw) at (\glbxxxy, \glbyyyw);
\coordinate (glbpppyx) at (\glbxxxy, \glbyyyx);
\coordinate (glbpppyy) at (\glbxxxy, \glbyyyy);
\coordinate (glbpppyz) at (\glbxxxy, \glbyyyz);
\coordinate (glbpppza) at (\glbxxxz, \glbyyya);
\coordinate (glbpppzb) at (\glbxxxz, \glbyyyb);
\coordinate (glbpppzc) at (\glbxxxz, \glbyyyc);
\coordinate (glbpppzd) at (\glbxxxz, \glbyyyd);
\coordinate (glbpppze) at (\glbxxxz, \glbyyye);
\coordinate (glbpppzf) at (\glbxxxz, \glbyyyf);
\coordinate (glbpppzg) at (\glbxxxz, \glbyyyg);
\coordinate (glbpppzh) at (\glbxxxz, \glbyyyh);
\coordinate (glbpppzi) at (\glbxxxz, \glbyyyi);
\coordinate (glbpppzj) at (\glbxxxz, \glbyyyj);
\coordinate (glbpppzk) at (\glbxxxz, \glbyyyk);
\coordinate (glbpppzl) at (\glbxxxz, \glbyyyl);
\coordinate (glbpppzm) at (\glbxxxz, \glbyyym);
\coordinate (glbpppzn) at (\glbxxxz, \glbyyyn);
\coordinate (glbpppzo) at (\glbxxxz, \glbyyyo);
\coordinate (glbpppzp) at (\glbxxxz, \glbyyyp);
\coordinate (glbpppzq) at (\glbxxxz, \glbyyyq);
\coordinate (glbpppzr) at (\glbxxxz, \glbyyyr);
\coordinate (glbpppzs) at (\glbxxxz, \glbyyys);
\coordinate (glbpppzt) at (\glbxxxz, \glbyyyt);
\coordinate (glbpppzu) at (\glbxxxz, \glbyyyu);
\coordinate (glbpppzv) at (\glbxxxz, \glbyyyv);
\coordinate (glbpppzw) at (\glbxxxz, \glbyyyw);
\coordinate (glbpppzx) at (\glbxxxz, \glbyyyx);
\coordinate (glbpppzy) at (\glbxxxz, \glbyyyy);
\coordinate (glbpppzz) at (\glbxxxz, \glbyyyz);

%\gangprintcoordinateat{(0,0)}{The last coordinate values: }{($(glbpppzz)$)}; 




% Draw related part of the coordinate system with dashed helplines with letters as background. 
%\coordinatebackgroundxy{gangliu}{b}{c}{v}{a}{b}{s};

\draw (gangliupppbr) -- (gangliupppsr);

\draw (gangliupppsr) 
      to [R = $R_4 \text{=} 56 \Omega$] 
      (gangliupppsp);
 
\draw (gangliupppsp) 
      to [full led = laser diode, label/align=rotate]
      (gangliupppsn) node [ground]{};


\node [nigfetd](nigfetd) at (gangliupppon) {F3055L};

\node [anchor=south] at (nigfetd.G) {G};
\node [anchor= west] at (nigfetd.D) {D};
\node [anchor= west] at (nigfetd.S) {S};

% To retrieve x- and y-component of the coordinates  (nigfetd.G), (nigfetd.D), and (nigfetd.S) separately. 
\getxyingivenunit{cm}{(nigfetd.G)}
                     {\nigfetdgx} {\nigfetdgy};
\getxyingivenunit{cm}{(nigfetd.D)}
                     {\nigfetddx} {\nigfetddy};
\getxyingivenunit{cm}{(nigfetd.S)}
                     {\nigfetdsx} {\nigfetdsy};

\draw  (gangliupppqr) -- 
       (gangliupppqq)
       to [Telmech=M1, n=motor]
       (\nigfetddx, \gangliuyyyq) --
       (nigfetd.D);

\node [xshift=-2mm] at (motor.block north east) {$-$};
\node [xshift= 2mm] at (motor.block south east) {$+$};

\draw  (gangliupppqq) -- 
       (gangliupppqp)
       to [full diode = 1N4001] 
       (\nigfetddx, \gangliuyyyp);

\draw  (nigfetd.S) -- 
       (\nigfetdsx, \gangliuyyyk)
          node [ground] {};

 
% To draw LM555
\draw [blue, line width=0.5mm] 
      (gangliuppphk) rectangle (gangliupppkq);
 
\node [blue, xshift=4mm] at (gangliupppio)
      {\underline{LM555}};

\draw (gangliupppir) -- 
      (gangliupppiq) node [anchor=north] {8};

\draw (gangliupppjr) -- 
      (gangliupppjq) node [anchor=north] {4};


\draw (nigfetd.G) 
      to [R, l_=$R_2 \text{=} 330 \Omega $] 
      (\gangliuxxxk, \nigfetdgy) 
      node [anchor=east] {3};
 
\draw (gangliupppkm) node [anchor=east] {1}  --
      (gangliupppom) ;

\draw (gangliupppkl) node [anchor=east] {5} 
      to [C, l_=$C_2  \text{=} 0.01 \mu F$] 
      (gangliupppnl) -- 
      (\nigfetdsx, \gangliuyyyl);


\draw (gangliupppdr) 
      to [R = $R_1 \text{=} 1k \Omega$] 
      (gangliupppdp) -- 
      (gangliuppphp) node [anchor=west] {7};
 
\draw (gangliupppdn) 
      to [full diode = 1N4001, label/align=rotate]
      (gangliupppdp);
 
\draw (gangliupppgp) 
      to [full diode = 1N4001, label/align=rotate]
      (gangliupppgn);
 

\draw (gangliupppgn) 
      to [potentiometer, l_=$R_3\text{=} 100k \Omega$,                                         n=mypot]
      (gangliupppdn);

\getxyingivenunit{cm}{(mypot.wiper)}
                     {\mypotwiperx}{\mypotwipery};


\draw (mypot.wiper) -- 
      (\mypotwiperx, \gangliuyyym) -- 
      (gangliuppphm) node [anchor=west] {6};

\draw  (\mypotwiperx, \gangliuyyym) -- 
       (\mypotwiperx, \gangliuyyyl) -- 
       (gangliuppphl) node [anchor=west] {2};
 

\draw  (\mypotwiperx, \gangliuyyyl) 
       to [C, n = capacitorl] 
       (\mypotwiperx, \gangliuyyyk) node[ground]{};

\node [anchor=north west, xshift=2mm, yshift=.7mm] 
      at (capacitorl) {$C_1 \text{=} 0.1 \mu F$};





%%%%%%%%%%-----------------------------------
%%%%%%%%%%-----------------------------------
%%%%%%%%  repeat here
%%%%%%%%%%-----------------------------------
%%%%%%%%%%-----------------------------------
% % https://github.com/LiuGangKingston/Nestable-coordinate-system-for-Tikz-circuits.git
% https://github.com/LiuGangKingston/Nestable-coordinate-system-for-Tikz-circuits.git


% https://github.com/LiuGangKingston/Nestable-coordinate-system-for-Tikz-circuits.git
% https://github.com/LiuGangKingston/Nestable-coordinate-system-for-Tikz-circuits.git


\pgfmathsetmacro{\totalgangliuxxx}{26}
\pgfmathsetmacro{\totalgangliuyyy}{26}
\pgfmathsetmacro{\gangliuxxxspacing}{1}
\pgfmathsetmacro{\gangliuyyyspacing}{1}
\pgfmathsetmacro{\gangliuxxxa}{-8}
\pgfmathsetmacro{\gangliuyyya}{-8}

\pgfmathsetmacro{\gangliuxxxb}{\gangliuxxxa + \gangliuxxxspacing + 0.0 }
\pgfmathsetmacro{\gangliuxxxc}{\gangliuxxxb + \gangliuxxxspacing + 0.0 }
\pgfmathsetmacro{\gangliuxxxd}{\gangliuxxxc + \gangliuxxxspacing + 0.0 }
\pgfmathsetmacro{\gangliuxxxe}{\gangliuxxxd + \gangliuxxxspacing + 0.0 }
\pgfmathsetmacro{\gangliuxxxf}{\gangliuxxxe + \gangliuxxxspacing + 0.0 }
\pgfmathsetmacro{\gangliuxxxg}{\gangliuxxxf + \gangliuxxxspacing + 0.0 }
\pgfmathsetmacro{\gangliuxxxh}{\gangliuxxxg + \gangliuxxxspacing + 0.0 }
\pgfmathsetmacro{\gangliuxxxi}{\gangliuxxxh + \gangliuxxxspacing + 0.0 }
\pgfmathsetmacro{\gangliuxxxj}{\gangliuxxxi + \gangliuxxxspacing + 0.0 }
\pgfmathsetmacro{\gangliuxxxk}{\gangliuxxxj + \gangliuxxxspacing + 0.0 }
\pgfmathsetmacro{\gangliuxxxl}{\gangliuxxxk + \gangliuxxxspacing + 0.0 }
\pgfmathsetmacro{\gangliuxxxm}{\gangliuxxxl + \gangliuxxxspacing + 0.0 }
\pgfmathsetmacro{\gangliuxxxn}{\gangliuxxxm + \gangliuxxxspacing + 0.0 }
\pgfmathsetmacro{\gangliuxxxo}{\gangliuxxxn + \gangliuxxxspacing + 0.0 }
\pgfmathsetmacro{\gangliuxxxp}{\gangliuxxxo + \gangliuxxxspacing + 0.0 }
\pgfmathsetmacro{\gangliuxxxq}{\gangliuxxxp + \gangliuxxxspacing + 0.0 }
\pgfmathsetmacro{\gangliuxxxr}{\gangliuxxxq + \gangliuxxxspacing + 0.0 }
\pgfmathsetmacro{\gangliuxxxs}{\gangliuxxxr + \gangliuxxxspacing + 0.0 }
\pgfmathsetmacro{\gangliuxxxt}{\gangliuxxxs + \gangliuxxxspacing + 0.0 }
\pgfmathsetmacro{\gangliuxxxu}{\gangliuxxxt + \gangliuxxxspacing + 0.0 }
\pgfmathsetmacro{\gangliuxxxv}{\gangliuxxxu + \gangliuxxxspacing + 0.0 }
\pgfmathsetmacro{\gangliuxxxw}{\gangliuxxxv + \gangliuxxxspacing + 0.0 }
\pgfmathsetmacro{\gangliuxxxx}{\gangliuxxxw + \gangliuxxxspacing + 0.0 }
\pgfmathsetmacro{\gangliuxxxy}{\gangliuxxxx + \gangliuxxxspacing + 0.0 }
\pgfmathsetmacro{\gangliuxxxz}{\gangliuxxxy + \gangliuxxxspacing + 0.0 }

\pgfmathsetmacro{\gangliuyyyb}{\gangliuyyya + \gangliuyyyspacing + 0.0 }
\pgfmathsetmacro{\gangliuyyyc}{\gangliuyyyb + \gangliuyyyspacing + 0.0 }
\pgfmathsetmacro{\gangliuyyyd}{\gangliuyyyc + \gangliuyyyspacing + 0.0 }
\pgfmathsetmacro{\gangliuyyye}{\gangliuyyyd + \gangliuyyyspacing + 0.0 }
\pgfmathsetmacro{\gangliuyyyf}{\gangliuyyye + \gangliuyyyspacing + 0.0 }
\pgfmathsetmacro{\gangliuyyyg}{\gangliuyyyf + \gangliuyyyspacing + 0.0 }
\pgfmathsetmacro{\gangliuyyyh}{\gangliuyyyg + \gangliuyyyspacing + 0.0 }
\pgfmathsetmacro{\gangliuyyyi}{\gangliuyyyh + \gangliuyyyspacing + 0.0 }
\pgfmathsetmacro{\gangliuyyyj}{\gangliuyyyi + \gangliuyyyspacing + 0.0 }
\pgfmathsetmacro{\gangliuyyyk}{\gangliuyyyj + \gangliuyyyspacing + 0.0 }
\pgfmathsetmacro{\gangliuyyyl}{\gangliuyyyk + \gangliuyyyspacing + 0.0 }
\pgfmathsetmacro{\gangliuyyym}{\gangliuyyyl + \gangliuyyyspacing + 0.0 }
\pgfmathsetmacro{\gangliuyyyn}{\gangliuyyym + \gangliuyyyspacing + 0.0 }
\pgfmathsetmacro{\gangliuyyyo}{\gangliuyyyn + \gangliuyyyspacing + 0.0 }
\pgfmathsetmacro{\gangliuyyyp}{\gangliuyyyo + \gangliuyyyspacing + 0.0 }
\pgfmathsetmacro{\gangliuyyyq}{\gangliuyyyp + \gangliuyyyspacing + 0.0 }
\pgfmathsetmacro{\gangliuyyyr}{\gangliuyyyq + \gangliuyyyspacing + 0.0 }
\pgfmathsetmacro{\gangliuyyys}{\gangliuyyyr + \gangliuyyyspacing + 0.0 }
\pgfmathsetmacro{\gangliuyyyt}{\gangliuyyys + \gangliuyyyspacing + 0.0 }
\pgfmathsetmacro{\gangliuyyyu}{\gangliuyyyt + \gangliuyyyspacing + 0.0 }
\pgfmathsetmacro{\gangliuyyyv}{\gangliuyyyu + \gangliuyyyspacing + 0.0 }
\pgfmathsetmacro{\gangliuyyyw}{\gangliuyyyv + \gangliuyyyspacing + 0.0 }
\pgfmathsetmacro{\gangliuyyyx}{\gangliuyyyw + \gangliuyyyspacing + 0.0 }
\pgfmathsetmacro{\gangliuyyyy}{\gangliuyyyx + \gangliuyyyspacing + 0.0 }
\pgfmathsetmacro{\gangliuyyyz}{\gangliuyyyy + \gangliuyyyspacing + 0.0 }

\coordinate (gangliupppaa) at (\gangliuxxxa, \gangliuyyya);
\coordinate (gangliupppab) at (\gangliuxxxa, \gangliuyyyb);
\coordinate (gangliupppac) at (\gangliuxxxa, \gangliuyyyc);
\coordinate (gangliupppad) at (\gangliuxxxa, \gangliuyyyd);
\coordinate (gangliupppae) at (\gangliuxxxa, \gangliuyyye);
\coordinate (gangliupppaf) at (\gangliuxxxa, \gangliuyyyf);
\coordinate (gangliupppag) at (\gangliuxxxa, \gangliuyyyg);
\coordinate (gangliupppah) at (\gangliuxxxa, \gangliuyyyh);
\coordinate (gangliupppai) at (\gangliuxxxa, \gangliuyyyi);
\coordinate (gangliupppaj) at (\gangliuxxxa, \gangliuyyyj);
\coordinate (gangliupppak) at (\gangliuxxxa, \gangliuyyyk);
\coordinate (gangliupppal) at (\gangliuxxxa, \gangliuyyyl);
\coordinate (gangliupppam) at (\gangliuxxxa, \gangliuyyym);
\coordinate (gangliupppan) at (\gangliuxxxa, \gangliuyyyn);
\coordinate (gangliupppao) at (\gangliuxxxa, \gangliuyyyo);
\coordinate (gangliupppap) at (\gangliuxxxa, \gangliuyyyp);
\coordinate (gangliupppaq) at (\gangliuxxxa, \gangliuyyyq);
\coordinate (gangliupppar) at (\gangliuxxxa, \gangliuyyyr);
\coordinate (gangliupppas) at (\gangliuxxxa, \gangliuyyys);
\coordinate (gangliupppat) at (\gangliuxxxa, \gangliuyyyt);
\coordinate (gangliupppau) at (\gangliuxxxa, \gangliuyyyu);
\coordinate (gangliupppav) at (\gangliuxxxa, \gangliuyyyv);
\coordinate (gangliupppaw) at (\gangliuxxxa, \gangliuyyyw);
\coordinate (gangliupppax) at (\gangliuxxxa, \gangliuyyyx);
\coordinate (gangliupppay) at (\gangliuxxxa, \gangliuyyyy);
\coordinate (gangliupppaz) at (\gangliuxxxa, \gangliuyyyz);
\coordinate (gangliupppba) at (\gangliuxxxb, \gangliuyyya);
\coordinate (gangliupppbb) at (\gangliuxxxb, \gangliuyyyb);
\coordinate (gangliupppbc) at (\gangliuxxxb, \gangliuyyyc);
\coordinate (gangliupppbd) at (\gangliuxxxb, \gangliuyyyd);
\coordinate (gangliupppbe) at (\gangliuxxxb, \gangliuyyye);
\coordinate (gangliupppbf) at (\gangliuxxxb, \gangliuyyyf);
\coordinate (gangliupppbg) at (\gangliuxxxb, \gangliuyyyg);
\coordinate (gangliupppbh) at (\gangliuxxxb, \gangliuyyyh);
\coordinate (gangliupppbi) at (\gangliuxxxb, \gangliuyyyi);
\coordinate (gangliupppbj) at (\gangliuxxxb, \gangliuyyyj);
\coordinate (gangliupppbk) at (\gangliuxxxb, \gangliuyyyk);
\coordinate (gangliupppbl) at (\gangliuxxxb, \gangliuyyyl);
\coordinate (gangliupppbm) at (\gangliuxxxb, \gangliuyyym);
\coordinate (gangliupppbn) at (\gangliuxxxb, \gangliuyyyn);
\coordinate (gangliupppbo) at (\gangliuxxxb, \gangliuyyyo);
\coordinate (gangliupppbp) at (\gangliuxxxb, \gangliuyyyp);
\coordinate (gangliupppbq) at (\gangliuxxxb, \gangliuyyyq);
\coordinate (gangliupppbr) at (\gangliuxxxb, \gangliuyyyr);
\coordinate (gangliupppbs) at (\gangliuxxxb, \gangliuyyys);
\coordinate (gangliupppbt) at (\gangliuxxxb, \gangliuyyyt);
\coordinate (gangliupppbu) at (\gangliuxxxb, \gangliuyyyu);
\coordinate (gangliupppbv) at (\gangliuxxxb, \gangliuyyyv);
\coordinate (gangliupppbw) at (\gangliuxxxb, \gangliuyyyw);
\coordinate (gangliupppbx) at (\gangliuxxxb, \gangliuyyyx);
\coordinate (gangliupppby) at (\gangliuxxxb, \gangliuyyyy);
\coordinate (gangliupppbz) at (\gangliuxxxb, \gangliuyyyz);
\coordinate (gangliupppca) at (\gangliuxxxc, \gangliuyyya);
\coordinate (gangliupppcb) at (\gangliuxxxc, \gangliuyyyb);
\coordinate (gangliupppcc) at (\gangliuxxxc, \gangliuyyyc);
\coordinate (gangliupppcd) at (\gangliuxxxc, \gangliuyyyd);
\coordinate (gangliupppce) at (\gangliuxxxc, \gangliuyyye);
\coordinate (gangliupppcf) at (\gangliuxxxc, \gangliuyyyf);
\coordinate (gangliupppcg) at (\gangliuxxxc, \gangliuyyyg);
\coordinate (gangliupppch) at (\gangliuxxxc, \gangliuyyyh);
\coordinate (gangliupppci) at (\gangliuxxxc, \gangliuyyyi);
\coordinate (gangliupppcj) at (\gangliuxxxc, \gangliuyyyj);
\coordinate (gangliupppck) at (\gangliuxxxc, \gangliuyyyk);
\coordinate (gangliupppcl) at (\gangliuxxxc, \gangliuyyyl);
\coordinate (gangliupppcm) at (\gangliuxxxc, \gangliuyyym);
\coordinate (gangliupppcn) at (\gangliuxxxc, \gangliuyyyn);
\coordinate (gangliupppco) at (\gangliuxxxc, \gangliuyyyo);
\coordinate (gangliupppcp) at (\gangliuxxxc, \gangliuyyyp);
\coordinate (gangliupppcq) at (\gangliuxxxc, \gangliuyyyq);
\coordinate (gangliupppcr) at (\gangliuxxxc, \gangliuyyyr);
\coordinate (gangliupppcs) at (\gangliuxxxc, \gangliuyyys);
\coordinate (gangliupppct) at (\gangliuxxxc, \gangliuyyyt);
\coordinate (gangliupppcu) at (\gangliuxxxc, \gangliuyyyu);
\coordinate (gangliupppcv) at (\gangliuxxxc, \gangliuyyyv);
\coordinate (gangliupppcw) at (\gangliuxxxc, \gangliuyyyw);
\coordinate (gangliupppcx) at (\gangliuxxxc, \gangliuyyyx);
\coordinate (gangliupppcy) at (\gangliuxxxc, \gangliuyyyy);
\coordinate (gangliupppcz) at (\gangliuxxxc, \gangliuyyyz);
\coordinate (gangliupppda) at (\gangliuxxxd, \gangliuyyya);
\coordinate (gangliupppdb) at (\gangliuxxxd, \gangliuyyyb);
\coordinate (gangliupppdc) at (\gangliuxxxd, \gangliuyyyc);
\coordinate (gangliupppdd) at (\gangliuxxxd, \gangliuyyyd);
\coordinate (gangliupppde) at (\gangliuxxxd, \gangliuyyye);
\coordinate (gangliupppdf) at (\gangliuxxxd, \gangliuyyyf);
\coordinate (gangliupppdg) at (\gangliuxxxd, \gangliuyyyg);
\coordinate (gangliupppdh) at (\gangliuxxxd, \gangliuyyyh);
\coordinate (gangliupppdi) at (\gangliuxxxd, \gangliuyyyi);
\coordinate (gangliupppdj) at (\gangliuxxxd, \gangliuyyyj);
\coordinate (gangliupppdk) at (\gangliuxxxd, \gangliuyyyk);
\coordinate (gangliupppdl) at (\gangliuxxxd, \gangliuyyyl);
\coordinate (gangliupppdm) at (\gangliuxxxd, \gangliuyyym);
\coordinate (gangliupppdn) at (\gangliuxxxd, \gangliuyyyn);
\coordinate (gangliupppdo) at (\gangliuxxxd, \gangliuyyyo);
\coordinate (gangliupppdp) at (\gangliuxxxd, \gangliuyyyp);
\coordinate (gangliupppdq) at (\gangliuxxxd, \gangliuyyyq);
\coordinate (gangliupppdr) at (\gangliuxxxd, \gangliuyyyr);
\coordinate (gangliupppds) at (\gangliuxxxd, \gangliuyyys);
\coordinate (gangliupppdt) at (\gangliuxxxd, \gangliuyyyt);
\coordinate (gangliupppdu) at (\gangliuxxxd, \gangliuyyyu);
\coordinate (gangliupppdv) at (\gangliuxxxd, \gangliuyyyv);
\coordinate (gangliupppdw) at (\gangliuxxxd, \gangliuyyyw);
\coordinate (gangliupppdx) at (\gangliuxxxd, \gangliuyyyx);
\coordinate (gangliupppdy) at (\gangliuxxxd, \gangliuyyyy);
\coordinate (gangliupppdz) at (\gangliuxxxd, \gangliuyyyz);
\coordinate (gangliupppea) at (\gangliuxxxe, \gangliuyyya);
\coordinate (gangliupppeb) at (\gangliuxxxe, \gangliuyyyb);
\coordinate (gangliupppec) at (\gangliuxxxe, \gangliuyyyc);
\coordinate (gangliuppped) at (\gangliuxxxe, \gangliuyyyd);
\coordinate (gangliupppee) at (\gangliuxxxe, \gangliuyyye);
\coordinate (gangliupppef) at (\gangliuxxxe, \gangliuyyyf);
\coordinate (gangliupppeg) at (\gangliuxxxe, \gangliuyyyg);
\coordinate (gangliupppeh) at (\gangliuxxxe, \gangliuyyyh);
\coordinate (gangliupppei) at (\gangliuxxxe, \gangliuyyyi);
\coordinate (gangliupppej) at (\gangliuxxxe, \gangliuyyyj);
\coordinate (gangliupppek) at (\gangliuxxxe, \gangliuyyyk);
\coordinate (gangliupppel) at (\gangliuxxxe, \gangliuyyyl);
\coordinate (gangliupppem) at (\gangliuxxxe, \gangliuyyym);
\coordinate (gangliupppen) at (\gangliuxxxe, \gangliuyyyn);
\coordinate (gangliupppeo) at (\gangliuxxxe, \gangliuyyyo);
\coordinate (gangliupppep) at (\gangliuxxxe, \gangliuyyyp);
\coordinate (gangliupppeq) at (\gangliuxxxe, \gangliuyyyq);
\coordinate (gangliuppper) at (\gangliuxxxe, \gangliuyyyr);
\coordinate (gangliupppes) at (\gangliuxxxe, \gangliuyyys);
\coordinate (gangliupppet) at (\gangliuxxxe, \gangliuyyyt);
\coordinate (gangliupppeu) at (\gangliuxxxe, \gangliuyyyu);
\coordinate (gangliupppev) at (\gangliuxxxe, \gangliuyyyv);
\coordinate (gangliupppew) at (\gangliuxxxe, \gangliuyyyw);
\coordinate (gangliupppex) at (\gangliuxxxe, \gangliuyyyx);
\coordinate (gangliupppey) at (\gangliuxxxe, \gangliuyyyy);
\coordinate (gangliupppez) at (\gangliuxxxe, \gangliuyyyz);
\coordinate (gangliupppfa) at (\gangliuxxxf, \gangliuyyya);
\coordinate (gangliupppfb) at (\gangliuxxxf, \gangliuyyyb);
\coordinate (gangliupppfc) at (\gangliuxxxf, \gangliuyyyc);
\coordinate (gangliupppfd) at (\gangliuxxxf, \gangliuyyyd);
\coordinate (gangliupppfe) at (\gangliuxxxf, \gangliuyyye);
\coordinate (gangliupppff) at (\gangliuxxxf, \gangliuyyyf);
\coordinate (gangliupppfg) at (\gangliuxxxf, \gangliuyyyg);
\coordinate (gangliupppfh) at (\gangliuxxxf, \gangliuyyyh);
\coordinate (gangliupppfi) at (\gangliuxxxf, \gangliuyyyi);
\coordinate (gangliupppfj) at (\gangliuxxxf, \gangliuyyyj);
\coordinate (gangliupppfk) at (\gangliuxxxf, \gangliuyyyk);
\coordinate (gangliupppfl) at (\gangliuxxxf, \gangliuyyyl);
\coordinate (gangliupppfm) at (\gangliuxxxf, \gangliuyyym);
\coordinate (gangliupppfn) at (\gangliuxxxf, \gangliuyyyn);
\coordinate (gangliupppfo) at (\gangliuxxxf, \gangliuyyyo);
\coordinate (gangliupppfp) at (\gangliuxxxf, \gangliuyyyp);
\coordinate (gangliupppfq) at (\gangliuxxxf, \gangliuyyyq);
\coordinate (gangliupppfr) at (\gangliuxxxf, \gangliuyyyr);
\coordinate (gangliupppfs) at (\gangliuxxxf, \gangliuyyys);
\coordinate (gangliupppft) at (\gangliuxxxf, \gangliuyyyt);
\coordinate (gangliupppfu) at (\gangliuxxxf, \gangliuyyyu);
\coordinate (gangliupppfv) at (\gangliuxxxf, \gangliuyyyv);
\coordinate (gangliupppfw) at (\gangliuxxxf, \gangliuyyyw);
\coordinate (gangliupppfx) at (\gangliuxxxf, \gangliuyyyx);
\coordinate (gangliupppfy) at (\gangliuxxxf, \gangliuyyyy);
\coordinate (gangliupppfz) at (\gangliuxxxf, \gangliuyyyz);
\coordinate (gangliupppga) at (\gangliuxxxg, \gangliuyyya);
\coordinate (gangliupppgb) at (\gangliuxxxg, \gangliuyyyb);
\coordinate (gangliupppgc) at (\gangliuxxxg, \gangliuyyyc);
\coordinate (gangliupppgd) at (\gangliuxxxg, \gangliuyyyd);
\coordinate (gangliupppge) at (\gangliuxxxg, \gangliuyyye);
\coordinate (gangliupppgf) at (\gangliuxxxg, \gangliuyyyf);
\coordinate (gangliupppgg) at (\gangliuxxxg, \gangliuyyyg);
\coordinate (gangliupppgh) at (\gangliuxxxg, \gangliuyyyh);
\coordinate (gangliupppgi) at (\gangliuxxxg, \gangliuyyyi);
\coordinate (gangliupppgj) at (\gangliuxxxg, \gangliuyyyj);
\coordinate (gangliupppgk) at (\gangliuxxxg, \gangliuyyyk);
\coordinate (gangliupppgl) at (\gangliuxxxg, \gangliuyyyl);
\coordinate (gangliupppgm) at (\gangliuxxxg, \gangliuyyym);
\coordinate (gangliupppgn) at (\gangliuxxxg, \gangliuyyyn);
\coordinate (gangliupppgo) at (\gangliuxxxg, \gangliuyyyo);
\coordinate (gangliupppgp) at (\gangliuxxxg, \gangliuyyyp);
\coordinate (gangliupppgq) at (\gangliuxxxg, \gangliuyyyq);
\coordinate (gangliupppgr) at (\gangliuxxxg, \gangliuyyyr);
\coordinate (gangliupppgs) at (\gangliuxxxg, \gangliuyyys);
\coordinate (gangliupppgt) at (\gangliuxxxg, \gangliuyyyt);
\coordinate (gangliupppgu) at (\gangliuxxxg, \gangliuyyyu);
\coordinate (gangliupppgv) at (\gangliuxxxg, \gangliuyyyv);
\coordinate (gangliupppgw) at (\gangliuxxxg, \gangliuyyyw);
\coordinate (gangliupppgx) at (\gangliuxxxg, \gangliuyyyx);
\coordinate (gangliupppgy) at (\gangliuxxxg, \gangliuyyyy);
\coordinate (gangliupppgz) at (\gangliuxxxg, \gangliuyyyz);
\coordinate (gangliupppha) at (\gangliuxxxh, \gangliuyyya);
\coordinate (gangliuppphb) at (\gangliuxxxh, \gangliuyyyb);
\coordinate (gangliuppphc) at (\gangliuxxxh, \gangliuyyyc);
\coordinate (gangliuppphd) at (\gangliuxxxh, \gangliuyyyd);
\coordinate (gangliuppphe) at (\gangliuxxxh, \gangliuyyye);
\coordinate (gangliuppphf) at (\gangliuxxxh, \gangliuyyyf);
\coordinate (gangliuppphg) at (\gangliuxxxh, \gangliuyyyg);
\coordinate (gangliuppphh) at (\gangliuxxxh, \gangliuyyyh);
\coordinate (gangliuppphi) at (\gangliuxxxh, \gangliuyyyi);
\coordinate (gangliuppphj) at (\gangliuxxxh, \gangliuyyyj);
\coordinate (gangliuppphk) at (\gangliuxxxh, \gangliuyyyk);
\coordinate (gangliuppphl) at (\gangliuxxxh, \gangliuyyyl);
\coordinate (gangliuppphm) at (\gangliuxxxh, \gangliuyyym);
\coordinate (gangliuppphn) at (\gangliuxxxh, \gangliuyyyn);
\coordinate (gangliupppho) at (\gangliuxxxh, \gangliuyyyo);
\coordinate (gangliuppphp) at (\gangliuxxxh, \gangliuyyyp);
\coordinate (gangliuppphq) at (\gangliuxxxh, \gangliuyyyq);
\coordinate (gangliuppphr) at (\gangliuxxxh, \gangliuyyyr);
\coordinate (gangliuppphs) at (\gangliuxxxh, \gangliuyyys);
\coordinate (gangliupppht) at (\gangliuxxxh, \gangliuyyyt);
\coordinate (gangliuppphu) at (\gangliuxxxh, \gangliuyyyu);
\coordinate (gangliuppphv) at (\gangliuxxxh, \gangliuyyyv);
\coordinate (gangliuppphw) at (\gangliuxxxh, \gangliuyyyw);
\coordinate (gangliuppphx) at (\gangliuxxxh, \gangliuyyyx);
\coordinate (gangliuppphy) at (\gangliuxxxh, \gangliuyyyy);
\coordinate (gangliuppphz) at (\gangliuxxxh, \gangliuyyyz);
\coordinate (gangliupppia) at (\gangliuxxxi, \gangliuyyya);
\coordinate (gangliupppib) at (\gangliuxxxi, \gangliuyyyb);
\coordinate (gangliupppic) at (\gangliuxxxi, \gangliuyyyc);
\coordinate (gangliupppid) at (\gangliuxxxi, \gangliuyyyd);
\coordinate (gangliupppie) at (\gangliuxxxi, \gangliuyyye);
\coordinate (gangliupppif) at (\gangliuxxxi, \gangliuyyyf);
\coordinate (gangliupppig) at (\gangliuxxxi, \gangliuyyyg);
\coordinate (gangliupppih) at (\gangliuxxxi, \gangliuyyyh);
\coordinate (gangliupppii) at (\gangliuxxxi, \gangliuyyyi);
\coordinate (gangliupppij) at (\gangliuxxxi, \gangliuyyyj);
\coordinate (gangliupppik) at (\gangliuxxxi, \gangliuyyyk);
\coordinate (gangliupppil) at (\gangliuxxxi, \gangliuyyyl);
\coordinate (gangliupppim) at (\gangliuxxxi, \gangliuyyym);
\coordinate (gangliupppin) at (\gangliuxxxi, \gangliuyyyn);
\coordinate (gangliupppio) at (\gangliuxxxi, \gangliuyyyo);
\coordinate (gangliupppip) at (\gangliuxxxi, \gangliuyyyp);
\coordinate (gangliupppiq) at (\gangliuxxxi, \gangliuyyyq);
\coordinate (gangliupppir) at (\gangliuxxxi, \gangliuyyyr);
\coordinate (gangliupppis) at (\gangliuxxxi, \gangliuyyys);
\coordinate (gangliupppit) at (\gangliuxxxi, \gangliuyyyt);
\coordinate (gangliupppiu) at (\gangliuxxxi, \gangliuyyyu);
\coordinate (gangliupppiv) at (\gangliuxxxi, \gangliuyyyv);
\coordinate (gangliupppiw) at (\gangliuxxxi, \gangliuyyyw);
\coordinate (gangliupppix) at (\gangliuxxxi, \gangliuyyyx);
\coordinate (gangliupppiy) at (\gangliuxxxi, \gangliuyyyy);
\coordinate (gangliupppiz) at (\gangliuxxxi, \gangliuyyyz);
\coordinate (gangliupppja) at (\gangliuxxxj, \gangliuyyya);
\coordinate (gangliupppjb) at (\gangliuxxxj, \gangliuyyyb);
\coordinate (gangliupppjc) at (\gangliuxxxj, \gangliuyyyc);
\coordinate (gangliupppjd) at (\gangliuxxxj, \gangliuyyyd);
\coordinate (gangliupppje) at (\gangliuxxxj, \gangliuyyye);
\coordinate (gangliupppjf) at (\gangliuxxxj, \gangliuyyyf);
\coordinate (gangliupppjg) at (\gangliuxxxj, \gangliuyyyg);
\coordinate (gangliupppjh) at (\gangliuxxxj, \gangliuyyyh);
\coordinate (gangliupppji) at (\gangliuxxxj, \gangliuyyyi);
\coordinate (gangliupppjj) at (\gangliuxxxj, \gangliuyyyj);
\coordinate (gangliupppjk) at (\gangliuxxxj, \gangliuyyyk);
\coordinate (gangliupppjl) at (\gangliuxxxj, \gangliuyyyl);
\coordinate (gangliupppjm) at (\gangliuxxxj, \gangliuyyym);
\coordinate (gangliupppjn) at (\gangliuxxxj, \gangliuyyyn);
\coordinate (gangliupppjo) at (\gangliuxxxj, \gangliuyyyo);
\coordinate (gangliupppjp) at (\gangliuxxxj, \gangliuyyyp);
\coordinate (gangliupppjq) at (\gangliuxxxj, \gangliuyyyq);
\coordinate (gangliupppjr) at (\gangliuxxxj, \gangliuyyyr);
\coordinate (gangliupppjs) at (\gangliuxxxj, \gangliuyyys);
\coordinate (gangliupppjt) at (\gangliuxxxj, \gangliuyyyt);
\coordinate (gangliupppju) at (\gangliuxxxj, \gangliuyyyu);
\coordinate (gangliupppjv) at (\gangliuxxxj, \gangliuyyyv);
\coordinate (gangliupppjw) at (\gangliuxxxj, \gangliuyyyw);
\coordinate (gangliupppjx) at (\gangliuxxxj, \gangliuyyyx);
\coordinate (gangliupppjy) at (\gangliuxxxj, \gangliuyyyy);
\coordinate (gangliupppjz) at (\gangliuxxxj, \gangliuyyyz);
\coordinate (gangliupppka) at (\gangliuxxxk, \gangliuyyya);
\coordinate (gangliupppkb) at (\gangliuxxxk, \gangliuyyyb);
\coordinate (gangliupppkc) at (\gangliuxxxk, \gangliuyyyc);
\coordinate (gangliupppkd) at (\gangliuxxxk, \gangliuyyyd);
\coordinate (gangliupppke) at (\gangliuxxxk, \gangliuyyye);
\coordinate (gangliupppkf) at (\gangliuxxxk, \gangliuyyyf);
\coordinate (gangliupppkg) at (\gangliuxxxk, \gangliuyyyg);
\coordinate (gangliupppkh) at (\gangliuxxxk, \gangliuyyyh);
\coordinate (gangliupppki) at (\gangliuxxxk, \gangliuyyyi);
\coordinate (gangliupppkj) at (\gangliuxxxk, \gangliuyyyj);
\coordinate (gangliupppkk) at (\gangliuxxxk, \gangliuyyyk);
\coordinate (gangliupppkl) at (\gangliuxxxk, \gangliuyyyl);
\coordinate (gangliupppkm) at (\gangliuxxxk, \gangliuyyym);
\coordinate (gangliupppkn) at (\gangliuxxxk, \gangliuyyyn);
\coordinate (gangliupppko) at (\gangliuxxxk, \gangliuyyyo);
\coordinate (gangliupppkp) at (\gangliuxxxk, \gangliuyyyp);
\coordinate (gangliupppkq) at (\gangliuxxxk, \gangliuyyyq);
\coordinate (gangliupppkr) at (\gangliuxxxk, \gangliuyyyr);
\coordinate (gangliupppks) at (\gangliuxxxk, \gangliuyyys);
\coordinate (gangliupppkt) at (\gangliuxxxk, \gangliuyyyt);
\coordinate (gangliupppku) at (\gangliuxxxk, \gangliuyyyu);
\coordinate (gangliupppkv) at (\gangliuxxxk, \gangliuyyyv);
\coordinate (gangliupppkw) at (\gangliuxxxk, \gangliuyyyw);
\coordinate (gangliupppkx) at (\gangliuxxxk, \gangliuyyyx);
\coordinate (gangliupppky) at (\gangliuxxxk, \gangliuyyyy);
\coordinate (gangliupppkz) at (\gangliuxxxk, \gangliuyyyz);
\coordinate (gangliupppla) at (\gangliuxxxl, \gangliuyyya);
\coordinate (gangliuppplb) at (\gangliuxxxl, \gangliuyyyb);
\coordinate (gangliuppplc) at (\gangliuxxxl, \gangliuyyyc);
\coordinate (gangliupppld) at (\gangliuxxxl, \gangliuyyyd);
\coordinate (gangliuppple) at (\gangliuxxxl, \gangliuyyye);
\coordinate (gangliuppplf) at (\gangliuxxxl, \gangliuyyyf);
\coordinate (gangliuppplg) at (\gangliuxxxl, \gangliuyyyg);
\coordinate (gangliuppplh) at (\gangliuxxxl, \gangliuyyyh);
\coordinate (gangliupppli) at (\gangliuxxxl, \gangliuyyyi);
\coordinate (gangliuppplj) at (\gangliuxxxl, \gangliuyyyj);
\coordinate (gangliuppplk) at (\gangliuxxxl, \gangliuyyyk);
\coordinate (gangliupppll) at (\gangliuxxxl, \gangliuyyyl);
\coordinate (gangliuppplm) at (\gangliuxxxl, \gangliuyyym);
\coordinate (gangliupppln) at (\gangliuxxxl, \gangliuyyyn);
\coordinate (gangliuppplo) at (\gangliuxxxl, \gangliuyyyo);
\coordinate (gangliuppplp) at (\gangliuxxxl, \gangliuyyyp);
\coordinate (gangliuppplq) at (\gangliuxxxl, \gangliuyyyq);
\coordinate (gangliuppplr) at (\gangliuxxxl, \gangliuyyyr);
\coordinate (gangliupppls) at (\gangliuxxxl, \gangliuyyys);
\coordinate (gangliuppplt) at (\gangliuxxxl, \gangliuyyyt);
\coordinate (gangliuppplu) at (\gangliuxxxl, \gangliuyyyu);
\coordinate (gangliuppplv) at (\gangliuxxxl, \gangliuyyyv);
\coordinate (gangliuppplw) at (\gangliuxxxl, \gangliuyyyw);
\coordinate (gangliuppplx) at (\gangliuxxxl, \gangliuyyyx);
\coordinate (gangliuppply) at (\gangliuxxxl, \gangliuyyyy);
\coordinate (gangliuppplz) at (\gangliuxxxl, \gangliuyyyz);
\coordinate (gangliupppma) at (\gangliuxxxm, \gangliuyyya);
\coordinate (gangliupppmb) at (\gangliuxxxm, \gangliuyyyb);
\coordinate (gangliupppmc) at (\gangliuxxxm, \gangliuyyyc);
\coordinate (gangliupppmd) at (\gangliuxxxm, \gangliuyyyd);
\coordinate (gangliupppme) at (\gangliuxxxm, \gangliuyyye);
\coordinate (gangliupppmf) at (\gangliuxxxm, \gangliuyyyf);
\coordinate (gangliupppmg) at (\gangliuxxxm, \gangliuyyyg);
\coordinate (gangliupppmh) at (\gangliuxxxm, \gangliuyyyh);
\coordinate (gangliupppmi) at (\gangliuxxxm, \gangliuyyyi);
\coordinate (gangliupppmj) at (\gangliuxxxm, \gangliuyyyj);
\coordinate (gangliupppmk) at (\gangliuxxxm, \gangliuyyyk);
\coordinate (gangliupppml) at (\gangliuxxxm, \gangliuyyyl);
\coordinate (gangliupppmm) at (\gangliuxxxm, \gangliuyyym);
\coordinate (gangliupppmn) at (\gangliuxxxm, \gangliuyyyn);
\coordinate (gangliupppmo) at (\gangliuxxxm, \gangliuyyyo);
\coordinate (gangliupppmp) at (\gangliuxxxm, \gangliuyyyp);
\coordinate (gangliupppmq) at (\gangliuxxxm, \gangliuyyyq);
\coordinate (gangliupppmr) at (\gangliuxxxm, \gangliuyyyr);
\coordinate (gangliupppms) at (\gangliuxxxm, \gangliuyyys);
\coordinate (gangliupppmt) at (\gangliuxxxm, \gangliuyyyt);
\coordinate (gangliupppmu) at (\gangliuxxxm, \gangliuyyyu);
\coordinate (gangliupppmv) at (\gangliuxxxm, \gangliuyyyv);
\coordinate (gangliupppmw) at (\gangliuxxxm, \gangliuyyyw);
\coordinate (gangliupppmx) at (\gangliuxxxm, \gangliuyyyx);
\coordinate (gangliupppmy) at (\gangliuxxxm, \gangliuyyyy);
\coordinate (gangliupppmz) at (\gangliuxxxm, \gangliuyyyz);
\coordinate (gangliupppna) at (\gangliuxxxn, \gangliuyyya);
\coordinate (gangliupppnb) at (\gangliuxxxn, \gangliuyyyb);
\coordinate (gangliupppnc) at (\gangliuxxxn, \gangliuyyyc);
\coordinate (gangliupppnd) at (\gangliuxxxn, \gangliuyyyd);
\coordinate (gangliupppne) at (\gangliuxxxn, \gangliuyyye);
\coordinate (gangliupppnf) at (\gangliuxxxn, \gangliuyyyf);
\coordinate (gangliupppng) at (\gangliuxxxn, \gangliuyyyg);
\coordinate (gangliupppnh) at (\gangliuxxxn, \gangliuyyyh);
\coordinate (gangliupppni) at (\gangliuxxxn, \gangliuyyyi);
\coordinate (gangliupppnj) at (\gangliuxxxn, \gangliuyyyj);
\coordinate (gangliupppnk) at (\gangliuxxxn, \gangliuyyyk);
\coordinate (gangliupppnl) at (\gangliuxxxn, \gangliuyyyl);
\coordinate (gangliupppnm) at (\gangliuxxxn, \gangliuyyym);
\coordinate (gangliupppnn) at (\gangliuxxxn, \gangliuyyyn);
\coordinate (gangliupppno) at (\gangliuxxxn, \gangliuyyyo);
\coordinate (gangliupppnp) at (\gangliuxxxn, \gangliuyyyp);
\coordinate (gangliupppnq) at (\gangliuxxxn, \gangliuyyyq);
\coordinate (gangliupppnr) at (\gangliuxxxn, \gangliuyyyr);
\coordinate (gangliupppns) at (\gangliuxxxn, \gangliuyyys);
\coordinate (gangliupppnt) at (\gangliuxxxn, \gangliuyyyt);
\coordinate (gangliupppnu) at (\gangliuxxxn, \gangliuyyyu);
\coordinate (gangliupppnv) at (\gangliuxxxn, \gangliuyyyv);
\coordinate (gangliupppnw) at (\gangliuxxxn, \gangliuyyyw);
\coordinate (gangliupppnx) at (\gangliuxxxn, \gangliuyyyx);
\coordinate (gangliupppny) at (\gangliuxxxn, \gangliuyyyy);
\coordinate (gangliupppnz) at (\gangliuxxxn, \gangliuyyyz);
\coordinate (gangliupppoa) at (\gangliuxxxo, \gangliuyyya);
\coordinate (gangliupppob) at (\gangliuxxxo, \gangliuyyyb);
\coordinate (gangliupppoc) at (\gangliuxxxo, \gangliuyyyc);
\coordinate (gangliupppod) at (\gangliuxxxo, \gangliuyyyd);
\coordinate (gangliupppoe) at (\gangliuxxxo, \gangliuyyye);
\coordinate (gangliupppof) at (\gangliuxxxo, \gangliuyyyf);
\coordinate (gangliupppog) at (\gangliuxxxo, \gangliuyyyg);
\coordinate (gangliupppoh) at (\gangliuxxxo, \gangliuyyyh);
\coordinate (gangliupppoi) at (\gangliuxxxo, \gangliuyyyi);
\coordinate (gangliupppoj) at (\gangliuxxxo, \gangliuyyyj);
\coordinate (gangliupppok) at (\gangliuxxxo, \gangliuyyyk);
\coordinate (gangliupppol) at (\gangliuxxxo, \gangliuyyyl);
\coordinate (gangliupppom) at (\gangliuxxxo, \gangliuyyym);
\coordinate (gangliupppon) at (\gangliuxxxo, \gangliuyyyn);
\coordinate (gangliupppoo) at (\gangliuxxxo, \gangliuyyyo);
\coordinate (gangliupppop) at (\gangliuxxxo, \gangliuyyyp);
\coordinate (gangliupppoq) at (\gangliuxxxo, \gangliuyyyq);
\coordinate (gangliupppor) at (\gangliuxxxo, \gangliuyyyr);
\coordinate (gangliupppos) at (\gangliuxxxo, \gangliuyyys);
\coordinate (gangliupppot) at (\gangliuxxxo, \gangliuyyyt);
\coordinate (gangliupppou) at (\gangliuxxxo, \gangliuyyyu);
\coordinate (gangliupppov) at (\gangliuxxxo, \gangliuyyyv);
\coordinate (gangliupppow) at (\gangliuxxxo, \gangliuyyyw);
\coordinate (gangliupppox) at (\gangliuxxxo, \gangliuyyyx);
\coordinate (gangliupppoy) at (\gangliuxxxo, \gangliuyyyy);
\coordinate (gangliupppoz) at (\gangliuxxxo, \gangliuyyyz);
\coordinate (gangliuppppa) at (\gangliuxxxp, \gangliuyyya);
\coordinate (gangliuppppb) at (\gangliuxxxp, \gangliuyyyb);
\coordinate (gangliuppppc) at (\gangliuxxxp, \gangliuyyyc);
\coordinate (gangliuppppd) at (\gangliuxxxp, \gangliuyyyd);
\coordinate (gangliuppppe) at (\gangliuxxxp, \gangliuyyye);
\coordinate (gangliuppppf) at (\gangliuxxxp, \gangliuyyyf);
\coordinate (gangliuppppg) at (\gangliuxxxp, \gangliuyyyg);
\coordinate (gangliupppph) at (\gangliuxxxp, \gangliuyyyh);
\coordinate (gangliuppppi) at (\gangliuxxxp, \gangliuyyyi);
\coordinate (gangliuppppj) at (\gangliuxxxp, \gangliuyyyj);
\coordinate (gangliuppppk) at (\gangliuxxxp, \gangliuyyyk);
\coordinate (gangliuppppl) at (\gangliuxxxp, \gangliuyyyl);
\coordinate (gangliuppppm) at (\gangliuxxxp, \gangliuyyym);
\coordinate (gangliuppppn) at (\gangliuxxxp, \gangliuyyyn);
\coordinate (gangliuppppo) at (\gangliuxxxp, \gangliuyyyo);
\coordinate (gangliuppppp) at (\gangliuxxxp, \gangliuyyyp);
\coordinate (gangliuppppq) at (\gangliuxxxp, \gangliuyyyq);
\coordinate (gangliuppppr) at (\gangliuxxxp, \gangliuyyyr);
\coordinate (gangliupppps) at (\gangliuxxxp, \gangliuyyys);
\coordinate (gangliuppppt) at (\gangliuxxxp, \gangliuyyyt);
\coordinate (gangliuppppu) at (\gangliuxxxp, \gangliuyyyu);
\coordinate (gangliuppppv) at (\gangliuxxxp, \gangliuyyyv);
\coordinate (gangliuppppw) at (\gangliuxxxp, \gangliuyyyw);
\coordinate (gangliuppppx) at (\gangliuxxxp, \gangliuyyyx);
\coordinate (gangliuppppy) at (\gangliuxxxp, \gangliuyyyy);
\coordinate (gangliuppppz) at (\gangliuxxxp, \gangliuyyyz);
\coordinate (gangliupppqa) at (\gangliuxxxq, \gangliuyyya);
\coordinate (gangliupppqb) at (\gangliuxxxq, \gangliuyyyb);
\coordinate (gangliupppqc) at (\gangliuxxxq, \gangliuyyyc);
\coordinate (gangliupppqd) at (\gangliuxxxq, \gangliuyyyd);
\coordinate (gangliupppqe) at (\gangliuxxxq, \gangliuyyye);
\coordinate (gangliupppqf) at (\gangliuxxxq, \gangliuyyyf);
\coordinate (gangliupppqg) at (\gangliuxxxq, \gangliuyyyg);
\coordinate (gangliupppqh) at (\gangliuxxxq, \gangliuyyyh);
\coordinate (gangliupppqi) at (\gangliuxxxq, \gangliuyyyi);
\coordinate (gangliupppqj) at (\gangliuxxxq, \gangliuyyyj);
\coordinate (gangliupppqk) at (\gangliuxxxq, \gangliuyyyk);
\coordinate (gangliupppql) at (\gangliuxxxq, \gangliuyyyl);
\coordinate (gangliupppqm) at (\gangliuxxxq, \gangliuyyym);
\coordinate (gangliupppqn) at (\gangliuxxxq, \gangliuyyyn);
\coordinate (gangliupppqo) at (\gangliuxxxq, \gangliuyyyo);
\coordinate (gangliupppqp) at (\gangliuxxxq, \gangliuyyyp);
\coordinate (gangliupppqq) at (\gangliuxxxq, \gangliuyyyq);
\coordinate (gangliupppqr) at (\gangliuxxxq, \gangliuyyyr);
\coordinate (gangliupppqs) at (\gangliuxxxq, \gangliuyyys);
\coordinate (gangliupppqt) at (\gangliuxxxq, \gangliuyyyt);
\coordinate (gangliupppqu) at (\gangliuxxxq, \gangliuyyyu);
\coordinate (gangliupppqv) at (\gangliuxxxq, \gangliuyyyv);
\coordinate (gangliupppqw) at (\gangliuxxxq, \gangliuyyyw);
\coordinate (gangliupppqx) at (\gangliuxxxq, \gangliuyyyx);
\coordinate (gangliupppqy) at (\gangliuxxxq, \gangliuyyyy);
\coordinate (gangliupppqz) at (\gangliuxxxq, \gangliuyyyz);
\coordinate (gangliupppra) at (\gangliuxxxr, \gangliuyyya);
\coordinate (gangliuppprb) at (\gangliuxxxr, \gangliuyyyb);
\coordinate (gangliuppprc) at (\gangliuxxxr, \gangliuyyyc);
\coordinate (gangliuppprd) at (\gangliuxxxr, \gangliuyyyd);
\coordinate (gangliupppre) at (\gangliuxxxr, \gangliuyyye);
\coordinate (gangliuppprf) at (\gangliuxxxr, \gangliuyyyf);
\coordinate (gangliuppprg) at (\gangliuxxxr, \gangliuyyyg);
\coordinate (gangliuppprh) at (\gangliuxxxr, \gangliuyyyh);
\coordinate (gangliupppri) at (\gangliuxxxr, \gangliuyyyi);
\coordinate (gangliuppprj) at (\gangliuxxxr, \gangliuyyyj);
\coordinate (gangliuppprk) at (\gangliuxxxr, \gangliuyyyk);
\coordinate (gangliuppprl) at (\gangliuxxxr, \gangliuyyyl);
\coordinate (gangliuppprm) at (\gangliuxxxr, \gangliuyyym);
\coordinate (gangliuppprn) at (\gangliuxxxr, \gangliuyyyn);
\coordinate (gangliupppro) at (\gangliuxxxr, \gangliuyyyo);
\coordinate (gangliuppprp) at (\gangliuxxxr, \gangliuyyyp);
\coordinate (gangliuppprq) at (\gangliuxxxr, \gangliuyyyq);
\coordinate (gangliuppprr) at (\gangliuxxxr, \gangliuyyyr);
\coordinate (gangliuppprs) at (\gangliuxxxr, \gangliuyyys);
\coordinate (gangliuppprt) at (\gangliuxxxr, \gangliuyyyt);
\coordinate (gangliupppru) at (\gangliuxxxr, \gangliuyyyu);
\coordinate (gangliuppprv) at (\gangliuxxxr, \gangliuyyyv);
\coordinate (gangliuppprw) at (\gangliuxxxr, \gangliuyyyw);
\coordinate (gangliuppprx) at (\gangliuxxxr, \gangliuyyyx);
\coordinate (gangliupppry) at (\gangliuxxxr, \gangliuyyyy);
\coordinate (gangliuppprz) at (\gangliuxxxr, \gangliuyyyz);
\coordinate (gangliupppsa) at (\gangliuxxxs, \gangliuyyya);
\coordinate (gangliupppsb) at (\gangliuxxxs, \gangliuyyyb);
\coordinate (gangliupppsc) at (\gangliuxxxs, \gangliuyyyc);
\coordinate (gangliupppsd) at (\gangliuxxxs, \gangliuyyyd);
\coordinate (gangliupppse) at (\gangliuxxxs, \gangliuyyye);
\coordinate (gangliupppsf) at (\gangliuxxxs, \gangliuyyyf);
\coordinate (gangliupppsg) at (\gangliuxxxs, \gangliuyyyg);
\coordinate (gangliupppsh) at (\gangliuxxxs, \gangliuyyyh);
\coordinate (gangliupppsi) at (\gangliuxxxs, \gangliuyyyi);
\coordinate (gangliupppsj) at (\gangliuxxxs, \gangliuyyyj);
\coordinate (gangliupppsk) at (\gangliuxxxs, \gangliuyyyk);
\coordinate (gangliupppsl) at (\gangliuxxxs, \gangliuyyyl);
\coordinate (gangliupppsm) at (\gangliuxxxs, \gangliuyyym);
\coordinate (gangliupppsn) at (\gangliuxxxs, \gangliuyyyn);
\coordinate (gangliupppso) at (\gangliuxxxs, \gangliuyyyo);
\coordinate (gangliupppsp) at (\gangliuxxxs, \gangliuyyyp);
\coordinate (gangliupppsq) at (\gangliuxxxs, \gangliuyyyq);
\coordinate (gangliupppsr) at (\gangliuxxxs, \gangliuyyyr);
\coordinate (gangliupppss) at (\gangliuxxxs, \gangliuyyys);
\coordinate (gangliupppst) at (\gangliuxxxs, \gangliuyyyt);
\coordinate (gangliupppsu) at (\gangliuxxxs, \gangliuyyyu);
\coordinate (gangliupppsv) at (\gangliuxxxs, \gangliuyyyv);
\coordinate (gangliupppsw) at (\gangliuxxxs, \gangliuyyyw);
\coordinate (gangliupppsx) at (\gangliuxxxs, \gangliuyyyx);
\coordinate (gangliupppsy) at (\gangliuxxxs, \gangliuyyyy);
\coordinate (gangliupppsz) at (\gangliuxxxs, \gangliuyyyz);
\coordinate (gangliupppta) at (\gangliuxxxt, \gangliuyyya);
\coordinate (gangliuppptb) at (\gangliuxxxt, \gangliuyyyb);
\coordinate (gangliuppptc) at (\gangliuxxxt, \gangliuyyyc);
\coordinate (gangliuppptd) at (\gangliuxxxt, \gangliuyyyd);
\coordinate (gangliupppte) at (\gangliuxxxt, \gangliuyyye);
\coordinate (gangliuppptf) at (\gangliuxxxt, \gangliuyyyf);
\coordinate (gangliuppptg) at (\gangliuxxxt, \gangliuyyyg);
\coordinate (gangliupppth) at (\gangliuxxxt, \gangliuyyyh);
\coordinate (gangliupppti) at (\gangliuxxxt, \gangliuyyyi);
\coordinate (gangliuppptj) at (\gangliuxxxt, \gangliuyyyj);
\coordinate (gangliuppptk) at (\gangliuxxxt, \gangliuyyyk);
\coordinate (gangliuppptl) at (\gangliuxxxt, \gangliuyyyl);
\coordinate (gangliuppptm) at (\gangliuxxxt, \gangliuyyym);
\coordinate (gangliuppptn) at (\gangliuxxxt, \gangliuyyyn);
\coordinate (gangliupppto) at (\gangliuxxxt, \gangliuyyyo);
\coordinate (gangliuppptp) at (\gangliuxxxt, \gangliuyyyp);
\coordinate (gangliuppptq) at (\gangliuxxxt, \gangliuyyyq);
\coordinate (gangliuppptr) at (\gangliuxxxt, \gangliuyyyr);
\coordinate (gangliupppts) at (\gangliuxxxt, \gangliuyyys);
\coordinate (gangliuppptt) at (\gangliuxxxt, \gangliuyyyt);
\coordinate (gangliuppptu) at (\gangliuxxxt, \gangliuyyyu);
\coordinate (gangliuppptv) at (\gangliuxxxt, \gangliuyyyv);
\coordinate (gangliuppptw) at (\gangliuxxxt, \gangliuyyyw);
\coordinate (gangliuppptx) at (\gangliuxxxt, \gangliuyyyx);
\coordinate (gangliupppty) at (\gangliuxxxt, \gangliuyyyy);
\coordinate (gangliuppptz) at (\gangliuxxxt, \gangliuyyyz);
\coordinate (gangliupppua) at (\gangliuxxxu, \gangliuyyya);
\coordinate (gangliupppub) at (\gangliuxxxu, \gangliuyyyb);
\coordinate (gangliupppuc) at (\gangliuxxxu, \gangliuyyyc);
\coordinate (gangliupppud) at (\gangliuxxxu, \gangliuyyyd);
\coordinate (gangliupppue) at (\gangliuxxxu, \gangliuyyye);
\coordinate (gangliupppuf) at (\gangliuxxxu, \gangliuyyyf);
\coordinate (gangliupppug) at (\gangliuxxxu, \gangliuyyyg);
\coordinate (gangliupppuh) at (\gangliuxxxu, \gangliuyyyh);
\coordinate (gangliupppui) at (\gangliuxxxu, \gangliuyyyi);
\coordinate (gangliupppuj) at (\gangliuxxxu, \gangliuyyyj);
\coordinate (gangliupppuk) at (\gangliuxxxu, \gangliuyyyk);
\coordinate (gangliupppul) at (\gangliuxxxu, \gangliuyyyl);
\coordinate (gangliupppum) at (\gangliuxxxu, \gangliuyyym);
\coordinate (gangliupppun) at (\gangliuxxxu, \gangliuyyyn);
\coordinate (gangliupppuo) at (\gangliuxxxu, \gangliuyyyo);
\coordinate (gangliupppup) at (\gangliuxxxu, \gangliuyyyp);
\coordinate (gangliupppuq) at (\gangliuxxxu, \gangliuyyyq);
\coordinate (gangliupppur) at (\gangliuxxxu, \gangliuyyyr);
\coordinate (gangliupppus) at (\gangliuxxxu, \gangliuyyys);
\coordinate (gangliuppput) at (\gangliuxxxu, \gangliuyyyt);
\coordinate (gangliupppuu) at (\gangliuxxxu, \gangliuyyyu);
\coordinate (gangliupppuv) at (\gangliuxxxu, \gangliuyyyv);
\coordinate (gangliupppuw) at (\gangliuxxxu, \gangliuyyyw);
\coordinate (gangliupppux) at (\gangliuxxxu, \gangliuyyyx);
\coordinate (gangliupppuy) at (\gangliuxxxu, \gangliuyyyy);
\coordinate (gangliupppuz) at (\gangliuxxxu, \gangliuyyyz);
\coordinate (gangliupppva) at (\gangliuxxxv, \gangliuyyya);
\coordinate (gangliupppvb) at (\gangliuxxxv, \gangliuyyyb);
\coordinate (gangliupppvc) at (\gangliuxxxv, \gangliuyyyc);
\coordinate (gangliupppvd) at (\gangliuxxxv, \gangliuyyyd);
\coordinate (gangliupppve) at (\gangliuxxxv, \gangliuyyye);
\coordinate (gangliupppvf) at (\gangliuxxxv, \gangliuyyyf);
\coordinate (gangliupppvg) at (\gangliuxxxv, \gangliuyyyg);
\coordinate (gangliupppvh) at (\gangliuxxxv, \gangliuyyyh);
\coordinate (gangliupppvi) at (\gangliuxxxv, \gangliuyyyi);
\coordinate (gangliupppvj) at (\gangliuxxxv, \gangliuyyyj);
\coordinate (gangliupppvk) at (\gangliuxxxv, \gangliuyyyk);
\coordinate (gangliupppvl) at (\gangliuxxxv, \gangliuyyyl);
\coordinate (gangliupppvm) at (\gangliuxxxv, \gangliuyyym);
\coordinate (gangliupppvn) at (\gangliuxxxv, \gangliuyyyn);
\coordinate (gangliupppvo) at (\gangliuxxxv, \gangliuyyyo);
\coordinate (gangliupppvp) at (\gangliuxxxv, \gangliuyyyp);
\coordinate (gangliupppvq) at (\gangliuxxxv, \gangliuyyyq);
\coordinate (gangliupppvr) at (\gangliuxxxv, \gangliuyyyr);
\coordinate (gangliupppvs) at (\gangliuxxxv, \gangliuyyys);
\coordinate (gangliupppvt) at (\gangliuxxxv, \gangliuyyyt);
\coordinate (gangliupppvu) at (\gangliuxxxv, \gangliuyyyu);
\coordinate (gangliupppvv) at (\gangliuxxxv, \gangliuyyyv);
\coordinate (gangliupppvw) at (\gangliuxxxv, \gangliuyyyw);
\coordinate (gangliupppvx) at (\gangliuxxxv, \gangliuyyyx);
\coordinate (gangliupppvy) at (\gangliuxxxv, \gangliuyyyy);
\coordinate (gangliupppvz) at (\gangliuxxxv, \gangliuyyyz);
\coordinate (gangliupppwa) at (\gangliuxxxw, \gangliuyyya);
\coordinate (gangliupppwb) at (\gangliuxxxw, \gangliuyyyb);
\coordinate (gangliupppwc) at (\gangliuxxxw, \gangliuyyyc);
\coordinate (gangliupppwd) at (\gangliuxxxw, \gangliuyyyd);
\coordinate (gangliupppwe) at (\gangliuxxxw, \gangliuyyye);
\coordinate (gangliupppwf) at (\gangliuxxxw, \gangliuyyyf);
\coordinate (gangliupppwg) at (\gangliuxxxw, \gangliuyyyg);
\coordinate (gangliupppwh) at (\gangliuxxxw, \gangliuyyyh);
\coordinate (gangliupppwi) at (\gangliuxxxw, \gangliuyyyi);
\coordinate (gangliupppwj) at (\gangliuxxxw, \gangliuyyyj);
\coordinate (gangliupppwk) at (\gangliuxxxw, \gangliuyyyk);
\coordinate (gangliupppwl) at (\gangliuxxxw, \gangliuyyyl);
\coordinate (gangliupppwm) at (\gangliuxxxw, \gangliuyyym);
\coordinate (gangliupppwn) at (\gangliuxxxw, \gangliuyyyn);
\coordinate (gangliupppwo) at (\gangliuxxxw, \gangliuyyyo);
\coordinate (gangliupppwp) at (\gangliuxxxw, \gangliuyyyp);
\coordinate (gangliupppwq) at (\gangliuxxxw, \gangliuyyyq);
\coordinate (gangliupppwr) at (\gangliuxxxw, \gangliuyyyr);
\coordinate (gangliupppws) at (\gangliuxxxw, \gangliuyyys);
\coordinate (gangliupppwt) at (\gangliuxxxw, \gangliuyyyt);
\coordinate (gangliupppwu) at (\gangliuxxxw, \gangliuyyyu);
\coordinate (gangliupppwv) at (\gangliuxxxw, \gangliuyyyv);
\coordinate (gangliupppww) at (\gangliuxxxw, \gangliuyyyw);
\coordinate (gangliupppwx) at (\gangliuxxxw, \gangliuyyyx);
\coordinate (gangliupppwy) at (\gangliuxxxw, \gangliuyyyy);
\coordinate (gangliupppwz) at (\gangliuxxxw, \gangliuyyyz);
\coordinate (gangliupppxa) at (\gangliuxxxx, \gangliuyyya);
\coordinate (gangliupppxb) at (\gangliuxxxx, \gangliuyyyb);
\coordinate (gangliupppxc) at (\gangliuxxxx, \gangliuyyyc);
\coordinate (gangliupppxd) at (\gangliuxxxx, \gangliuyyyd);
\coordinate (gangliupppxe) at (\gangliuxxxx, \gangliuyyye);
\coordinate (gangliupppxf) at (\gangliuxxxx, \gangliuyyyf);
\coordinate (gangliupppxg) at (\gangliuxxxx, \gangliuyyyg);
\coordinate (gangliupppxh) at (\gangliuxxxx, \gangliuyyyh);
\coordinate (gangliupppxi) at (\gangliuxxxx, \gangliuyyyi);
\coordinate (gangliupppxj) at (\gangliuxxxx, \gangliuyyyj);
\coordinate (gangliupppxk) at (\gangliuxxxx, \gangliuyyyk);
\coordinate (gangliupppxl) at (\gangliuxxxx, \gangliuyyyl);
\coordinate (gangliupppxm) at (\gangliuxxxx, \gangliuyyym);
\coordinate (gangliupppxn) at (\gangliuxxxx, \gangliuyyyn);
\coordinate (gangliupppxo) at (\gangliuxxxx, \gangliuyyyo);
\coordinate (gangliupppxp) at (\gangliuxxxx, \gangliuyyyp);
\coordinate (gangliupppxq) at (\gangliuxxxx, \gangliuyyyq);
\coordinate (gangliupppxr) at (\gangliuxxxx, \gangliuyyyr);
\coordinate (gangliupppxs) at (\gangliuxxxx, \gangliuyyys);
\coordinate (gangliupppxt) at (\gangliuxxxx, \gangliuyyyt);
\coordinate (gangliupppxu) at (\gangliuxxxx, \gangliuyyyu);
\coordinate (gangliupppxv) at (\gangliuxxxx, \gangliuyyyv);
\coordinate (gangliupppxw) at (\gangliuxxxx, \gangliuyyyw);
\coordinate (gangliupppxx) at (\gangliuxxxx, \gangliuyyyx);
\coordinate (gangliupppxy) at (\gangliuxxxx, \gangliuyyyy);
\coordinate (gangliupppxz) at (\gangliuxxxx, \gangliuyyyz);
\coordinate (gangliupppya) at (\gangliuxxxy, \gangliuyyya);
\coordinate (gangliupppyb) at (\gangliuxxxy, \gangliuyyyb);
\coordinate (gangliupppyc) at (\gangliuxxxy, \gangliuyyyc);
\coordinate (gangliupppyd) at (\gangliuxxxy, \gangliuyyyd);
\coordinate (gangliupppye) at (\gangliuxxxy, \gangliuyyye);
\coordinate (gangliupppyf) at (\gangliuxxxy, \gangliuyyyf);
\coordinate (gangliupppyg) at (\gangliuxxxy, \gangliuyyyg);
\coordinate (gangliupppyh) at (\gangliuxxxy, \gangliuyyyh);
\coordinate (gangliupppyi) at (\gangliuxxxy, \gangliuyyyi);
\coordinate (gangliupppyj) at (\gangliuxxxy, \gangliuyyyj);
\coordinate (gangliupppyk) at (\gangliuxxxy, \gangliuyyyk);
\coordinate (gangliupppyl) at (\gangliuxxxy, \gangliuyyyl);
\coordinate (gangliupppym) at (\gangliuxxxy, \gangliuyyym);
\coordinate (gangliupppyn) at (\gangliuxxxy, \gangliuyyyn);
\coordinate (gangliupppyo) at (\gangliuxxxy, \gangliuyyyo);
\coordinate (gangliupppyp) at (\gangliuxxxy, \gangliuyyyp);
\coordinate (gangliupppyq) at (\gangliuxxxy, \gangliuyyyq);
\coordinate (gangliupppyr) at (\gangliuxxxy, \gangliuyyyr);
\coordinate (gangliupppys) at (\gangliuxxxy, \gangliuyyys);
\coordinate (gangliupppyt) at (\gangliuxxxy, \gangliuyyyt);
\coordinate (gangliupppyu) at (\gangliuxxxy, \gangliuyyyu);
\coordinate (gangliupppyv) at (\gangliuxxxy, \gangliuyyyv);
\coordinate (gangliupppyw) at (\gangliuxxxy, \gangliuyyyw);
\coordinate (gangliupppyx) at (\gangliuxxxy, \gangliuyyyx);
\coordinate (gangliupppyy) at (\gangliuxxxy, \gangliuyyyy);
\coordinate (gangliupppyz) at (\gangliuxxxy, \gangliuyyyz);
\coordinate (gangliupppza) at (\gangliuxxxz, \gangliuyyya);
\coordinate (gangliupppzb) at (\gangliuxxxz, \gangliuyyyb);
\coordinate (gangliupppzc) at (\gangliuxxxz, \gangliuyyyc);
\coordinate (gangliupppzd) at (\gangliuxxxz, \gangliuyyyd);
\coordinate (gangliupppze) at (\gangliuxxxz, \gangliuyyye);
\coordinate (gangliupppzf) at (\gangliuxxxz, \gangliuyyyf);
\coordinate (gangliupppzg) at (\gangliuxxxz, \gangliuyyyg);
\coordinate (gangliupppzh) at (\gangliuxxxz, \gangliuyyyh);
\coordinate (gangliupppzi) at (\gangliuxxxz, \gangliuyyyi);
\coordinate (gangliupppzj) at (\gangliuxxxz, \gangliuyyyj);
\coordinate (gangliupppzk) at (\gangliuxxxz, \gangliuyyyk);
\coordinate (gangliupppzl) at (\gangliuxxxz, \gangliuyyyl);
\coordinate (gangliupppzm) at (\gangliuxxxz, \gangliuyyym);
\coordinate (gangliupppzn) at (\gangliuxxxz, \gangliuyyyn);
\coordinate (gangliupppzo) at (\gangliuxxxz, \gangliuyyyo);
\coordinate (gangliupppzp) at (\gangliuxxxz, \gangliuyyyp);
\coordinate (gangliupppzq) at (\gangliuxxxz, \gangliuyyyq);
\coordinate (gangliupppzr) at (\gangliuxxxz, \gangliuyyyr);
\coordinate (gangliupppzs) at (\gangliuxxxz, \gangliuyyys);
\coordinate (gangliupppzt) at (\gangliuxxxz, \gangliuyyyt);
\coordinate (gangliupppzu) at (\gangliuxxxz, \gangliuyyyu);
\coordinate (gangliupppzv) at (\gangliuxxxz, \gangliuyyyv);
\coordinate (gangliupppzw) at (\gangliuxxxz, \gangliuyyyw);
\coordinate (gangliupppzx) at (\gangliuxxxz, \gangliuyyyx);
\coordinate (gangliupppzy) at (\gangliuxxxz, \gangliuyyyy);
\coordinate (gangliupppzz) at (\gangliuxxxz, \gangliuyyyz);

%\gangprintcoordinateat{(0,0)}{The last coordinate values: }{($(gangliupppzz)$)}; 



\pgfmathsetmacro{\totalglaxxx}{26}
\pgfmathsetmacro{\totalglayyy}{26}
\pgfmathsetmacro{\glaxxxspacing}{1}
\pgfmathsetmacro{\glayyyspacing}{1}
\pgfmathsetmacro{\glaxxxa}{-8}
\pgfmathsetmacro{\glayyya}{\gangliuxxxa -8.0}

\pgfmathsetmacro{\glaxxxb}{\glaxxxa + \glaxxxspacing + 0.0 }
\pgfmathsetmacro{\glaxxxc}{\glaxxxb + \glaxxxspacing + 0.0 }
\pgfmathsetmacro{\glaxxxd}{\glaxxxc + \glaxxxspacing + 0.0 }
\pgfmathsetmacro{\glaxxxe}{\glaxxxd + \glaxxxspacing + 0.0 }
\pgfmathsetmacro{\glaxxxf}{\glaxxxe + \glaxxxspacing + 0.0 }
\pgfmathsetmacro{\glaxxxg}{\glaxxxf + \glaxxxspacing + 0.0 }
\pgfmathsetmacro{\glaxxxh}{\glaxxxg + \glaxxxspacing + 0.0 }
\pgfmathsetmacro{\glaxxxi}{\glaxxxh + \glaxxxspacing + 0.0 }
\pgfmathsetmacro{\glaxxxj}{\glaxxxi + \glaxxxspacing + 0.0 }
\pgfmathsetmacro{\glaxxxk}{\glaxxxj + \glaxxxspacing + 0.0 }
\pgfmathsetmacro{\glaxxxl}{\glaxxxk + \glaxxxspacing + 0.0 }
\pgfmathsetmacro{\glaxxxm}{\glaxxxl + \glaxxxspacing + 0.0 }
\pgfmathsetmacro{\glaxxxn}{\glaxxxm + \glaxxxspacing + 0.0 }
\pgfmathsetmacro{\glaxxxo}{\glaxxxn + \glaxxxspacing + 0.0 }
\pgfmathsetmacro{\glaxxxp}{\glaxxxo + \glaxxxspacing + 0.0 }
\pgfmathsetmacro{\glaxxxq}{\glaxxxp + \glaxxxspacing + 0.0 }
\pgfmathsetmacro{\glaxxxr}{\glaxxxq + \glaxxxspacing + 0.0 }
\pgfmathsetmacro{\glaxxxs}{\glaxxxr + \glaxxxspacing + 0.0 }
\pgfmathsetmacro{\glaxxxt}{\glaxxxs + \glaxxxspacing + 0.0 }
\pgfmathsetmacro{\glaxxxu}{\glaxxxt + \glaxxxspacing + 0.0 }
\pgfmathsetmacro{\glaxxxv}{\glaxxxu + \glaxxxspacing + 0.0 }
\pgfmathsetmacro{\glaxxxw}{\glaxxxv + \glaxxxspacing + 0.0 }
\pgfmathsetmacro{\glaxxxx}{\glaxxxw + \glaxxxspacing + 0.0 }
\pgfmathsetmacro{\glaxxxy}{\glaxxxx + \glaxxxspacing + 0.0 }
\pgfmathsetmacro{\glaxxxz}{\glaxxxy + \glaxxxspacing + 0.0 }

\pgfmathsetmacro{\glayyyb}{\glayyya + \glayyyspacing + 0.0 }
\pgfmathsetmacro{\glayyyc}{\glayyyb + \glayyyspacing + 0.0 }
\pgfmathsetmacro{\glayyyd}{\glayyyc + \glayyyspacing + 0.0 }
\pgfmathsetmacro{\glayyye}{\glayyyd + \glayyyspacing + 0.0 }
\pgfmathsetmacro{\glayyyf}{\glayyye + \glayyyspacing + 0.0 }
\pgfmathsetmacro{\glayyyg}{\glayyyf + \glayyyspacing + 0.0 }
\pgfmathsetmacro{\glayyyh}{\glayyyg + \glayyyspacing + 0.0 }
\pgfmathsetmacro{\glayyyi}{\glayyyh + \glayyyspacing + 0.0 }
\pgfmathsetmacro{\glayyyj}{\glayyyi + \glayyyspacing + 0.0 }
\pgfmathsetmacro{\glayyyk}{\glayyyj + \glayyyspacing + 0.0 }
\pgfmathsetmacro{\glayyyl}{\glayyyk + \glayyyspacing + 0.0 }
\pgfmathsetmacro{\glayyym}{\glayyyl + \glayyyspacing + 0.0 }
\pgfmathsetmacro{\glayyyn}{\glayyym + \glayyyspacing + 0.0 }
\pgfmathsetmacro{\glayyyo}{\glayyyn + \glayyyspacing + 0.0 }
\pgfmathsetmacro{\glayyyp}{\glayyyo + \glayyyspacing + 0.0 }
\pgfmathsetmacro{\glayyyq}{\glayyyp + \glayyyspacing + 0.0 }
\pgfmathsetmacro{\glayyyr}{\glayyyq + \glayyyspacing + 0.0 }
\pgfmathsetmacro{\glayyys}{\glayyyr + \glayyyspacing + 0.0 }
\pgfmathsetmacro{\glayyyt}{\glayyys + \glayyyspacing + 0.0 }
\pgfmathsetmacro{\glayyyu}{\glayyyt + \glayyyspacing + 0.0 }
\pgfmathsetmacro{\glayyyv}{\glayyyu + \glayyyspacing + 0.0 }
\pgfmathsetmacro{\glayyyw}{\glayyyv + \glayyyspacing + 0.0 }
\pgfmathsetmacro{\glayyyx}{\glayyyw + \glayyyspacing + 0.0 }
\pgfmathsetmacro{\glayyyy}{\glayyyx + \glayyyspacing + 0.0 }
\pgfmathsetmacro{\glayyyz}{\glayyyy + \glayyyspacing + 0.0 }

\coordinate (glapppaa) at (\glaxxxa, \glayyya);
\coordinate (glapppab) at (\glaxxxa, \glayyyb);
\coordinate (glapppac) at (\glaxxxa, \glayyyc);
\coordinate (glapppad) at (\glaxxxa, \glayyyd);
\coordinate (glapppae) at (\glaxxxa, \glayyye);
\coordinate (glapppaf) at (\glaxxxa, \glayyyf);
\coordinate (glapppag) at (\glaxxxa, \glayyyg);
\coordinate (glapppah) at (\glaxxxa, \glayyyh);
\coordinate (glapppai) at (\glaxxxa, \glayyyi);
\coordinate (glapppaj) at (\glaxxxa, \glayyyj);
\coordinate (glapppak) at (\glaxxxa, \glayyyk);
\coordinate (glapppal) at (\glaxxxa, \glayyyl);
\coordinate (glapppam) at (\glaxxxa, \glayyym);
\coordinate (glapppan) at (\glaxxxa, \glayyyn);
\coordinate (glapppao) at (\glaxxxa, \glayyyo);
\coordinate (glapppap) at (\glaxxxa, \glayyyp);
\coordinate (glapppaq) at (\glaxxxa, \glayyyq);
\coordinate (glapppar) at (\glaxxxa, \glayyyr);
\coordinate (glapppas) at (\glaxxxa, \glayyys);
\coordinate (glapppat) at (\glaxxxa, \glayyyt);
\coordinate (glapppau) at (\glaxxxa, \glayyyu);
\coordinate (glapppav) at (\glaxxxa, \glayyyv);
\coordinate (glapppaw) at (\glaxxxa, \glayyyw);
\coordinate (glapppax) at (\glaxxxa, \glayyyx);
\coordinate (glapppay) at (\glaxxxa, \glayyyy);
\coordinate (glapppaz) at (\glaxxxa, \glayyyz);
\coordinate (glapppba) at (\glaxxxb, \glayyya);
\coordinate (glapppbb) at (\glaxxxb, \glayyyb);
\coordinate (glapppbc) at (\glaxxxb, \glayyyc);
\coordinate (glapppbd) at (\glaxxxb, \glayyyd);
\coordinate (glapppbe) at (\glaxxxb, \glayyye);
\coordinate (glapppbf) at (\glaxxxb, \glayyyf);
\coordinate (glapppbg) at (\glaxxxb, \glayyyg);
\coordinate (glapppbh) at (\glaxxxb, \glayyyh);
\coordinate (glapppbi) at (\glaxxxb, \glayyyi);
\coordinate (glapppbj) at (\glaxxxb, \glayyyj);
\coordinate (glapppbk) at (\glaxxxb, \glayyyk);
\coordinate (glapppbl) at (\glaxxxb, \glayyyl);
\coordinate (glapppbm) at (\glaxxxb, \glayyym);
\coordinate (glapppbn) at (\glaxxxb, \glayyyn);
\coordinate (glapppbo) at (\glaxxxb, \glayyyo);
\coordinate (glapppbp) at (\glaxxxb, \glayyyp);
\coordinate (glapppbq) at (\glaxxxb, \glayyyq);
\coordinate (glapppbr) at (\glaxxxb, \glayyyr);
\coordinate (glapppbs) at (\glaxxxb, \glayyys);
\coordinate (glapppbt) at (\glaxxxb, \glayyyt);
\coordinate (glapppbu) at (\glaxxxb, \glayyyu);
\coordinate (glapppbv) at (\glaxxxb, \glayyyv);
\coordinate (glapppbw) at (\glaxxxb, \glayyyw);
\coordinate (glapppbx) at (\glaxxxb, \glayyyx);
\coordinate (glapppby) at (\glaxxxb, \glayyyy);
\coordinate (glapppbz) at (\glaxxxb, \glayyyz);
\coordinate (glapppca) at (\glaxxxc, \glayyya);
\coordinate (glapppcb) at (\glaxxxc, \glayyyb);
\coordinate (glapppcc) at (\glaxxxc, \glayyyc);
\coordinate (glapppcd) at (\glaxxxc, \glayyyd);
\coordinate (glapppce) at (\glaxxxc, \glayyye);
\coordinate (glapppcf) at (\glaxxxc, \glayyyf);
\coordinate (glapppcg) at (\glaxxxc, \glayyyg);
\coordinate (glapppch) at (\glaxxxc, \glayyyh);
\coordinate (glapppci) at (\glaxxxc, \glayyyi);
\coordinate (glapppcj) at (\glaxxxc, \glayyyj);
\coordinate (glapppck) at (\glaxxxc, \glayyyk);
\coordinate (glapppcl) at (\glaxxxc, \glayyyl);
\coordinate (glapppcm) at (\glaxxxc, \glayyym);
\coordinate (glapppcn) at (\glaxxxc, \glayyyn);
\coordinate (glapppco) at (\glaxxxc, \glayyyo);
\coordinate (glapppcp) at (\glaxxxc, \glayyyp);
\coordinate (glapppcq) at (\glaxxxc, \glayyyq);
\coordinate (glapppcr) at (\glaxxxc, \glayyyr);
\coordinate (glapppcs) at (\glaxxxc, \glayyys);
\coordinate (glapppct) at (\glaxxxc, \glayyyt);
\coordinate (glapppcu) at (\glaxxxc, \glayyyu);
\coordinate (glapppcv) at (\glaxxxc, \glayyyv);
\coordinate (glapppcw) at (\glaxxxc, \glayyyw);
\coordinate (glapppcx) at (\glaxxxc, \glayyyx);
\coordinate (glapppcy) at (\glaxxxc, \glayyyy);
\coordinate (glapppcz) at (\glaxxxc, \glayyyz);
\coordinate (glapppda) at (\glaxxxd, \glayyya);
\coordinate (glapppdb) at (\glaxxxd, \glayyyb);
\coordinate (glapppdc) at (\glaxxxd, \glayyyc);
\coordinate (glapppdd) at (\glaxxxd, \glayyyd);
\coordinate (glapppde) at (\glaxxxd, \glayyye);
\coordinate (glapppdf) at (\glaxxxd, \glayyyf);
\coordinate (glapppdg) at (\glaxxxd, \glayyyg);
\coordinate (glapppdh) at (\glaxxxd, \glayyyh);
\coordinate (glapppdi) at (\glaxxxd, \glayyyi);
\coordinate (glapppdj) at (\glaxxxd, \glayyyj);
\coordinate (glapppdk) at (\glaxxxd, \glayyyk);
\coordinate (glapppdl) at (\glaxxxd, \glayyyl);
\coordinate (glapppdm) at (\glaxxxd, \glayyym);
\coordinate (glapppdn) at (\glaxxxd, \glayyyn);
\coordinate (glapppdo) at (\glaxxxd, \glayyyo);
\coordinate (glapppdp) at (\glaxxxd, \glayyyp);
\coordinate (glapppdq) at (\glaxxxd, \glayyyq);
\coordinate (glapppdr) at (\glaxxxd, \glayyyr);
\coordinate (glapppds) at (\glaxxxd, \glayyys);
\coordinate (glapppdt) at (\glaxxxd, \glayyyt);
\coordinate (glapppdu) at (\glaxxxd, \glayyyu);
\coordinate (glapppdv) at (\glaxxxd, \glayyyv);
\coordinate (glapppdw) at (\glaxxxd, \glayyyw);
\coordinate (glapppdx) at (\glaxxxd, \glayyyx);
\coordinate (glapppdy) at (\glaxxxd, \glayyyy);
\coordinate (glapppdz) at (\glaxxxd, \glayyyz);
\coordinate (glapppea) at (\glaxxxe, \glayyya);
\coordinate (glapppeb) at (\glaxxxe, \glayyyb);
\coordinate (glapppec) at (\glaxxxe, \glayyyc);
\coordinate (glappped) at (\glaxxxe, \glayyyd);
\coordinate (glapppee) at (\glaxxxe, \glayyye);
\coordinate (glapppef) at (\glaxxxe, \glayyyf);
\coordinate (glapppeg) at (\glaxxxe, \glayyyg);
\coordinate (glapppeh) at (\glaxxxe, \glayyyh);
\coordinate (glapppei) at (\glaxxxe, \glayyyi);
\coordinate (glapppej) at (\glaxxxe, \glayyyj);
\coordinate (glapppek) at (\glaxxxe, \glayyyk);
\coordinate (glapppel) at (\glaxxxe, \glayyyl);
\coordinate (glapppem) at (\glaxxxe, \glayyym);
\coordinate (glapppen) at (\glaxxxe, \glayyyn);
\coordinate (glapppeo) at (\glaxxxe, \glayyyo);
\coordinate (glapppep) at (\glaxxxe, \glayyyp);
\coordinate (glapppeq) at (\glaxxxe, \glayyyq);
\coordinate (glappper) at (\glaxxxe, \glayyyr);
\coordinate (glapppes) at (\glaxxxe, \glayyys);
\coordinate (glapppet) at (\glaxxxe, \glayyyt);
\coordinate (glapppeu) at (\glaxxxe, \glayyyu);
\coordinate (glapppev) at (\glaxxxe, \glayyyv);
\coordinate (glapppew) at (\glaxxxe, \glayyyw);
\coordinate (glapppex) at (\glaxxxe, \glayyyx);
\coordinate (glapppey) at (\glaxxxe, \glayyyy);
\coordinate (glapppez) at (\glaxxxe, \glayyyz);
\coordinate (glapppfa) at (\glaxxxf, \glayyya);
\coordinate (glapppfb) at (\glaxxxf, \glayyyb);
\coordinate (glapppfc) at (\glaxxxf, \glayyyc);
\coordinate (glapppfd) at (\glaxxxf, \glayyyd);
\coordinate (glapppfe) at (\glaxxxf, \glayyye);
\coordinate (glapppff) at (\glaxxxf, \glayyyf);
\coordinate (glapppfg) at (\glaxxxf, \glayyyg);
\coordinate (glapppfh) at (\glaxxxf, \glayyyh);
\coordinate (glapppfi) at (\glaxxxf, \glayyyi);
\coordinate (glapppfj) at (\glaxxxf, \glayyyj);
\coordinate (glapppfk) at (\glaxxxf, \glayyyk);
\coordinate (glapppfl) at (\glaxxxf, \glayyyl);
\coordinate (glapppfm) at (\glaxxxf, \glayyym);
\coordinate (glapppfn) at (\glaxxxf, \glayyyn);
\coordinate (glapppfo) at (\glaxxxf, \glayyyo);
\coordinate (glapppfp) at (\glaxxxf, \glayyyp);
\coordinate (glapppfq) at (\glaxxxf, \glayyyq);
\coordinate (glapppfr) at (\glaxxxf, \glayyyr);
\coordinate (glapppfs) at (\glaxxxf, \glayyys);
\coordinate (glapppft) at (\glaxxxf, \glayyyt);
\coordinate (glapppfu) at (\glaxxxf, \glayyyu);
\coordinate (glapppfv) at (\glaxxxf, \glayyyv);
\coordinate (glapppfw) at (\glaxxxf, \glayyyw);
\coordinate (glapppfx) at (\glaxxxf, \glayyyx);
\coordinate (glapppfy) at (\glaxxxf, \glayyyy);
\coordinate (glapppfz) at (\glaxxxf, \glayyyz);
\coordinate (glapppga) at (\glaxxxg, \glayyya);
\coordinate (glapppgb) at (\glaxxxg, \glayyyb);
\coordinate (glapppgc) at (\glaxxxg, \glayyyc);
\coordinate (glapppgd) at (\glaxxxg, \glayyyd);
\coordinate (glapppge) at (\glaxxxg, \glayyye);
\coordinate (glapppgf) at (\glaxxxg, \glayyyf);
\coordinate (glapppgg) at (\glaxxxg, \glayyyg);
\coordinate (glapppgh) at (\glaxxxg, \glayyyh);
\coordinate (glapppgi) at (\glaxxxg, \glayyyi);
\coordinate (glapppgj) at (\glaxxxg, \glayyyj);
\coordinate (glapppgk) at (\glaxxxg, \glayyyk);
\coordinate (glapppgl) at (\glaxxxg, \glayyyl);
\coordinate (glapppgm) at (\glaxxxg, \glayyym);
\coordinate (glapppgn) at (\glaxxxg, \glayyyn);
\coordinate (glapppgo) at (\glaxxxg, \glayyyo);
\coordinate (glapppgp) at (\glaxxxg, \glayyyp);
\coordinate (glapppgq) at (\glaxxxg, \glayyyq);
\coordinate (glapppgr) at (\glaxxxg, \glayyyr);
\coordinate (glapppgs) at (\glaxxxg, \glayyys);
\coordinate (glapppgt) at (\glaxxxg, \glayyyt);
\coordinate (glapppgu) at (\glaxxxg, \glayyyu);
\coordinate (glapppgv) at (\glaxxxg, \glayyyv);
\coordinate (glapppgw) at (\glaxxxg, \glayyyw);
\coordinate (glapppgx) at (\glaxxxg, \glayyyx);
\coordinate (glapppgy) at (\glaxxxg, \glayyyy);
\coordinate (glapppgz) at (\glaxxxg, \glayyyz);
\coordinate (glapppha) at (\glaxxxh, \glayyya);
\coordinate (glappphb) at (\glaxxxh, \glayyyb);
\coordinate (glappphc) at (\glaxxxh, \glayyyc);
\coordinate (glappphd) at (\glaxxxh, \glayyyd);
\coordinate (glappphe) at (\glaxxxh, \glayyye);
\coordinate (glappphf) at (\glaxxxh, \glayyyf);
\coordinate (glappphg) at (\glaxxxh, \glayyyg);
\coordinate (glappphh) at (\glaxxxh, \glayyyh);
\coordinate (glappphi) at (\glaxxxh, \glayyyi);
\coordinate (glappphj) at (\glaxxxh, \glayyyj);
\coordinate (glappphk) at (\glaxxxh, \glayyyk);
\coordinate (glappphl) at (\glaxxxh, \glayyyl);
\coordinate (glappphm) at (\glaxxxh, \glayyym);
\coordinate (glappphn) at (\glaxxxh, \glayyyn);
\coordinate (glapppho) at (\glaxxxh, \glayyyo);
\coordinate (glappphp) at (\glaxxxh, \glayyyp);
\coordinate (glappphq) at (\glaxxxh, \glayyyq);
\coordinate (glappphr) at (\glaxxxh, \glayyyr);
\coordinate (glappphs) at (\glaxxxh, \glayyys);
\coordinate (glapppht) at (\glaxxxh, \glayyyt);
\coordinate (glappphu) at (\glaxxxh, \glayyyu);
\coordinate (glappphv) at (\glaxxxh, \glayyyv);
\coordinate (glappphw) at (\glaxxxh, \glayyyw);
\coordinate (glappphx) at (\glaxxxh, \glayyyx);
\coordinate (glappphy) at (\glaxxxh, \glayyyy);
\coordinate (glappphz) at (\glaxxxh, \glayyyz);
\coordinate (glapppia) at (\glaxxxi, \glayyya);
\coordinate (glapppib) at (\glaxxxi, \glayyyb);
\coordinate (glapppic) at (\glaxxxi, \glayyyc);
\coordinate (glapppid) at (\glaxxxi, \glayyyd);
\coordinate (glapppie) at (\glaxxxi, \glayyye);
\coordinate (glapppif) at (\glaxxxi, \glayyyf);
\coordinate (glapppig) at (\glaxxxi, \glayyyg);
\coordinate (glapppih) at (\glaxxxi, \glayyyh);
\coordinate (glapppii) at (\glaxxxi, \glayyyi);
\coordinate (glapppij) at (\glaxxxi, \glayyyj);
\coordinate (glapppik) at (\glaxxxi, \glayyyk);
\coordinate (glapppil) at (\glaxxxi, \glayyyl);
\coordinate (glapppim) at (\glaxxxi, \glayyym);
\coordinate (glapppin) at (\glaxxxi, \glayyyn);
\coordinate (glapppio) at (\glaxxxi, \glayyyo);
\coordinate (glapppip) at (\glaxxxi, \glayyyp);
\coordinate (glapppiq) at (\glaxxxi, \glayyyq);
\coordinate (glapppir) at (\glaxxxi, \glayyyr);
\coordinate (glapppis) at (\glaxxxi, \glayyys);
\coordinate (glapppit) at (\glaxxxi, \glayyyt);
\coordinate (glapppiu) at (\glaxxxi, \glayyyu);
\coordinate (glapppiv) at (\glaxxxi, \glayyyv);
\coordinate (glapppiw) at (\glaxxxi, \glayyyw);
\coordinate (glapppix) at (\glaxxxi, \glayyyx);
\coordinate (glapppiy) at (\glaxxxi, \glayyyy);
\coordinate (glapppiz) at (\glaxxxi, \glayyyz);
\coordinate (glapppja) at (\glaxxxj, \glayyya);
\coordinate (glapppjb) at (\glaxxxj, \glayyyb);
\coordinate (glapppjc) at (\glaxxxj, \glayyyc);
\coordinate (glapppjd) at (\glaxxxj, \glayyyd);
\coordinate (glapppje) at (\glaxxxj, \glayyye);
\coordinate (glapppjf) at (\glaxxxj, \glayyyf);
\coordinate (glapppjg) at (\glaxxxj, \glayyyg);
\coordinate (glapppjh) at (\glaxxxj, \glayyyh);
\coordinate (glapppji) at (\glaxxxj, \glayyyi);
\coordinate (glapppjj) at (\glaxxxj, \glayyyj);
\coordinate (glapppjk) at (\glaxxxj, \glayyyk);
\coordinate (glapppjl) at (\glaxxxj, \glayyyl);
\coordinate (glapppjm) at (\glaxxxj, \glayyym);
\coordinate (glapppjn) at (\glaxxxj, \glayyyn);
\coordinate (glapppjo) at (\glaxxxj, \glayyyo);
\coordinate (glapppjp) at (\glaxxxj, \glayyyp);
\coordinate (glapppjq) at (\glaxxxj, \glayyyq);
\coordinate (glapppjr) at (\glaxxxj, \glayyyr);
\coordinate (glapppjs) at (\glaxxxj, \glayyys);
\coordinate (glapppjt) at (\glaxxxj, \glayyyt);
\coordinate (glapppju) at (\glaxxxj, \glayyyu);
\coordinate (glapppjv) at (\glaxxxj, \glayyyv);
\coordinate (glapppjw) at (\glaxxxj, \glayyyw);
\coordinate (glapppjx) at (\glaxxxj, \glayyyx);
\coordinate (glapppjy) at (\glaxxxj, \glayyyy);
\coordinate (glapppjz) at (\glaxxxj, \glayyyz);
\coordinate (glapppka) at (\glaxxxk, \glayyya);
\coordinate (glapppkb) at (\glaxxxk, \glayyyb);
\coordinate (glapppkc) at (\glaxxxk, \glayyyc);
\coordinate (glapppkd) at (\glaxxxk, \glayyyd);
\coordinate (glapppke) at (\glaxxxk, \glayyye);
\coordinate (glapppkf) at (\glaxxxk, \glayyyf);
\coordinate (glapppkg) at (\glaxxxk, \glayyyg);
\coordinate (glapppkh) at (\glaxxxk, \glayyyh);
\coordinate (glapppki) at (\glaxxxk, \glayyyi);
\coordinate (glapppkj) at (\glaxxxk, \glayyyj);
\coordinate (glapppkk) at (\glaxxxk, \glayyyk);
\coordinate (glapppkl) at (\glaxxxk, \glayyyl);
\coordinate (glapppkm) at (\glaxxxk, \glayyym);
\coordinate (glapppkn) at (\glaxxxk, \glayyyn);
\coordinate (glapppko) at (\glaxxxk, \glayyyo);
\coordinate (glapppkp) at (\glaxxxk, \glayyyp);
\coordinate (glapppkq) at (\glaxxxk, \glayyyq);
\coordinate (glapppkr) at (\glaxxxk, \glayyyr);
\coordinate (glapppks) at (\glaxxxk, \glayyys);
\coordinate (glapppkt) at (\glaxxxk, \glayyyt);
\coordinate (glapppku) at (\glaxxxk, \glayyyu);
\coordinate (glapppkv) at (\glaxxxk, \glayyyv);
\coordinate (glapppkw) at (\glaxxxk, \glayyyw);
\coordinate (glapppkx) at (\glaxxxk, \glayyyx);
\coordinate (glapppky) at (\glaxxxk, \glayyyy);
\coordinate (glapppkz) at (\glaxxxk, \glayyyz);
\coordinate (glapppla) at (\glaxxxl, \glayyya);
\coordinate (glappplb) at (\glaxxxl, \glayyyb);
\coordinate (glappplc) at (\glaxxxl, \glayyyc);
\coordinate (glapppld) at (\glaxxxl, \glayyyd);
\coordinate (glappple) at (\glaxxxl, \glayyye);
\coordinate (glappplf) at (\glaxxxl, \glayyyf);
\coordinate (glappplg) at (\glaxxxl, \glayyyg);
\coordinate (glappplh) at (\glaxxxl, \glayyyh);
\coordinate (glapppli) at (\glaxxxl, \glayyyi);
\coordinate (glappplj) at (\glaxxxl, \glayyyj);
\coordinate (glappplk) at (\glaxxxl, \glayyyk);
\coordinate (glapppll) at (\glaxxxl, \glayyyl);
\coordinate (glappplm) at (\glaxxxl, \glayyym);
\coordinate (glapppln) at (\glaxxxl, \glayyyn);
\coordinate (glappplo) at (\glaxxxl, \glayyyo);
\coordinate (glappplp) at (\glaxxxl, \glayyyp);
\coordinate (glappplq) at (\glaxxxl, \glayyyq);
\coordinate (glappplr) at (\glaxxxl, \glayyyr);
\coordinate (glapppls) at (\glaxxxl, \glayyys);
\coordinate (glappplt) at (\glaxxxl, \glayyyt);
\coordinate (glappplu) at (\glaxxxl, \glayyyu);
\coordinate (glappplv) at (\glaxxxl, \glayyyv);
\coordinate (glappplw) at (\glaxxxl, \glayyyw);
\coordinate (glappplx) at (\glaxxxl, \glayyyx);
\coordinate (glappply) at (\glaxxxl, \glayyyy);
\coordinate (glappplz) at (\glaxxxl, \glayyyz);
\coordinate (glapppma) at (\glaxxxm, \glayyya);
\coordinate (glapppmb) at (\glaxxxm, \glayyyb);
\coordinate (glapppmc) at (\glaxxxm, \glayyyc);
\coordinate (glapppmd) at (\glaxxxm, \glayyyd);
\coordinate (glapppme) at (\glaxxxm, \glayyye);
\coordinate (glapppmf) at (\glaxxxm, \glayyyf);
\coordinate (glapppmg) at (\glaxxxm, \glayyyg);
\coordinate (glapppmh) at (\glaxxxm, \glayyyh);
\coordinate (glapppmi) at (\glaxxxm, \glayyyi);
\coordinate (glapppmj) at (\glaxxxm, \glayyyj);
\coordinate (glapppmk) at (\glaxxxm, \glayyyk);
\coordinate (glapppml) at (\glaxxxm, \glayyyl);
\coordinate (glapppmm) at (\glaxxxm, \glayyym);
\coordinate (glapppmn) at (\glaxxxm, \glayyyn);
\coordinate (glapppmo) at (\glaxxxm, \glayyyo);
\coordinate (glapppmp) at (\glaxxxm, \glayyyp);
\coordinate (glapppmq) at (\glaxxxm, \glayyyq);
\coordinate (glapppmr) at (\glaxxxm, \glayyyr);
\coordinate (glapppms) at (\glaxxxm, \glayyys);
\coordinate (glapppmt) at (\glaxxxm, \glayyyt);
\coordinate (glapppmu) at (\glaxxxm, \glayyyu);
\coordinate (glapppmv) at (\glaxxxm, \glayyyv);
\coordinate (glapppmw) at (\glaxxxm, \glayyyw);
\coordinate (glapppmx) at (\glaxxxm, \glayyyx);
\coordinate (glapppmy) at (\glaxxxm, \glayyyy);
\coordinate (glapppmz) at (\glaxxxm, \glayyyz);
\coordinate (glapppna) at (\glaxxxn, \glayyya);
\coordinate (glapppnb) at (\glaxxxn, \glayyyb);
\coordinate (glapppnc) at (\glaxxxn, \glayyyc);
\coordinate (glapppnd) at (\glaxxxn, \glayyyd);
\coordinate (glapppne) at (\glaxxxn, \glayyye);
\coordinate (glapppnf) at (\glaxxxn, \glayyyf);
\coordinate (glapppng) at (\glaxxxn, \glayyyg);
\coordinate (glapppnh) at (\glaxxxn, \glayyyh);
\coordinate (glapppni) at (\glaxxxn, \glayyyi);
\coordinate (glapppnj) at (\glaxxxn, \glayyyj);
\coordinate (glapppnk) at (\glaxxxn, \glayyyk);
\coordinate (glapppnl) at (\glaxxxn, \glayyyl);
\coordinate (glapppnm) at (\glaxxxn, \glayyym);
\coordinate (glapppnn) at (\glaxxxn, \glayyyn);
\coordinate (glapppno) at (\glaxxxn, \glayyyo);
\coordinate (glapppnp) at (\glaxxxn, \glayyyp);
\coordinate (glapppnq) at (\glaxxxn, \glayyyq);
\coordinate (glapppnr) at (\glaxxxn, \glayyyr);
\coordinate (glapppns) at (\glaxxxn, \glayyys);
\coordinate (glapppnt) at (\glaxxxn, \glayyyt);
\coordinate (glapppnu) at (\glaxxxn, \glayyyu);
\coordinate (glapppnv) at (\glaxxxn, \glayyyv);
\coordinate (glapppnw) at (\glaxxxn, \glayyyw);
\coordinate (glapppnx) at (\glaxxxn, \glayyyx);
\coordinate (glapppny) at (\glaxxxn, \glayyyy);
\coordinate (glapppnz) at (\glaxxxn, \glayyyz);
\coordinate (glapppoa) at (\glaxxxo, \glayyya);
\coordinate (glapppob) at (\glaxxxo, \glayyyb);
\coordinate (glapppoc) at (\glaxxxo, \glayyyc);
\coordinate (glapppod) at (\glaxxxo, \glayyyd);
\coordinate (glapppoe) at (\glaxxxo, \glayyye);
\coordinate (glapppof) at (\glaxxxo, \glayyyf);
\coordinate (glapppog) at (\glaxxxo, \glayyyg);
\coordinate (glapppoh) at (\glaxxxo, \glayyyh);
\coordinate (glapppoi) at (\glaxxxo, \glayyyi);
\coordinate (glapppoj) at (\glaxxxo, \glayyyj);
\coordinate (glapppok) at (\glaxxxo, \glayyyk);
\coordinate (glapppol) at (\glaxxxo, \glayyyl);
\coordinate (glapppom) at (\glaxxxo, \glayyym);
\coordinate (glapppon) at (\glaxxxo, \glayyyn);
\coordinate (glapppoo) at (\glaxxxo, \glayyyo);
\coordinate (glapppop) at (\glaxxxo, \glayyyp);
\coordinate (glapppoq) at (\glaxxxo, \glayyyq);
\coordinate (glapppor) at (\glaxxxo, \glayyyr);
\coordinate (glapppos) at (\glaxxxo, \glayyys);
\coordinate (glapppot) at (\glaxxxo, \glayyyt);
\coordinate (glapppou) at (\glaxxxo, \glayyyu);
\coordinate (glapppov) at (\glaxxxo, \glayyyv);
\coordinate (glapppow) at (\glaxxxo, \glayyyw);
\coordinate (glapppox) at (\glaxxxo, \glayyyx);
\coordinate (glapppoy) at (\glaxxxo, \glayyyy);
\coordinate (glapppoz) at (\glaxxxo, \glayyyz);
\coordinate (glappppa) at (\glaxxxp, \glayyya);
\coordinate (glappppb) at (\glaxxxp, \glayyyb);
\coordinate (glappppc) at (\glaxxxp, \glayyyc);
\coordinate (glappppd) at (\glaxxxp, \glayyyd);
\coordinate (glappppe) at (\glaxxxp, \glayyye);
\coordinate (glappppf) at (\glaxxxp, \glayyyf);
\coordinate (glappppg) at (\glaxxxp, \glayyyg);
\coordinate (glapppph) at (\glaxxxp, \glayyyh);
\coordinate (glappppi) at (\glaxxxp, \glayyyi);
\coordinate (glappppj) at (\glaxxxp, \glayyyj);
\coordinate (glappppk) at (\glaxxxp, \glayyyk);
\coordinate (glappppl) at (\glaxxxp, \glayyyl);
\coordinate (glappppm) at (\glaxxxp, \glayyym);
\coordinate (glappppn) at (\glaxxxp, \glayyyn);
\coordinate (glappppo) at (\glaxxxp, \glayyyo);
\coordinate (glappppp) at (\glaxxxp, \glayyyp);
\coordinate (glappppq) at (\glaxxxp, \glayyyq);
\coordinate (glappppr) at (\glaxxxp, \glayyyr);
\coordinate (glapppps) at (\glaxxxp, \glayyys);
\coordinate (glappppt) at (\glaxxxp, \glayyyt);
\coordinate (glappppu) at (\glaxxxp, \glayyyu);
\coordinate (glappppv) at (\glaxxxp, \glayyyv);
\coordinate (glappppw) at (\glaxxxp, \glayyyw);
\coordinate (glappppx) at (\glaxxxp, \glayyyx);
\coordinate (glappppy) at (\glaxxxp, \glayyyy);
\coordinate (glappppz) at (\glaxxxp, \glayyyz);
\coordinate (glapppqa) at (\glaxxxq, \glayyya);
\coordinate (glapppqb) at (\glaxxxq, \glayyyb);
\coordinate (glapppqc) at (\glaxxxq, \glayyyc);
\coordinate (glapppqd) at (\glaxxxq, \glayyyd);
\coordinate (glapppqe) at (\glaxxxq, \glayyye);
\coordinate (glapppqf) at (\glaxxxq, \glayyyf);
\coordinate (glapppqg) at (\glaxxxq, \glayyyg);
\coordinate (glapppqh) at (\glaxxxq, \glayyyh);
\coordinate (glapppqi) at (\glaxxxq, \glayyyi);
\coordinate (glapppqj) at (\glaxxxq, \glayyyj);
\coordinate (glapppqk) at (\glaxxxq, \glayyyk);
\coordinate (glapppql) at (\glaxxxq, \glayyyl);
\coordinate (glapppqm) at (\glaxxxq, \glayyym);
\coordinate (glapppqn) at (\glaxxxq, \glayyyn);
\coordinate (glapppqo) at (\glaxxxq, \glayyyo);
\coordinate (glapppqp) at (\glaxxxq, \glayyyp);
\coordinate (glapppqq) at (\glaxxxq, \glayyyq);
\coordinate (glapppqr) at (\glaxxxq, \glayyyr);
\coordinate (glapppqs) at (\glaxxxq, \glayyys);
\coordinate (glapppqt) at (\glaxxxq, \glayyyt);
\coordinate (glapppqu) at (\glaxxxq, \glayyyu);
\coordinate (glapppqv) at (\glaxxxq, \glayyyv);
\coordinate (glapppqw) at (\glaxxxq, \glayyyw);
\coordinate (glapppqx) at (\glaxxxq, \glayyyx);
\coordinate (glapppqy) at (\glaxxxq, \glayyyy);
\coordinate (glapppqz) at (\glaxxxq, \glayyyz);
\coordinate (glapppra) at (\glaxxxr, \glayyya);
\coordinate (glappprb) at (\glaxxxr, \glayyyb);
\coordinate (glappprc) at (\glaxxxr, \glayyyc);
\coordinate (glappprd) at (\glaxxxr, \glayyyd);
\coordinate (glapppre) at (\glaxxxr, \glayyye);
\coordinate (glappprf) at (\glaxxxr, \glayyyf);
\coordinate (glappprg) at (\glaxxxr, \glayyyg);
\coordinate (glappprh) at (\glaxxxr, \glayyyh);
\coordinate (glapppri) at (\glaxxxr, \glayyyi);
\coordinate (glappprj) at (\glaxxxr, \glayyyj);
\coordinate (glappprk) at (\glaxxxr, \glayyyk);
\coordinate (glappprl) at (\glaxxxr, \glayyyl);
\coordinate (glappprm) at (\glaxxxr, \glayyym);
\coordinate (glappprn) at (\glaxxxr, \glayyyn);
\coordinate (glapppro) at (\glaxxxr, \glayyyo);
\coordinate (glappprp) at (\glaxxxr, \glayyyp);
\coordinate (glappprq) at (\glaxxxr, \glayyyq);
\coordinate (glappprr) at (\glaxxxr, \glayyyr);
\coordinate (glappprs) at (\glaxxxr, \glayyys);
\coordinate (glappprt) at (\glaxxxr, \glayyyt);
\coordinate (glapppru) at (\glaxxxr, \glayyyu);
\coordinate (glappprv) at (\glaxxxr, \glayyyv);
\coordinate (glappprw) at (\glaxxxr, \glayyyw);
\coordinate (glappprx) at (\glaxxxr, \glayyyx);
\coordinate (glapppry) at (\glaxxxr, \glayyyy);
\coordinate (glappprz) at (\glaxxxr, \glayyyz);
\coordinate (glapppsa) at (\glaxxxs, \glayyya);
\coordinate (glapppsb) at (\glaxxxs, \glayyyb);
\coordinate (glapppsc) at (\glaxxxs, \glayyyc);
\coordinate (glapppsd) at (\glaxxxs, \glayyyd);
\coordinate (glapppse) at (\glaxxxs, \glayyye);
\coordinate (glapppsf) at (\glaxxxs, \glayyyf);
\coordinate (glapppsg) at (\glaxxxs, \glayyyg);
\coordinate (glapppsh) at (\glaxxxs, \glayyyh);
\coordinate (glapppsi) at (\glaxxxs, \glayyyi);
\coordinate (glapppsj) at (\glaxxxs, \glayyyj);
\coordinate (glapppsk) at (\glaxxxs, \glayyyk);
\coordinate (glapppsl) at (\glaxxxs, \glayyyl);
\coordinate (glapppsm) at (\glaxxxs, \glayyym);
\coordinate (glapppsn) at (\glaxxxs, \glayyyn);
\coordinate (glapppso) at (\glaxxxs, \glayyyo);
\coordinate (glapppsp) at (\glaxxxs, \glayyyp);
\coordinate (glapppsq) at (\glaxxxs, \glayyyq);
\coordinate (glapppsr) at (\glaxxxs, \glayyyr);
\coordinate (glapppss) at (\glaxxxs, \glayyys);
\coordinate (glapppst) at (\glaxxxs, \glayyyt);
\coordinate (glapppsu) at (\glaxxxs, \glayyyu);
\coordinate (glapppsv) at (\glaxxxs, \glayyyv);
\coordinate (glapppsw) at (\glaxxxs, \glayyyw);
\coordinate (glapppsx) at (\glaxxxs, \glayyyx);
\coordinate (glapppsy) at (\glaxxxs, \glayyyy);
\coordinate (glapppsz) at (\glaxxxs, \glayyyz);
\coordinate (glapppta) at (\glaxxxt, \glayyya);
\coordinate (glappptb) at (\glaxxxt, \glayyyb);
\coordinate (glappptc) at (\glaxxxt, \glayyyc);
\coordinate (glappptd) at (\glaxxxt, \glayyyd);
\coordinate (glapppte) at (\glaxxxt, \glayyye);
\coordinate (glappptf) at (\glaxxxt, \glayyyf);
\coordinate (glappptg) at (\glaxxxt, \glayyyg);
\coordinate (glapppth) at (\glaxxxt, \glayyyh);
\coordinate (glapppti) at (\glaxxxt, \glayyyi);
\coordinate (glappptj) at (\glaxxxt, \glayyyj);
\coordinate (glappptk) at (\glaxxxt, \glayyyk);
\coordinate (glappptl) at (\glaxxxt, \glayyyl);
\coordinate (glappptm) at (\glaxxxt, \glayyym);
\coordinate (glappptn) at (\glaxxxt, \glayyyn);
\coordinate (glapppto) at (\glaxxxt, \glayyyo);
\coordinate (glappptp) at (\glaxxxt, \glayyyp);
\coordinate (glappptq) at (\glaxxxt, \glayyyq);
\coordinate (glappptr) at (\glaxxxt, \glayyyr);
\coordinate (glapppts) at (\glaxxxt, \glayyys);
\coordinate (glappptt) at (\glaxxxt, \glayyyt);
\coordinate (glappptu) at (\glaxxxt, \glayyyu);
\coordinate (glappptv) at (\glaxxxt, \glayyyv);
\coordinate (glappptw) at (\glaxxxt, \glayyyw);
\coordinate (glappptx) at (\glaxxxt, \glayyyx);
\coordinate (glapppty) at (\glaxxxt, \glayyyy);
\coordinate (glappptz) at (\glaxxxt, \glayyyz);
\coordinate (glapppua) at (\glaxxxu, \glayyya);
\coordinate (glapppub) at (\glaxxxu, \glayyyb);
\coordinate (glapppuc) at (\glaxxxu, \glayyyc);
\coordinate (glapppud) at (\glaxxxu, \glayyyd);
\coordinate (glapppue) at (\glaxxxu, \glayyye);
\coordinate (glapppuf) at (\glaxxxu, \glayyyf);
\coordinate (glapppug) at (\glaxxxu, \glayyyg);
\coordinate (glapppuh) at (\glaxxxu, \glayyyh);
\coordinate (glapppui) at (\glaxxxu, \glayyyi);
\coordinate (glapppuj) at (\glaxxxu, \glayyyj);
\coordinate (glapppuk) at (\glaxxxu, \glayyyk);
\coordinate (glapppul) at (\glaxxxu, \glayyyl);
\coordinate (glapppum) at (\glaxxxu, \glayyym);
\coordinate (glapppun) at (\glaxxxu, \glayyyn);
\coordinate (glapppuo) at (\glaxxxu, \glayyyo);
\coordinate (glapppup) at (\glaxxxu, \glayyyp);
\coordinate (glapppuq) at (\glaxxxu, \glayyyq);
\coordinate (glapppur) at (\glaxxxu, \glayyyr);
\coordinate (glapppus) at (\glaxxxu, \glayyys);
\coordinate (glappput) at (\glaxxxu, \glayyyt);
\coordinate (glapppuu) at (\glaxxxu, \glayyyu);
\coordinate (glapppuv) at (\glaxxxu, \glayyyv);
\coordinate (glapppuw) at (\glaxxxu, \glayyyw);
\coordinate (glapppux) at (\glaxxxu, \glayyyx);
\coordinate (glapppuy) at (\glaxxxu, \glayyyy);
\coordinate (glapppuz) at (\glaxxxu, \glayyyz);
\coordinate (glapppva) at (\glaxxxv, \glayyya);
\coordinate (glapppvb) at (\glaxxxv, \glayyyb);
\coordinate (glapppvc) at (\glaxxxv, \glayyyc);
\coordinate (glapppvd) at (\glaxxxv, \glayyyd);
\coordinate (glapppve) at (\glaxxxv, \glayyye);
\coordinate (glapppvf) at (\glaxxxv, \glayyyf);
\coordinate (glapppvg) at (\glaxxxv, \glayyyg);
\coordinate (glapppvh) at (\glaxxxv, \glayyyh);
\coordinate (glapppvi) at (\glaxxxv, \glayyyi);
\coordinate (glapppvj) at (\glaxxxv, \glayyyj);
\coordinate (glapppvk) at (\glaxxxv, \glayyyk);
\coordinate (glapppvl) at (\glaxxxv, \glayyyl);
\coordinate (glapppvm) at (\glaxxxv, \glayyym);
\coordinate (glapppvn) at (\glaxxxv, \glayyyn);
\coordinate (glapppvo) at (\glaxxxv, \glayyyo);
\coordinate (glapppvp) at (\glaxxxv, \glayyyp);
\coordinate (glapppvq) at (\glaxxxv, \glayyyq);
\coordinate (glapppvr) at (\glaxxxv, \glayyyr);
\coordinate (glapppvs) at (\glaxxxv, \glayyys);
\coordinate (glapppvt) at (\glaxxxv, \glayyyt);
\coordinate (glapppvu) at (\glaxxxv, \glayyyu);
\coordinate (glapppvv) at (\glaxxxv, \glayyyv);
\coordinate (glapppvw) at (\glaxxxv, \glayyyw);
\coordinate (glapppvx) at (\glaxxxv, \glayyyx);
\coordinate (glapppvy) at (\glaxxxv, \glayyyy);
\coordinate (glapppvz) at (\glaxxxv, \glayyyz);
\coordinate (glapppwa) at (\glaxxxw, \glayyya);
\coordinate (glapppwb) at (\glaxxxw, \glayyyb);
\coordinate (glapppwc) at (\glaxxxw, \glayyyc);
\coordinate (glapppwd) at (\glaxxxw, \glayyyd);
\coordinate (glapppwe) at (\glaxxxw, \glayyye);
\coordinate (glapppwf) at (\glaxxxw, \glayyyf);
\coordinate (glapppwg) at (\glaxxxw, \glayyyg);
\coordinate (glapppwh) at (\glaxxxw, \glayyyh);
\coordinate (glapppwi) at (\glaxxxw, \glayyyi);
\coordinate (glapppwj) at (\glaxxxw, \glayyyj);
\coordinate (glapppwk) at (\glaxxxw, \glayyyk);
\coordinate (glapppwl) at (\glaxxxw, \glayyyl);
\coordinate (glapppwm) at (\glaxxxw, \glayyym);
\coordinate (glapppwn) at (\glaxxxw, \glayyyn);
\coordinate (glapppwo) at (\glaxxxw, \glayyyo);
\coordinate (glapppwp) at (\glaxxxw, \glayyyp);
\coordinate (glapppwq) at (\glaxxxw, \glayyyq);
\coordinate (glapppwr) at (\glaxxxw, \glayyyr);
\coordinate (glapppws) at (\glaxxxw, \glayyys);
\coordinate (glapppwt) at (\glaxxxw, \glayyyt);
\coordinate (glapppwu) at (\glaxxxw, \glayyyu);
\coordinate (glapppwv) at (\glaxxxw, \glayyyv);
\coordinate (glapppww) at (\glaxxxw, \glayyyw);
\coordinate (glapppwx) at (\glaxxxw, \glayyyx);
\coordinate (glapppwy) at (\glaxxxw, \glayyyy);
\coordinate (glapppwz) at (\glaxxxw, \glayyyz);
\coordinate (glapppxa) at (\glaxxxx, \glayyya);
\coordinate (glapppxb) at (\glaxxxx, \glayyyb);
\coordinate (glapppxc) at (\glaxxxx, \glayyyc);
\coordinate (glapppxd) at (\glaxxxx, \glayyyd);
\coordinate (glapppxe) at (\glaxxxx, \glayyye);
\coordinate (glapppxf) at (\glaxxxx, \glayyyf);
\coordinate (glapppxg) at (\glaxxxx, \glayyyg);
\coordinate (glapppxh) at (\glaxxxx, \glayyyh);
\coordinate (glapppxi) at (\glaxxxx, \glayyyi);
\coordinate (glapppxj) at (\glaxxxx, \glayyyj);
\coordinate (glapppxk) at (\glaxxxx, \glayyyk);
\coordinate (glapppxl) at (\glaxxxx, \glayyyl);
\coordinate (glapppxm) at (\glaxxxx, \glayyym);
\coordinate (glapppxn) at (\glaxxxx, \glayyyn);
\coordinate (glapppxo) at (\glaxxxx, \glayyyo);
\coordinate (glapppxp) at (\glaxxxx, \glayyyp);
\coordinate (glapppxq) at (\glaxxxx, \glayyyq);
\coordinate (glapppxr) at (\glaxxxx, \glayyyr);
\coordinate (glapppxs) at (\glaxxxx, \glayyys);
\coordinate (glapppxt) at (\glaxxxx, \glayyyt);
\coordinate (glapppxu) at (\glaxxxx, \glayyyu);
\coordinate (glapppxv) at (\glaxxxx, \glayyyv);
\coordinate (glapppxw) at (\glaxxxx, \glayyyw);
\coordinate (glapppxx) at (\glaxxxx, \glayyyx);
\coordinate (glapppxy) at (\glaxxxx, \glayyyy);
\coordinate (glapppxz) at (\glaxxxx, \glayyyz);
\coordinate (glapppya) at (\glaxxxy, \glayyya);
\coordinate (glapppyb) at (\glaxxxy, \glayyyb);
\coordinate (glapppyc) at (\glaxxxy, \glayyyc);
\coordinate (glapppyd) at (\glaxxxy, \glayyyd);
\coordinate (glapppye) at (\glaxxxy, \glayyye);
\coordinate (glapppyf) at (\glaxxxy, \glayyyf);
\coordinate (glapppyg) at (\glaxxxy, \glayyyg);
\coordinate (glapppyh) at (\glaxxxy, \glayyyh);
\coordinate (glapppyi) at (\glaxxxy, \glayyyi);
\coordinate (glapppyj) at (\glaxxxy, \glayyyj);
\coordinate (glapppyk) at (\glaxxxy, \glayyyk);
\coordinate (glapppyl) at (\glaxxxy, \glayyyl);
\coordinate (glapppym) at (\glaxxxy, \glayyym);
\coordinate (glapppyn) at (\glaxxxy, \glayyyn);
\coordinate (glapppyo) at (\glaxxxy, \glayyyo);
\coordinate (glapppyp) at (\glaxxxy, \glayyyp);
\coordinate (glapppyq) at (\glaxxxy, \glayyyq);
\coordinate (glapppyr) at (\glaxxxy, \glayyyr);
\coordinate (glapppys) at (\glaxxxy, \glayyys);
\coordinate (glapppyt) at (\glaxxxy, \glayyyt);
\coordinate (glapppyu) at (\glaxxxy, \glayyyu);
\coordinate (glapppyv) at (\glaxxxy, \glayyyv);
\coordinate (glapppyw) at (\glaxxxy, \glayyyw);
\coordinate (glapppyx) at (\glaxxxy, \glayyyx);
\coordinate (glapppyy) at (\glaxxxy, \glayyyy);
\coordinate (glapppyz) at (\glaxxxy, \glayyyz);
\coordinate (glapppza) at (\glaxxxz, \glayyya);
\coordinate (glapppzb) at (\glaxxxz, \glayyyb);
\coordinate (glapppzc) at (\glaxxxz, \glayyyc);
\coordinate (glapppzd) at (\glaxxxz, \glayyyd);
\coordinate (glapppze) at (\glaxxxz, \glayyye);
\coordinate (glapppzf) at (\glaxxxz, \glayyyf);
\coordinate (glapppzg) at (\glaxxxz, \glayyyg);
\coordinate (glapppzh) at (\glaxxxz, \glayyyh);
\coordinate (glapppzi) at (\glaxxxz, \glayyyi);
\coordinate (glapppzj) at (\glaxxxz, \glayyyj);
\coordinate (glapppzk) at (\glaxxxz, \glayyyk);
\coordinate (glapppzl) at (\glaxxxz, \glayyyl);
\coordinate (glapppzm) at (\glaxxxz, \glayyym);
\coordinate (glapppzn) at (\glaxxxz, \glayyyn);
\coordinate (glapppzo) at (\glaxxxz, \glayyyo);
\coordinate (glapppzp) at (\glaxxxz, \glayyyp);
\coordinate (glapppzq) at (\glaxxxz, \glayyyq);
\coordinate (glapppzr) at (\glaxxxz, \glayyyr);
\coordinate (glapppzs) at (\glaxxxz, \glayyys);
\coordinate (glapppzt) at (\glaxxxz, \glayyyt);
\coordinate (glapppzu) at (\glaxxxz, \glayyyu);
\coordinate (glapppzv) at (\glaxxxz, \glayyyv);
\coordinate (glapppzw) at (\glaxxxz, \glayyyw);
\coordinate (glapppzx) at (\glaxxxz, \glayyyx);
\coordinate (glapppzy) at (\glaxxxz, \glayyyy);
\coordinate (glapppzz) at (\glaxxxz, \glayyyz);

%\gangprintcoordinateat{(0,0)}{The last coordinate values: }{($(glapppzz)$)}; 








% Draw related part of the coordinate system with dashed helplines with letters as background. 
%\coordinatebackgroundxy{gla}{b}{c}{v}{a}{b}{s};

\draw (glapppbr) -- (glapppjr);

\node [nigfetd](nigfetd) at (glapppon) {F3055L};

\node [anchor=south] at (nigfetd.G) {G};
\node [anchor= west] at (nigfetd.D) {D};
\node [anchor= west] at (nigfetd.S) {S};

% To retrieve x- and y-component of the coordinates  (nigfetd.G), (nigfetd.D), and (nigfetd.S) separately. 
\getxyingivenunit{cm}{(nigfetd.G)}
                     {\nigfetdgx} {\nigfetdgy};
\getxyingivenunit{cm}{(nigfetd.D)}
                     {\nigfetddx} {\nigfetddy};
\getxyingivenunit{cm}{(nigfetd.S)}
                     {\nigfetdsx} {\nigfetdsy};

\draw  (glapppqr) -- 
       (glapppqq)
       to [Telmech=M2, n=motor]
       (\nigfetddx, \glayyyq) --
       (nigfetd.D);

\node [xshift=-2mm] at (motor.block north east) {$-$};
\node [xshift= 2mm] at (motor.block south east) {$+$};

\draw  (glapppqq) -- 
       (glapppqp)
       to [full diode = 1N4001] 
       (\nigfetddx, \glayyyp);

\draw  (nigfetd.S) -- 
       (\nigfetdsx, \glayyyk)
          node [ground] {};

 
% To draw LM555
\draw [blue, line width=0.5mm] 
      (glappphk) rectangle (glapppkq);
 
\node [blue, xshift=4mm] at (glapppio)
      {\underline{LM555}};

\draw (glapppir) -- 
      (glapppiq) node [anchor=north] {8};

\draw (glapppjr) -- 
      (glapppjq) node [anchor=north] {4};


\draw (nigfetd.G) 
      to [R, l_=$R_2 \text{=} 330 \Omega $] 
      (\glaxxxk, \nigfetdgy) 
      node [anchor=east] {3};
 
\draw (glapppkm) node [anchor=east] {1}  --
      (glapppom) ;

\draw (glapppkl) node [anchor=east] {5} 
      to [C, l_=$C_2  \text{=} 0.01 \mu F$] 
      (glapppnl) -- 
      (\nigfetdsx, \glayyyl);


\draw (glapppdr) 
      to [R = $R_1 \text{=} 1k \Omega$] 
      (glapppdp) -- 
      (glappphp) node [anchor=west] {7};
 
\draw (glapppdn) 
      to [full diode = 1N4001, label/align=rotate]
      (glapppdp);
 
\draw (glapppgp) 
      to [full diode = 1N4001, label/align=rotate]
      (glapppgn);
 

\draw (glapppgn) 
      to [potentiometer, l_=$R_3\text{=} 100k \Omega$,                                         n=mypot]
      (glapppdn);

\getxyingivenunit{cm}{(mypot.wiper)}
                     {\mypotwiperx}{\mypotwipery};


\draw (mypot.wiper) -- 
      (\mypotwiperx, \glayyym) -- 
      (glappphm) node [anchor=west] {6};

\draw  (\mypotwiperx, \glayyym) -- 
       (\mypotwiperx, \glayyyl) -- 
       (glappphl) node [anchor=west] {2};
 

\draw  (\mypotwiperx, \glayyyl) 
       to [C, n = capacitorl] 
       (\mypotwiperx, \glayyyk) node[ground]{};

\node [anchor=north west, xshift=2mm, yshift=.7mm] 
      at (capacitorl) {$C_1 \text{=} 0.1 \mu F$};





%%%%%%%%%%-----------------------------------
%%%%%%%%%%-----------------------------------
%%%%%%%%  repeat here
%%%%%%%%%%-----------------------------------
%%%%%%%%%%-----------------------------------



% Draw related part of the coordinate system with dashed helplines with letters as background. 
%\coordinatebackgroundxy{glb}{b}{c}{v}{a}{b}{s};

\draw (glbpppbr) -- (glbpppjr);

\node [nigfetd](nigfetd) at (glbpppon) {F3055L};

\node [anchor=south] at (nigfetd.G) {G};
\node [anchor= west] at (nigfetd.D) {D};
\node [anchor= west] at (nigfetd.S) {S};

% To retrieve x- and y-component of the coordinates  (nigfetd.G), (nigfetd.D), and (nigfetd.S) separately. 
\getxyingivenunit{cm}{(nigfetd.G)}
                     {\nigfetdgx} {\nigfetdgy};
\getxyingivenunit{cm}{(nigfetd.D)}
                     {\nigfetddx} {\nigfetddy};
\getxyingivenunit{cm}{(nigfetd.S)}
                     {\nigfetdsx} {\nigfetdsy};

\draw  (glbpppqr) -- 
       (glbpppqq)
       to [Telmech=M3, n=motor]
       (\nigfetddx, \glbyyyq) --
       (nigfetd.D);

\node [xshift=-2mm] at (motor.block north east) {$-$};
\node [xshift= 2mm] at (motor.block south east) {$+$};

\draw  (glbpppqq) -- 
       (glbpppqp)
       to [full diode = 1N4001] 
       (\nigfetddx, \glbyyyp);

\draw  (nigfetd.S) -- 
       (\nigfetdsx, \glbyyyk)
          node [ground] {};

 
% To draw LM555
\draw [blue, line width=0.5mm] 
      (glbppphk) rectangle (glbpppkq);
 
\node [blue, xshift=4mm] at (glbpppio)
      {\underline{LM555}};

\draw (glbpppir) -- 
      (glbpppiq) node [anchor=north] {8};

\draw (glbpppjr) -- 
      (glbpppjq) node [anchor=north] {4};


\draw (nigfetd.G) 
      to [R, l_=$R_2 \text{=} 330 \Omega $] 
      (\glbxxxk, \nigfetdgy) 
      node [anchor=east] {3};
 
\draw (glbpppkm) node [anchor=east] {1}  --
      (glbpppom) ;

\draw (glbpppkl) node [anchor=east] {5} 
      to [C, l_=$C_2  \text{=} 0.01 \mu F$] 
      (glbpppnl) -- 
      (\nigfetdsx, \glbyyyl);


\draw (glbpppdr) 
      to [R = $R_1 \text{=} 1k \Omega$] 
      (glbpppdp) -- 
      (glbppphp) node [anchor=west] {7};
 
\draw (glbpppdn) 
      to [full diode = 1N4001, label/align=rotate]
      (glbpppdp);
 
\draw (glbpppgp) 
      to [full diode = 1N4001, label/align=rotate]
      (glbpppgn);
 

\draw (glbpppgn) 
      to [potentiometer, l_=$R_3\text{=} 100k \Omega$,                                         n=mypot]
      (glbpppdn);

\getxyingivenunit{cm}{(mypot.wiper)}
                     {\mypotwiperx}{\mypotwipery};


\draw (mypot.wiper) -- 
      (\mypotwiperx, \glbyyym) -- 
      (glbppphm) node [anchor=west] {6};

\draw  (\mypotwiperx, \glbyyym) -- 
       (\mypotwiperx, \glbyyyl) -- 
       (glbppphl) node [anchor=west] {2};
 

\draw  (\mypotwiperx, \glbyyyl) 
       to [C, n = capacitorl] 
       (\mypotwiperx, \glbyyyk) node[ground]{};

\node [anchor=north west, xshift=2mm, yshift=.7mm] 
      at (capacitorl) {$C_1 \text{=} 0.1 \mu F$};


\draw [] (gangliupppqp) -- (glapppqr);
\draw [] (glapppqp) -- (glbpppqr);
\draw [o-] (gangliupppbs) node [anchor = south] {5V}  -- (glbpppbr);

\draw [white, *-*] (\gangliuxxxa - 3.0, \gangliuyyyv)   -- (gangliupppkv) ;


\end{circuitikz}



\end{document}
